%#!luajitlatex
\documentclass{ltjsarticle}
\usepackage[ipaex]{luatexja-preset}


\def\test#1{\vrule\hbox spread\zw{#1}\vrule}

\def\testH#1{\vtop{\hsize=21\zw #1%
\par\vrule{%
  \ltjsetparameter{kanjiskip=0pt plus 3\zw}%
  \hbox to 20\zw{あ「い」う,えお}%
}\vrule\par
\test{あ(…)ああ}\qquad
\test{あ(¥)ああ}\par
\test{あ(%)ああ}\qquad
\test{あ(ー)ああ}\par
\test{あ(あ)ああ}\qquad
\test{あ(ア)ああ}\par
\test{あ(漢)ああ}\qquad
\test{あ(0)ああ}\par
\test{… ………}\qquad
\test{¥ ¥ %}\par
\test{% % あ}\qquad
\test{ー ー あ}\par
\test{あ あ ア}\qquad
\test{0 0 ー}\par
\test{佐々十郎}\qquad
\test{大村 崑}\par
\test{岡 八郎}\qquad
\test{花紀 京}}\par\bigskip}

\begin{document}
1行目の例は「TeXでDTP―min10.tfmやjis.tfmの問題点」\hfil\break
(渡邉たけしさん,\texttt{http://www.dab.hi-ho.ne.jp/t-wara/tex/min10.html})から引用.

1行目以外の例は「プリセットの「文字組アキ量設定」における設定値の齟齬と回避策」\hfil\break
(なんでやねんDTPさん,\texttt{http://d.hatena.ne.jp/works014/20150926})から引用.


\parindent0pt
{\gt\bf 標準メトリック}

\testH\yoko
\testH\tate

\newpage
\addjfontfeatures{Kanjiskip=False}

{\gt\bf\verb+\addjfontfeatures{Kanjiskip=False}+ 指定(旧バージョンの組み方)}

\testH\yoko
\testH\tate


\newpage
\addjfontfeatures{Kanjiskip}

{\gt\bf さらに~\verb+\addjfontfeatures{Kanjiskip}+ 指定(現行の組み方)}

\testH\yoko
\testH\tate


\end{document}
%#! lualatex
\documentclass{article}
\usepackage[margin=15mm]{geometry}

\usepackage{luatexja}
\usepackage{luatexja-fontspec}

\defaultfontfeatures{Numbers=OldStyle}
\setmainfont{TeXGyreTermes}
\setsansfont{TeXGyreHeros}

\defaultjfontfeatures{Scale=1.2}
\setmainjfont[BoldFont=IPAexGothic]{IPAexMincho} % default: JFM=ujis
\setsansjfont{IPAexGothic}

\newjfontfamily\ipajisninety[CJKShape=JIS1990]{IPAexMincho}

\begin{document}

\section{fontspecのテスト}

※テストのため,欧文文字に対して和文文字のサイズを大きくしています.

\noindent 通常:「あいうえお」、(かきくけこ)。{\bf 太字}と{\gt ゴシック}。

{\addjfontfeatures{JFM=mono}
\noindent 等幅:「あいうえお」、(かきくけこ)。{\bf 太字}と{\gt ゴシック}。
}

\subsection{JIS2004}
逢芦飴溢茨鰯淫迂厩噂餌襖迦牙廻恢晦蟹葛鞄釜翰翫徽
祇汲灸笈卿饗僅喰櫛屑粂祁隙倦捲牽鍵諺巷梗膏鵠甑叉
榊薩鯖錆鮫餐杓灼酋楯薯藷哨鞘杖蝕訊逗摺撰煎煽穿箭
詮噌遡揃遜腿蛸辿樽歎註瀦捗槌鎚辻挺鄭擢溺兎堵屠賭
瀞遁謎灘楢禰牌這秤駁箸叛挽誹樋稗逼謬豹廟瀕斧蔽瞥
蔑篇娩鞭庖蓬鱒迄儲餅籾爺鑓愈猷漣煉簾榔屢冤叟咬嘲
囀徘扁棘橙狡甕甦疼祟竈筵篝腱艘芒虔蜃蠅訝靄靱騙鴉

\subsection{JIS1990}
{\ipajisninety
逢芦飴溢茨鰯淫迂厩噂餌襖迦牙廻恢晦蟹葛鞄釜翰翫徽
祇汲灸笈卿饗僅喰櫛屑粂祁隙倦捲牽鍵諺巷梗膏鵠甑叉
榊薩鯖錆鮫餐杓灼酋楯薯藷哨鞘杖蝕訊逗摺撰煎煽穿箭
詮噌遡揃遜腿蛸辿樽歎註瀦捗槌鎚辻挺鄭擢溺兎堵屠賭
瀞遁謎灘楢禰牌這秤駁箸叛挽誹樋稗逼謬豹廟瀕斧蔽瞥
蔑篇娩鞭庖蓬鱒迄儲餅籾爺鑓愈猷漣煉簾榔屢冤叟咬嘲
囀徘扁棘橙狡甕甦疼祟竈筵篝腱艘芒虔蜃蠅訝靄靱騙鴉

\noindent
\addjfontfeatures{LetterSpace=50}
逢芦飴溢茨鰯淫迂厩噂餌襖迦牙廻恢晦蟹葛鞄釜翰翫徽
祇汲灸笈卿饗僅喰櫛屑粂祁隙倦捲牽鍵諺巷梗膏鵠甑叉
榊薩鯖錆鮫餐杓灼酋楯薯藷哨鞘杖蝕訊逗摺撰煎煽穿箭
詮噌遡揃遜腿蛸辿樽歎註瀦捗槌鎚辻挺鄭擢溺兎堵屠賭
瀞遁謎灘楢禰牌這秤駁箸叛挽誹樋稗逼謬豹廟瀕斧蔽瞥
蔑篇娩鞭庖蓬鱒迄儲餅籾爺鑓愈猷漣煉簾榔屢冤叟咬嘲
囀徘扁棘橙狡甕甦疼祟竈筵篝腱艘芒虔蜃蠅訝靄靱騙鴉
}

\vspace{1\zw}

\subsection{Kerning}
{\jfontspec[NoEmbed]{Ryumin-Light}

アノ ← Kerning=Off

\noindent{\addjfontfeatures{Kerning=On} アノ ← Kerning=On}
}

\subsection{unicode}

設定依存:「\char"201C」「\char"010F」%"
常に和文:「\ltjjachar"201C」「\ltjjachar"010F」%"
常に欧文:「\ltjalchar"201C」「\ltjalchar"010F」%"

{\tracingall\let\char=\ltjalchar\textquotedblleft}あ\textquotedblleft あ%
{\let\char=\ltjjachar\textquotedblleft}あ“あ

\begin{itemize}
\item hoge
\begin{itemize}
\item hoge\textendash\ltjjachar"2013\ltjalchar"2013
\begin{itemize}
\item hoge
\begin{itemize}
\item hoge|\textperiodcentered|\ltjjachar"00B7|\ltjalchar"00B7|・|
\end{itemize}
\end{itemize}
\end{itemize}
\end{itemize}

\addfontfeatures{Color=BD6D8F}
\addjfontfeatures{Color=28AFCB,CJKShape=JIS1990}
あいうえお芦漢字ABC
\[
あいうえお芦\int_0^∞ e^{-x^2}\,dx\mathrm{ABC}\textrm{ABC}\textsf{ABC}
\]

\end{document}

%#! lualatex
\documentclass{article}

\makeatletter
\let\ltj@orig@DeclareFontShape@=\DeclareFontShape@
\def\DeclareFontShape@#1#2#3#4#5#6{%
\immediate\typeout{DFS |#1|#2|#3|#4|#5|#6|}%
\ltj@orig@DeclareFontShape@{#1}{#2}{#3}{#4}{#5}{#6}%
}
\makeatother


\usepackage{luatexja}
\usepackage{luatexja-fontspec,booktabs,array,lltjext}

\defaultfontfeatures{Numbers=OldStyle}
\setmainfont{TeXGyreTermes}
\setsansfont{TeXGyreHeros}

\defaultjfontfeatures{Scale=1.2}
\setmainjfont[BoldFont=IPAexGothic]{IPAexMincho} % default: JFM=ujis
\setsansjfont{IPAexGothic}

\newjfontfamily\ipajisninety[CJKShape=JIS1990]{IPAexMincho}

\normalsize
\usepackage[margin=25mm,a4paper]{geometry}

\begin{document}

\section{fontspecのテスト}

※テストのため,欧文文字に対して和文文字のサイズを大きくしています.

\noindent 通常:「あいうえお」、(かきくけこ)。{\bf 太字}と{\gt ゴシック}。

%{\addjfontfeatures{JFM=mono}
%\noindent 等幅:「あいうえお」、(かきくけこ)。{\bf 太字}と{\gt ゴシック}。
%}


\def\R{%
逢芦飴溢茨鰯淫迂厩噂餌襖迦牙廻恢晦蟹葛鞄釜翰翫徽
祇汲灸笈卿饗僅喰櫛屑粂祁隙倦捲牽鍵諺巷梗膏鵠甑叉
榊薩鯖錆鮫餐杓灼酋楯薯藷哨鞘杖蝕訊逗摺撰煎煽穿箭
詮噌遡揃遜腿蛸辿樽歎註瀦捗槌鎚辻挺鄭擢溺兎堵屠賭
瀞遁謎灘楢禰牌這秤駁箸叛挽誹樋稗逼謬豹廟瀕斧蔽瞥
蔑篇娩鞭庖蓬鱒迄儲餅籾爺鑓愈猷漣煉簾榔屢冤叟咬嘲
囀徘扁棘橙狡甕甦疼祟竈筵篝腱艘芒虔蜃蠅訝靄靱騙鴉
}

\subsection{JIS2004}
\R

\medskip\hrule\medskip
\vbox{\hsize=7\zw\tate\parindent1\zw\R}



\subsection{JIS1990}
{\ipajisninety
\R

\medskip\hrule\medskip
\vbox{\hsize=7\zw\tate\parindent1\zw\R}}

\subsection{Kerning}
{\jfontspec[NoEmbed]{Ryumin-Light}

アノ ← Kerning=Off

\noindent{\addjfontfeatures{Kerning=On} アノ ← Kerning=On}
}

\subsection{unicode}

設定依存:「\char"201C」「\char"010F」%"
常に和文:「\ltjjachar"201C」「\ltjjachar"010F」%"
常に欧文:「\ltjalchar"201C」「\ltjalchar"010F」%"

{\let\char=\ltjalchar\textquotedblleft}あ\textquotedblleft あ%
{\let\char=\ltjjachar\textquotedblleft}あ“あ

\begin{itemize}
\item hoge
\begin{itemize}
\item hoge\textendash\ltjjachar"2013\ltjalchar"2013
\begin{itemize}
\item hoge
\begin{itemize}
\item hoge|\textperiodcentered|\ltjjachar"00B7|\ltjalchar"00B7|・|
\end{itemize}
\end{itemize}
\end{itemize}
\end{itemize}

\newpage
\section{fontspecによる非埋め込みフォント}
\newjfontfamily\test[BoldFont=GothicBBB-Medium, NoEmbed]{Ryumin-Light}

%{\test
%\begin{tabular}<y>{ll>{\itshape}l>{\slshape}l}
%\toprule
%&Upright&Italic&Slanted\\
%\midrule
%Regular&あいうえお&あいうえお&あいうえお\
%Bold   &\textbf{あいうえお}&\textbf{あいうえお}&\textbf{あいうえお}\\
%\bottomrule
%\end{tabular}}

\section{JFMの違い}
\newjfontfamily\test[ TateFont=KozGoPr6N-Light ]{KozMinPr6N-Bold}

\def\yttest{\leavevmode%
\parbox<y>{20\zw}{%
「ちょっと」,“チェック”しちゃった!あいう
}%
\parbox<t>{20\zw}{%
「ちょっと」,“チェック”しちゃった!あいう}%
}

\test
\addjfontfeatures{CJKShape=JIS1990,Color=FF0000}
\yttest
\noindent

%\vskip-\dimexpr20\zw+\lineskip\relax
\noindent
\addjfontfeatures{Color=0000FF, YokoFeatures={ Color=0080FF, JFM=min}, TateFeatures={JFM=tmin}}
\yttest


\end{document}
%#!luatex
\input luatexja-core.sty

\def\head#1{\medskip\noindent{\bf\tengt ■ #1}\par}
\jfont\rml={psft:Ryumin-Light:jfm=ujis} at 10pt
\rml あ\inhibitglue\char"201Cあ←Ryumin-Light

\jfont\rml={file:ipam.ttf:jfm=ujis} at 10pt
\rml あ\inhibitglue\char"201Cあ←ipam

\jfont\rml={file:KozMinPr6N-Regular.otf:jfm=ujis} at 10pt
\rml あ\inhibitglue\char"201Cあ←KozMinPr6N-Regular

{\scrollmode
\globaljfont\rml={psft:GothicBBB-Medium:jfm=bad} at 10pt % must be error
}
\rml あ123 % \rml は未定義となる
{\tt\meaning\rml}

\bigskip
\font\rml={file:ipaexg.ttf} at 10pt\rml

ここからは,欧文文字フォントはIPA EXゴシック


\head{文字範囲の代入/取得テスト}

\ltjdefcharrange{2}{`あ,"E0-"FF}
「\char"F4」は2番の文字範囲なので,和文扱いのはず

{iso8859-1 和文扱い:\ltjsetparameter{jacharrange={1}}%
\ltjgetparameter{jacharrange}{1}%
§ ¶ ° £ ¥ \char"F4}

{iso8859-1 欧文扱い:\ltjsetparameter{jacharrange={-1}}%
\ltjgetparameter{jacharrange}{1}%
§ ¶ ° £ ¥ \char"F4}


\head{文字範囲の状況取得}
\ltjgetparameter{jcharrange}{0}
\ltjgetparameter{jcharrange}{1}
\ltjgetparameter{jcharrange}{2}

\head{文字コード→文字範囲}
\ltjgetparameter{chartorange}{`い} % must be 217
\ltjgetparameter{chartorange}{`§} % must be 1
\ltjgetparameter{chartorange}{"F7} % must be 2
\ltjgetparameter{chartorange}{-1}  % must be error "

\medskip
\ltjsetparameter{jacharrange={-217}}
ほとんど欧文扱い.2番は別(「あ」)
\ltjsetparameter{jacharrange={218}}
和文扱いにもどる

\vfill\eject
\jfont\rmlh={psft:Ryumin-Light:jfm=hang} at 10pt
\head{‘lineend’検証}
\jfont\sixgt={psft:GothicBBB-Medium:jfm=ujis} at 6pt
\font\sixtt=cmtt10 at 6pt


\def\dumplist#1{\par\noindent\leavevmode
\hbox to 0.2\hsize{\copy#1\hss}%
\vbox{\hsize=0.6\hsize\sixtt\baselineskip=7.2pt\sixgt\let\\=\relax
\directlua{ltj.ext_show_node_list(tex.getbox(#1).head, '\\par', tex.print)}\hrule}}

\setbox0=\hbox{\rmlh あえあいえ}
\dumplist0

\medskip
\ltjsetparameter{postbreakpenalty={`あ,100}}
\ltjsetparameter{postbreakpenalty={`い,100}}
\ltjsetparameter{postbreakpenalty={`う,100}}
\ltjsetparameter{postbreakpenalty={`え,100}}
\setbox0=\hbox{\rmlh あえあいえ}
\dumplist0

\head{italic correction}

\setbox0=\hbox{x{\rmlh お\/}({\rmlh お\/}あ}
\dumplist0

\setbox0=\hbox{x{\it f\/}({\it g\/}あ}
\dumplist0

\end

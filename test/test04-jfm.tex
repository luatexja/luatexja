%#!luatex
\input luatexja-core.sty

\def\head#1{\medskip\penalty-100\noindent{\bf\tengt ■ #1}\par\penalty10000}
\jfont\rml={psft:Ryumin-Light:jfm=ujis} at 10pt
\rml あ\inhibitglue\char"201Cあ・い←Ryumin-Light

\jfont\rml={file:ipam.ttf:jfm=ujis} at 10pt
\rml あ\inhibitglue\char"201Cあ・い←IPA\hbox{}明朝 (ipam)

\jfont\rml={file:KozMinPr6N-Regular.otf:jfm=ujis} at 10pt
\rml あ\inhibitglue\char"201Cあ・い←KozMinPr6N-Regular

\jfont\rml={file:ipamp.ttf:jfm=ujis} at 10pt
\rml あ\inhibitglue\char"201Cあ・い←IPA P\hbox{}明朝 (ipamp)

{\scrollmode
\globaljfont\rml={psft:GothicBBB-Medium:jfm=bad} at 10pt % must be error
}
\rml あ123 % \rml は未定義となる
{\tt\meaning\rml}

\bigskip
\font\rml={file:ipaexg.ttf} at 10pt\rml

ここからは,欧文文字フォントはIPA EXゴシック


\head{文字範囲の代入/取得テスト}

\ltjdefcharrange{2}{`あ,"E0-"FF}
「\char"F4」は2番の文字範囲なので,和文扱いのはず

{iso8859-1 和文扱い:\ltjsetparameter{jacharrange={1}}%
\ltjgetparameter{jacharrange}{1}%
§ ¶ ° £ ¥ \char"F4}

{iso8859-1 欧文扱い:\ltjsetparameter{jacharrange={-1}}%
\ltjgetparameter{jacharrange}{1}%
§ ¶ ° £ ¥ \char"F4}


\head{文字範囲の状況取得}
\ltjgetparameter{jacharrange}{0}
\ltjgetparameter{jacharrange}{1}
\ltjgetparameter{jacharrange}{2}

\head{文字コード→文字範囲}
\ltjgetparameter{chartorange}{`い} % must be 217
\ltjgetparameter{chartorange}{`§} % must be 1
\ltjgetparameter{chartorange}{"F7} % must be 2
\ltjgetparameter{chartorange}{-1}  % must be error "

\medskip
\ltjsetparameter{jacharrange={-217}}
ほとんど欧文扱い.2番は別(「あ」)
\ltjsetparameter{jacharrange={218}}
和文扱いにもどる

\head{Ticket \#25121}
\setbox0=\hbox{\ltjsetparameter{kanjiskip=-5pt}
あいうえお\hbox{\ltjsetparameter{kanjiskip=12pt}かきくけこ}さしすせそ\par a}\copy0
\tracingonline=0\showboxdepth=10000\showboxbreadth=10000\tracingonline=1\showbox0

{\ltjsetparameter{kanjiskip=3pt}
あいうえお{\ltjsetparameter{kanjiskip=-5pt}% this setting is ignored
かきくけこ}さしすせそ\par}

{\ltjsetparameter{kanjiskip=3pt}
あいうえおさしすせそ}\par


\vfill\eject
\noindent{\bf\gt  以下はJFMグルー挿入検証}
\jfont\rmlh={psft:Ryumin-Light:jfm=test} at 10pt
\jfont\sixgt={psft:GothicBBB-Medium:jfm=ujis} at 6pt
\font\sixtt=cmtt10 at 6pt

\def\dumplist#1{\par\noindent\leavevmode
\hbox to 0.2\hsize{\copy#1\hss}%
\vbox{\hsize=0.6\hsize\sixtt\baselineskip=7.2pt\sixgt\let\\=\relax
\directlua{ltj.ext_show_node_list(tex.getbox(#1).head, '\\par', tex.print)}\hrule}}

\head{JA--JA (penなし)}

\setbox0=\hbox{\rmlh あア}\dumplist0

\setbox0=\hbox{\rmlh あイ}\dumplist0

\setbox0=\hbox{\rmlh あウ}\dumplist0

\setbox0=\hbox{\rmlh いア}\dumplist0

\setbox0=\hbox{\rmlh いイ}\dumplist0

\setbox0=\hbox{\rmlh いウ}\dumplist0

\head{JA--EN (penなし)}

\setbox0=\hbox{\rmlh あa}\dumplist0

\setbox0=\hbox{\rmlh いa}\dumplist0

\setbox0=\hbox{\rmlh うa}\dumplist0

\setbox0=\hbox{\rmlh えa}\dumplist0

\setbox0=\hbox{\rmlh おa}\dumplist0

\head{EN--JA (penなし)}

\setbox0=\hbox{\rmlh aあ}\dumplist0

\setbox0=\hbox{\rmlh aい}\dumplist0

\setbox0=\hbox{\rmlh aう}\dumplist0

\head{JA--明kern (penなし)}

\setbox0=\hbox{\rmlh あ\kern3pt}\dumplist0

\setbox0=\hbox{\rmlh い\kern3pt}\dumplist0

\setbox0=\hbox{\rmlh う\kern3pt}\dumplist0

\head{明kern--JA (penなし)}

\setbox0=\hbox{\rmlh \kern3ptあ}\dumplist0

\setbox0=\hbox{\rmlh \kern3ptい}\dumplist0

\setbox0=\hbox{\rmlh \kern3ptう}\dumplist0

\head{JA--hbox (penなし)}

\setbox0=\hbox{\rmlh あ\hbox{}}\dumplist0

\setbox0=\hbox{\rmlh い\hbox{}}\dumplist0

\setbox0=\hbox{\rmlh う\hbox{}}\dumplist0

\setbox0=\hbox{\rmlh え\hbox{}}\dumplist0

\head{hbox--JA (penなし)}

\setbox0=\hbox{\rmlh \hbox{}あ}\dumplist0

\setbox0=\hbox{\rmlh \hbox{}い}\dumplist0

\setbox0=\hbox{\rmlh \hbox{}う}\dumplist0

\head{JA--penalty (penなし)} TODO: この場合の挙動はこれで良いか?

\setbox0=\hbox{\rmlh あ\penalty567}\dumplist0

\setbox0=\hbox{\rmlh い\penalty567}\dumplist0

\setbox0=\hbox{\rmlh う\penalty567}\dumplist0

\setbox0=\hbox{\rmlh え\penalty567}\dumplist0

\head{penalty--JA (penなし)} TODO: この場合の挙動はこれで良いか?

\setbox0=\hbox{\rmlh \penalty567あ}\dumplist0

\setbox0=\hbox{\rmlh \penalty567い}\dumplist0

\setbox0=\hbox{\rmlh \penalty567う}\dumplist0

{\vfill\eject%

\ltjsetparameter{prebreakpenalty={`(,123}}
\ltjsetparameter{postbreakpenalty={`あ,123}}
\ltjsetparameter{postbreakpenalty={`い,123}}
\ltjsetparameter{postbreakpenalty={`う,123}}
\ltjsetparameter{postbreakpenalty={`え,123}}
\ltjsetparameter{postbreakpenalty={`お,123}}
\ltjsetparameter{postbreakpenalty={`a,321}}
\head{JA--JA (penあり)}

\setbox0=\hbox{\rmlh あア}\dumplist0

\setbox0=\hbox{\rmlh あイ}\dumplist0

\setbox0=\hbox{\rmlh あウ}\dumplist0

\setbox0=\hbox{\rmlh いア}\dumplist0

\setbox0=\hbox{\rmlh いイ}\dumplist0

\setbox0=\hbox{\rmlh いウ}\dumplist0

\head{JA--EN (penあり)}

\setbox0=\hbox{\rmlh あa}\dumplist0

\setbox0=\hbox{\rmlh いa}\dumplist0

\setbox0=\hbox{\rmlh うa}\dumplist0

\setbox0=\hbox{\rmlh えa}\dumplist0

\setbox0=\hbox{\rmlh おa}\dumplist0

\head{EN--JA (penあり)}

\setbox0=\hbox{\rmlh aあ}\dumplist0

\setbox0=\hbox{\rmlh aい}\dumplist0

\setbox0=\hbox{\rmlh aう}\dumplist0

\head{italic correction}

\setbox0=\hbox{\it f\/(\/あ}\dumplist0

}

\vfill\eject
\noindent{\gt\bf [x]kanjiskipの挿入}

\head{kanjiskip from JFM, autospacing (JA--JA)}
\setbox0=\hbox{あ\rmlh あ}\dumplist0

{\ltjsetparameter{kanjiskip=\maxdimen}
\setbox0=\hbox{あ\rmlh あ}\dumplist0
\jfont\rmlk={psft:Ryumin-Light:jfm=test} at 12pt
\setbox0=\hbox{\rmlk は\rmlh あ}\dumplist0}

\def\naspc{\ltjsetparameter{autospacing=false}\relax}
\def\naxspc{\ltjsetparameter{autoxspacing=false}\relax}
\setbox0=\hbox{あ\naspc ああ}\dumplist0

\head{kanjiskip from JFM/autoxspacing (JA--EN)}
\setbox0=\hbox{\rmlh まx}\dumplist0

\setbox0=\hbox{\rmlh ま\naxspc x}\dumplist0

\setbox0=\hbox{\naxspc\rmlh まx}\dumplist0

{\ltjsetparameter{xkanjiskip=\maxdimen}\setbox0=\hbox{\rmlh まx}\dumplist0}

{\ltjsetparameter{jaxspmode={`ま,preonly}}
\setbox0=\hbox{\rmlh まx}\dumplist0}
{\ltjsetparameter{alxspmode={`x,postonly}}
\setbox0=\hbox{\rmlh まx}\dumplist0}

\head{kanjiskip from JFM/autoxspacing (EN--JA)}
\setbox0=\hbox{\rmlh xま}\dumplist0

\setbox0=\hbox{\rmlh x\naxspc ま}\dumplist0

\setbox0=\hbox{\naxspc\rmlh xま}\dumplist0

{\ltjsetparameter{xkanjiskip=\maxdimen}\setbox0=\hbox{\rmlh xま}\dumplist0}

{\ltjsetparameter{jaxspmode={`ま,postonly}}
\setbox0=\hbox{\rmlh xま}\dumplist0}
{\ltjsetparameter{alxspmode={`x,inhibit}}
\setbox0=\hbox{\rmlh xまおx}\dumplist0}

\head{more than one penalty}

\setbox0=\hbox{\rmlh お\penalty1701\penalty1701\penalty1701い}
\dumplist0
\setbox0=\hbox{\rmlh お\penalty1701\penalty1701\penalty1701お}
\dumplist0
\setbox0=\hbox{\rmlh あ\penalty1701\penalty1701\penalty1701い}
\dumplist0
\setbox0=\hbox{\rmlh あ\penalty1701\penalty1701\penalty1701お}
\dumplist0

\setbox0=\hbox{\tenrm あ\/\v{A}あ}
\dumplist0

\end

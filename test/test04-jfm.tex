%#!luatex
\input luatexja-core.sty

\jfont\rml={psft:Ryumin-Light:jfm=ujis} at 10pt
\rml あ\inhibitglue\char"201Cあ←Ryumin-Light

\jfont\rml={file:ipam.ttf:jfm=ujis} at 10pt
\rml あ\inhibitglue\char"201Cあ←ipam

\jfont\rml={file:KozMinPr6N-Regular.otf:jfm=ujis} at 10pt
\rml あ\inhibitglue\char"201Cあ←KozMinPr6N-Regular

\scrollmode
\jfont\rml={psft:GothicBBB-Medium:jfm=bad} at 10pt % must be error
\rml あ123 % \rml は未定義となる

\bigskip
\font\rml={file:ipaexg.ttf} at 10pt\rml

ここからは,欧文文字フォントはIPA EXゴシック


\medskip{\tengt 文字範囲の代入/取得テスト}

\defcharrange{2}{`あ,"E0-"FF}
「\char"F4」は2番の文字範囲なので,和文扱いのはず

{iso8859-1 和文扱い:\setjaparameter{jcharrange={1}}%
\getjaparameter{jcharrange}{1}%
§ ¶ ° £ ¥ \char"F4}

{iso8859-1 欧文扱い:\setjaparameter{jcharrange={-1}}%
\getjaparameter{jcharrange}{1}%
§ ¶ ° £ ¥ \char"F4}


\medskip{\tengt 文字範囲の状況取得}
\getjaparameter{jcharrange}{0}
\getjaparameter{jcharrange}{1}
\getjaparameter{jcharrange}{2}

\medskip{\tengt 文字コード→文字範囲}
\getjaparameter{chartorange}{`い} % must be 217
\getjaparameter{chartorange}{`§} % must be 1
\getjaparameter{chartorange}{"F7} % must be 2
\getjaparameter{chartorange}{-1}  % must be -1

\medskip
\setjaparameter{jcharrange={-217}}
ほとんど欧文扱い.2番は別(「あ」)
\setjaparameter{jcharrange={218}}
和文扱いにもどる
\end

%#! luajitlatex
%%% a test of ruby.
\documentclass[b5paper,10pt]{ltjsarticle}
\usepackage{luatexja-otf,amsmath,fontspec}
\usepackage{unicode-math}
\setmathfont{XITS Math}
\setmainfont[Ligatures=TeX]{TeX Gyre Termes}
\setsansfont[Ligatures=TeX]{TeX Gyre Heros}
\setmonofont[Ligatures=TeX]{LMMono10-Regular}
\usepackage{luatexja-ruby,showexpl,booktabs}
\lstset{preset=\huge,basicstyle=\ttfamily}
\fboxsep=0mm

\def\kata{\ltjsetruby{kata}} 
% 肩つきルビ用簡易設定.グループルビでは使用しないこと
\def\notalign{\setkeys[ltj]{ruby}{stretchhead = {1}{2}{1}, stretchend = {1}{2}{1}}}
% 行頭・行末で揃えない

\def\Node#1#2{\,\vcenter{\hbox{\fboxsep=1pt\fbox{\vbox{\small\halign{\hfil##\hfil\cr
  #1\mathstrut\cr\noalign{\hrule height.4pt}\strut#2\cr}}}}\,}}
\def\LuaTeX{Lua\TeX}

% 三分ルビ用
\DeclareFontShape{JY3}{mc}{mc}{n}{<-> [0.92487] 
  psft:Ryumin-Light:extend=0.67;jfm=ujisc33}{}

\title{\texttt{luatexja-ruby.sty}}
\begin{document}
\catcode`\<=13
\def<#1>{{\normalfont\rm\itshape$\langle$#1\/$\rangle$}}
%\fontsize{13.19873}{15}\selectfont%

\maketitle

\texttt{luatexja-ruby.sty} は,\LuaTeX-jaの機能を利用してルビの組版処理を行う追加パッケージである.
\section{使い方}
パッケージ読み込みは,\verb+\usepackage{luatexja-ruby}+ で良い.
plain \LuaTeX でのロードはまだサポートしておらず,
\LaTeXe のみサポート.

\subsection{用語}
「進入(intrusion)」「突出(protrusion)」という用語は,ZRさんによるpxrubricaパッケージでの用法に準ずる.

\begin{quotation}
進入あり:と\ltjruby{暁}{あかつき}の

進入なし:{\setkeys[ltj]{ruby}{mode=0}と\ltjruby{暁}{あかつき}の}

突出あり:{\setkeys[ltj]{ruby}{stretch={1}{2}{1}}\ltjruby{聴衆}{ちようしゆう}}

突出なし:{\setkeys[ltj]{ruby}{stretch={0}{2}{0}}\ltjruby{聴衆}{ちようしゆう}}
\end{quotation}

\subsection{命令}
\paragraph{\tt\textbackslash ltjruby}
ルビ出力用命令の本体.\verb+\ruby+ という別名を定義している.
\begin{quote}
\tt \textbackslash ltjruby[<option>]\{親|文|字\}\{おや|も|じ\}
\end{quote}
のように使用する.

第2・第3引数内の「\verb+|+」はグループの区切りを表す.グループの数は両者で一致しなければならず,
\verb+\ltjruby{紋章}{もん|しよう}+ のようにはできない.
\begin{itemize}
\item 1グループのみのルビ(単純グループルビ)はグループルビとして組まれる.そのため,
もしモノルビを使いたければ,面倒でも
\begin{LTXexample}[width=0.4\textwidth]
の\ltjruby{紋}{もん}\ltjruby{章}{しよう}が
\end{LTXexample}
のように,複数回使用すること.
\item 全てのグループにおいて「ルビ文字列の長さは親文字列以下」であれば,
単純グループルビの並びとして扱われる.すなわち,次の2行は全くの等価となる.
\begin{verbatim}
\ltjruby{普|通|車}{ふ|つう|しや}
\ltjruby{普}{ふ}\ltjruby{通}{つう}\ltjruby{車}{しや}
\end{verbatim}
\item 複数グループかつ上記の条件を満たさない場合は,
所謂「可動グループルビ」であり,グループの切れ目で改行が可能となる.
例えば
\begin{verbatim}
…の\ltjruby{表|現|力}{ひよう|げん|りよく}は…
\end{verbatim}
という入力からは得られる組版結果は,次のいずれかになる.
\begin{center}
 \begin{tabular}{ll}
 \toprule
 改行なし(行中形)&\Large …の\ltjruby{表|現|力}{ひよう|げん|りよく}は…\\
 直前で改行&\Large \vrule \ltjruby[stretch=011]{表|現|力}{ひよう|げん|りよく}は…\\
&\Large …の\ltjruby[stretch=110]{表}{ひよう}
    \vrule\ltjruby[stretch=011]{現|力}{げん|りよく}は…\\
&\Large …の\ltjruby[stretch=110]{表|現}{ひよう|げん}
    \vrule\ltjruby[stretch=011]{力}{りよく}は…\\
 直後に改行&\Large …の\ltjruby[stretch=110]{表|現|力}{ひよう|げん|りよく}\vrule\\
\bottomrule
 \end{tabular}
\end{center}

2ブロック以上をまとめて組むときは,全体を1つのグループルビのように組版する(JIS~X~4051と同様).
『日本語組版処理の要件』では,
附属書Fに「熟語の構成,さらにその熟語の前後にくる文字の種類を考慮して配置する方法」として
別の方法を解説しているが,こちらの方法は現時点ではサポートしない.

\end{itemize}


さて,<option>には以下の内容をkey-valueリストで指定可能である:
\begin{description}
\def\makelabel#1{\tt#1}
\item[intrusionpre=<real>] 前進入許容量をルビ全角単位で指定.
負の長さを指定した場合は,ルビの状況や直前の文字に応じた自動指定を意味する.
デフォルト値は負(つまり,自動指定).

\item[intrusionpost=<real>] 同様に,後進入許容量を指定する.デフォルト値は負(つまり,自動指定).

\item[mode] 進入処理のモードを表すbit vector.下位2\,bitは,\texttt{intrusionpre}や
\texttt{intrusionpost}が負であるの場合にしか効力を発揮しない.デフォルト値は$0001_2 = 1$.
\begin{description}
 \item[bit 0] 進入を無効にするならば0,有効にするならば1.
 \item[bit 1] 前進入許容量$B$と後進入許容量$A$が異なった場合,
そのまま処理する場合は0,小さい方に揃えるならば1.
 \item[bit 2--3] ルビ文字の突出量$x$から実際の前進入量$b$,後進入量$a$の計算方法を指定する.
親文字の文字数が$k+1$,親文字の前に入る空白量・間の空白量・後ろの空白量の比が$p:q:r$のとき,
\begin{description}
 \item[00] $b=\min\{B, xp/[p+kq+r]\}$, \ $a=\min\{A, xr/[p+kq+r]\}$
 \item[01] $b=\min(B, x)$, \ $a=\min[A, \max(x-b,0)]$
 \item[10] $a=\min(A, x)$, \ $b=\min[B, \max(x-a,0)]$
 \item[11] $M=\min(B,A)$とおく.もし$x\le 2M$ならば$b=a=x/2$.そうでなければ
\[
 b=\min\left(B, \frac x2 + \frac{(x-2M)p}{p+kp+r}\right),\qquad 
 a=\min\left(A, \frac x2 + \frac{(x-2M)r}{p+kp+r}\right)
\]
\end{description}

組み方の具体例を実際に示す.例示のため,平仮名にはルビが1字まで,「立」にはルビを
0.5字分までかけてよいことにしている.
\begin{description}
\item[00]{\setkeys[ltj]{ruby}{mode=1}%
\ltjsetparameter{rubypreintrusion={`立,0.5}, rubypostintrusion={`立,0.5}}%
は\ltjruby{美}{うつく}しい
  \quad は\ltjruby{聴衆}{ちようしゆう}と\quad
は\ltjruby{暁}{あかつき}立\quad
は\ltjruby{聴衆}{ちようしゆう}立\par}
\item[01]{\setkeys[ltj]{ruby}{mode=5}%
\ltjsetparameter{rubypreintrusion={`立,0.5}, rubypostintrusion={`立,0.5}}%
は\ltjruby{美}{うつく}しい
  \quad は\ltjruby{聴衆}{ちようしゆう}と\quad
は\ltjruby{暁}{あかつき}立\quad
は\ltjruby{聴衆}{ちようしゆう}立\par}
\item[10]{\setkeys[ltj]{ruby}{mode=9}%
\ltjsetparameter{rubypreintrusion={`立,0.5}, rubypostintrusion={`立,0.5}}%
は\ltjruby{美}{うつく}しい
  \quad は\ltjruby{聴衆}{ちようしゆう}と\quad
は\ltjruby{暁}{あかつき}立\quad
は\ltjruby{聴衆}{ちようしゆう}立\par}
\item[11]{\setkeys[ltj]{ruby}{mode=13}%
\ltjsetparameter{rubypreintrusion={`立,0.5}, rubypostintrusion={`立,0.5}}%
は\ltjruby{美}{うつく}しい\quad
は\ltjruby{聴衆}{ちようしゆう}と\quad
は\ltjruby{暁}{あかつき}立\quad
は\ltjruby{聴衆}{ちようしゆう}立\par}
\end{description} 

\end{description}
\item[stretchruby=\{<left>\}\{<middle>\}\{<right>\}] 親文字の合計長が
ルビ文字の合計長より長い時に,ルビ文字間に入れる空白の割合であり,それぞれ0--7の自然数で指定する.
デフォルト値は\ \verb+{1}{2}{1}+ である.

<left>はルビ文字の先頭までの空き量,<middle>はルビ文字間の空き量,<right>はルビ文字の末尾からの
空き量(の比)を表す.以下が例である.
\begin{LTXexample}[width=0.3\textwidth]
\Large
\ltjruby[stretchruby=123,maxmargin=2]%
  {◯◯◯◯}{◆◆}
\end{LTXexample}
\item[stretch=\{<left>\}\{<middle>\}\{<right>\}]
行中形でルビ文字の方が長い場合,親文字の前・中・後に入れる空白の割合.
デフォルト値は\ \verb+{1}{2}{1}+ である.
\item[stretchhead=\{<left>\}\{<middle>\}\{<right>\}] 行頭形〜.
デフォルト値は\ \verb+{0}{1}{1}+ である.
\item[stretchend=\{<left>\}\{<middle>\}\{<right>\}] 行末形〜.
デフォルト値は\ \verb+{1}{1}{0}+ である.

\item[maxmargin=<real>] 親文字の方がルビより長い時に,ルビの先頭と親文字の先頭,及び
ルビ末尾と親文字の末尾の間に許される最大の空白量.\emph{親文字全角単位}で指定し,デフォルト値は0.5.
\item[rubysize=<real>] ルビ文字の親文字に対する大きさ.デフォルト値は0.5.

\medskip
\item[naka] 以下のオプションを同時に設定する.主に中付きルビを組むときに用いる.
\begin{verbatim}
mode=1, stretch=121, stretchruby=121
\end{verbatim}
\item[kata] 同様に,肩付きルビ用に \verb+mode=9, stretch=121, stretchruby=001+ を設定.
\item[ekata] pxrubricaパッケージで言う「拡張肩付き」用に,次を設定する.
\begin{verbatim}
intrusionpre=0, mode=1, stretch=001, stretchruby=001
\end{verbatim}
\end{description}
\paragraph{\tt\textbackslash ltjsetruby\{<option>\}}
<option>の規定値を指定する.デフォルト値は各項目の所で既に説明してあるが,
\begin{verbatim}
stretchruby={1}{2}{1}, stretch = {1}{2}{1},
stretchhead = {0}{1}{1}, stretchend = {1}{1}{0},
intrusionpre = -1, intrusionpost = -1, maxmargin=0.5, 
mode = 1, rubysize = 0.5, kenten=\ltjalchar`•
\end{verbatim}
である.

\paragraph{\texttt{\textbackslash ltjsetparamater} に追加されるキー}
\begin{description}
\item[\textsf{rubypreintrusion}\tt =\{<chr\_code>, <pre\_int>\}]
文字<chr\_code> に,その\emph{直後}のルビによって掛けられるルビ文字列の最大長をルビ全角単位で指定.
\item[\textsf{rubypostintrusion}\tt =\{<chr\_code>, <post\_int>\}]
文字<chr\_code> に,その\emph{直後}のルビによって掛けられるルビ文字列の最大長をルビ全角単位で指定.
\end{description} 
デフォルト値は,\textsf{rubypreintrusion},~\textsf{rubypostintrusion}とも
以下の文字に対しては1,その他の文字については0である:
\begin{quote}
 平仮名(\texttt{U+3040}--\texttt{U+309F}),カギ括弧「」,読点「,」「、」,中黒「・」
\end{quote}

\section{注意点}
\begin{itemize}
\item ルビ文字のはみ出しが繋がらないようにする処理(図3.82)には注意.
例えば,
\begin{center}\Large
\ltjsetparameter{rubypreintrusion={`◆,1}, 
  rubypostintrusion={`◆,1}}
\ltjruby{陵}{りよう}◆\ltjruby{陵}{みささぎ}
\end{center}
において,後者の「\ltjruby{陵}{みささぎ}」のルビが前の「◆」にかかる量は次のように決まる:
\begin{enumerate}
\item 1回目の実行では,行分割前に「\ltjruby{陵}{りよう}」の後側進入量は前もって知ることはでき
      ない.なので,「\ltjruby{陵}{りよう}」は行中形で組まれるものとして
「\ltjruby{陵}{みささぎ}」前側進入許容量は
\[
 \underbrace{0.5\,\mathrm{zw}}_{\text{元々の許容量}}
-\underbrace{0.25\,\mathrm{zw}}_{\text{前のルビの後側進入量}}=0.25\,\mathrm{zw}
\]
となる.なお,行分割後,「\ltjruby{陵}{りよう}」が実際に組まれた時に使われた後側進入量は
auxファイルに記述される.
\item 2回目以降の実行では,auxファイルに保存された「\ltjruby{陵}{りよう}」の後側進入量
を用いて,「\ltjruby{陵}{みささぎ}」前側進入許容量を計算する.
\end{enumerate}
なお,auxファイルに保存する際,各 \verb+\ltjruby+ 命令の呼び出しを識別するキーが必要になるが,
そのキーとしては単純に「何個目の \verb+\ltjruby+ 命令か」である.

\item 実装方法の都合上,ルビの直前・直後・途中で2箇所以上の改行が起きる場合
(以下のパーツの組み方が出てくる)に対応できない.
\begin{center}\small
\begin{tabular}{ll}
\toprule
\multicolumn{1}{c}{\sf 組み方}&\multicolumn{1}{c}{\sf サンプル}\\
\midrule
単独1&
\huge
\vrule{\color{blue}\gt\ltjruby[]{流}{りゆう}}\vrule\\
単独2&
\huge
\vrule{\color{blue}\gt\ltjruby[]{暢}{ちよう}}\vrule\\
単独$(1+2)$&
\huge
\vrule{\color{blue}\gt\ltjruby[stretch=010]{流|暢}{りゆう|ちよう}}\vrule\\
\bottomrule
\end{tabular}
\end{center}

\item 段落がルビで終わった場合,そのルビが行末形で組まれることはない.
これは,段落の「本当の」末尾には \verb+\penalty10000\parfillskip+ があるためで,
ルビ処理用に作った最後のグルー(下の説明では$g_2$)が消去されないことによる.

\verb+\parfillskip+ の長さ(や,場合によっては \verb+\rightskip+)を実測し,
それによって処理を変えるのも可能だが,そのようなことはしなかった.
段落がルビで終わることは普通ない(最低でも句点が続くだろう)と思うからである.
\end{itemize}

\section{実装の大まかな方法}
次の例で説明する.
\begin{LTXexample}
……を\ltjruby{流|暢}{りゆう|ちよう}に……
\end{LTXexample}

\begin{enumerate}
\item \verb|\ltjruby|コマンド自体は,一旦次のnode listを値とするwhatsit~$W$を作って,
現在の水平リストへと挿入する(必要ならば\verb|\leavevmode|も実行):
\[
 \Node{whatsit $w$}{value: 2}\longrightarrow 
 \Node{hlist $s_1$}{「りゆう」}\longrightarrow
 \Node{hlist $p_1$}{「流」}\longrightarrow
 \Node{hlist $s_2$}{「ちよう」}\longrightarrow
 \Node{hlist $p_2$}{「暢」}
\]
ここで,最初の$w$の値2は,ルビが2つのパーツ「\ltjruby{流}{りゆう}」「\ltjruby{暢}{ちよう}」からなっていることを
表している.この値を$\mathit{cmp}$とおこう.
$s_i$達の中の文字は既にルビの大きさである.
\item \LuaTeX-jaの和文処理グルー挿入処理において,
このwhatsit~$W$はまとめて
「先頭が『流』,最後が『暢』であるようなhboxを \verb|\unhbox| で展開したもの」と扱われる.
言い換えれば,ルビ部分を無視した単なる「流暢」という和文文字の並びとして扱われる\footnote{「流」「暢」の間のグルーは既に入っている,と扱われる.}.
次のサンプルを参照
\begin{LTXexample}
\leavevmode\hbox{.}A\\
%↑xkanjiskip 
\ltjruby{.}{}A
%↑2分
\end{LTXexample}
\item 和文処理グルーの挿入が終わった後で,可動グループルビのためのノードの挿入に入る.
\begin{enumerate}
\item $W$の前後に$2\mathit{cmp}+1=5$個のノードが挿入され,$W$の周辺は次のようなノード列になる.
\begin{align*}
 (\text{other nodes})&\longrightarrow
 \Node{glue $g_0$}{}\longrightarrow \Node{whatsit~$W$}{元からある}\longrightarrow \Node{rule $r_1$}{}
\\&\longrightarrow
 \Node{glue $g_1$}{}\longrightarrow \Node{rule $r_2$}{}\longrightarrow 
 \Node{glue $g_2$}{}\longrightarrow (\text{other nodes})
\end{align*}

\item このようにノードを挿入する目的は,\TeX の行分割処理自体に影響を加えずに可動グループルビ
を実現させることにある.
\begin{gather*}
 (\text{other nodes})\longrightarrow
 \Node{glue $g_0$}{}\longrightarrow \Node{whatsit~$W$}{元からある}\longrightarrow \Node{rule $r_1$}{}\\
\noalign{\hrulefill 行の境目\hrulefill}
\Node{rule $r_2$}{}\longrightarrow 
 \Node{glue $g_2$}{}\longrightarrow (\text{other nodes})
\end{gather*}
のようになったとしたら,「\ltjruby{流}{りゆう}」「\ltjruby{暢}{ちよう}」の間で行分割が起きた,ということがわかり,
$g_i$,~$r_i$達のノードを適切に置き換えればよい(後で詳しく説明する).

\item なお,$r_i$達の高さ・深さは組み上がった後のそれである.
$g_i$,~$r_i$達の幅は,以下の対応に沿って算出する.

\begin{center}\small
\begin{tabular}{cllll}
\toprule
\multicolumn{1}{c}{\sf node名}&\multicolumn{1}{c}{\sf 組み方}&\multicolumn{1}{c}{\sf サンプル}%
&\multicolumn{1}{c}{\sf 対応するノード並び}\\
\midrule
$n_1$&行末1グループ&
\huge
\fbox{を}{\color{blue}\gt\ltjruby[intrusionpre=1,mode=5,stretch=110]{流}{りゆう}}\vrule
&
$g_0\rightarrow W\rightarrow r_1$\\
$n_2$&行末2グループ&
\huge
\fbox{を}{\color{blue}\gt\ltjruby[intrusionpre=1,mode=1,stretch=110]{流|暢}{りゆう|ちよう}}\vrule
&
$g_0\rightarrow W\rightarrow r_1\rightarrow g_2 \rightarrow r_2$\\
$n_3$&行頭1グループ&
\huge
\vrule{\color{blue}\gt\ltjruby[intrusionpost=1,mode=1,stretch=011]{暢}{ちよう}}\fbox{に}
&
$r_2\rightarrow g_2$\\
$n_4$&行頭2グループ&
\huge
\vrule{\color{blue}\gt\ltjruby[intrusionpost=1,mode=1,stretch=011]{流|暢}{りゆう|ちよう}}\fbox{に}
&
$W\rightarrow r_1\rightarrow g_2 \rightarrow r_2\rightarrow g_2$\\
$n_5$&行中&
\huge
\fbox{を}{\color{blue}\gt\ltjruby[intrusionpost=0.5,intrusionpre=0.5,mode=1]{流|暢}{りゆう|ちよう}}\fbox{に}
&
$g_0\rightarrow W\rightarrow r_1\rightarrow g_2 \rightarrow r_2\rightarrow g_2$\\

\bottomrule
\end{tabular}
\end{center}

\medskip

例えばこの場合,$n_5$に対して
\[
 g_0+r_1+g_2+r_2+g_2 = 3\,\mathrm{zw}-(0.25\,\mathrm{zw}\times 2)=2.5\,\mathrm{zw}
\]
という方程式が立つ(zwは親文字全角の幅,進入量込).
$n_1$から$n_5$まで計5本の方程式が立つが,これらはGau\ss の消去法で解くことができて
$g_i$,~$r_i$達の幅が求まる.
\item また,ルビ処理を統括しているwhatsit~$W$の値も
\[
 \Node{whatsit $w$}{value: 2}\longrightarrow 
 \Node{vlist $n_1$}{末1}\longrightarrow
 \Node{vlist $n_2$}{末2}\longrightarrow
 \Node{vlist $n_3$}{頭1}\longrightarrow
 \Node{vlist $n_4$}{頭2}\longrightarrow
 \Node{vlist $n_5$}{中}
\]
に置き換えておく.

\end{enumerate}

\item \LuaTeX の行分割処理を普通に行う.
\item 行分割の結果に従って,$g_i$,~$r_i$達を適切に置換する.

例えば行分割の結果
\begin{gather*}
 (\text{other nodes})\longrightarrow
 \Node{glue $g_0$}{}\longrightarrow \Node{whatsit~$W$}{元からある}\longrightarrow \Node{rule $r_1$}{}
  \tag{行A}\\
\noalign{\hrulefill 行の境目\hrulefill}
\Node{rule $r_2$}{}\longrightarrow 
 \Node{glue $g_2$}{}\longrightarrow (\text{other nodes})\tag{行B}
\end{gather*}
のようになったとしよう.
\begin{enumerate}
\item 処理は段落の上の行から順番に行われる.行Aの処理がまわってきたとしよう.

\item 行Aの先頭から順番に眺めていく.すると「whatsit~$W$由来」のノード,$g_0$,~$W$,~$r_1$が見つかり,
行Aはここで終わっている.

まず,行Aのhboxの中身からwhatsit~$W$を消去(リストから取り除くだけで,$W$のメモリを解放するわけではない)する.
$g_0$,~($W$,)~$r_1$というノードの並びは,「行末1グループ」$n_1$に対応しているので,
$g_0$,~$r_1$を行Aから除去・メモリ解放し,代わりに$n_1$を行Aの中身に追加する.

\item 次に行Bの処理にうつる.行Aでルビの処理は完了していない(2パーツのルビなのにまだ1パーツ目しか使っていないからである)ので,
「whatsit~$W$由来」のノードがいくつか残っているはずである.

案の定,$r_2$,~$g_2$というノード列が見つかった.これは「行頭1グループ」$n_3$に対応しているので,
$r_2$,~$g_2$を行Bから除去・メモリ解放し,代わりに$n_3$を行Bの中身に挿入する.

\item これで2パーツとも使い切ったことになるので,
隔離しておいた$W$を,(使われなかった$n_2$,~$n_4$,~$n_5$などと共に)メモリ解放する.結果として
次のようになった:
\begin{gather*}
 (\text{other nodes})\longrightarrow
 \Node{vlist $n_1$}{末1}\tag{行A}\\
\noalign{\hrulefill 行の境目\hrulefill}
\Node{vlist $n_3$}{頭1}\longrightarrow (\text{other nodes})\tag{行B}
\end{gather*}
\end{enumerate}
\end{enumerate}


\section{いくつかの例}
\def\ltjrubytest{\ltjruby{黄金橋}{ゴールデンゲートブリッジ}\relax}

\setbox0=\vbox{\hsize=22\zw%
ああああ\ltjrubytest いうえおかきくけこ
あ\ltjrubytest いうえおかきくけこ
あ\ltjrubytest いうえおかきくけこ
あ\ltjrubytest いうえおかきくけこ
あ\ltjrubytest いうえおかきくけこ
あ\ltjrubytest いうえおかきくけこ}
%\directlua{ltj.ext_show_node_list(tex.box[0], '? ', print)}
\fbox{\box0}

\def\ltjrubytest{\ltjruby{国府津}{こうづ}\relax}
% グループルビ

\setbox0=\vbox{\hsize=18\zw%
あ\ltjrubytest いうえおかきくけこ
あ\ltjrubytest いうえおかきくけこ
あ\ltjrubytest いうえおかきくけこ
あ\ltjrubytest いうえおかきくけこ
あ\ltjrubytest いうえおかきくけこ
あ\ltjrubytest いうえおかきくけこ}
%\directlua{ltj.ext_show_node_list(tex.box[0], '? ', print)}
\fbox{\box0}

\def\ltjrubytest{\ltjruby{●●|◆}{◆◆◆◆◆◆|●●●}\relax}
\setbox0=\vbox{\hsize=19\zw%
あ\ltjrubytest いうえおかきくけこ
あ\ltjrubytest いうえおかきくけこ
あ\ltjrubytest いうえおかきくけこ
あ\ltjrubytest いうえおかきくけこイ
あ\ltjrubytest いうえおかきくけこ
あ\ltjrubytest いうえおかきくけこウ
あ\ltjrubytest いうえおかきくけこエ
あ\ltjrubytest いうえおかきくけこ
あ\ltjrubytest いうえおかきくけこ
あ\ltjrubytest いうえおかきくけこ}
%\directlua{ltj.ext_show_node_list(tex.box[0], '? ', print)}
\fbox{\box0}

\def\ltjrubytest{\ltjruby{●●|□}{◆◆◆|●●●}\relax}
\setbox0=\vbox{\hsize=19\zw%
あ\ltjrubytest いうえおかきくけこ
あ\ltjrubytest いうえおかきくけこ
あ\ltjrubytest いうえおかきくけこ
あ\ltjrubytest いうえおかきくけこイ
あ\ltjrubytest いうえおかきくけこ
あ\ltjrubytest いうえおかきくけこウ
あ\ltjrubytest いうえおかきくけこエ
あ\ltjrubytest いうえおかきくけこ
あ\ltjrubytest いうえおかきくけこ
あ\ltjrubytest いうえおかきくけこ}
%\directlua{ltj.ext_show_node_list(tex.box[0], '? ', print)}
\fbox{\box0}

\def\ltjrubytest{\ltjruby{異|様}{い|よう}\relax}
\setbox0=\vbox{\hsize=19\zw%
あ\ltjrubytest いうえくけこ
あ\ltjrubytest いうえくけこ
あ\ltjrubytest いうえくけこ
あ\ltjrubytest いうえくけこイ
あ\ltjrubytest いうえおかきくけこ}
%\directlua{ltj.ext_show_node_list(tex.box[0], '? ', print)}
\fbox{\box0}

\def\ltjrubytest{\ltjruby{□|■|□}{■■|□□□|■■}\relax}

\setbox0=\vbox{\hsize=23\zw%
あ\ltjrubytest いうえおかきくけこうえおかきくけこ
あ\ltjrubytest いうえおかきくけこうえおかきくけこ
あ\ltjrubytest いう□おかきくけこうえおかきくけこ
あ\ltjrubytest いう□おかきくけこうえおかきくけこ
あ\ltjrubytest いう□おかきくけこうえおかきくけこ
あ\ltjrubytest いうえおかきくけこ}
\fbox{\box0}

\begin{description}
\def\sample{又\ltjruby{承}{うけたまわ}る\quad \ltjruby{疎}{そ}\quad は\ltjruby{俄}{にわか}勉強
  \quad 後\ltjruby{俄}{にわか}勉強\quad は\ltjruby{暁}{あかつき}に}
\item[標準] \sample
\item[肩つき] {\kata\sample}
\item[拡張肩つき]{\ltjsetruby{ekata}\sample}
\end{description}


\newpage
\section{jlreq 20120403の例}

\obeylines\newcommand*{\噂}{\CID{7642}}
%================================== 横組
\paragraph{3.3.1節}\ \par
3.49 \ltjruby{君|子}{くん|し}は\ltjruby{和}{わ}して\ltjruby{同}{どう}ぜず
3.50 \ltjruby{人}{ひと}に\ltjruby{誨}{おし}えて\ltjruby{倦}{う}まず\par% 中付き
3.51 \ltjruby{鬼}{き}\ltjruby{門}{もん}の\ltjruby{方}{ほう}\ltjruby{角}{がく}を% モノ中付き
\ltjruby{凝}{ぎょう}\ltjruby{視}{し}する
3.52 \ltjruby{鬼|門}{き|もん}の\ltjruby{方|角}{ほう|がく}を% 熟語(グループ扱い)
\ltjruby{凝|視}{ぎよう|し}する
3.53 \ltjruby{茅場町}{かやばちよう}\quad\ltjruby{茅場}{かやば}\ltjruby{町}{ちよう}\par% 複合語
% 3.53右の説明はこれでいいのか?
3.54 \ltjruby{紫陽花}{あじさい}\quad\ltjruby{坩堝}{るつぼ}\quad\ltjruby{田舎}{いなか}\par% 熟字訓 
3.55 \ltjruby{模型}{モデル}\quad\ltjruby{顧客}{クライアント}\quad% カタカナルビ
\ltjruby{境界面}{インターフエース}\quad\ltjruby{避難所}{アジール}
3.56 \ltjruby{編集者}{editor}\quad \ltjruby{editor}{エディター}% 欧文

\paragraph{3.3.3節}\ \par
3.58 に\ltjruby{幟}{のぼり}を\quad{\kata に\ltjruby{幟}{のぼり}を}%
  \quad \ltjruby{韋}{い}\ltjruby{編}{へん}\ltjruby{三}{さん}\ltjruby{絶}{ぜつ}
3.59 に\ltjruby{幟}{\kanjiseries{mc}\selectfont のぼり}を
3.60 \ltjruby{韋}{い}\ltjruby{編}{へん}\ltjruby{三}{さん}\ltjruby{絶}{ぜつ}\quad%
{\setkeys[ltj]{ruby}{rubysize=0.4}% 文字サイズ変更
  \ltjruby{韋}{い}\ltjruby{編}{へん}\ltjruby{三}{さん}\ltjruby{絶}{ぜつ}}

\paragraph{3.3.4節}\ \par
3.61図(両側ルビ)はまだ未サポートにより省略

\paragraph{3.3.5節 モノルビ}\ \par
3.62 の\ltjruby{葯}{やく}に
3.63 版面の\ltjruby{地}{ち}に\quad{\kata 版面の\ltjruby{地}{ち}に}
3.64× {\kata 版面の\ltjruby{地}{ち}に}(まだ縦組み未サポート)\par% 横組み肩つき
3.65 の\ltjruby{砦}{とりで}に\quad{\kata の\ltjruby{砦}{とりで}に}
{\kata 3.66上 の\ltjruby{旬}{しゆん}に\quad 後\ltjruby{旬}{しゆん}に
3.66下 の\ltjruby{旬}{しゆん}又\quad 後\ltjruby{旬}{しゆん}又\par}


\paragraph{3.3.6節 グループルビ}\ \par
3.67 は\ltjruby{冊子体}{コーデツクス}と
3.68 \ltjruby{模型}{モデル}\quad \ltjruby{利用許諾}{ライセンス}
3.69 {\setkeys[ltj]{ruby}{stretchruby=010}% 両端を揃える流儀
  \ltjruby{模型}{モデル}\quad \ltjruby{利用許諾}{ライセンス}}
3.70 \ltjruby{なげきの聖母像}{ピエタ}←自動調整
3.71 \ltjruby{顧客}{クライアント}\quad \ltjruby{境界面}{インターフエース}
3.72 {\setkeys[ltj]{ruby}{stretch=010, stretchhead = 010, stretchend = 010}% はみ出さない流儀
  \ltjruby{顧客}{クライアント}\quad \ltjruby{境界面}{インターフエース}}


\paragraph{3.3.7節 熟語ルビ}\ \par
3.73左 \ltjruby{杞|憂}{き|ゆう}\quad \ltjruby{畏|怖}{い|ふ}
3.73右 {\kata\ltjruby{杞|憂}{き|ゆう}\quad \ltjruby{畏|怖}{い|ふ}}
3.74 の\ltjruby{流|儀}{りゆう|ぎ}を\quad の\ltjruby{無|常}{む|じよう}を\quad%
の\ltjruby{成|就}{じよう|じゆ}を\quad
3.74 の\ltjruby{紋|章}{もん|しよう}を\quad の\ltjruby{象|徴}{しよう|ちよう}を

3.75 {\kata の\ltjruby{流|儀}{りゆう|ぎ}を\quad の\ltjruby{無|常}{む|じよう}を\quad%
の\ltjruby{成|就}{じよう|じゆ}を\quad 
3.75 の\ltjruby{紋|章}{もん|しよう}を\quad の\ltjruby{象|徴}{しよう|ちよう}を}

3.76× の\ltjruby{流}{りゆう}\ltjruby{儀}{ぎ}を\quad の\ltjruby{無}{む}\ltjruby{常}{じよう}を\quad%

要調整 3.77\ {\notalign%この図では揃えない
\hbox{\vrule\vbox{\hsize=10\zw あああああああの%
  \ltjruby{流|儀}{りゆう|ぎ}がある.}\vrule}\quad%
% ↑どうすれば改行されるのだろう
\hbox{\vrule\vbox{\hsize=5\zw ……の\ltjruby{無|常}{む|じよう}を}\vrule}}

\paragraph{3.3.8節 ルビはみ出し}\ \par
3.78 \ltjruby{人}{ひと}は\ltjruby{死}{し}して\ltjruby{名}{な}を\ltjruby{残}{のこ}す\par% ベタ
要調整3.79 漢字の部首には\ltjruby{偏}{へん}・\ltjruby{冠}{かんむり}・\ltjruby{脚}{きやく}・%
\ltjruby{旁}{つくり}がある
要調整3.79 漢字の部首には\ltjruby{偏}{へん},\ltjruby{冠}{かんむり},\ltjruby{脚}{きやく},%
\ltjruby{旁}{つくり}がある
3.79 この\ltjruby{\噂}{うわさ}の好きな人は%
\ltjruby{懐}{ふところ}ぐあいもよく、\ltjruby{檜}{ひのき}を
3.80 漢字の部首には「\ltjruby{偏}{へん}」「\ltjruby{冠}{かんむり}」「\ltjruby{脚}{きやく}」%
「\ltjruby{旁}{つくり}」がある
3.80 この\ltjruby{\噂}{うわさ}好きな人は\ltjruby{懐}{ふところ}具合もよく、\ltjruby{檜}{ひのき}材を
要調整3.81× に\ltjruby{暁}{あかつき}の\kern-1\zw の\ltjruby{趣}{おもむき}を
3.82 に\ltjruby{暁}{あかつき}の\ltjruby{趣}{おもむき}を
{%
  \ltjsetparameter{rubypostintrusion={`好,0.5}}
  \ltjsetparameter{rubypostintrusion={`具,0.5}}
  \ltjsetparameter{rubypostintrusion={`材,0.5}}
3,83 この\ltjruby{\噂}{うわさ}の好きな人は\ltjruby{懐}{ふところ}ぐあいもよく、\ltjruby{檜}{ひのき}を
3.83 この\ltjruby{\噂}{うわさ}好きな人は\ltjruby{懐}{ふところ}具合もよく、\ltjruby{檜}{ひのき}材を
}
{\catcode`\<12%
  \makeatletter\count@="3040\loop\relax\ifnum \count@<"30A0%
  \ltjsetparameter{rubypreintrusion={\the\count@,0}, %
    rubypostintrusion={\the\count@,0}}%
  \advance\count@1 \repeat
3.84 この\ltjruby{\噂}{うわさ}の好きな人は\ltjruby{懐}{ふところ}ぐあいもよく、\ltjruby{檜}{ひのき}を
3.84 この\ltjruby{\噂}{うわさ}好きな人は\ltjruby{懐}{ふところ}具合もよく、\ltjruby{檜}{ひのき}材を
}
要調整3.85\ {\notalign%この図では揃えない
\hbox{\vrule\vbox{\hsize=15\zw% なぜ行末形にならない!
あああああああああああああの\ltjruby{徑}{こみち}\penalty-1000をあああああああ%
あああああああああああああああああいの\ltjruby{徑}{こみち}ああ}\vrule}}

3.86\ %
\hbox{\vrule\vbox{\hsize=15\zw
ああああああああああああの\ltjruby{徑}{こみち}をあああああああ%
あああああああ\ltjruby{徑}{こみち}を}\vrule}

3.87\ %
\hbox{\vrule\vbox{\hsize=15\zw%
ああああああああああああの\ltjruby{飾り}{アクセサリー}等あああああああ%
あああああああああああああああ共\ltjruby{飾り}{アクセサリー}あ.}\vrule}


\paragraph{圏点の例(常用漢字表前書きより)}
  \ltjgetparameter{chartorange}{`﹅}%
\ltjgetparameter{jacharrange}{6}
この表は,法令,公⽤⽂書,新聞,雑誌,放送など,⼀般の社会⽣活におい
て\kenten[kenten=﹅]{現代の国語を書き表す}場合の\kenten{漢字使⽤の⽬安を⽰す}ものである。


\newpage
{\Large 要調整}

\paragraph{F.1--2節}\ \par

F.01 {\kata\ltjruby{治|癒}{ち|ゆ}\quad\ltjruby{模|索}{も|さく}\quad%
\ltjruby{遷|移}{せん|い}\quad\ltjruby{混|沌}{こん|とん}}
F.01中 \ltjruby{治|癒}{ち|ゆ}\quad\ltjruby{模|索}{も|さく}\quad%
\ltjruby{遷|移}{せん|い}\quad\ltjruby{混|沌}{こん|とん}

F.02 \ltjruby{橋|頭|堡}{きよう|とう|ほ}

F.03 {\kata\ltjruby{凝|視}{ぎよう|し}\quad\ltjruby{調|理|師}{ちよう|り|し}\quad%
\ltjruby{思|春|期}{し|しゆん|き}\quad\ltjruby{管|状|花}{かん|じよう|か}\quad%
\ltjruby{蒸|気|船}{じよう|き|せん}}
F.03 \ltjruby{凝|視}{ぎよう|し}\quad\ltjruby{調|理|師}{ちよう|り|し}\quad%
\ltjruby{思|春|期}{し|しゆん|き}\quad\ltjruby{管|状|花}{かん|じよう|か}\quad%
\ltjruby{蒸|気|船}{じよう|き|せん}

F.04 {\kata\ltjruby{未|熟}{み|じゆく}\quad\ltjruby{法|華|経}{ほ|け|きよう}\quad%
\ltjruby{顕|微|鏡}{けん|び|きよう}\quad\ltjruby{課|徴|金}{か|ちよう|きん}\quad%
\ltjruby{古|戦|場}{こ|せん|じよう}}
F.04 \ltjruby{未|熟}{み|じゆく}\quad\ltjruby{法|華|経}{ほ|け|きよう}\quad%
\ltjruby{顕|微|鏡}{けん|び|きよう}\quad\ltjruby{課|徴|金}{か|ちよう|きん}\quad%
\ltjruby{古|戦|場}{こ|せん|じよう}

F.05 の\ltjruby{峻|別}{しゆん|べつ}は

F.06以降は未チェック

\end{document}

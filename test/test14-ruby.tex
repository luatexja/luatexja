%#! lualatex
%%% a test of ruby.
\documentclass[b5paper,10pt]{ltjsarticle}
\usepackage{fontspec,luacode,luatexja-otf,amsmath}
\usepackage{unicode-math}
\setmathfont{XITS Math}
\setmainfont{TeX Gyre Termes}
\setsansfont{TeX Gyre Heros}
\setmonofont{LMMono10-Regular}
\usepackage{luatexja-ruby,showexpl,booktabs}
\lstset{preset=\huge,basicstyle=\ttfamily}
\fboxsep=0mm
%\usepackage{lua-visual-debug}

\def\kata{\ltjsetruby{kata}} 
% 肩つきルビ用簡易設定.グループルビでは使用しないこと
\def\notalign{\setkeys[ltj]{ruby}{stretchhead = {1}{2}{1}, stretchend = {1}{2}{1}}}
% 行頭・行末で揃えない

\def\Node#1#2{\,\vcenter{\hbox{\fboxsep=1pt\fbox{\vbox{\small\halign{\hfil##\hfil\cr
  #1\mathstrut\cr\noalign{\hrule height.4pt}\strut#2\cr}}}}\,}}
\def\LuaTeX{Lua\TeX}

% 三分ルビ用
\DeclareFontShape{JY3}{mc}{mc}{n}{<-> [0.92487] 
  psft:Ryumin-Light:extend=0.67;jfm=ujisc33}{}

\title{ルビ実装のサンプル}
\begin{document}
\maketitle

\section{使い方}
\begin{itemize}
\item パッケージ読み込み:\verb+\usepackage{luatexja-ruby}+のみ.\LaTeXe のみサポート.
\item 命令:
\begin{itemize}
\item \verb+\ltjruby[option]{親|文|字}{おや|も|じ}+
\item \verb|\ltjsetruby{option}|
\end{itemize}
\item \verb|\ltjsetparameter|に追加のキー
\begin{itemize}
\item \verb+rubypreintrusion={char_code, pre_int}+:全角単位
\item \verb+rubypostintrusion={char_code, post_int}+:全角単位
\end{itemize}
\item \verb|\ltjsetruby{option}|他でサポートするオプション:後ろに指定したものが優先
\begin{itemize}
\item \verb+mode+:進入処理のモードを表すbit vector.意味はluatexja-ruby.sty参照.
\item \verb+intrusionpre=<real>+:前進入量指定.ルビ全角単位で指定.
負の長さは文字種に応じた自動指定を意味.
\item \verb+intrusionpost=<real>+:後進入量指定.
\item \verb+stretchhead={left}{middle}{right}+:行頭形でルビ文字の合計長が親文字の合計長より
長い時に,親文字間に入れる空白の割合.それぞれ0--7の自然数で指定.
\item \verb+stretchend={left}{middle}{right}+:行末形〜.
\item \verb+stretch={left}{middle}{right}+:行中形〜.
\item \verb+stretchruby={left}{middle}{right}+:ルビ文字の方が長い時の空白の割合.
\item \verb+maxmargin=<real>+:親文字の方がルビより長い時に,ルビの先頭と親文字の先頭に許される
最大の空白量.親文字全角単位.
\item \verb+rubysize=<real>+:ルビ文字の親文字に対する大きさ.
\end{itemize}
\end{itemize}

\section{実装の大まかな方法}
次の例で説明する.
\begin{LTXexample}
……を\ruby{流|暢}{りゆう|ちよう}に……
\end{LTXexample}

\begin{enumerate}
\item \verb|\ruby|コマンド自体は,一旦次のnode listを値とするwhatsit~$W$を作って,
現在の水平リストへと挿入する(必要ならば\verb|\leavevmode|も実行):
\[
 \Node{whatsit $w$}{value: 2}\longrightarrow 
 \Node{hlist $r_1$}{「りゆう」}\longrightarrow
 \Node{hlist $p_1$}{「流」}\longrightarrow
 \Node{hlist $r_2$}{「ちよう」}\longrightarrow
 \Node{hlist $p_2$}{「暢」}
\]
ここで,最初の$w$の値2は,ルビが2つのパーツ「\ruby{流}{りゆう}」「\ruby{暢}{ちよう}」からなっていることを
表している.$r_i$達の中の文字は既にルビの大きさである.この値を$\mathit{cmp}$とおこう.
\item \LuaTeX-jaの和文処理グルー挿入処理においては,「先頭が『流』,最後が『暢』であるようなhboxを\verb|\unhbox|で展開したもの」と扱われる.
言い換えれば,ルビ部分を無視した単なる「流暢」という和文文字の並びとして扱われる\footnote{「流」「暢」の間のグルーは既に入っている,と扱われる.}.
次のサンプルを参照
\begin{LTXexample}
\leavevmode\hbox{.}A\\
%↑xkanjiskip 
\ruby{.}{}A
%↑2分
\end{LTXexample}
\item 和文処理グルーの挿入が終わった後で,可動グループルビのためのノードの挿入に入る.
\begin{enumerate}
\item $W$の周辺は次のようなノード列になる.
\begin{align*}
 (\text{other nodes})&\longrightarrow
 \Node{glue $g_0$}{}\longrightarrow \Node{whatsit~$W$}{元からある}\longrightarrow \Node{rule $r_1$}{}
\\&\longrightarrow
 \Node{glue $g_1$}{}\longrightarrow \Node{rule $r_2$}{}\longrightarrow 
 \Node{glue $g_2$}{}\longrightarrow (\text{other nodes})
\end{align*}
$W$の前後に$2\mathit{cmp}+1=5$個のノードが挿入される.

\item このようにノードを挿入する目的は,\TeX の行分割処理自体に影響を加えずに可動グループルビ
を実現させることにある.
\begin{gather*}
 (\text{other nodes})\longrightarrow
 \Node{glue $g_0$}{}\longrightarrow \Node{whatsit~$W$}{元からある}\longrightarrow \Node{rule $r_1$}{}\\
\noalign{\hrulefill 行の境目\hrulefill}
\Node{rule $r_2$}{}\longrightarrow 
 \Node{glue $g_2$}{}\longrightarrow (\text{other nodes})
\end{gather*}
のようになったとしたら,「\ruby{流}{りゆう}」「\ruby{暢}{ちよう}」の間で行分割が起きた,ということがわかり,
$g_i$,~$r_i$達のノードを適切に置き換えればよい(後で詳しく説明する).

\item なお,$r_i$達の高さ・深さは組み上がった後のそれであり,現状では親文字部分の高さ・深さと一致する.
$g_i$,~$r_i$達の幅は,以下の対応に沿って計算される.

\begin{center}\small
\begin{tabular}{cllll}
\toprule
\multicolumn{1}{c}{\sf node名}&\multicolumn{1}{c}{\sf 組み方}&\multicolumn{1}{c}{\sf サンプル}%
&\multicolumn{1}{c}{\sf 対応するノード並び}\\
\midrule
$n_1$&行末1グループ&
\huge
\fbox{を}{\color{blue}\gt\ruby[intrusionpre=1,mode=5,stretch=110]{流}{りゆう}}\vrule
&
$g_0\rightarrow W\rightarrow r_1$\\
$n_2$&行末2グループ&
\huge
\fbox{を}{\color{blue}\gt\ruby[intrusionpre=1,mode=1,stretch=110]{流|暢}{りゆう|ちよう}}\vrule
&
$g_0\rightarrow W\rightarrow r_1\rightarrow g_2 \rightarrow r_2$\\
$n_3$&行頭1グループ&
\huge
\vrule{\color{blue}\gt\ruby[intrusionpost=1,mode=1,stretch=011]{暢}{ちよう}}\fbox{に}
&
$r_2\rightarrow g_2$\\
$n_4$&行頭2グループ&
\huge
\vrule{\color{blue}\gt\ruby[intrusionpost=1,mode=1,stretch=011]{流|暢}{りゆう|ちよう}}\fbox{に}
&
$W\rightarrow r_1\rightarrow g_2 \rightarrow r_2\rightarrow g_2$\\
$n_5$&行中&
\huge
\fbox{を}{\color{blue}\gt\ruby[intrusionpost=0.5,intrusionpre=0.5,mode=1]{流|暢}{りゆう|ちよう}}\fbox{に}
&
$g_0\rightarrow W\rightarrow r_1\rightarrow g_2 \rightarrow r_2\rightarrow g_2$\\

\bottomrule
\end{tabular}
\end{center}


この5元連立方程式を解いて$g_i$,~$r_i$達の幅を求める.
なお,進入長は予め計算しておく.例えばこの場合,
\[
 g_0+r_1+g_2+r_2+g_2 = 3\,\mathrm{zw}-(0.25\,\mathrm{zw}\times 2)=2.5\,\mathrm{zw}
\]
である(zwは親文字全角の幅).

\item また,ルビ処理を統括しているwhatsit~$W$の値も
\[
 \Node{whatsit $w$}{value: 2}\longrightarrow 
 \Node{vlist $n_1$}{末1}\longrightarrow
 \Node{vlist $n_2$}{末2}\longrightarrow
 \Node{vlist $n_3$}{頭1}\longrightarrow
 \Node{vlist $n_4$}{頭2}\longrightarrow
 \Node{vlist $n_5$}{中}
\]
に置き換えておく.
\end{enumerate}
\item \LuaTeX の行分割処理を普通に行う.
\item 行分割の結果に従って,$g_i$,~$r_i$達を適切に置換する.

例えば行分割の結果
\begin{gather*}
 (\text{other nodes})\longrightarrow
 \Node{glue $g_0$}{}\longrightarrow \Node{whatsit~$W$}{元からある}\longrightarrow \Node{rule $r_1$}{}
  \tag{行A}\\
\noalign{\hrulefill 行の境目\hrulefill}
\Node{rule $r_2$}{}\longrightarrow 
 \Node{glue $g_2$}{}\longrightarrow (\text{other nodes})\tag{行B}
\end{gather*}
のようになったとしよう.
\begin{enumerate}
\item 処理は段落の上の行から順番に行われる.行Aの処理がまわってきたとしよう.

\item 行Aの先頭から順番に眺めていく.すると「whatsit~$W$由来」のノード,$g_0$,~$W$,~$r_1$が見つかり,
行Aはここで終わっている.

まず,行Aのhboxの中身からwhatsit~$W$を消去(リストから取り除くだけで,$W$のメモリを解放するわけではない)する.
$g_0$,~($W$,)~$r_1$というノードの並びは,「行末1グループ」$n_1$に対応しているので,
$g_0$,~$r_1$を行Aから除去・メモリ解放し,代わりに$n_1$を行Aの中身に追加する.

\item 次に行Bの処理にうつる.行Aでルビの処理は完了していない(2パーツのルビなのにまだ1パーツ目しか使っていないからである)ので,
「whatsit~$W$由来」のノードがいくつか残っているはずである.

案の定,$r_2$,~$g_2$というノード列が見つかった.これは「行頭1グループ」$n_3$に対応しているので,
$r_2$,~$g_2$を行Bから除去・メモリ解放し,代わりに$n_3$を行Bの中身に挿入する.

\item これで2パーツとも使い切ったことになるので,
隔離しておいた$W$を,(使われなかった$n_2$,~$n_4$,~$n_5$などと共に)メモリ解放する.結果として
次のようになった:
\begin{gather*}
 (\text{other nodes})\longrightarrow
 \Node{vlist $n_1$}{末1}\tag{行A}\\
\noalign{\hrulefill 行の境目\hrulefill}
\Node{vlist $n_3$}{頭1}\longrightarrow (\text{other nodes})\tag{行B}
\end{gather*}
\end{enumerate}
\end{enumerate}

\section{TODO}
\begin{itemize}
\item 熟語ルビの配置が「モノルビの連続」「グループルビ」のどちらかしかない(行分割は可能).
「一字だけかけても良い」なんてどうやるんだろう.
\item 図3.82のように,ルビ文字のはみ出しが繋がらないようにする処理が不安定である.
例えば,
\begin{center}\Large
\ruby{陵}{りよう}と\ruby{陵}{みささぎ}
\end{center}
において,後者の「\ruby{陵}{みささぎ}」のルビは前の「と」にかかっていない.
これは次の理由によるものである:
\begin{enumerate}
\item ルビのところで行分割を正しく実装するためには,行分割前に行頭・行中・行末形の幅を
決定しておかねばならない.
\item 「\ruby{陵}{りよう}」の後側進入量は,行頭形のとき(ルビ全角)が最大となる.
\item 「\ruby{陵}{みささぎ}」のルビ文字のはみ出しが重ならないように,前側進入許容量は
\[
 \underbrace{0.5\,\mathrm{zw}}_{\text{元々の許容量}}
-\underbrace{0.5\,\mathrm{zw}}_{\text{前のルビの後側進入量の最大値}}=0
\]
となる.そのため,「\ruby{陵}{みささぎ}」のルビ文字は「と」にはかからない.
\end{enumerate}
1パスでは厳しいだろうが,2パスでなら
全てのルビにおいて「どの形で出力されたか」をauxに保存することでは改善されると思う.
しかし,どのように保存するか
(具体的には,各\verb|\ruby|命令の呼びだしをどういうキーに変換するか)が問題だ.
\item 実装方法の都合上,ルビの直前・直後・途中で2箇所以上の改行が起きる場合(以下のパーツの組み方が出てくる)に対応できない.
\begin{center}\small
\begin{tabular}{ll}
\toprule
\multicolumn{1}{c}{\sf 組み方}&\multicolumn{1}{c}{\sf サンプル}\\
\midrule
単独1&
\huge
\vrule{\color{blue}\gt\ruby[]{流}{りゆう}}\vrule\\
単独2&
\huge
\vrule{\color{blue}\gt\ruby[]{暢}{ちよう}}\vrule\\
単独$(1+2)$&
\huge
\vrule{\color{blue}\gt\ruby[stretch=010]{流|暢}{りゆう|ちよう}}\vrule\\
\bottomrule
\end{tabular}
\end{center}

\end{itemize}

\paragraph{進入(intrusion)と突出(protrusion)}\ 

進入ありの例:\quad と\ruby{暁}{あかつき}の

進入なしの例:\quad {\setkeys[ltj]{ruby}{mode=0}と\ruby{暁}{あかつき}の}

突出ありの例:\quad {\setkeys[ltj]{ruby}{stretch={1}{2}{1}}\ruby{聴衆}{ちようしゆう}}

突出なしの例:\quad {\setkeys[ltj]{ruby}{stretch={0}{2}{0}}\ruby{聴衆}{ちようしゆう}}


\paragraph{mode}\ 

mode 00xx: {\setkeys[ltj]{ruby}{mode=1}は\ruby{美}{うつく}しい
  \quad は\ruby{聴衆}{ちようしゆう}と}

mode 01xx: {\setkeys[ltj]{ruby}{mode=5}は\ruby{美}{うつく}しい
  \quad は\ruby{聴衆}{ちようしゆう}と}

mode 10xx: {\setkeys[ltj]{ruby}{mode=9}は\ruby{美}{うつく}しい
  \quad は\ruby{聴衆}{ちようしゆう}と}

mode 11xx: {\setkeys[ltj]{ruby}{mode=13}%
\ltjsetparameter{rubypreintrusion={`立,0.5}, 
  rubypostintrusion={`立,0.5}}は\ruby{暁}{あかつき}立\quad
は\ruby{聴衆}{ちようしゆう}立\par}
↑ここでは「立」には0.5字がけを許容

中付きでは00xx, 11xxを選択して下さい


\section{いくつかの例}
\def\rubytest{\ruby{黄金橋}{ゴールデンゲートブリッジ}\relax}

\setbox0=\vbox{\hsize=22\zw%
ああああ\rubytest いうえおかきくけこ
あ\rubytest いうえおかきくけこ
あ\rubytest いうえおかきくけこ
あ\rubytest いうえおかきくけこ
あ\rubytest いうえおかきくけこ
あ\rubytest いうえおかきくけこ}
%\directlua{ltj.ext_show_node_list(tex.box[0], '? ', print)}
\fbox{\box0}

\def\rubytest{\ruby{国府津}{こうづ}\relax}
% グループルビ

\setbox0=\vbox{\hsize=18\zw%
あ\rubytest いうえおかきくけこ
あ\rubytest いうえおかきくけこ
あ\rubytest いうえおかきくけこ
あ\rubytest いうえおかきくけこ
あ\rubytest いうえおかきくけこ
あ\rubytest いうえおかきくけこ}
%\directlua{ltj.ext_show_node_list(tex.box[0], '? ', print)}
\fbox{\box0}

\def\rubytest{\ruby{●●|◆}{◆◆◆◆◆◆|●●●}\relax}
\setbox0=\vbox{\hsize=19\zw%
あ\rubytest いうえおかきくけこ
あ\rubytest いうえおかきくけこ
あ\rubytest いうえおかきくけこ
あ\rubytest いうえおかきくけこイ
あ\rubytest いうえおかきくけこ
あ\rubytest いうえおかきくけこウ
あ\rubytest いうえおかきくけこエ
あ\rubytest いうえおかきくけこ
あ\rubytest いうえおかきくけこ
あ\rubytest いうえおかきくけこ}
%\directlua{ltj.ext_show_node_list(tex.box[0], '? ', print)}
\fbox{\box0}

\def\rubytest{\ruby{●●|□}{◆◆◆|●●●}\relax}
\setbox0=\vbox{\hsize=19\zw%
あ\rubytest いうえおかきくけこ
あ\rubytest いうえおかきくけこ
あ\rubytest いうえおかきくけこ
あ\rubytest いうえおかきくけこイ
あ\rubytest いうえおかきくけこ
あ\rubytest いうえおかきくけこウ
あ\rubytest いうえおかきくけこエ
あ\rubytest いうえおかきくけこ
あ\rubytest いうえおかきくけこ
あ\rubytest いうえおかきくけこ}
%\directlua{ltj.ext_show_node_list(tex.box[0], '? ', print)}
\fbox{\box0}

\def\rubytest{\ruby{異|様}{い|よう}\relax}
\setbox0=\vbox{\hsize=19\zw%
あ\rubytest いうえくけこ
あ\rubytest いうえくけこ
あ\rubytest いうえくけこ
あ\rubytest いうえくけこイ
あ\rubytest いうえおかきくけこ}
%\directlua{ltj.ext_show_node_list(tex.box[0], '? ', print)}
\fbox{\box0}

\def\rubytest{\ruby{□|■|□}{■■|□□□|■■}\relax}

\setbox0=\vbox{\hsize=23\zw%
あ\rubytest いうえおかきくけこうえおかきくけこ
あ\rubytest いうえおかきくけこうえおかきくけこ
あ\rubytest いう□おかきくけこうえおかきくけこ
あ\rubytest いう□おかきくけこうえおかきくけこ
あ\rubytest いう□おかきくけこうえおかきくけこ
あ\rubytest いうえおかきくけこ}
\fbox{\box0}

\begin{description}
\def\sample{又\ruby{承}{うけたまわ}る\quad \ruby{疎}{そ}\quad は\ruby{俄}{にわか}勉強
  \quad 後\ruby{俄}{にわか}勉強}
\item[標準] \sample
\item[肩つき] {\kata\sample}
\item[拡張肩つき]{\ltjsetruby{ekata}\sample}
\end{description}


\newpage
\section{jlreq 20120403の例}

\obeylines\newcommand*{\噂}{\CID{7642}}
%================================== 横組
\paragraph{3.3.1節}\ \par
3.49 \ruby{君|子}{くん|し}は\ruby{和}{わ}して\ruby{同}{どう}ぜず
3.50 \ruby{人}{ひと}に\ruby{誨}{おし}えて\ruby{倦}{う}まず\par% 中付き
3.51 \ruby{鬼}{き}\ruby{門}{もん}の\ruby{方}{ほう}\ruby{角}{がく}を% モノ中付き
\ruby{凝}{ぎょう}\ruby{視}{し}する
3.52 \ruby{鬼|門}{き|もん}の\ruby{方|角}{ほう|がく}を% 熟語(グループ扱い)
\ruby{凝|視}{ぎよう|し}する
3.53 \ruby{茅場町}{かやばちよう}\quad\ruby{茅場}{かやば}\ruby{町}{ちよう}\par% 複合語
3.54 \ruby{紫陽花}{あじさい}\quad\ruby{坩堝}{るつぼ}\quad\ruby{田舎}{いなか}\par% 熟字訓 
3.55 \ruby{模型}{モデル}\quad\ruby{顧客}{クライアント}\quad% カタカナルビ
\ruby{境界面}{インターフエース}\quad\ruby{避難所}{アジール}
3.56 \ruby{編集者}{editor}\quad \ruby{editor}{エディター}% 欧文

\paragraph{3.3.3節}\ \par
3.58 に\ruby{幟}{のぼり}を\quad{\kata に\ruby{幟}{のぼり}を}%
  \quad \ruby{韋}{い}\ruby{編}{へん}\ruby{三}{さん}\ruby{絶}{ぜつ}
3.59 に\ruby{幟}{\kanjiseries{mc}\selectfont のぼり}を
3.60 \ruby{韋}{い}\ruby{編}{へん}\ruby{三}{さん}\ruby{絶}{ぜつ}\quad%
{\setkeys[ltj]{ruby}{rubysize=0.4}% 文字サイズ変更
  \ruby{韋}{い}\ruby{編}{へん}\ruby{三}{さん}\ruby{絶}{ぜつ}}

\paragraph{3.3.4節}\ \par
3.61図(両側ルビ)はまだ未サポートにより省略

\paragraph{3.3.5節 モノルビ}\ \par
3.62 の\ruby{葯}{やく}に
3.63 版面の\ruby{地}{ち}に\quad{\kata 版面の\ruby{地}{ち}に}
3.64× {\kata 版面の\ruby{地}{ち}に}(まだ縦組み未サポート)\par% 横組み肩つき
3.65 の\ruby{砦}{とりで}に\quad{\kata の\ruby{砦}{とりで}に}
{\kata 3.66上 の\ruby{旬}{しゆん}に\quad 後\ruby{旬}{しゆん}に
3.66下 の\ruby{旬}{しゆん}又\quad 後\ruby{旬}{しゆん}又\par}


\paragraph{3.3.6節 グループルビ}\ \par
3.67 は\ruby{冊子体}{コーデツクス}と
3.68 \ruby{模型}{モデル}\quad \ruby{利用許諾}{ライセンス}
3.69 {\setkeys[ltj]{ruby}{stretchruby=010}% 両端を揃える流儀
  \ruby{模型}{モデル}\quad \ruby{利用許諾}{ライセンス}}
3.70 \ruby{なげきの聖母像}{ビエタ}←自動調整
3.71 \ruby{顧客}{クライアント}\quad \ruby{境界面}{インターフエース}
3.72 {\setkeys[ltj]{ruby}{stretch=010, stretchhead = 010, stretchend = 010}% はみ出さない流儀
  \ruby{顧客}{クライアント}\quad \ruby{境界面}{インターフエース}}


\paragraph{3.3.7節 熟語ルビ}\ \par
3.73左 \ruby{杞|憂}{き|ゆう}\quad \ruby{畏|怖}{い|ふ}
3.73右 {\kata\ruby{杞|憂}{き|ゆう}\quad \ruby{畏|怖}{い|ふ}}
3.74 の\ruby{流|儀}{りゆう|ぎ}を\quad の\ruby{無|常}{む|じよう}を\quad%
の\ruby{成|就}{じよう|じゆ}を\quad
3.74 の\ruby{紋|章}{もん|しよう}を\quad の\ruby{象|徴}{しよう|ちよう}を

要調整3.75 {\kata の\ruby{流|儀}{りゆう|ぎ}を\quad の\ruby{無|常}{む|じよう}を\quad%
の\ruby{成|就}{じよう|じゆ}を\quad 
要調整3.75 の\ruby{紋|章}{もん|しよう}を\quad の\ruby{象|徴}{しよう|ちよう}を}

3.76× の\ruby{流}{りゆう}\ruby{儀}{ぎ}を\quad の\ruby{無}{む}\ruby{常}{じよう}を\quad%

3.77\ {\notalign%この図では揃えない
\hbox{\vrule\vbox{\hsize=5\zw ……の\ruby{流|儀}{りゆう|ぎ}を}\vrule}\quad%
% ↑どうすれば改行されるのだろう
\hbox{\vrule\vbox{\hsize=5\zw ……の\ruby{無|常}{む|じよう}を}\vrule}}

\paragraph{3.3.8節 ルビはみ出し}\ \par
3.78 \ruby{人}{ひと}は\ruby{死}{し}して\ruby{名}{な}を\ruby{残}{のこ}す\par% ベタ
要調整3.79 漢字の部首には\ruby{偏}{へん}・\ruby{冠}{かんむり}・\ruby{脚}{きやく}・%
\ruby{旁}{つくり}がある
要調整3.79 漢字の部首には\ruby{偏}{へん},\ruby{冠}{かんむり},\ruby{脚}{きやく},%
\ruby{旁}{つくり}がある
3.79 この\ruby{\噂}{うわさ}の好きな人は%
\ruby{懐}{ふところ}ぐあいもよく、\ruby{檜}{ひのき}を
3.80 漢字の部首には「\ruby{偏}{へん}」「\ruby{冠}{かんむり}」「\ruby{脚}{きやく}」%
「\ruby{旁}{つくり}」がある
3.80 この\ruby{\噂}{うわさ}好きな人は\ruby{懐}{ふところ}具合もよく、\ruby{檜}{ひのき}材を
要調整3.81× に\ruby{暁}{あかつき}の\kern-1\zw の\ruby{趣}{おもむき}を
3.82 に\ruby{暁}{あかつき}の\ruby{趣}{おもむき}を
{%
  \ltjsetparameter{rubypostintrusion={`好,0.5}}
  \ltjsetparameter{rubypostintrusion={`具,0.5}}
  \ltjsetparameter{rubypostintrusion={`材,0.5}}
3,83 この\ruby{\噂}{うわさ}の好きな人は\ruby{懐}{ふところ}ぐあいもよく、\ruby{檜}{ひのき}を
3.83 この\ruby{\噂}{うわさ}好きな人は\ruby{懐}{ふところ}具合もよく、\ruby{檜}{ひのき}材を
}
{%
  \makeatletter\count@="3040\loop\relax\ifnum \count@<"30A0%
  \ltjsetparameter{rubypreintrusion={\the\count@,0}, %
    rubypostintrusion={\the\count@,0}}%
  \advance\count@1 \repeat
3.84 この\ruby{\噂}{うわさ}の好きな人は\ruby{懐}{ふところ}ぐあいもよく、\ruby{檜}{ひのき}を
3.84 この\ruby{\噂}{うわさ}好きな人は\ruby{懐}{ふところ}具合もよく、\ruby{檜}{ひのき}材を
}
要調整3.85\ {\notalign%この図では揃えない
\hbox{\vrule\vbox{\hsize=15\zw% なぜ行末形にならない!
あああああああああああの\penalty10000\ruby{徑}{こみち}をあああああああ%
ああああああああああああああああいの\ruby{徑}{こみち}}\vrule}}

3.86\ %
\hbox{\vrule\vbox{\hsize=15\zw
ああああああああああああの\ruby{徑}{こみち}をあああああああ%
あああああああ\ruby{徑}{こみち}を}\vrule}


\newpage
{\Large 要調整}

\paragraph{F.1--2節}\ \par

F.01 {\kata\ruby{治|癒}{ち|ゆ}\quad\ruby{模|索}{も|さく}\quad%
\ruby{遷|移}{せん|い}\quad\ruby{混|沌}{こん|とん}}
F.01中 \ruby{治|癒}{ち|ゆ}\quad\ruby{模|索}{も|さく}\quad%
\ruby{遷|移}{せん|い}\quad\ruby{混|沌}{こん|とん}

F.02 \ruby{橋|頭|堡}{きよう|とう|ほ}

F.03 {\kata\ruby{凝|視}{ぎよう|し}\quad\ruby{調|理|師}{ちよう|り|し}\quad%
\ruby{思|春|期}{し|しゆん|き}\quad\ruby{管|状|花}{かん|じよう|か}\quad%
\ruby{蒸|気|船}{じよう|き|せん}}
F.03 \ruby{凝|視}{ぎよう|し}\quad\ruby{調|理|師}{ちよう|り|し}\quad%
\ruby{思|春|期}{し|しゆん|き}\quad\ruby{管|状|花}{かん|じよう|か}\quad%
\ruby{蒸|気|船}{じよう|き|せん}

F.04 {\kata\ruby{未|熟}{み|じゆく}\quad\ruby{法|華|経}{ほ|け|きよう}\quad%
\ruby{顕|微|鏡}{けん|び|きよう}\quad\ruby{課|徴|金}{か|ちよう|きん}\quad%
\ruby{古|戦|場}{こ|せん|じよう}}
F.04 \ruby{未|熟}{み|じゆく}\quad\ruby{法|華|経}{ほ|け|きよう}\quad%
\ruby{顕|微|鏡}{けん|び|きよう}\quad\ruby{課|徴|金}{か|ちよう|きん}\quad%
\ruby{古|戦|場}{こ|せん|じよう}

F.05 の\ruby{峻|別}{しゆん|べつ}は

F.06以降は未チェック
\end{document}

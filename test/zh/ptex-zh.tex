\documentclass{zh-article}
\usepackage{geometry}
\geometry{twoside,left=23mm,width=170mm,right=17mm,top=25mm,height=231mm,bottom=32mm}
\def\JTeX{\leavevmode\hbox{\lower.5ex\hbox{J}\kern-.18em\TeX}}
\def\pTeX{p\TeX}
\def\upTeX{up\TeX}
\def\LuaTeX{Lua\TeX}
\def\cwTeX{cw\TeX}
\def\PuTeX{P\kern-.1667em\lower.5ex\hbox{U}\kern-.1667em\TeX}
\begin{document}
\title{高性能排版:从\pTeX到\LuaTeX-ja}
\author{马起园}
\date{2012年3月}
\maketitle
\section{引言}
\subsection{CJK语言排版综述}
CJK是英文中日韩(Chinese, Japanese, Korean)的缩写。CJK语言的排版因为涉及到大量象形字和诸多技术处理细节,一直是\TeX中的排版难题。自\TeX发明以来,有多种排版CJK语言的方法:
\begin{itemize}
\item 次字体框架(subfont scheme),将CJK字体分割为字符为256个或者更少的字符为单位的字体。次字体框架的弊病为不能在不同的次级字体中增加胶(glue)和出格(kern)。
\item 对\TeX进行扩展,例如扩展至支持Unicode或者JIS之类,可以处理CJK。
\item 根据扩展的\TeX来处理CJK排版。
\end{itemize}

由这三种方法,衍生出了大量的CJK版本的\TeX扩展(为讨论方便,此处略去韩国扩展部分)。pmC/pmJ,\cwTeX,\PuTeX,CCT,TY,\JTeX,\pTeX,\Omega-CJK,CJK,xeCJK,zhspacing。

本文主要讨论日本对\TeX的贡献。
\subsection{日本的\TeX扩展历史}
1987年,NTT的斋藤康己开发了\JTeX\cite{1},使用次字体框架(Subfont Scheme)技术,使用33个字体,每个字体包含256个字符。

同年,ASCII公司的大野俊治和苍泽良一开发了ASCII Nihongo \TeX。

1990年,滨野尚人对ASCII Nihongo \TeX扩展了直行排版功能,此版更名为\pTeX。

1995年,\pTeX升级到\TeX3扩展。

2007年,田中琢尔对\pTeX进行Unicode扩展,可以处理CJK区块。

2011年,LuaTeX-ja项目成立。
\section{\pTeX/\upTeX细节}
\subsection{汉字间隙}
汉字间隙在\pTeX中定义为\verb!\kanjiskip!,初始值为\verb!0pt plus .4pt minus.4pt!,有如下定义:
$$	字宽 \times \frac{1}{单行字数 - 1}	$$
\subsection{汉字与西文}
\verb!\noautospacing!和\verb!\autospacing!

汉字与西文之间:\verb!\inhibitxspcode!和\verb!\xspcode!

西文与汉字之间:\verb!\xspcode!和\verb!\inhibitxspcode!
\subsection{JFM}
\verb!\inhibitglue!

和文フォントのメトリック情報から、自動的に挿入されるグルーの挿入を禁止します。このプリミティブを挿入した箇所にのみ有効です。 
\subsection{记号类}
\verb!\xspcode!和\verb!\inhibitxspcode!

\begin{itemize}
\item   0  漢字と英字間の処理を禁止する。
\item   1  文字と直前の英字との間にだけスペースの挿入を禁止する。
\item   2  直後の英字との間にだけスペースの挿入を禁止する。
\item   3  前後の英字との間に対して、スペースの挿入を許可する。
\end{itemize}
\begin{verbatim}
    \xspcode`(=1    \xspcode`)=2     \xspcode`[=1
    \xspcode`]=2    \xspcode``=1     \xspcode`'=2
    \xspcode`;=2    \xspcode`,=2     \xspcode`.=2

    \inhibitxspcode`、=1   \inhibitxspcode`。=1
    \inhibitxspcode`,=1   \inhibitxspcode`.=1
    \inhibitxspcode`;=1   \inhibitxspcode`?=1
    \inhibitxspcode`(=2   \inhibitxspcode`)=1
    \inhibitxspcode`[=2   \inhibitxspcode`]=1
    \inhibitxspcode`{=2   \inhibitxspcode`}=1
    \inhibitxspcode`‘=2   \inhibitxspcode`’=1
    \inhibitxspcode`“=2   \inhibitxspcode`”=1
    \inhibitxspcode`〔=2   \inhibitxspcode`〕=1
    \inhibitxspcode`〈=2   \inhibitxspcode`〉=1
    \inhibitxspcode`《=2   \inhibitxspcode`》=1
    \inhibitxspcode`「=2   \inhibitxspcode`」=1
    \inhibitxspcode`『=2   \inhibitxspcode`』=1
    \inhibitxspcode`【=2   \inhibitxspcode`】=1
    \inhibitxspcode`―=0   \inhibitxspcode`~=0
    \inhibitxspcode`…=0   \inhibitxspcode`¥=0
    \inhibitxspcode`°=1   \inhibitxspcode`′=1
    \inhibitxspcode`″=1
\end{verbatim}
\section{\LuaTeX-ja:\pTeX的\LuaTeX实现。}

\begin{thebibliography}{99}
\bibitem{1}奥村晴彦,{\it \pTeX\ and Japanese Typesetting\/}, The Asian Journal of \TeX\ 2 (2008), no. 1, 43-51
\bibitem{2}斋藤康己,{\it Report on \JTeX: A Japanese \TeX\/}, TUGboat 8 (1987), no. 2, 103-116.
\bibitem{3}滨野尚人,{\it Vertical typesetting with \TeX\/}, TUGboat 11 (1990), no. 3, 346-352.
\bibitem{4}田中琢尔,{\it \upTeX\/}. \texttt{http://homepage3.nifty.com/ttk/comp/tex/uptex.html}
\bibitem{5}ASCII MEDIA WORKS, {\hei アスキー日本語\TeX\ (\pTeX)}. \texttt{http://ascii.asciimw.jp/pb/ptex/}
\bibitem{6}赵珍焕,奥村晴彦,{\it Typesetting CJK Languages with Omega\/}, \TeX, XML, and Digital Typography, Lecture Notes in Computer Science, vol.~3130, Springer, 2004, 139--148.
\end{thebibliography}
\end{document}

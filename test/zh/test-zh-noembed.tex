%#!lualatex
%% This is a test source based on test-zh-maqiyuan.tex,
%% which is created by MaQiYuan(1113706230@qq.com, ClerkMa@gmail.com).

\documentclass{article}
\usepackage{luatexja}
\DeclareYokoKanjiEncoding{ZH}{}{}
\DeclareKanjiEncodingDefaults{}{}
\DeclareErrorKanjiFont{ZH}{song}{m}{n}{10}
\DeclareKanjiSubstitution{ZH}{song}{m}{n}
%
\newcommand\songdefault{song}
\newcommand\heidefault{hei}
\renewcommand\kanjiencodingdefault{ZH}
\renewcommand\kanjifamilydefault{\songdefault}
\renewcommand\kanjiseriesdefault{\mddefault}
\renewcommand\kanjishapedefault{\updefault}

\DeclareKanjiFamily{ZH}{song}{}
\DeclareFontShape{ZH}{song}{m}{n}{<-> psft:STSong-Light:cid=Adobe-GB1-5;jfm=jis}{}
\DeclareFontShape{JY3}{mc}{m}{n}{<-> psft:Ryumin-Light:cid=Adobe-Japan1-6;jfm=jis}{}

\fontencoding{ZH}\selectfont
\begin{document}
\section{CMap: Adobe-GB1-5}
\kanjiencoding{ZH}\kanjifamily{song}\selectfont
君不見,黃河之水天上來,奔流到海不復回!君不見,高堂明鏡悲白髮,朝如青絲暮成雪!人生得意須盡歡,莫使金樽空對月。天生我材必有用,千金散盡還復來。烹羊宰牛且為樂,會須一飲三百杯。
岑夫子,丹丘生。進酒君莫停。與君歌一曲,請君為我傾耳聽。鐘鼓饌玉不足貴,但願長醉不用醒。古來聖賢皆寂寞,惟有飲者留其名。陳王昔時宴平樂,斗酒十千恣歡謔。主人何為言少錢?
徑須沽酒對君酌。五花馬,千金裘。呼兒將出換美酒,與爾同銷萬古愁。


\section{CMap: Adobe-Japan1-6}
\kanjiencoding{JY3}\kanjifamily{mc}\selectfont
君不見,黃河之水天上來,奔流到海不復回!君不見,高堂明鏡悲白髮,朝如青絲暮成雪!人生得意須盡歡,莫使金樽空對月。天生我材必有用,千金散盡還復來。烹羊宰牛且為樂,會須一飲三百杯。
岑夫子,丹丘生。進酒君莫停。與君歌一曲,請君為我傾耳聽。鐘鼓饌玉不足貴,但願長醉不用醒。古來聖賢皆寂寞,惟有飲者留其名。陳王昔時宴平樂,斗酒十千恣歡謔。主人何為言少錢?
徑須沽酒對君酌。五花馬,千金裘。呼兒將出換美酒,與爾同銷萬古愁。

\end{document}

\input luatexja-core.sty

\jfont\tensong={AR PL SungtiL GB:jfm=quanjiao}
\rm\tensong

\parindent=2\zw
\parskip=0pt
\pdfpagewidth=210mm
\pdfpageheight=297mm
\hsize=35\zw
\hoffset=\dimexpr(\pdfpagewidth-\hsize)/2-1in\relax
\vsize=246mm

xeCJK是一个XeLaTeX宏包,用于排版CJK文字,
包括字体选择和标点控制等。主要特点:

\par\medskip
\item{1.}分别设置CJK和英文字体;
\item{2.}自动忽略CJK文字间的空格而保留其它空格,
允许在非标点汉字和英文字母(a--z, A--Z)间断行;
\item{3.}提供多种标点处理方式:全角式、半角式、开明式、行末半角式;
\item{4.}自动调整中英文间空白。
\par\medskip

xeCJK是在CCT和CJK包基础上发展起来的,支持多种标点格式。
例如,“标点挤压”。
xeCJK是在CCT和CJK包基础上发展起来的,支持多种标点格式。
例如,“标点挤压”。
\par\bigskip

\hrule\bigskip

\newcount\tcount
\tcount=0 \loop
xeCJK是一个XeLaTeX宏包,用于排版CJK文字,
包括字体选择和标点控制等。主要特点:
分别设置CJK和英文字体;
自动忽略CJK文字间的空格而保留其它空格,
允许在非标点汉字和英文字母(a--z, A--Z)间断行;
提供多种标点处理方式:全角式、半角式、开明式、行末半角式;
自动调整中英文间空白。
xeCJK是在CCT和CJK包基础上发展起来的,支持多种标点格式。
例如,“标点挤压”。
xeCJK是在CCT和CJK包基础上发展起来的,支持多种标点格式。
例如,“标点挤压”。
\advance\tcount1
\ifnum\tcount<10 \repeat

\bye

\input luatexja-core.sty
\beginsection A. CMap: Adobe-CNS1-5

%http://www.gutenberg.org/cache/epub/24041/pg24041.html
\jfont\testA={psft:AdobeMingStd-Light:cid=Adobe-CNS1-5;jfm=jis}
\jfont\testB={psft:AdobeFanHeitiStd-Bold:cid=Adobe-CNS1-5;jfm=jis}
\testB
\noindent洛神賦\par
\smallskip
\noindent曹植\par
\testA
\medskip
黃初三年,余朝京師,還濟洛川。古人有言,斯水之神,名曰宓妃。感宋玉對楚王神女之事,遂作斯賦。其辭曰:余從京域,言歸東藩。背伊闕,越轘轅,經通谷,陵景山。日既西傾,車殆馬煩。爾乃稅駕乎蘅皋,秣駟乎芝田,容与乎陽林,流眄乎洛川。于是精移神駭,忽焉思散。俯則末察,仰以殊觀,睹一麗人,于岩之畔。乃援御者而告之曰:“爾有覿于彼者乎?彼何人斯?若此之艷也!”御者對曰:“臣聞河洛之神,名曰宓妃。然則君王所見,無乃日乎?其狀若何?臣愿聞之。”余告之曰:“其形也,翩若惊鴻,婉若游龍。榮曜秋菊,華茂春松。仿佛兮若輕云之蔽月,飄飄兮若流風之回雪。遠而望之,皎若太陽升朝霞;迫而察之,灼若芙蕖出淥波。襛纖得衷,修短合度。肩若削成,腰如約素。延頸秀項,皓質呈露。芳澤無加,鉛華弗御。云髻峨峨,修眉聯娟。丹唇外朗,皓齒內鮮,明眸善睞,靨輔承權。瑰姿艷逸,儀靜体閒。柔情綽態,媚于語言。奇服曠世,骨像應圖。披羅衣之璀粲兮,珥瑤碧之華琚。戴金翠之首飾,綴明珠以耀軀。踐遠游之文履,曳霧綃之輕裾。微幽蘭之芳藹兮,步踟躕于山隅。于是忽焉縱体,以遨以嬉。左倚采旄,右蔭桂旗。壤皓腕于神滸兮,采湍瀨之玄芝。余情悅其淑美兮,心振蕩而不怡。無良媒以接歡兮,托微波而通辭。愿誠素之先達兮,解玉佩以要之。嗟佳人之信修,羌習禮而明詩。抗瓊珶以和予兮,指潛淵而為期。執眷眷之款實兮,懼斯靈之我欺。感交甫之棄言兮,悵猶豫而狐疑。收和顏而靜志兮,申禮防以自持。于是洛靈感焉,徙倚彷徨,神光离合,乍陰乍陽。竦輕軀以鶴立,若將飛而未翔。踐椒涂之郁烈,步蘅薄而流芳。超長吟以永慕兮,聲哀厲而彌長。爾乃眾靈雜遢,命儔嘯侶,或戲清流,或翔神渚,或采明珠,或拾翠羽。從南湘之二妃,攜漢濱之游女。歎匏瓜之無匹兮,詠牽牛之獨處。揚輕褂之猗靡兮,翳修袖以延佇。休迅飛鳧,飄忽若神,陵波微步,羅襪生塵。動無常則,若危若安。進止難期,若往若還。轉眄流精,光潤玉顏。含辭未吐,气若幽蘭。華容婀娜,令我忘餐。于是屏翳收風,川后靜波。馮夷鳴鼓,女媧清歌。騰文魚以警乘,鳴玉鸞以偕逝。六龍儼其齊首,載云車之容裔,鯨鯢踊而夾轂,水禽翔而為衛。于是越北沚。過南岡,紆素領,回清陽,動朱唇以徐言,陳交接之大綱。恨人神之道殊兮,怨盛年之莫當。抗羅袂以掩涕兮,淚流襟之浪浪。悼良會之永絕兮。哀一逝而异鄉。無微情以效愛兮,獻江南之明璫。雖潛處于太陽,長寄心于君王。忽不悟其所舍,悵神宵而蔽光。于是背下陵高,足往神留,遺情想像,顧望怀愁。冀靈体之复形,御輕舟而上溯。浮長川而忘返,思綿綿督。夜耿耿而不寐,沾繁霜而至曙。命仆夫而就駕,吾將歸乎東路。攬騑轡以抗策,悵盤桓而不能去。
\beginsection B. CMap: Adobe-GB1-5

\jfont\testA={psft:SimSun:cid=Adobe-GB1-5;jfm=jis}
\jfont\testB={psft:SimHei:cid=Adobe-GB1-5;jfm=jis}
\testB
\noindent 绿\par
\smallskip
\noindent朱自清\par
\medskip
\testA
我第二次到仙岩的时候,我惊诧于梅雨潭的绿了。

梅雨潭是一个瀑布潭。仙瀑有三个瀑布,梅雨瀑最低。走到山边,便听见花花花花的声音;抬起头,镶在两条湿湿的黑边儿里的,一带白而发亮的水便呈现于眼前了。

我们先到梅雨亭。梅雨亭正对着那条瀑布;坐在亭边,不必仰头,便可见它的全体了。亭下深深的便是梅雨潭。这个亭踞在突出的一角的岩石上,上下都空空儿的;仿佛一只苍鹰展着翼翅浮在天宇中一般。三面都是山,像半个环儿拥着;人如在井底了。这是一个秋季的薄阴的天气。微微的云在我们顶上流着;岩面与草丛都从润湿中透出几分油油的绿意。而瀑布也似乎分外的响了。那瀑布从上面冲下,仿佛已被扯成大小的几绺;不复是一幅整齐而平滑的布。岩上有许多棱角;瀑流经过时,作急剧的撞击,便飞花碎玉般乱溅着了。那溅着的水花,晶莹而多芒;远望去,像一朵朵小小的白梅,微雨似的纷纷落着。据说,这就是梅雨潭之所以得名了。但我觉得像杨花,格外确切些。轻风起来时,点点随风飘散,那更是杨花了。——这时偶然有几点送入我们温暖的怀里,便倏的钻了进去,再也寻它不着。

梅雨潭闪闪的绿色招引着我们;我们开始追捉她那离合的神光了。揪着草,攀着乱石,小心探身下去,又鞠躬过了一个石穹门,便到了汪汪一碧的潭边了。瀑布在襟袖之间;但我的心中已没有瀑布了。我的心随潭水的绿而摇荡。那醉人的绿呀,仿佛一张极大极大的荷叶铺着,满是奇异的绿呀。我想张开两臂抱住她;但这是怎样一个妄想呀。--站在水边,望到那面,居然觉着有些远呢!这平铺着,厚积着的绿,着实可爱。她松松的皱缬着,像少妇拖着的裙幅;她轻轻的摆弄着,像跳动的初恋的处女的心;她滑滑的明亮着,像涂了“明油”一般,有鸡蛋清那样软,那样嫩,令人想着所曾触过的最嫩的皮肤;她又不杂些儿法滓,宛然一块温润的碧玉,只清清的一色--但你却看不透她!我曾见过北京什刹海指地的绿杨,脱不了鹅黄的底子,似乎太淡了。我又曾见过杭州虎跑寺旁高峻而深密的“绿壁”,重叠着无穷的碧草与绿叶的,那又似乎太浓了。其余呢,西湖的波太明了,秦淮河的又太暗了。可爱的,我将什么来比拟你呢?我怎么比拟得出呢?大约潭是很深的、故能蕴蓄着这样奇异的绿;仿佛蔚蓝的天融了一块在里面似的,这才这般的鲜润呀。——那醉人的绿呀!我若能裁你以为带,我将赠给那轻盈的舞女;她必能临风飘举了。我若能挹你以为眼,我将赠给那善歌的盲妹;她必明眸善睐了。我舍不得你;我怎舍得你呢?我用手拍着你,抚摩着你,如同一个十二三岁的小姑娘。我又掬你入口,便是吻着她了。我送你一个名字,我从此叫你“女儿绿”,好么?

我第二次到仙岩的时候,我不禁惊诧于梅雨潭的绿了。 
\beginsection C. CMap: Adobe-Japan1-6

%http://www.aozora.gr.jp/cards/000879/files/4872_21839.html
\jfont\testA={psft:Ryumin-Light:cid=Adobe-Japan1-6;jfm=jis}
\jfont\testB={psft:GothicBBB-Medium:cid=Adobe-Japan1-6;jfm=jis}
\testB

\noindent 愛読書の印象\par
\smallskip
\noindent 芥川龍之介\par
\medskip
\testA
子供の時の愛読書は「西遊記」が第一である。これ等は今日でも僕の愛読書である。比喩談としてこれほどの傑作は、西洋には一つもないであらうと思ふ。名高いバンヤンの「天路歴程」なども到底この「西遊記」の敵ではない。それから「水滸伝」も愛読書の一つである。これも今以て愛読してゐる。一時は「水滸伝」の中の一百八人の豪傑の名前を悉く諳記(あんき)してゐたことがある。その時分でも押川春浪氏の冒険小説や何かよりもこの「水滸伝」だの「西遊記」だのといふ方が遥かに僕に面白かつた。

中学へ入学前から徳富蘆花氏の「自然と人生」や樗牛の「平家雑感」や小島烏水氏の「日本山水論」を愛読した。同時に、夏目さんの「猫」や鏡花氏の「風流線」や緑雨の「あられ酒」を愛読した。だから人の事は笑へない。僕にも「文章倶楽部」の「青年文士録」の中にあるやうな「トルストイ、坪内士行、大町桂月」時代があつた。

中学を卒業してから色んな本を読んだけれども、特に愛読した本といふものはないが、概して云ふと、ワイルドとかゴーチエとかいふやうな絢爛(けんらん)とした小説が好きであつた。それは僕の気質からも来てゐるであらうけれども、一つは慥(たし)かに日本の自然主義的な小説に厭きた反動であらうと思ふ。ところが、高等学校を卒業する前後から、どういふものか趣味や物の見方に大きな曲折が起つて、前に言つたワイルドとかゴーチエとかといふ作家のものがひどくいやになつた。ストリンドベルクなどに傾倒したのはこの頃である。その時分の僕の心持からいふと、ミケエロ・アンヂエロ風な力を持つてゐない芸術はすべて瓦礫のやうに感じられた。これは当時読んだ「ジヤンクリストフ」などの影響であつたらうと思ふ。

さういふ心持が大学を卒業する後までも続いたが、段々燃えるやうな力の崇拝もうすらいで、一年前から静かな力のある書物に最も心を惹かれるやうになつてゐる。但、静かなと言つてもたゞ静かだけでも力のないものには余り興味がない。スタンダールやメリメエや日本物で西鶴などの小説はこの点で今の僕には面白くもあり、又ためにもなる本である。

序ながら附け加へておくが、此間「ジヤンクリストフ」を出して読んで見たが、昔ほど感興が乗らなかつた。あの時分の本はだめなのかと思つたが、「アンナカレニナ」を出して二三章読んで見たら、これは昔のやうに有難い気がした。
\beginsection D. CMap: Adobe-Korea1-2

\jfont\test={psft:Batang:cid=Adobe-Korea1-2;jfm=jis}
\test

진달래꽃\par

나 보기가 역겨워진달래꽃\par
가실 때에는\par
말없이 고이 보내 드리오리다.\par

영변에 약산\par
진달래꽃,\par
아름 따다 가실 길에 뿌리오리다.\par 

가시는 걸음 걸음\par
놓인 그 꽃을\par
사뿐히 즈려 밟고 가시옵소서.\par 

나 보기가 역겨워\par
가실 때에는\par
죽어도 아니 눈물 흘리오리다.
\end
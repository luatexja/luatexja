%#!lualatex

\documentclass{ltjsarticle}
\usepackage{luatexja-fontspec,luatexja-adjust,xcolor,amsmath,amssymb}
\DeclareKanjiFamily{JY3}{piyo}{}
\DeclareFontShape{JY3}{piyo}{m}{n}{<-> s*[\Cjascale] file:ipaexg.ttf:jfm=jlreq}{}

\def\sq{%
  \hbox to 1\zw{\hss\fboxsep=-.5\fboxrule\fbox{%
   \hskip\dimexpr-.5\zw-.2pt\vrule width.4pt height.08\zw depth.12\zw%
  \hskip\dimexpr.5\zw-.2pt\relax}\hss}}
\def\sb{\hbox to 1\zw{\hss\fboxsep=-.5\fboxrule\fbox{%
  ■\hskip\dimexpr-.5\zw-.2pt\vrule width.4pt height.08\zw depth.12\zw%
  \hskip\dimexpr.5\zw-.2pt\relax}\hss}}
\newbox\gridbox
\setbox\gridbox=\hbox to 20\zw{\sq\sq\sq\sq\sb\sq\sq\sq\sq\sb\sq\sq\sq\sq\sb\sq\sq\sq\sq\sb}
\def\outbox#1{%
  \leavevmode\hbox to 2em{\tt #1\hss}\vrule
  \textcolor{cyan!50!white}{\copy\gridbox}\hskip-20\zw\copy0\vrule\par
}

\ltjdisableadjust
\long\def\testbox#1{%
  \textcolor{red!80!black}{\ltjenableadjust[priority=true, lineend=extended]
    \setbox0=\vbox{%
      \hsize=20\zw#1%
	}\ltjdisableadjust\outbox{ON}}\par
  \textcolor{black!90!white}{%
    \ltjdisableadjust\setbox0=\vbox{%
	  \hsize=20\zw#1%
    }\outbox{OFF}}\par\medskip
}

\usepackage[textwidth=52\zw,lines=47,centering]{geometry}
\parindent0pt
\begin{document}
\jfontspec[YokoFeatures={JFM=hang}]{ipam.ttf}

\ltjsetparameter{kanjiskip=.0\zw plus .4pt minus .5pt}

\twocolumn[{\tt kanjiskip: \ltjgetparameter{kanjiskip}

\ltjsetparameter{xkanjiskip=.25\zw plus .25\zw minus .125\zw}
xkanjiskip: \ltjgetparameter{xkanjiskip}}

このテストでは,行末の句読点・中点類の位置調整を有効にした
\texttt{jfm-hang.lua}を用いている.
{\begin{itemize}
\item 句点は,調整量に合わせて,ぶら下げ,全角取りの2種類から選択される.
\item 読点は,調整量に合わせて,ぶら下げ,二分取り,全角取りの3種類から選択される.
\item 中点類は,行末に四分空きを追加することのみ対応.
詰める際の「直前の四分空きも取る」は未実装,

\item \texttt{lineend=true}のときは,\TeX による行分割後に行末文字の位置調整が行われる.
行われる条件は,
\begin{description}
\item[最終行以外] 無限大の伸長度を持つグルーが関わっていない
\item[最終行] 無限大の伸長度を持つグルーは\texttt{\textbackslash parfillskip}のみで,かつ
\begin{align*}
 \min\{(\hbox{許される最小の行末文字と行末の間}),0\}
  &\leq(\hbox{\texttt{\textbackslash parfillskip}のこの行における実際の長さ})\\
  &\leq\max\{(\hbox{許される最大の行末文字と行末の間}),0\}
\end{align*}
となっている
\end{description}

\item \texttt{lineend=extended}のときは,\TeX による行分割の時点で行末位置の文字調整を考慮
      する.但し,段落の最後の文字については例外的に行わず,代わりに
上の「\texttt{lineend=true}の場合」の最終行のときと同じ補正を行う.
\end{itemize}}]

\testbox{%
◆◆◆◆◆◆◆◆◆◆◆◆◆◆◆◆◆◆◆◆
%あいうえおかきくけこさしすせそたちつてと
}

\testbox{%
あいうえおかきくけこ「「さしすせそたちつて
}

\testbox{%
あうえおかきAI M M Dこさ\texttt{DO i=1,10}『
}

\testbox{%
「\texttt{\textbackslash expandafter}ユーザの集い」が開催された
}

\testbox{%
あいうえおきくけこ「」さ123456そたちつて
}

\typeout{あいうえお}


\def\pTeX{p\kern-.2em\TeX}
\testbox{%
日本で\pTeX,p\LaTeX がよく使われている。
}

中点類の空き詰めは括弧類より優先
\typeout{中点類の空き詰め}

\testbox{%
あいうえおかきくけ・こさしすせそたち「「あ
}

次の例では\verb+\parfillskip+を0にしている

\testbox{%
あいうえおかきくけこさしすせそたちつて・
}
\testbox{%
\parfillskip0ptあいうえおかきくけこさしすせそたちつて・
}

行末の句点
\typeout{行末の句点}

\testbox{%
あいうえおかきくけこさしすせそたちつて.
}
\testbox{%
あいうえおかきくけこさしすせそたちつ\vrule width .25\zw て.
}
\testbox{%
あいうえおかきくけこさしすせそたちつ\vrule width .5\zw て.
}
\testbox{%
あいうえおかきくけこさしすせそたちつ\vrule width .75\zw て.
}
\testbox{%
あいうえおかきくけこさしすせそたちつ\vrule width 1\zw て.
}

行末の読点
\typeout{行末の読点}

\testbox{%
あいうえおかきくけこさしすせそたちつて,
}
\testbox{%
あいうえおかきくけこさしすせそたちつ\vrule width .25\zw て,
}
\testbox{%
あいうえおかきくけこさしすせそたちつ\vrule width .5\zw て,
}
\testbox{%
あいうえおかきくけこさしすせそたちつ\vrule width .75\zw て,
}
\testbox{%
あいうえおかきくけこさしすせそたちつ\vrule width 1\zw て,
}


伸び縮みで異なる優先度.

以下の例では,「ぱ」と鍵括弧の間は自然長・伸び・縮み全部半角.
\texttt{kanjiskip}より伸びる時の優先度は高く,
縮むときの優先度は低い.

{\ltjsetparameter{kanjiskip=0pt plus 1.5pt minus 1.5pt}
\testbox{%
  \parfillskip0ptあいうえおかきくけこ\<\vrule width7.5\zw\<ぱ「
}
\testbox{%
  \parfillskip0ptあいうえおかきくけこ\<\vrule width9\zw\<ぱ「
}
}

\newpage
xkanjiskip手動挿入.

\texttt{\string\insertxkanjiskip}で挿入されたグルーは\\
\texttt{\string\hskip\string\ltjgetparameter\{xkanjiskip\}}\insertxkanjiskip によるグルー%
(下段)とは違い,
優先度付き行長調整でも通常のxkanjiskipと同等の挙動をする.

{\ltjsetparameter{xkanjiskip=10pt plus 50pt minus 50pt,kanjiskip=0pt plus 5pt minus 5pt}%
\testbox{%
  \vrule width2.5\zw ◆◆◆◆◆◆◆◆\insertxkanjiskip a\null ◆◆◆◆◆◆◆a%
}
\testbox{%
  \vrule width2.5\zw ◆◆◆◆◆◆◆◆\hskip\ltjgetparameter{xkanjiskip}a\null ◆◆◆◆◆◆◆a%
}
}

\bigskip

次ページ以降の出典:
  Wikisource日本語版「竹取物語」(一部),2016/08/11閲覧\\
{\catcode`\%=11\texttt{https://ja.wikisource.org/wiki/%E7%AB%B9%E5%8F%96%E7%89%A9%E8%AA%9E}}

\twocolumn
\def\USTCON{\hbox{USTCON}}
\small\newdimen\R \R=25\zw
\def\sample#1{\small\hsize=\R\jfontspec[YokoFeatures={JFM=hang}]{ipam.ttf}
{\centering\scriptsize\textbf{\ttfamily #1}\par}\parindent1\zw%
\ltjsetparameter{kanjiskip=.0\zw plus .4pt minus .5pt}
かやうにて、御心を互に慰め給ふほどに、三年ばかりありて、春の初より、かぐや姫月のおもしろう出でたるを見て、常よりも物思ひたるさまなり。ある人の「月の顔見るは忌むこと。」ゝ制しけれども、ともすればひとまには月を見ていみじく泣き給ふ。七月のもちの月にいで居て、切に物思へるけしきなり。近く使はるゝ人々、竹取の翁に告げていはく、「かぐや姫例も月をあはれがり給ひけれども、この頃となりてはたゞ事にも侍らざンめり。いみじく思し歎くことあるべし。よく〳〵見奉らせ給へ。」といふを聞きて、かぐや姫にいふやう、「なでふ心ちすれば、かく物を思ひたるさまにて月を見給ふぞ。うましき世に。」といふ。かぐや姫、「月を見れば世の中こゝろぼそくあはれに侍り。なでふ物をか歎き侍るべき。」といふ。かぐや姫のある所に至りて見れば、なほ物思へるけしきなり。これを見て、「あが佛何事を思ひ給ふぞ。思すらんこと何事ぞ。」といへば、「思ふこともなし。物なん心細く覺ゆる。」といへば、翁、「月な見給ひそ。これを見給へば物思すけしきはあるぞ。」といへば、「いかでか月を見ずにはあらん。」とて、なほ月出づれば、いで居つゝ歎き思へり。夕暗には物思はぬ氣色なり。月の程になりぬれば、猶時々はうち歎きなきなどす。是をつかふものども、「猶物思すことあるべし。」とさゝやけど、親を始めて何事とも知らず。八月十五日ばかりの月にいで居て、かぐや姫いといたく泣き給ふ。人めも今はつゝみ給はず泣き給ふ。これを見て、親どもゝ「何事ぞ。」と問ひさわぐ。かぐや姫なく〳〵いふ、「さき〳〵も申さんと思ひしかども、『かならず心惑はし給はんものぞ。』と思ひて、今まで過し侍りつるなり。『さのみやは。』とてうち出で侍りぬるぞ。おのが身はこの國の人にもあらず、月の都の人なり。それを昔の契なりけるによりてなん、この世界にはまうで來りける。今は歸るべきになりにければ、この月の十五日に、かのもとの國より迎に人々まうでこんず。さらずまかりぬべければ、思し歎かんが悲しきことを、この春より思ひ歎き侍るなり。」といひて、いみじく泣く。翁「こはなでふことをの給ふぞ。竹の中より見つけきこえたりしかど、菜種の大さおはせしを、我丈たち並ぶまで養ひ奉りたる我子を、何人か迎へ聞えん。まさに許さんや。」といひて、「我こそ死なめ。」とて、泣きのゝしることいと堪へがたげなり。かぐや姫のいはく、「月の都の人にて父母あり。片時の間とてかの國よりまうでこしかども、かくこの國には數多の年を經ぬるになんありける。かの國の父母の事もおぼえず。こゝにはかく久しく遊び聞えてならひ奉れり。いみじからん心地もせず、悲しくのみなんある。されど己が心ならず罷りなんとする。」といひて、諸共にいみじう泣く。つかはるゝ人々も年頃ならひて、立ち別れなんことを、心ばへなどあてやかに美しかりつることを見ならひて、戀しからんことの堪へがたく、湯水も飮まれず、同じ心に歎しがりけり。この事を帝きこしめして、竹取が家に御使つかはさせ給ふ。御使に竹取いで逢ひて、泣くこと限なし。この事を歎くに、髪も白く腰も屈り目もたゞれにけり。翁今年は五十許なりけれども、「物思には片時になん老になりにける。」と見ゆ。御使仰事とて翁にいはく、「いと心苦しく物思ふなるは、誠にか。」と仰せ給ふ。

}


\ltjenableadjust[lineend=extended, priority=true]
\setbox40000=\vtop{\sample{lineend=extended, priority=true}}
\ltjdisableadjust
\ltjenableadjust[lineend=true, priority=false, priority=true]
\setbox40002=\vtop{\sample{linened=true,priority=true}}
\ltjdisableadjust
\ltjenableadjust[lineend=false, priority=false, priority=true]
\setbox40004=\vtop{\sample{lineend=false,priority=true}}
\ltjdisableadjust
\ltjenableadjust[lineend=extended, priority=false]
\setbox40010=\vtop{\sample{lineend=extended, priority=false}}
\ltjdisableadjust
\ltjenableadjust[lineend=true, priority=false, priority=false]
\setbox40012=\vtop{\sample{linened=true,priority=false}}
\ltjdisableadjust
\ltjenableadjust[lineend=false, priority=false, priority=false]
\setbox40014=\vtop{\sample{lineend=false,priority=false}}
\ltjdisableadjust

{\catcode`\#=12
\directlua{%
  function gb(a)
     local t = tex.getbox(a)
     local x = {}
     for n in node.traverse_id(node.id('hlist'), t.head) do
        local b = n.glue_order>0 and 0 or math.floor(100*n.glue_set^3+0.5);
        if b<=12 then x[#x+1]={2,b}    % decent
        elseif n.glue_sign==1 and b>=100 then x[#x+1]={0,b} %very loose
        elseif n.glue_sign==1 then x[#x+1]={1,b} % loose
        else   x[#x+1]={3,b} end %tight
     end
     x[0]={2, 0}
     local d = 0
     for i=1,#x do
       d = d + math.floor((tex.linepenalty + x[i][2])^2+0.5)
       if math.abs(x[i][1]-x[i-1][1])>=1 then d = d + tex.adjdemerits end
     end
     tex.sprint(-2,tostring(d) )
  end
}}
\protected\def\getbadness#1{\par\medskip\textcolor{blue}{\small demerits: \directlua{gb(#1)}}}

\noindent
\vrule\copy40010\vrule\relax\getbadness{40010}

\medskip
\noindent
\vrule\copy40012\vrule\relax\getbadness{40012}

\newpage
\noindent
\vrule\copy40014\vrule\relax\getbadness{40014}

\R28\zw
\ltjenableadjust[lineend=extended, priority=true]
\setbox40000=\vtop{\sample{lineend=extended, priority=true}}
\ltjdisableadjust
\ltjenableadjust[lineend=true, priority=false, priority=true]
\setbox40002=\vtop{\sample{linened=true,priority=true}}
\ltjdisableadjust
\ltjenableadjust[lineend=false, priority=false, priority=true]
\setbox40004=\vtop{\sample{lineend=false,priority=true}}
\ltjdisableadjust
\ltjenableadjust[lineend=extended, priority=false]
\setbox40010=\vtop{\sample{lineend=extended, priority=false}}
\ltjdisableadjust
\ltjenableadjust[lineend=true, priority=false, priority=false]
\setbox40012=\vtop{\sample{linened=true,priority=false}}
\ltjdisableadjust
\ltjenableadjust[lineend=false, priority=false, priority=false]
\setbox40014=\vtop{\sample{lineend=false,priority=false}}
\ltjdisableadjust

\noindent
\vrule\copy40010\vrule\relax\getbadness{40010}

\newpage
\noindent
\vrule\copy40012\vrule\relax\getbadness{40012}
\newpage
\noindent
\vrule\copy40014\vrule\relax\getbadness{40014}



\newpage

\def\MYGUARDA#1{ \kern-\zw#1\kern-\zw  \ignorespaces}
\def\MYGUARDB#1{{\kanjifamily{mc}\selectfont  }%
        \kern-\zw#1\kern-\zw{\kanjifamily{mc}\selectfont  }\ignorespaces}
\def\MYGUARDC#1{\ltjghostjachar#1\ltjghostjachar\ignorespaces}
\def\MYVRULE{\raisebox{-2pt}{漢}}

\paragraph{jfm: ujis}\ 

\leavevmode\hbox to 15\zw{◆\MYVRULE ◆◆◆◆$f$◆}\par
\leavevmode\hbox to 15\zw{◆\MYGUARDA{\MYVRULE}◆◆◆◆$f$◆} A\par
\leavevmode\hbox to 15\zw{◆\MYGUARDB{\MYVRULE}◆◆◆◆$f$◆} B\par
\leavevmode\hbox to 15\zw{◆\MYGUARDC{\MYVRULE}◆◆◆◆$f$◆} C\par
\leavevmode\hbox to 15\zw{◆◇◆◆◆◆$f$◆} 比較用\par

\paragraph{jfm: jlreq}\ 

\kanjifamily{piyo}\selectfont
\leavevmode\hbox to 15\zw{◆\MYVRULE ◆◆◆◆$f$◆}\par
\leavevmode\hbox to 15\zw{◆\MYGUARDA{\MYVRULE}◆◆◆◆$f$◆} A\par
\leavevmode\hbox to 15\zw{◆\MYGUARDB{\MYVRULE}◆◆◆◆$f$◆} B\par
\leavevmode\hbox to 15\zw{◆\MYGUARDC{\MYVRULE}◆◆◆◆$f$◆} C\par
\leavevmode\hbox to 15\zw{◆◇◆◆◆◆$f$◆} 比較用\par

\ltjsetparameter{xkanjiskip=10pt}

ab\ltjghostjachar A\ltjghostjachar de\ltjghostjachar 漢f

ab\ltjghostbeforejachar A\ltjghostbeforejachar de\ltjghostbeforejachar 漢\ltjghostbeforejachar  f

ab\ltjghostafterjachar A\ltjghostafterjachar de\ltjghostafterjachar 漢\ltjghostafterjachar  f
​
\end{document}

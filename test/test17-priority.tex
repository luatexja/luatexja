%#!lualatex

\documentclass{ltjsarticle}
\usepackage{luatexja-fontspec,luatexja-adjust, luacode,xcolor}

\def\sq{%
  \hbox to 1\zw{\hss\fboxsep=-.5\fboxrule\fbox{%
   \hskip\dimexpr-.5\zw-.2pt\vrule width.4pt height.08\zw depth.12\zw%
  \hskip\dimexpr.5\zw-.2pt\relax}\hss}}
\def\sb{\hbox to 1\zw{\hss\fboxsep=-.5\fboxrule\fbox{%
  ■\hskip\dimexpr-.5\zw-.2pt\vrule width.4pt height.08\zw depth.12\zw%
  \hskip\dimexpr.5\zw-.2pt\relax}\hss}}
\newbox\gridbox
\setbox\gridbox=\hbox to 20\zw{\sq\sq\sq\sq\sb\sq\sq\sq\sq\sb\sq\sq\sq\sq\sb\sq\sq\sq\sq\sb}
\def\outbox#1{%
  \leavevmode\hbox to 2em{\tt #1\hss}\vrule
  \textcolor{cyan!50!white}{\copy\gridbox}\hskip-20\zw\copy0\vrule\par
}

\ltjdisableadjust
\long\def\testbox#1{%
  \textcolor{red!80!black}{\ltjenableadjust[all, lineend=extended]
    \setbox0=\vbox{%
      \hsize=20\zw#1%
	}\ltjdisableadjust\outbox{ON}}\par
  \textcolor{black!90!white}{%
    \ltjdisableadjust\setbox0=\vbox{%
	  \hsize=20\zw#1%
    }\outbox{OFF}}\par\medskip
}

\parindent0pt
\begin{document}
\jfontspec[YokoFeatures={JFM=hang}]{ipam.ttf}

\ltjsetparameter{kanjiskip=.0\zw plus .4pt minus .4pt}
{\tt kanjiskip: \ltjgetparameter{kanjiskip}

\ltjsetparameter{xkanjiskip=.25\zw plus .25\zw minus .125\zw}
xkanjiskip: \ltjgetparameter{xkanjiskip}}

このテストでは,行末の句読点・中点類の位置調整を有効にした
\texttt{jfm-hang.lua}を用いている.
\begin{itemize}
\item 句点は,調整量に合わせて,ぶら下げ,全角取りの2種類から選択される.
\item 読点は,調整量に合わせて,ぶら下げ,二分取り,全角取りの3種類から選択される.
\item 中点類は,行末に四分空きを追加することのみ対応.
詰める際の「直前の四分空きも取る」は未実装,

\item \texttt{lineend=true}のときは,\TeX による行分割後に行末文字の位置調整が行われる.
行われる条件は,
\begin{description}
\item[最終行以外] 無限大の伸長度を持つグルーが関わっていない
\item[最終行] 無限大の伸長度を持つグルーは\verb+\parfillskip+のみで,かつ
\[
 (\hbox{許される最小の行末文字と行末の間})\leq
 (\hbox{\verb+\parfillskip+のこの行における実際の長さ})\leq
 (\hbox{許される最大の行末文字と行末の間})
\]
となっている
\end{description}

\item \texttt{lineend=extended}のときは,\TeX による行分割の時点で行末位置の文字調整を考慮
      する.但し,段落の最後の文字については例外的に行わず,代わりに
上の「\texttt{lineend=true}の場合」の最終行のときと同じ補正を行う.
\end{itemize}

\testbox{%
◆◆◆◆◆◆◆◆◆◆◆◆◆◆◆◆◆◆◆◆
%あいうえおかきくけこさしすせそたちつてと
}

\testbox{%
あいうえおかきくけこ「「さしすせそたちつて
}

\testbox{%
あうえおかきAI M M Dこさ\texttt{DO i=1,10}『
}

\testbox{%
「\texttt{\textbackslash expandafter}ユーザの集い」が開催された
}

\testbox{%
あいうえおきくけこ「」さ123456そたちつて
}

\typeout{あいうえお}


\def\pTeX{p\kern-.2em\TeX}
\testbox{%
日本で\pTeX,p\LaTeX がよく使われている。
}

中点類の空き詰めは括弧類より優先
\typeout{中点類の空き詰め}

\testbox{%
あいうえおかきくけ・こさしすせそたち「「あ
}

句読点類・中点類の後ろ空き

\testbox{%
日本では\pTeX,p\LaTeX が使われている。
}
\testbox{%
あいうえおかきくけこさしすせそたちつて・
}

\newpage
行末の句点
\typeout{行末の句点}

\testbox{%
あいうえおかきくけこさしすせそたちつて.
}
\testbox{%
あいうえおかきくけこさしすせそたちつ\vrule width .25\zw て.
}
\testbox{%
あいうえおかきくけこさしすせそたちつ\vrule width .5\zw て.
}
\testbox{%
あいうえおかきくけこさしすせそたちつ\vrule width .75\zw て.
}
\testbox{%
あいうえおかきくけこさしすせそたちつ\vrule width 1\zw て.
}

行末の読点
\typeout{行末の読点}

\testbox{%
あいうえおかきくけこさしすせそたちつて,
}
\testbox{%
あいうえおかきくけこさしすせそたちつ\vrule width .25\zw て,
}
\testbox{%
あいうえおかきくけこさしすせそたちつ\vrule width .5\zw て,
}
\testbox{%
あいうえおかきくけこさしすせそたちつ\vrule width .75\zw て,
}
\testbox{%
あいうえおかきくけこさしすせそたちつ\vrule width 1\zw て,
}

\newpage

\end{document}

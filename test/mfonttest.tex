%#!lualatex
\documentclass{ltjsarticle}

\begin{document}

%%%%%%%% plain TeX
\jfont\jaA=KozGoPr6N-Regular.otf:jfm=ujis at 9.24872pt
\jfont\jaB=KozMinPr6N-Bold.otf:jfm=ujis   at 9.24872pt
\jfont\jaC=KozGoPr6N-Bold.otf:jfm=ujis    at 9.24872pt

\jaA 
あア漢字% 置換なし
\ltjdeclarealtfont\jaA\jaB{"3000-"30FF}
あア漢字% 「あ」のみ置換(段落末尾の状況が全段落で通用)
\ltjdeclarealtfont\jaA\jaA{"30A0-"30FF}
あア漢字% 「あ」のみ置換

\ltjdeclarealtfont\jaA\jaB{"3000-"30FF}
\ltjdeclarealtfont\jaA\jaC{`漢}
あア漢字% 「あ」「ア」「漢」置換
\mc
あア漢字% 置換なし
\jaA 
あア漢字% 「あ」「ア」「漢」置換

\jaA 
あア漢字% 置換なし,\ltjclearaltfont の効力はこの段落全部なので
\ltjclearaltfont\jaA
あア漢字% 置換なし

%%%%%%%% LaTeX

\DeclareAlternateKanjiFont{JY3}{mc}{m}{n}{JY3}{gt}{m}{n}{`い,`う}
\mc
あいうえお%             \selectfont しないと有効にはならない
\selectfont あいうえお% 「い」「う」が置換
{\Large  あいうえお}

{%
  \DeclareAlternateKanjiFont{JY3}{mc}{m}{n}{JY3}{mc}{m}{n}{`い}%
}% always global
\DeclareAlternateKanjiFont{JY3}{mc}{m}{n}{JY3}{gt}{m}{n}{`お}
\selectfont あいうえお% 「う」「お」が置換
{\Large  あいうえお}

\ClearAlternateKanjiFont{JY3}{mc}{m}{n}
{\Large  あいうえお}%    置換なし
あいうえお%              まだ「う」「お」が置換のまま

\end{document}

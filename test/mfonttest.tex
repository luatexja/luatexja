%#!lualatex
\documentclass{ltjsarticle}

%%% ltj-charrange.lua に本ドキュメント末尾のようなパッチが必要

\makeatletter
%%%%%%%%%%%%%%%%%%%%%%%%%%%%%%%%
% 42 番の文字範囲を「かな書体」用として宣言
\ltjdefcharrange{42}{"3000-"30FF, "31F0-"31FF, "FF00-"FFEF, "1B000-"1B0FF}
% 3000-303F CJK Symbols and Punctuation
% 3040-309F Hiragana
% 30A0-30FF Katakana
% 31F0-31FF Katakana Phonetic Extensions
% FF00-FFEF Halfwidth and Fullwidth Forms
% 1B000-1B0FF Kana Supplement

% 42 番の文字範囲を和文文字の範囲とする
\ltjsetparameter{jacharrange={+42}}

%% 42 にあまり大きい意味はない.既に使われていないもののうち
%% 適当に大きいものをとった

% 「現在のかな用書体」のフォント番号を格納する attribute
\newluatexattribute\ltj@curkanafnt

%%%%%%%%%%%%%%%%%%%%%%%%%%%%%%%%
% ここから Lua code
\usepackage{luacode}

\begin{luacode}
local ltjc = luatexja.charrange

local attr_curjfnt = luatexbase.attributes['ltj@curjfnt']
local attr_curkfnt = luatexbase.attributes['ltj@curkanafnt']
local attr_icflag = luatexbase.attributes['ltj@icflag']
local has_attr = node.has_attribute
local set_attr = node.set_attribute
local id_glyph = node.id('glyph')

local function font_replace_42(head)
   for p in node.traverse_id(id_glyph, head) do
      if (has_attr(p, attr_icflag) or 0)<=0 and ltjc.jcr_table_main[p.char]==42 then
         set_attr(p, attr_curjfnt, 
            has_attr(p, attr_curkfnt) or (has_attr(p, attr_curjfnt) or p.font))
         -- 42 番の文字範囲に即する文字は,(もし和文ならば)かな書体で組まれる
      end
   end
   return head
end
luatexbase.add_to_callback('hyphenate', 
   function (head,tail)
      return font_replace_42(head)
   end,'replace_42', 1)

\end{luacode}

%%%%%%%%%%%%%%%%%%%%%%%%%%%%%%%%
% ユーザ命令: \setkanafont <書体選択命令>, \unsetkanafont

\def\setkanafont#1{%
  \bgroup#1\directlua{luatexja.mfont_temp = \the\ltj@curjfnt}\egroup
  \ltj@curkanafnt=\directlua{tex.sprint(luatexja.mfont_temp)}\relax
}
\def\unsetkanafont{\ltj@curkanafnt=-"7FFFFFFF\relax}%"

\makeatother

\usepackage{luatexja-fontspec}
\setmainjfont{ipam.ttf}
\setsansjfont{ipag.ttf}
\newjfontfamily\kanaX{mogamb.ttc}% かな用書体1: MogaMincho Bold
%%% ↑Y. Oz Vox (http://yozvox.web.fc2.com/) より入手可能
\newjfontfamily\kanaY{ipag.ttf}%   かな用書体2: IPA ゴシック

\begin{document}
% \setkanafont はTeX のグルーピングに従う
あいう漢字
{\setkanafont\kanaY あいう漢字}%
あいう漢字
{\setkanafont\kanaX あいう漢字
  {\setkanafont\kanaY あいう漢字}
あいう漢字}

% \unsetkanafont で解除可能
{\setkanafont\kanaY あいう漢字
  {\unsetkanafont あいう漢字}% ここだけ IPA 明朝
あいう漢字}


% ゴシック体に太明朝のカタカナを合わせる
{\gtfamily\setkanafont\kanaX あいう漢字}

% \setkanafont はメインの和文フォントに追従しない
% サイズ変更にも追従しないので注意
{\setkanafont\kanaY あいう漢字\gtfamily あいう漢字\Large あいうえお漢字}

% サイズ変更命令には手動で \setkanafont を実行しないといけない
{\setkanafont\kanaY あいう漢字\Large\setkanafont\kanaY あいう漢字}


\gtfamily\setkanafont\kanaX
これは,意味のないサンプルテキストです.「スペーシングはうまく行ってるかな?」

\unsetkanafont
これは,意味のないサンプルテキストです.「スペーシングはうまく行ってるかな?」

\end{document}


--- src/ltj-charrange.lua.orig	2013-12-29 16:58:52.482148630 +0900
+++ src/ltj-charrange.lua	2013-12-25 20:26:17.879807073 +0900
@@ -28,8 +28,7 @@
 --         external    1  2       216, (out of range): 'other'
 
 -- initialize
-jcr_table_main = {}
-local jcr_table_main = jcr_table_main
+local jcr_table_main = {}
 local jcr_cjk = 0; local jcr_noncjk = 1; local ucs_out = 0x110000
 
 for i=0x80 ,0xFF      do jcr_table_main[i]=1 end

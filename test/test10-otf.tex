%#! lualatex
\documentclass{bxjsarticle}
\usepackage{luatexja}
\usepackage{luatexja-otf}
\usepackage{luatexja-fontspec}
\begin{document}

※このファイルはフォントを埋め込んでいないため,代替されるフォントによっては正しく
表示されないでしょう.

森\UTF{9DD7}外と内田百\UTF{9592}とが\UTF{9AD9}島屋に行くところを想像した。

\CID{7652}飾区の\CID{13706}野屋

\section*{Adobe-Japan1-5で追加された文字を使った例}
\begin{itemize}
  %% \item 「\゜か」,「\゜き」,「\゜く」,「\゜け」,「\゜こ」,
  %%       「\゜カ」,「\゜キ」,「\゜ク」,「\゜ケ」,「\゜コ」は鼻濁音を表す。
  \item Macintosh用キーボードの\UTF{2318}(Command key)を押す。
  \item \UTF{2672}を心がけよう。
\end{itemize}

\section*{Adobe-Japan1-6で追加された文字を使った例}
\begin{itemize}
  \item ほげほげ番組\CID{20556}
  \item ほげほげフェスティバル\CID{20656}
  \item \CID{20939}(Bq: becquerel)は放射能の強さを表す単位である。
  \item フラーレン(fullerene) C$_{60}$は
        サッカーボール状(\CID{20957})の構造をしている。
  \item \UTF{9B87}とは岩魚(イワナ)のことであり,嘉魚とも書く。
\end{itemize}

\section*{要検討}
現在の実装だと,対応する Unicode の符号位置を取得して単純に \verb|\char| に
渡しているだけなので,以下の問題がある.
\begin{itemize}
\item 欧文文字範囲に設定されている文字は CID で出力しても欧文フォントになる.
  例: あ\CID{18}1あ
\item OpenType の feature が指定されているとそちらが優先される.
  例: {\jfontspec[NoEmbed,CJKShape=JIS1978]{Ryumin-Light}\CID{3056}\CID{8267}}
  (左は一点,右は二点になってほしい.)
\end{itemize}

\end{document}

%#! lualatex
\documentclass{bxjsarticle}
\usepackage{luacode}
\usepackage{luatexja}
\usepackage{luatexja-otf}
\begin{document}

※このファイルはフォントを埋め込んでいないため,代替されるフォントによっては正しく
表示されないでしょう.

森\UTF{9DD7}外と内田百\UTF{9592}とが\UTF{9AD9}島屋に行くところを想像した。

\CID{7652}飾区の\CID{13706}野屋

\section*{Adobe-Japan1-5で追加された文字を使った例}
\begin{itemize}
  \item 「\゜か」,「\゜き」,「\゜く」,「\゜け」,「\゜こ」,
        「\゜カ」,「\゜キ」,「\゜ク」,「\゜ケ」,「\゜コ」は鼻濁音を表す。
  \item Macintosh用キーボードの\UTF{2318}(Command key)を押す。
  \item \UTF{2672}を心がけよう。
\end{itemize}

\section*{Adobe-Japan1-6で追加された文字を使った例}
\begin{itemize}
  \item ほげほげ番組\CID{20556}
  \item ほげほげフェスティバル\CID{20656}
  \item \CID{20939}(Bq: becquerel)は放射能の強さを表す単位である。
  \item フラーレン(fullerene) C$_{60}$は
        サッカーボール状(\CID{20957})の構造をしている。
  \item \UTF{9B87}とは岩魚(イワナ)のことであり,嘉魚とも書く。
\end{itemize}

\section*{ajmacros}

\begin{enumerate}\renewcommand{\labelenumi}{\ajLabel\ajKuroKaku{enumi}}
\item その1
\item その2
\item その3
\end{enumerate}

{\bf(速報)}世界陸上\CID{20660},\ajLig{ボルト}がフライングで失格.

\ajHankaku{半角カタカナひらがな} ←JFM の問題か,半角ひらがなが全角幅で出てしまう.

\newpage

\section*{Adobe-Japan1-6 全グリフ}

{\footnotesize
% 表の作成は Lua でサボってしまう.
% \CID{0} (.notdef) は luaotfload がマッピングしてくれないみたい.
\begin{luacode*}
for i = 0, 23057 do
   if i % 50 == 0 then
      tex.print(string.format("\\noindent%05d", i))
   end
   if i % 10 == 0 then
      tex.print(" ")
   end
   tex.print("\\hbox to 1\\zw{\\CID{"..tostring(i).."}}%")
   if i % 50 == 49 then
      tex.print("\\\\")
   end
end
\end{luacode*}
}

\end{document}

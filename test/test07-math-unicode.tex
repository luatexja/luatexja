%#!lualatex
\documentclass{ltjsarticle}
\usepackage[a4paper]{geometry}
\makeatletter\ltj@alljachar %←全部の(>=U+0080な)文字を和文文字扱いに!

\usepackage{luatexja-fontspec}
\usepackage{unicode-math}


\setmathfont{XITSMath}
\setmainjfont{IPAGothic}

\makeatletter
\reDeclareMathAlphabet{\mathtestA}{\mathrm}{\mathmc}
\reDeclareMathAlphabet{\mathtestB}{\mathrm}{\mathtestb}
\reDeclareMathAlphabet{\mathtestC}{\mathtesta}{\mathmc}
\reDeclareMathAlphabet{\mathtestD}{\mathtesta}{\mathtestb}
\reDeclareMathAlphabet{\mathtestE}{\mathtestE}{\mathmc}
\begin{document}\makeatletter

\[
 \frac1{1^2}+\frac1{2^2}+\cdots=
\sum_{n=1}^\infty \frac1{n^s}=\zeta(2)=\frac{\pi^2}{6}
=\frac16\cdot \pi\times\pi,\qquad a_1,\dots,a_n.
\]

$\hslash$

○×○漢×漢\times え\ltjjachar`\“え\ltjalchar`\“え\textquotedblleft え“え←数式外では和文文字扱いのままになっている.

\kanjifamily{mc}\selectfont
$あいうえおabcde a^\mathrm{b}$

mathrm: $\mathrm{\alpha あいうえおabcde}^\mathrm{b}$

mathbf: $\mathbf{\alpha あいうえおabcde}$

mathmc: $\mathmc{\alpha あいうえおabcde}$

mathgt: $\mathgt{\alpha あいうえおabcde}$

\tt\meaning\mathtestA

\tt\meaning\mathtestB

\tt\meaning\mathtestC

\tt\meaning\mathtestD

\tt\meaning\mathtestE
\end{document}

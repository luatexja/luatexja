%#!lualatex
\documentclass{ltjarticle}
\begin{document}
\noindent 漢字漢字ちょっと\textgt{チェック}

「漢字←全角二分下がり?\\
「あいうえお」\textgt{「かきくけこ」}{\Large 「}

次の例は「\texttt{min10}フォントについて」(乙部厳己)中のp.~8から拝借.

\def\g#1#2#3{\leavevmode\vbox{\hsize=100pt%
\hrule height 1pt depth 0pt
\vskip#3pt\hbox{\jfont\e=file:#1:jfm=#2 at 20pt\e あいうえお}\vskip-#3pt
\hrule height 0pt depth 1pt}: #1, jfm=#2, yjabaselineshift=#3pt\par\bigskip}

\g{KozGoPr6N-Medium.otf}{ujis}{0}
\g{KozGoPr6N-Medium.otf}{jis}{0.7563636}
\g{KozGoPr6N-Medium.otf}{jis}{0}
\g{ipaexg.ttf}{ujis}{0}
\g{ipaexg.ttf}{jis}{0.7563636}
\g{ipaexg.ttf}{jis}{0}
\g{hgrgm.ttc}{ujis}{0}
\g{hgrgm.ttc}{jis}{0}
\g{hgrgm.ttc}{jis}{0.7563636}

\end{document}
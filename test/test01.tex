%#! time luatex test01
\input luatexja-core.sty

\jfont\jisninety={file:ipaexm.ttf:script=latn;+jp90;jfm=ujis}
\jfont\jisfour={file:ipaexm.ttf:script=latn;+jp04;jfm=ujis}
\jfont\jisexpt={file:ipaexm.ttf:script=latn;+expt;jfm=ujis}
\jfont\jishwid={file:ipaexm.ttf:script=latn;+hwid;jfm=ujis}
%\font\tmihwid={file:ipaexm.ttf:script=latn;+hwid}
\jfont\jisnalt={file:ipaexm.ttf:script=latn;+nalt;jfm=ujis}
\jfont\jistrad={file:ipaexm.ttf:script=latn;+trad;jfm=ujis}
\jfont\jissups={file:ipaexm.ttf:script=latn;+sups;jfm=ujis}
\jfont\jisliga={file:ipaexm.ttf:script=latn;+liga;jfm=ujis}
%\font\tmiliga={file:ipaexm.ttf:script=latn;+liga}
\jfont\jisvert={file:ipaexm.ttf:script=latn;+vert;jfm=ujis}
\parskip=\smallskipamount\parindent=1\zw

{\noindent\bf\tengt jfm-ujis.lua を使用}

\bigskip

{\noindent\bf\tengt ■luaotf\/load による feature との共存状況}

{\tentt expt} feature: 剥→{\jisexpt 剥}

{\tentt jp90} feature: 辻→{\jisninety 辻}

{\tentt jp04} feature: 辻→{\jisfour 辻}

{\tentt hwid} feature: アイウエ→{\jishwid アイウエ}\hfil\break

{\tentt nalt} feature: 男→{\jisnalt 男}

{\tentt trad} feature: 医学→{\jistrad 医学}

{\tentt sups} feature: 注1注1→{\jissups 注1注1}\hfil\break
↑まだ上付き文字用の文字クラスを作ってない.

{\tentt liga} feature: か゚き゚く゚け゚こ゚→{\jisliga か゚き゚く゚け゚こ゚}\hfil\break
↑合字用の半濁点({\tentt U+309A})を用いれば成功する.単体用({\tentt U+309C})では失敗する.%比較:{\tmiliga か゜き゜く゜け゜こ゜}

{\tentt vert} feature: あ(㌢㍍),い→{\jisvert あ(㌢㍍),い}\hfil\break
↑縦組み時に気にすればいいか.

\bigskip

\noindent あいうえお

「あいうえお←全角下がりが正しい({\tt'boxbdd'}のテスト1)

{\tt'boxbdd'}のテスト2: \vrule\hbox{「」}\vrule ←正しい実装ならば2本の罫線の間は全角幅


\bigskip
{\noindent\bf\tengt ■{\tt differentjfm}による挙動の違い}

\jfont\tenMa={file:ipaexm.ttf:jfm=ujis}
\jfont\tenMb={file:ipaexm.ttf:jfm=ujis;jfmvar=a}
\jfont\tenGa={file:ipaexg.ttf:jfm=ujis} at 15pt
\jfont\tenGb={file:ipaexg.ttf:jfm=ujis;jfmvar=a} at 15pt\relax

\def\djtest{%
{\tt Ma-Ma}: \setbox0=\hbox{{\tenMa )}{\tenMa  ・}}\hbox to 90pt{\copy0\hss(\the\wd0)},
{\tt Ma-Mb}: \setbox0=\hbox{{\tenMa )}{\tenMb ・}}\hbox to 90pt{\copy0\hss(\the\wd0)},
{\tt Mb-Mb}: \setbox0=\hbox{{\tenMb )}{\tenMb ・}}\hbox to 90pt{\copy0\hss(\the\wd0)},

{\tt Ma-Ga}: \setbox0=\hbox{{\tenMa )}{\tenGa ・}}\hbox to 90pt{\copy0\hss(\the\wd0)},
{\tt Ma-Gb}: \setbox0=\hbox{{\tenMa )}{\tenGb ・}}\hbox to 90pt{\copy0\hss(\the\wd0)},
{\tt Mb-Gb}: \setbox0=\hbox{{\tenMb )}{\tenGb ・}}\hbox to 90pt{\copy0\hss(\the\wd0)}%
}

{\noindent\bf\tengt {\tt differentjfm=small}(小さい方)}

\ltjsetparameter{differentjfm=small}\djtest

{\noindent\bf\tengt {\tt differentjfm=large}(大きい方)}

\ltjsetparameter{differentjfm=LARGE}\djtest

{\noindent\bf\tengt {\tt differentjfm=average}(default,平均)}

\ltjsetparameter{differentjfm=AVERAGE}\djtest

{\noindent\bf\tengt {\tt differentjfm=both}(両方挿入)}

\ltjsetparameter{differentjfm=Both}\djtest

\bigskip
{\noindent\bf\tengt ■ujis, jis, min}

\def\test#1{\jfont\mx={file:ipaexm.ttf:jfm=#1}
\par\noindent\hbox to 30pt{#1}%
\hbox to 250pt{\mx ちょっとチェックしちゃった!が,まあいいです。\hss}%
\hbox{\mx「大」、「小」,‘国’“国”――……}}

\test{ujis}
\test{jis}
\test{min}
\end


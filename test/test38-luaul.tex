\documentclass{article}
\makeatletter
\directlua{jlreq = { burasage = true } }
\def\ltj@stdyokojfm{jlreq}
\usepackage{luatexja-adjust}
\usepackage{lua-ul}
\ltjenableadjust[priority=true,lineend=extended]
\newunderlinetype\beginUnderBLine{\cleaders\hbox{%
 \vrule height -0.1em depth 0.25em width 0.3\zw
}}

\NewDocumentCommand{\underBLine}{+m}{{\beginUnderBLine#1}}
\textwidth=20\zw

\begin{document}
\parindent0pt
\underBLine{あああ【いいい】ううう}\par\underBLine{「かかか」きく・けけ}\par
\underBLine{ああ(ほげ)いいい},\underBLine{ほげ,ふが.ぴよ」}

\underLine{ほげほげ}.ふが.\par % これはうまく機能する
あいうえ\underLine{お123いああああああああああああああああああああああ%
ああああああああああああああああああああ}かきくけこ\par % 波線がページ端まで伸びてしまう
ほげ,\underLine{あいうお}あああええええええええ1238ええええ$y=x$ええええええええええ
ええええええええええx

\def\R#1{\par \underBLine{あああああ\vrule width #1\zw ああああああああああああああ.あああ}}

\R{0}
\R{0.25}
\R{0.5}
\R{0.75}
\R{1}
\R{1.25}
\R{1.5}

\par\bigskip

\ltjenableadjust[priority=false,lineend=true]
\underBLine{あああ【いいい】ううう}\par\underBLine{「かかか」きく・けけ}\par
\underBLine{ああ(ほげ)いいい},\underBLine{ほげ,ふが.ぴよ」}

\R{0}
\R{0.25}
\R{0.5}
\R{0.75}
\R{1}
\R{1.25}
\R{1.5}


\end{document}
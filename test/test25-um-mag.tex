\documentclass[14pt,ja=standard,nomag*,lualatex]{bxjsarticle}
\usepackage{ifptex,iftex}
\usepackage{amsmath}
\ifptex
\usepackage{lmodern}
\else
\usepackage{unicode-math}
\fi
\title{テスト}
\author{テスト 太郎}
\begin{document}
\maketitle
\section{オイラーの公式}
オイラーの公式を式\ref{eq:eular}に示す。
\begin{equation}\label{eq:eular}
  e^{i\theta} = \cos\theta + i\sin\theta
\end{equation}
これに$\theta = \pi$を代入すると式\ref{eq:eular2}を得る。
\begin{equation}\label{eq:eular2}
  e^{-i\pi} = -1
\end{equation}
\end{document}

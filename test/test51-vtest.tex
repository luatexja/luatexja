%#!luatex

\input luatexja.sty
%\input lua-visual-debug.sty

\catcode`\@=11
\newdimen\@tempdima
\newbox\@tempboxa
\newdimen\fboxrule
\newdimen\fboxsep
\fboxrule=0.4pt\fboxsep=0pt
\long\def\fbox#1{%
  \leavevmode
  \setbox\@tempboxa\hbox{\kern\fboxsep{#1}\kern\fboxsep}%
  \@frameb@x\relax}
\def\@frameb@x#1{%
  \@tempdima\fboxrule
  \advance\@tempdima\fboxsep
  \advance\@tempdima\dp\@tempboxa
  \hbox{%
    \lower\@tempdima\hbox{%
      \vbox{%
        \hrule height\fboxrule
        \hbox{%
          \vrule width\fboxrule
          #1%
          \vbox{%
            \vskip\fboxsep
            \box\@tempboxa
            \vskip\fboxsep}%
          #1%
          \vrule width\fboxrule}%
        \hrule height\fboxrule}%
                          }%
        }%
}
\protected\def\LaTeX{L\kern-.36em%
        {\setbox\z@\hbox{T}
         \vbox to\ht\z@{\hbox{\sevenrm A}%
                        \vss}%
        }%
        \kern-.15em%
        \TeX}

\tentgt

\hbox{\yoko
横水平Hxy\hbox{\yoko 横水平Hxy}かき◆
\hbox{\tate 縦垂平Hxy}◆おおおお
\vbox{\yoko\hsize=30mm 横垂直Hxyああああああああああ}かき◆
\vbox{\tate\hsize=30mm  縦垂直Hxy\hfill ああ\break ああああああああ}◆ああああ
}
\vfill\eject
\hbox{\tate
縦水平Hxy\hbox{\yoko 横水平Hxy}かき◆
\hbox{\tate 縦水平Hxy}◆おおおお
\vbox{\yoko\hsize=30mm 横垂直Hxyああああああああああ}かき◆
\vbox{\tate\hsize=30mm  縦垂直Hxy\hfill ああ\break ああああああああ}◆ああああ
}

\vfill\eject


\setbox0=\vbox{\yoko\hsize=100mm
横垂直Hxy\hbox{\yoko 横水平Hxy}かき◆
これは,意味のないサンプルテキストです.
\hbox{\tate 縦水平Hxy}◆おおおお
これは,意味のないサンプルテキストです.
\vbox{\yoko\hsize=50mm 横垂直Hxyあああああああああああああああああああ}かき◆
これは,意味のないサンプルテキストです.
\vbox{\tate\hsize=50mm 縦垂直Hxyあああああああああああああああああああ}◆ああああ
これは,意味のないサンプルテキスト\hbox{\tate 縦水平Hxy}◆おおおおです.
}
\copy0
%{\showboxbreadth10000\showboxdepth10000
%\showbox0}

\vfill\eject
\vbox{\tate\hsize=100mm
◆◆◆Hxy\hbox{\yoko 横水平Hxy}かき◆
これは,意味のないサンプルテキストです.
\hbox{\tate 縦水平Hxy}◆おおおお
これは,意味のないサンプルテキストです.
\vbox{\yoko\hsize=50mm 横垂直Hxyあああああああああああああああああああ}かき◆
これは,意味のないサンプルテキストです.
\vbox{\tate\hsize=50mm  縦垂直Hxyあああああああああああああああああああ}◆ああああ
これは,意味のないサンプルテキスト\hbox{\tate 縦水平Hxy}◆おおおお
です.
}


\vbox{\tate\hsize100mm
\tfont\f=KozMinPr6N-Regular.otf:+vert;jfm=ujisv \f

\ltjsetparameter{yalbaselineshift=.25em}\baselineskip15pt
\LaTeX の特徴として、テキストファイルであるソースコードを入力として処理することでDVIや
PDFなどの表示形式を出力として得ることが挙げられる。

\LaTeX の最大の長所は、\TeX に由来する高品質で自由度の高い組版処理能力である。組版処理能力
は一般向けの出版物の作成にも充分に耐えられるものであり、実際の出版例もある。
中でも数式組版の品質が高い。さらに、数式専用の命令文(コマンド)が用意されているので、
単純なソースコードで高品質な数式表示を得られる。そのため、数式を多く含む自然科学系や工学系
の出版物などでは、\LaTeX 形式での投稿が標準的なフォーマットとされていることも多い。

くわえて、ソースコードに詳細な設定を記述することで、文書のスタイル(表示形式、見栄え、たと
えばレイアウトやフォントなど)を自在に調節できる。また、ソースコード上で文書内容・文書構造
(章・節・段落や強調箇所など)と文書スタイルの設定との分離が可能である。そのため、同一文書
内で終始一貫したスタイをル保つことができる。また、同じ文書内容でレイアウトやフォントだけを
変えることができるなど、文書内容データの再利用性が高い。一度スタイル設定を決めてしまえば、
あとは文書内容の記述に専念することができるという利点もある。論文などの投稿では、学術雑
誌のスタイルを記述したファイル(パッケージファイル)を Web からダウンロードして利用するこ
とで、統一されたスタイルで論文xを投稿することができ、論文の投稿者と雑誌の編集者とのやり取り
を簡略化できる。Kile、TeXShop、TeXworks、EasyTeX、WinShell などの \TeX 用エ
ディタや、野鳥(やてふ、YaTeX)、TeXlipse、KaTeX(花鳥、かてふ)、祝鳥
(のりてふ)、M's TeX Helper 2など の\TeX 用テキストエディタマクロを兼用することに
よって、より効率的な文章作成が可能ともなる。またMapleやMathematicaなどでドキュメントを
作成し、\TeX 形式で出力することも可能である。

}
\end


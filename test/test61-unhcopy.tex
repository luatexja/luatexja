\documentclass{minimal}
\usepackage{luatexja}

\begin{document}
\output={\setbox60000=\box255\deadcycles=0\relax}
\newcount\mycnt
\typeout{00> \the\ht\strutbox}

\everypar{\strut}

%%% \expandafter...\relax

\directlua{
  socket=require'socket'
  my_a=socket.gettime()
}

\mycnt=0
\loop\ifnum\mycnt<100000\relax
  \leavevmode TEST\par
  \advance\mycnt by1\relax
\repeat

\directlua{
  my_b=socket.gettime()
}
\typeout{01> \the\ht\strutbox, \directlua{tex.sprint(my_b-my_a)}}

%%% \expandafter...(space)
\makeatletter
\protected\def\ltj@@unhbox{\ltj@reset@globaldefs\afterassignment\ltj@@unhbox@\ltj@tempcnta}
\protected\def\ltj@@unhcopy{\ltj@reset@globaldefs\afterassignment\ltj@@unhcopy@\ltj@tempcnta}
\def\ltj@@unhbox@#1{%
  \protected\def\ltj@@unhbox@{\ltj@@lua@unboxcheckdir\expandafter\ltj@@orig@unhbox\the\ltj@tempcnta#1}%
}
\def\ltj@@unhcopy@#1{%
  \protected\def\ltj@@unhcopy@{%
    \ltj@@lua@uncopycheckdir\expandafter\ltj@@orig@unhcopy\the\ltj@tempcnta#1
    \ltj@@lua@uncopy@restore@whatsit}%
}
\ltj@@unhbox@{ }\ltj@@unhcopy@{ }
\let\unhbox\ltj@@unhbox %% PRIMITIVE
\let\unhcopy\ltj@@unhcopy %% PRIMITIVE

\directlua{
  my_a=socket.gettime()
}

\mycnt=0
\loop\ifnum\mycnt<100000\relax
  \leavevmode TEST\par
  \advance\mycnt by1\relax
\repeat

\directlua{
  my_b=socket.gettime()
}
\typeout{02> \the\ht\strutbox, \directlua{tex.sprint(my_b-my_a)}}

%%% \begingroup...\endgroup
\makeatletter
\protected\def\ltj@@unhbox{\begingroup\ltj@reset@globaldefs\afterassignment\ltj@@unhbox@\ltj@tempcnta}
\protected\def\ltj@@unhbox@{\ltj@@lua@unboxcheckdir\ltj@@orig@unhbox\ltj@tempcnta\endgroup}
\protected\def\ltj@@unhcopy{\begingroup\ltj@reset@globaldefs\afterassignment\ltj@@unhcopy@\ltj@tempcnta}
\protected\def\ltj@@unhcopy@{%
  \ltj@@lua@uncopycheckdir\ltj@@orig@unhcopy\ltj@tempcnta
  \ltj@@lua@uncopy@restore@whatsit\endgroup}%
\let\unhbox\ltj@@unhbox %% PRIMITIVE
\let\unhcopy\ltj@@unhcopy %% PRIMITIVE

\directlua{
  my_a=socket.gettime()
}

\mycnt=0
\loop\ifnum\mycnt<100000\relax
  \leavevmode TEST\par
  \advance\mycnt by1\relax
\repeat

\directlua{
  my_b=socket.gettime()
}
\typeout{03> \the\ht\strutbox, \directlua{tex.sprint(my_b-my_a)}}


\end{document}


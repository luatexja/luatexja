%#!lualatex
\documentclass{ltjsarticle}
\usepackage{luatexja-fontspec,booktabs,array}

\def\MJI[#1]#2{#2\char\numexpr "E0100+#1\relax}%"
\def\IVSL#1{\directlua{ivs.list_ivs('#1')}}
\begin{document}
\jfontspec{ipamjm.ttf} % IPA MJ 明朝

例文はZRさんのブログ記事「ipamjmパッケージでアレしてみた」\footnote{%
\verb+http://d.hatena.ne.jp/zrbabbler/20131214/1387029624+}より引用.

\paragraph{標準状態では……}
\begin{quote}
\LARGE
渡邉󠄏さんとか   % { } の中は U+9089 U+E010F
渡𫟪󠄂さんとか。% { } の中は U+2B7EA U+E0102
\end{quote}

\paragraph{IVS処理コードをここで読み込んだ.}\ 
\directlua{dofile('ivs.lua')}

\begin{quote}
\LARGE
\MJI[15]{邉}\MJI[25]{邉}\MJI[27]{邉}\MJI[26]{邉}\MJI[26]{邉}\MJI[16]{邊}
\MJI[18]{邊}\MJI[2]{𫟪}\MJI[17]{邊}〓\\
\MJI[28]{邉}\MJI[29]{邉}\MJI[23]{邉}\MJI[15]{邊}\MJI[8]{邊}\MJI[20]{邉}
\MJI[24]{邉}\MJI[19]{邉}\MJI[18]{邉}\MJI[16]{邉}\\
\MJI[14]{邊}\MJI[10]{邊}\MJI[12]{邊}\MJI[11]{邊}\MJI[13]{邊}\MJI[9]{邊}
\MJI[0]{𫟪}\MJI[1]{𫟪}〓\MJI[21]{邉}
\end{quote}
「〓」はMJ番号を直接指定していたところなので,とりあえず無視している.
なぜか𫟪(U+2B7EA)のIVSが機能していないようだが,フォント側にその記述がない,ということ?

\begin{quote}
\LARGE
渡邉󠄏さんとか   % { } の中は U+9089 U+E010F
渡𫟪󠄂さんとか。% { } の中は U+2B7EA U+E0102
\end{quote}

\begin{center}
\Large
\begin{tabular}{c>{\tt}ll}
\toprule
文字&Unicode&IVS\\
\midrule
邉&U+9089&\IVSL{邉}\\
邊&U+908A&\IVSL{邊}\\
𫟪&U+2B7EA&\IVSL{𫟪}\\
\bottomrule
\end{tabular}
\end{center}
\end{document}
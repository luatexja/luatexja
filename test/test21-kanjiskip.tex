%#!luajittex
\input luatexja.sty
\input luatexja-adjust.sty
%\input lua-visual-debug.sty
\def\test#1{\vrule\hbox spread\zw{#1}\vrule}

\baselineskip=1.75\zw

\def\testH#1{\vtop{\hsize=21\zw #1%
\par\vrule{%
  \ltjsetparameter{kanjiskip=0pt plus 3\zw}%
  \hbox to 20\zw{あ「い」う,えお}%
}\vrule\par
\test{あ(…)ああ}\qquad
\test{あ(¥)ああ}\par
\test{あ(%)ああ}\qquad
\test{あ(ー)ああ}\par
\test{あ(あ)ああ}\qquad
\test{あ(ア)ああ}\par
\test{あ(漢)ああ}\qquad
\test{あ(0)ああ}\par
\test{… ………}\qquad
\test{¥ ¥ %}\par
\test{% % あ}\qquad
\test{ー ー あ}\par
\test{あ あ ア}\qquad
\test{0 0 ー}\par
\test{佐々十郎}\qquad
\test{大村 崑}\par
\test{岡 八郎}\qquad
\test{花紀 京}}\par\bigskip}

1行目の例は「TeXでDTP―min10.tfmやjis.tfmの問題点」\hfil\break
(渡邉たけしさん,{\tt http://www.dab.hi-ho.ne.jp/t-wara/tex/min10.html})から引用.

1行目以外の例は「プリセットの「文字組アキ量設定」における設定値の齟齬と回避策」\hfil\break
(なんでやねんDTPさん,{\tt http://d.hatena.ne.jp/works014/20150926})から引用.

\parindent0pt
{\gt\bf 標準メトリック}

\testH\yoko
\testH\tate

\vfill\eject

{\gt\bf 標準メトリックに加えて {\tt -ltjksp} 指定(旧バージョンの組み方)}

\jfont\G=file:ipaexm.ttf:jfm=ujis;-ltjksp at 9.62216pt
\tfont\H=file:ipaexm.ttf:jfm=ujisv;-ltjksp at 9.62216pt
\G\H
\testH\yoko
\testH\tate


\bye

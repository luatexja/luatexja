%#! luatex
\input luatexja-core.sty

\loadjfontmetric{mt}{ujis}
\jfont\tenmin={psft:Ryumin-Light:jfm=mt}\tenmin
\jfont\tengoth={psft:GothicBBB-Medium:jfm=mt}
\jfont\jisse={psft:Ryumin-Light:script=latn;+jp78;jfm=mt}
\jfont\jisexpt={psft:Ryumin-Light:script=latn;+expt;jfm=mt}
\jfont\jishwid={psft:Ryumin-Light:script=latn;+hwid;jfm=mt}
%\font\tmihwid={psft:Ryumin-Light:script=latn;+hwid}
\jfont\jisnalt={psft:Ryumin-Light:script=latn;+nalt;jfm=mt}
\jfont\jistrad={psft:Ryumin-Light:script=latn;+trad;jfm=mt}
\jfont\jissups={psft:Ryumin-Light:script=latn;+sups;jfm=mt}
\jfont\jisliga={psft:Ryumin-Light:script=latn;+liga;jfm=mt}
%\font\tmiliga={psft:Ryumin-Light:script=latn;+liga}
\jfont\jisvert={psft:Ryumin-Light:script=latn;+vert;jfm=mt}
\parskip=\smallskipamount\parindent=1\zw

{\noindent\bf\tengoth luatj-ujis.lua を使用}

\bigskip

{\noindent\bf\tengoth luaotf\/load による feature との共存状況(非埋め込み版)}

{\noindent\bf\tengoth 注: 表示はビューワによって実際に用いられるフォントに依存します.}

{\tentt expt} feature: 剥→{\jisexpt 剥}

{\tentt jp78} feature: 辻→{\jisse 辻}

{\tentt hwid} feature: アイウエ→{\jishwid アイウエ}\hfil\break
↑文字クラスが変わらないので幅も変わらない.

{\tentt nalt} feature: 男→{\jisnalt 男}

{\tentt trad} feature: 医学→{\jistrad 医学}

{\tentt sups} feature: 注1注1→{\jissups 注1注1}\hfil\break
↑文字クラスが変わらないので幅も変わらない.

{\tentt liga} feature: か゜き゜く゜け゜こ゜→{\jisliga か゜き゜く゜け゜こ゜}\hfil\break
↑なぜかうまくいかない.%比較:{\tmiliga か゜き゜く゜け゜こ゜}

{\tentt vert} feature: あ(㌢㍍),い→{\jisvert あ(㌢㍍),い}\hfil\break
↑縦組み時に気にすればいいか.

\bigskip

\noindent あいうえお

「あいうえお←全角下がりが正しい({\tt'boxbdd'}のテスト1)

{\tt'boxbdd'}のテスト2: \vrule\hbox{「」}\vrule ←正しい実装ならば2本の罫線の間は全角幅

\end


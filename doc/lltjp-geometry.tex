%#! luajitlatex
\documentclass[a4paper,10pt]{ltjsarticle}
\usepackage[textwidth=45\zw, lines=45, footskip=2\zh, verbose]{geometry}
\usepackage{listings,amsmath,booktabs,lltjext}
\usepackage[match]{luatexja-fontspec}
\usepackage{unicode-math}
\setmathfont{XITS Math}
\setmainfont{TeX Gyre Termes}
\setsansfont[Scale=0.95]{TeX Gyre Heros}
\usepackage[kozuka-pr6n]{luatexja-preset}\normalsize
\def\emph#1{\textbf{\textgt{\mathversion{bold}#1}}}
\def\headfont{\normalfont\bfseries\gtfamily}
\def\pTeX{p\kern-.05em\TeX}
\def\cs#1{\text{\texttt{\char`\\#1}}}

\lstset{
  basicstyle=\ttfamily, basewidth=0.5em,
}
\makeatletter\let\SX@Info=\relax\makeatother
\fboxsep=0mm

\usepackage{hyperref,bookmark,xcolor}
\hypersetup{%
	unicode,
	colorlinks,
	allbordercolors=1 1 1,
	allcolors=blue,
	pdftitle={lltjp-geometryパッケージ}
}

\def\LuaTeX{Lua\TeX}
\definecolor{blue}{rgb}{0, 0.25, 1}

\title{\textsf{lltjp-geometry}パッケージ}
\author{\LuaTeX-jaプロジェクト%
  \thanks{\url{http://osdn.jp/projects/luatex-ja/wiki/FrontPage}}}
\begin{document}
\maketitle

ページレイアウトの設定として,\href{http://www.ctan.org/pkg/geometry}%
{\textsf{geometry}パッケージ}が有名であるが,
これはp\LaTeX・\LuaTeX-jaの縦組クラスでは利用が不可能という問題があった.
本文書で解説する\textsf{lltjp-geometry}パッケージは,\textsf{geometry}パッケージを
縦組クラスに対応させるパッチである.


\section{利用方法}
\textsf{lltjp-geometry}パッケージは,\LuaTeX-jaに標準で含まれている.
本パッケージの動作には\href{http://www.ctan.org/pkg/ifluatex}{\textsf{ifluatex}},
\href{http://www.ctan.org/pkg/filehook}{\textsf{filehook}}パッケージが必要である.

\subsection{\LuaTeX-ja}
\LuaTeX-jaでは,\textsf{geometry}パッケージ読み込み時に
自動的に\textsf{lltjp-geometry}パッケージが読み込まれ,ユーザは何もしなくても良い.
\LuaTeX-jaで横組クラスを利用する時でも,
\textsf{lltjp-geometry}パッケージは自動的に動作を停止するので,横組時の挙動が
変わってしまうことはない.

\subsection{\pTeX 系列}
\pTeX 系列では,\textsf{tarticle}, \textsf{tbook}, \textsf{treport}といった
\emph{縦組クラスを使う場合}にのみ,
\begin{lstlisting}
\usepackage{lltjp-geometry}
\usepackage[...]{geometry}
\end{lstlisting}
のように\emph{\textsf{geometry}パッケージの前}に読み込む.
\pTeX 系列では使用クラスが横組か縦組かの自動判定を行えない\footnote{%
  標準縦組クラスでは,\cs{begin\{document\}} の内部で組方向を縦組に変更するので,
  プリアンブル中で判定できない.
}ので,\emph{横組クラスで読み込んではならない}.

パッケージオプションは存在しない.

\section{\textsf{lltjp-geometry}使用時の注意事項}

\subsection{\texttt{twoside}指定時}
縦組の本は通常右綴じである.これを反映し,
\texttt{twoside} オプション指定時には
\begin{itemize}
\item \texttt{left}, \texttt{lmargin} は小口側の余白,
\texttt{right}, \texttt{rmargin} はノド側の余白を指す.
\item 左右余白比 \texttt{hmarginratio} の標準値は$3:2$に変更.
\item \texttt{bindingoffset} は\emph{右側}に余白を確保する.
\end{itemize}
と変更している.

\subsection{傍注}
縦組の場合,傍注は本文の上下に配置される\footnote{%
  二段組の場合は上下共に,一段組の場合は標準では下側だが,
  \texttt{reversemp} が指定されたときには上側に配置される.
}.これにより,\emph{\texttt{includemp}(や \texttt{includeall})が
未指定の場合,傍注はヘッダやフッタに重なる}.
\texttt{includemp} 指定時は,\cs{footskip}, \cs{headsep} のいずれか
(二段組の場合は両方)を$\cs{marginparwidth} + \cs{marginparsep}$だけ
増加させる.

\section{\texttt{lines}オプションに関する注意事項}
本節の内容は,\textsf{lltjp-geometry}パッケージを読み込まない場合,
つまり,横組クラスで\textsf{geometry}パッケージを普通に使用した場合にも
当てはまる注意事項である.

\subsection{\textsf{fontspec}パッケージとの干渉}
\pTeX 系列では,次のように\textsf{fontenc}パッケージ読み込み直後に
\textsf{geometry}パッケージを用いてレイアウトを設定すると,
\texttt{lines} による指定が正しく働かないという症状が生じる:
\begin{lstlisting}
\documentclass{article}
\usepackage{geometry}
\usepackage{fontspec}
\geometry{lines=20}
\begin{document}
hoge\typeout{\the\topskip, \the\baselineskip, \the\textheight}
\end{document}
\end{lstlisting}
\cs{typeout} で \cs{topskip}, \cs{baselineskip}, \cs{textheight} の値を調べると
\[
 \frac{\cs{textheight} - \cs{topskip}}{\cs{baselineskip}} = 15.8\dot 3
\]
となることがわかるから,1ページには16行分入らないことがわかる.

これは,\textsf{fontspec} の読み込みによって \cs{baselineskip} がなぜか
10\,ptに変えられてしまい,\cs{geometry} 命令はその値に従って本文領域の高さを計算するためで
ある.とりあえずの対策は,
\cs{normalsize} によって \cs{baselineskip} を正しい値に再設定し,その後
レイアウトを設定すれば良い:
\begin{lstlisting}
\usepackage{geometry}
\usepackage{fontspec}
\normalsize\geometry{lines=20}
\end{lstlisting}

なお,同様の症状は\textsf{newtxtext}パッケージなどでも発生するようなので,
\pTeX 系列といえども無縁ではない.


\subsection{\cs{maxdepth} の調整}
\LaTeX では,
最後の行の深さ\rensuji{$d$}と
本文領域の上端から最後の行のベースラインまでの距離\rensuji{$f$}に対し,
\[
 \cs{textheight} = f+\max(0, d-\cs{maxdepth})
\]
が成り立つ.

\pTeX 系列の標準縦組クラス\textsf{[u]tarticle}等,
及びそれを\LuaTeX-ja用に移植した\textsf{ltjtarticle}等では,
 \cs{topskip} は\emph{横組時における}全角空白の高さ7.77588\,pt\footnote{%
  標準の\texttt{10pt}オプション指定時.以下同じ.
  ところで,この量は公称フォントサイズの10\,ptか,もしくは
  全角空白の高さと深さを合わせた値の9.16446\,ptの間違いではないか,と筆者は考えている.
  なお,奥村晴彦氏の\href{https://oku.edu.mie-u.ac.jp/~okumura/jsclasses/}%
  {p\LaTeXe 新ドキュメントクラス}では公称ポイントサイズ10\,ptに設定されている.
}であり,\cs{maxdepth} はその半分の値(従って3.88794\,pt)である.

いくつかのフォントについて,その中の文字の深さの最大値を見てみると
表\ref{tab:baseline}のようになっている.
\begin{table}[tb]
 \layoutfloat[c]{%
 \begin{tabular}<y>{ll}
  \toprule
  \bfseries フォント(10\,pt)&\bfseries 深さ(pt単位)\\
  \midrule
  横組用の標準和文フォント(\pTeX)&1.38855\\
  縦組用の標準和文フォント(\pTeX)&4.58221\\
  \midrule
  Computer Modern Roman 10\,pt&2.5\\
  Computer Modern Sans Serif 10\,pt&2.5\\
  Times Roman (\texttt{ptmr8t})&2.16492\\
  Helvetica Bold Oblique (\texttt{phvbo8t})&2.22491\\
  Palatino (\texttt{pplr8t})&2.75989\\
  \bottomrule
 \end{tabular}}
 \pcaption{いくつかのフォント中の,文字の深さの最大値
    \label{tab:baseline}}
\end{table}
欧文フォントのベースラインは,そのままでは和文との組み合わせが悪いので,
さらに$\textsf{tbaselineshift}=3.41666\,\textrm{pt}$だけ下がることを考えると,
最後の行に和文文字が来た場合はほぼ確実に深さが \cs{maxdepth} を超えてしまうことになる.
従って,本文領域を「\rensuji{$n$}行分」として指定するときによく使われる
\begin{equation}
   \cs{textheight} = \cs{topskip} + (n-1)\cs{baselineskip}
 \label{eq:nline}
\end{equation}
は\textsf{tarticle}クラスのデフォルトでは通用しない.

通常の地の文のみの文章においてほぼ確実に\eqref{eq:nline}が成り立つようにするため,
\textsf{lltjp-geometry}では\emph{\texttt{lines}オプション指定時のみ} \cs{maxdepth} の値が
最低でも
\begin{quote}
公称ポイントサイズの半分に,欧文ベースラインのシフト量を加えた値%
\footnote{\textsf{tarticle}の場合だと,
$5\,\textrm{pt} + 3.41666\,\textrm{pt}=8.41666\,\textrm{pt}$である.}
\end{quote}
になるようにしている.\texttt{lines} オプション非指定時にはこのような調整は
行われない.


\subsection{見かけ上の基本版面の位置}
\LaTeX では,
本文の一行目のベースラインは,本文領域の「上端」から
 \ \cs{topskip}\ だけ「下がった」ところに来ることになっている.
あまり \cs{topskip} が小さいと,ユーザが大きい文字サイズを指定した時に1行目のベースライン
位置が狂う危険があるため,
\textsf{geometry}パッケージでは
\begin{quote}
 \texttt{lines} オプション指定時,\cs{topskip} の値を最低でも
 \cs{strutbox} の高さ($0.7\cs{baselineskip}$)まで引き上げる
\end{quote}
という仕様になっている.

縦組の場合は,\cs{strutbox} に対応するボックスは \cs{tstrutbox} であるため,
\textsf{lltjp-geometry}では
\begin{quote}
 \texttt{lines} オプション指定時,\cs{topskip} の値を最低でも
 \emph{\cs{tstrutbox} の高さ($\cs{baselineskip}/2$)}まで引き上げる
\end{quote}
という挙動にした.見かけ上は \cs{topskip} の値制限が緩くなったが,前節で述べたように
欧文フォントのベースラインは和文に合うように下にずらされるので,
実用上は問題は起きないだろう.

前節の \cs{maxdepth} の調整も考え合わせると,\emph{\LaTeX が認識する本文領域と,
実際の見た目の基本版面の位置とは異なる}ことに注意してほしい.

\medskip
例えばA4縦を縦組で,公称フォントサイズ10\,pt,行送り18\,pt,30行左右中央
というレイアウトにするため,
\begin{lstlisting}
\documentclass{tarticle}
\usepackage{lltjp-geometry}
\baselineskip=18pt
\usepackage[a4paper,hcentering,lines=30]{geometry}
\end{lstlisting}
と指定すると,実際には以下のように設定される.
\begin{itemize}
 \item \cs{topskip} は \cs{tstrutbox} の高さ8.5\,ptに設定される.
 \item 本文領域の「高さ」 \cs{textheight} は
\[
 \cs{topskip} + (30-1)\cs{baselineskip} = 530.5\,\textrm{pt}.
\]
 \item 従って,左余白と右余白は
\[
 \frac{210\,\textrm{mm}-\cs{textheight}}2 = 33.50394\,\textrm{pt}.
\]
\end{itemize}
しかし,実際にはページの最初の行のベースラインは,本文領域の右端から
\ \cs{topskip} だけ左にずれたところにあり,
一方ページの最終行のベースラインは本文領域の左端にある.
縦組和文フォントのベースラインは文字の左右中央を通ることから,
従って,\emph{見た目で言えば,右余白の方が$\cs{topskip} =8.5\,\textrm{pt}$だけ
大きい}ということになってしまう\footnote{%
  同様に,横組で \texttt{vcentering} を指定すると,見かけでは
  $\cs{topskip}-\cs{Cht}+\cs{Cdp}$だけ上余白が大きいように見える.
}.


\end{document}

%#! time luatex jfmglue
\input s1sty.tex % style file
\newcount\ncount
\def\node#1{
  \ifnum\ncount=0 \ncount=1\else\longrightarrow\fi
  \setbox0=\hbox{\kern.5em$\mathstrut#1$\kern.5em}\dp0=0pt
  \hbox{\vrule
    $\vcenter{\hsize=\wd0\hrule\kern.5ex\copy0\kern.5ex\hrule}$%
    \vrule\kern.1em}}
\def\nk{{\rm kern}\ }
\def\ng{{\rm glue}\ }
\def\np{{\rm penalty}\ }
\def\z{\,{\rm zw}}
\jfont\tenmini={file:ipam.ttf:slant=0.5;jfm=ujis} at 13\jQ

\centerline{\big Lua\TeX-ja 和文処理グルーについて}\bigskip
\centerline{\large\the\year/\the\month/\the\day}\medskip

本文書では,Lua\TeX-jaが(現時点において)和文処理に関わるglue/kernを
どのように挿入するかの内部処理について説明する.

\beginsection 予備知識

説明に入る前に,段落やhboxの中身は,\TeX の内部ではnode達による
リストとして表現されていることに注意する.nodeの種類については,
\hbox{\it The\ Lua\TeX\ Reference\/}の第8章を参照して欲しい.代表的なものを挙げると,

\item {\it glyph\_node}: 文字(合字も含む)を表現する.和文処理グルーを挿入する際には,
既に各{\it glyph\_node}が欧文文字のものか和文文字のものか区別がついている.
\item {\it glue\_node}: glueを表す.
\item {\it kern\_node}: kernを表す.各|kern_node|には|subtype|という値があり,
次の3種類を区別できるようになっている.
\itemitem 0: 欧文用TFM由来
\itemitem 1: 明示的な|\kern|か,イタリック補正 (|\/|) によるもの
\itemitem 2: 非数式アクセント用文字の左右位置調整のためのもの
\item {\it penalty\_node}: penaltyを表す.
\item {\it hlist\_node}: hbox(水平ボックス)を表す.
\enditem

以後,次のように,nodeがどのように連続しているかを表すことにする.
$$
\node{a}\node{b}_{\rm I}\node{c}
$$
右下についている添字は,Lua\TeX-ja においてそのnodeの役割を区別するためにつけられた
値であり,次のようになっている.
\begintt
I: イタリック補正由来の\kern
T: \[x]kanjiskipに置換されうる\kern
J: JFM由来のglue/kern
K: 禁則処理用penalty
E: 
KS: \kanjiskip
XS: \xkanjiskip
\endtt


\beginsection JFM由来グルーの挿入 ({\tt luatexja-jfmglue.lua})

JFM由来グルーの処理は,「連続する2つのnodeの間に何を入れるか」という単位で行われる.
そのため,
\item node生成を伴わないもの(グループ境界,|\relax|等)は全て無視される.
\item 一方,node生成を伴うものは全て「透過しない」.
例えば,次のソースにおいて,閉じ括弧と開き括弧の間に入る物は,
左と右とで異なる:
\begintt
)(  )\hbox{}(
\endtt
\item JFM由来グルーの挿入禁止を行う|\inhibitglue|は,内部では専用のnodeを作ること
によって実装している.この|\inhibitglue|用nodeは透過する.
\enditem
以下,$q$, $p$を連続するnodeとする.


\beginparagraph 2つの和文文字の間

この場合,グルー挿入に関係する量は次の通りである.これら3つの量の値によって,
$q$と$p$の間に何が挿入されるかが決定される.これらの記号は他の場合にも用いる.
\item $g$: JFMで指定された,$q$と$p$の間に入るglue/kern.
JFMで規定されていないときは$\emptyset$と書こう.
両ノードで使われているJFMが異なる時の$g$の決定方法は,後に記述する.
\item $w$: JFMで指定された,「$q$の直後で改行が行われた場合,
$q$と行末の間に入るカーン量」の値.

$g-w$で,$g$の自然長を$w$だけ減算したglue/kernを表すことにする.

\item $P$: $q$に対する行末禁則用ペナルティ (post-break penalty) と,
$p$に対する行頭禁則用ペナルティ (pre-break penalty) との和.どちらも
設定されていないときは0となる.
\enditem

設計方針としては,
\item JFM由来で入るものがkernの場合,この場所では行分割は許さない.
\item そうでない場合,(penaltyの値$P$があるが)この場所での行分割は可能である.
\enditem
である.さて,次が実際の場合わけである:

\enum $w\neq 0$, $g=\emptyset$のとき
$$\ncount=0
\node{q}
\node{\nk w}_{\rm E}
\node{\np P}_{\rm K}
\node{\nk{-w}}_{\rm T}
\node{p}
$$

\enum $w\neq 0$, $g\neq\emptyset$のとき:
$$\ncount=0
\node{q}
\node{\nk w}_{\rm E}
\node{\np P}_{\rm K}
\node{g-w}_{\rm J}
\node{p}
$$

\enum $w=0$, $g$: kernのとき
$$\ncount=0
\node{q}
\node{g}_{\rm J}
\node{p}
$$

\enum $w=0$, $g$: glueのとき
$$\ncount=0
\node{q}
\node{\np P}_{\rm K}
\node{g}_{\rm J}
\node{p}
$$

\enum $w=0$, $g=\emptyset$, $P\neq 0$のとき
$$\ncount=0
\node{q}
\node{\np P}_{\rm K}
\node{p}
$$

\enum $w=0$, $g=\emptyset$, $P=0$のとき
$$\ncount=0
\node{q}
\node{p}
$$
\enditem

なお,両ノードで使われているJFMが異なる時の$g$の決定方法であるが,
\enum $g_{\rm L}$を,$q$に使用されているJFMにおける,「$q$と文字|'diffmet'|」の間に
入るglue/kernの値とする.
\enum $g_{\rm R}$を,$p$に使用されているJFMにおける,「文字|'diffmet'|と$p$」の間に
入るglue/kernの値とする.
\enum 両方から,実際に入る$g$の値を計算する.
\itemitem $g_L$, $g_R$の少なくとも片方が$\emptyset$のときは,$\emptyset$でない方を
そのまま採用する.
\itemitem 両方とも$\emptyset$でない場合は,|differentjfm|の値にそって$g$の値を計算する.
\enditem

\beginparagraph 和文文字と(和文文字,kern以外のnode)の間

「和文文字の間」の場合に対して,以下が異なる:
\item $g$は,$q$に使用されているJFMにおける,「$q$と文字|'jcharbdd'|」の間に
入るglue/kernの値である.
\item $p$が{\bf penaltyでない場合}は,いつもこの位置で行分割できるようにするため,
case~6 ($w$, $P=0$, $g=\emptyset$) の場合にも,$q$と$p$の間には
0という値のpenaltyが入る.即ち,次のようになる.
$$\ncount=0
\node{q}
\node{\np 0}_{\rm K}
\node{p}
$$
\enditem

\beginparagraph (和文文字,kern以外のnode)と和文文字の間

この場合も,基本的には「和文文字の間」と似ているが,以下が異なる:
\item $g$は,$p$のJFMにおける,「文字|'jcharbdd'|と$p$」の間に
入るglue/kernの値である.
\item 常に$w=0$である.
\item いつもこの位置で行分割できるようにするため,
case~6 ($w$, $P=0$, $g=\emptyset$) の場合にも,$q$と$p$の間には
0という値のpenaltyが入る.
\enditem

即ち,次の3通りになる.
\enum $g$: kernのとき
$$\ncount=0
\node{q}
\node{g}_{\rm J}
\node{p}
$$

\enum $g$: glueのとき
$$\ncount=0
\node{q}
\node{\np P}_{\rm K}
\node{g}_{\rm J}
\node{p}
$$

\enum $g=\emptyset$のとき
$$\ncount=0
\node{q}
\node{\np P}_{\rm K}
\node{p}
$$
\enditem

\beginparagraph 和文文字とkernの間,kernと和文文字の間

和文文字の後にkernが続いた場合,あるいはkernの後に和文文字が続いた場合,
この間で行分割はできないものとしている.そのため,
以下の3ケースに限られる:
\enum $g$: kernのとき
$$\ncount=0
\node{q}
\node{g}_{\rm J}
\node{p}
$$

\enum $g$: glueのとき
$$\ncount=0
\node{q}
\node{\np 10000}_{\rm K}
\node{g}_{\rm J}
\node{p}
$$

\enum $g=\emptyset$のとき
$$\ncount=0
\node{q}
\node{p}
$$
\enditem

なお,ここでの$g$は,
\item kernが前だった場合は,
$q$のJFMにおける,「$q$と|'jcharbdd'|」の間に
入るglue/kernの値.
\item kernが後だった場合は,
$p$のJFMにおける,「|'jcharbdd'|と$p$」の間に
入るglue/kernの値.
\enditem

\beginparagraph 要検討の箇所

私が推測するに,欧文では,
\item 単語内ではフォントは変わらない.
\item 単語内では,明示的に/ハイフネーションにより挿入されたdiscretionary break以外では
行分割がおきない.
\enditem
という事情があるため,TFM由来のkernや合字処理は(nodeを生成しないもの以外は)
何も透過しないという状態になっているものと思われます.

そのため,JFMグルー等の仕様を考える場合,欧文でいう「単語」に対応するようなものは
何か,というのを考える必要があります.現実装では,素直に欧文の合字処理と同様のものであると考え,
透過するnodeはない,という仕様にしています.
しかし,|\[x]kanjiskip|の処理と共通にしてしまうというのも
考え方によってはありかもしれません.

\bigskip
\item {\bf イタリック補正のkernの周囲}

例えば,|jfm-ujis.lua|では,|'jcharbdd'|は文字クラス
0であるため,今の実装では,\hfil\break
「|)\/(|」という入力からは,次のnodeの並びを得る:
\setbox1=\hbox{|\/|}%
$$
\node{\hbox{)}}
\node{\np 10000}_{\rm K}
\node{\ng 0.5\z_{-0.5}}_{\rm J}
\node{\nk \copy1}
\node{\ng 0.5\z_{-0.5}}_{\rm J}
\node{\hbox{(}}
$$
一方,イタリック補正をJFM由来グルーが透過するとしたならば,当然
$$
\node{\hbox{)}}
\node{\nk \copy1}
\node{\ng 0.5\z_{-0.5}}
\node{\hbox{(}}
$$
となる(実際の組版イメージでは,
「{\tenmini 斜め)}\hbox{}(」「{\tenmini 斜め)}(」).どちらにするか?



\item {\bf penaltyの周囲}

これも,例えば次の設定の下では,「|)\penalty1701(|」からは以下を得る:
\itemitem 「)」と行末の間に$-0.5\z$だけkernを入れる.
\itemitem 「)」「(」の行頭/行末禁則用penaltyの値はどれも1000.
$$
\halign{$#$\hfil&$#$\hfil&$#$\hfil&$#$\hfil&$#$\hfil\cr
\node{\hbox{)}}&\ncount=1
\node{\nk {-0.5}\z}_{\rm E}&\ncount=1
\node{\np 1000}_{\rm K}&\ncount=1
\node{\ng 1\z_{-0.5}}_{\rm J}\cr
&\ncount=1\node{\np 1701}&\ncount=1
\node{\np 1000}_{\rm K}&\ncount=1
\node{\ng 0.5\z_{-0.5}}_{\rm J}&\ncount=1
\node{\hbox{(}}\cr}
$$
\leftskip2\zw
例えばpenaltyを合算することとした場合,上の入力例では本来「)」「(」の間に
1701のpenaltyがあるのだから,
$$
\halign{$#$\hfil&$#$\hfil&$#$\hfil&$#$\hfil&$#$\hfil&$#$\hfil\cr
\node{\hbox{)}}&\ncount=1
\node{\nk {-0.5}\z}_{\rm E}&\ncount=1
\node{\np 3701}_{\rm K}&\ncount=1
\node{\ng 0.5\z_{-0.5}}_{\rm J}&\ncount=1
\node{\hbox{(}}\cr
\node{\hbox{)}}&\ncount=1
\node{\nk {-0.5}\z}_{\rm E}&\ncount=1
\node{\np 3701}_{\rm K}&\ncount=1
\node{\ng 1\z_{-0.5}}_{\rm J}&\ncount=1
\node{\ng 0.5\z_{-0.5}}_{\rm J}&\ncount=1
\node{\hbox{(}}\cr
}
$$
のどちらか(上はpenaltyを透過する場合,下は透過しない場合)にするのが
良いと思われます.


\item {\bf discretionary breakの取り扱い}

discretionary break ({\it disc\_node})は,行分割時の行末の内容<pre>,
行頭の内容<post>,それに行分割しないときの内容<no_break>の3つをリストの形で
持っている.|linebreak.w|を見る限り,Lua\TeX でも<pre>, <post>, <no_break>の
中身にはglueやpenaltyを許容していないようだ.

現行の実装では,<pre>, <post>, <no_break>のどれも,和文フォントへの置換の
段階からして行われていない(だから中身は全部欧文扱いとなる).
{\small 単純にサボっていました|^^;|}
<pre>, <post>, <no_break>の中身にglueやpenaltyが許容されないことから,
これらに対する和文用処理の方法として,次の2種類が挙げられる.
私は前者で良いのではないかと思っているのだが…….
\itemitem (現行のまま)discretionary breakの中身に和文文字はないものと想定する.
例えば<pre>の中身に和文文字を入れたい場合は,<pre>の中身全体を
必ずhboxで括ることとする.
\itemitem 「glueを挿入」を全部「自然長だけを取り出したkernを挿入」に置き換え,
普段の和文処理グルー挿入処理を流用する.

\enditem

\beginsection {\tt \char"5C[x]kanjiskip}の挿入%"

現実装の|\[x]kanjiskip|の挿入の方針として,
\item JFMグルーが挿入されていないところに「標準の空き量」として挿入する.
\item 実際の段落/hboxの内容に即して,組版イメージの見た目に関係のないところは透過する.
\enditem

|\[x]kanjiskip|挿入処理では,次の3つのnodeを用いている.
$$\ncount=0
\node{\it nr}\longrightarrow\cdots
\node{\it nq}\longrightarrow\cdots
\node{\it np}
$$
\item $\it nr$と$\it np$の間に|\[x]kanjiskip|を挿入しようとする.
\item 実際にnodeの形で挿入しようとする場所は$\it nq$の直後である.
\item $\it nr$, $\it nq$は異なるnodeとは限らない.
\item $\it np$はリストの先頭から末尾までループで渡る.その過程で
$\it nr$, $\it nq$を適宜更新し,実際のnode挿入処理を行っている.
\enditem

ループの中で,以下の場合には$\it nr$は変化せず,${\it nq}\leftarrow {\it np}$となる.
つまり,これらのnodeに対して|\[x]kanjiskip|は透過する:
\item $\it np$がpenaltyの場合
\item $\it np$がkernであって,予備知識の項目で述べられた値が
$$
\hbox{I (イタリック補正),E(行末との間),T(一時的)}
$$
であるもの.後者2つはJFMグルーの挿入で入るものなので,
ユーザは「イタリック補正は透過」と考えればよい.
\item $\it np$がアクセント由来のもの.この場合は,$\it nq$は変化せず,
${\it np}\leftarrow {\it next}({\it next}({\it np}))$となる.
\item $\it np$がinsertion, mark, |\vadjust|, whatsitのnodeである場合.
これらは水平リストからは消え去る運命にある.
\enditem

\end

%#! time luatex jfmglue
\input s1sty.tex % style file
\newcount\ncount
\def\node#1{
  \ifnum\ncount=0 \ncount=1\else\longrightarrow\fi
  \setbox0=\hbox{\kern.5em$\mathstrut#1$\kern.5em}\dp0=0pt
  \hbox{\vrule
    $\vcenter{\hsize=\wd0\hrule\kern.5ex\copy0\kern.5ex\hrule}$%
    \vrule\kern.1em}{}}
\def\nk{{\rm kern}\ }
\def\ng{{\rm glue}\ }
\def\np{{\rm penalty}\ }
\def\z{\,{\rm zw}}
\jfont\tenmini={file:ipam.ttf:slant=0.5;jfm=ujis} at 13\jQ
\def\mibox#1{\hbox{\it #1\/}}\def\IT#1{{\it #1\/}}
\ltjsetparameter{alxspmode={`\$,allow}}
\ltjsetparameter{alxspmode={`\\,allow}}
\ltjsetparameter{alxspmode={`\',allow}}

\centerline{\big Lua\TeX-ja 和文処理グルーについて}\bigskip
\centerline{\large\the\year/\the\month/\the\day}\medskip

本文書では,Lua\TeX-jaが(現時点において)和文処理に関わるglue/kernを
どのように挿入するかの内部処理について説明する.

\bigskip
{\large\bf\noindent これは仕様・内部処理の提案の1つにしかすぎません.最終的にこのようになる
保証はどこにもありませんし,現時点でのLuaコードが本文書に従っている保証もありません.
バグが混入している可能性も大きいです.}

\beginsection 予備知識

説明に入る前に,段落やhboxの中身は,\TeX の内部ではnode達による
リストとして表現されていることに注意する.nodeの種類については,
\IT{The\ Lua\TeX\ Reference}の第8章を参照して欲しい.代表的なものを挙げると,

\item \IT{glyph\_node}: 文字(合字も含む)を表現する.和文処理グルーを挿入する際には,
既に各\IT{glyph\_node}が欧文文字のものか和文文字のものか区別がついている.また,
しばしば\IT{glyph\_node}~$p$と,それの表す文字の文字コード$p.\mibox{char}$とを同一視する.
\item \IT{glue\_node}: glueを表す.
\item \IT{kern\_node}: kernを表す.各|kern_node|には|subtype|という値があり,
次の3種類を区別できるようになっている.
\itemitem 0: 欧文用TFM由来
\itemitem 1: 明示的な|\kern|か,イタリック補正 (|\/|) によるもの
\itemitem 2: |\accent|による非数式アクセント用文字の左右位置調整のためのもの
\item \IT{penalty\_node}: penaltyを表す.
\item \IT{hlist\_node}: hbox(水平ボックス)を表す.
\enditem


\item 次のように,nodeがどのように連続しているかを表すことにする.
$$
\node{a}\node{b}_{1}\node{c}
$$
下添字は,Lua\TeX-jaにおいてそのnodeの役割を区別するためにつけられた
値(icflagと呼ぼう)であり,次のようになっている.
$$
\vbox{\halign{#:\ \hfil&#\hfil\quad&#:\ \hfil&#\hfil\cr
1&イタリック補正由来のkern &
2&幅補正のため,hboxにカプセル化された和文文字\cr
3&禁則処理用penalty&
4&JFM由来のglue/kern\cr
5&「行末」との間に入るkern&
6&|\kanjiskip|用glue\cr
7&|\xkanjiskip|用glue&
8&既に処理対象となったnode\cr
15&リスト先頭/末尾に入るglue/kern/penalty\span\omit\span\cr
}}
$$
和文処理グルーの挿入処理に一度通されたnodeは,みなicflagが3以上となることに注意.

なお,上添字はnodeのsubtypeを表す.
\item {\sf jaxspmode}のようなサンセリフ体で,|\ltjsetparameter|で設定可能なパラメタ値を表す.
\item タイプライタ体の|\kanjiskip|, |\xkanjiskip|は,
それぞれ「和文間空白」「和欧文間空白」の意味で抽象的に用いている.
\item nil値は$\emptyset$と書く.
\enditem

\beginsection 方針など

本バージョンにおいては,JFM由来グルーと|\[x]kanjiskip|の挿入は同じ段階で行われる.
大雑把に言うと,
$$
\vbox{\hsize=0.85\hsize\bf 和文処理グルーの挿入処理では,以下は存在しないものとして扱われる:
\item 「文字に付属する」アクセントやイタリック補正.
\item 行中数式の内部.
\item 実際の組版中には現れないinsertion, vadjust, mark, whatsit node達.
\enditem}
$$

\beginparagraph 和文文字の「自然長」(JFMにおける{\tt width}の指定値)について

p\TeX においては,和文文字の行頭と行末に自動的にglueやkernをおくことはできなかったことから,
JFMにおける文字幅の意味は,
$$
\hbox{\vbox{\hsize=0.8\hsize \noindent
「その文字が行頭におかれるときの,版面左端の位置」を左端,\hfill\break
「その文字が行末におかれるときの,版面右端の位置」を右端としたときの幅}}
$$
というように,明確な意味があるものであった.例えば,乙部さんによるぶら下げ組パッケージ ({\tt burasage.lzh})
においては,句読点類(「|,、.。|」の4文字)の文字幅は0.0となっている.

一方,Lua\TeX-ja においては,和文文字が行末にきた場合,その文字と行末の間にkernを挿入することができる:
例えば,前に挙げた4文字についてぶら下げ組をしたいのであれば,

\begintt
   [1701] = {
      chars = { 'lineend' }
   },
   [42] = {
      chars = { ',', '、', '.', '。' },
      align = 'left', left = 0.0, down = 0.0,
      width = 0.5, height = 0.88, depth = 0.12, italic=0.0,
      kern = { [1701] = -0.5, ...}
   }, ...
\endtt
のように,「文字|'lineend'|との間に負のkernをおく」ように指定すればよい.
そのため,p\TeX と比較すると,JFMにおける|width|の指定値に絶対的な意味はあまりないことになる.
{\small 行頭にもkern をおけるようにするかどうかは検討中である.}


\beginparagraph グルーの挿入単位「塊」

和文処理グルーの挿入処理は,
ごく大雑把にいうと,「連続する2つのnodeの間に何を挿入するか」の繰り返しである.
実際の挿入処理は,「隣り合った2つの『塊』\IT{Nq}, \IT{Np}の間に何を入れるか」を
単位として行われる.

\medskip\noindent{\bf 定義}\quad
「{\bf 塊}」(\IT{Nn}などで表す)とは,次の4つのいずれかの条件を満たすnode(達のリスト)
のことである:
\enum icflagが3以上15未満であるnode達の連続からなるリスト.

このようなnode達は,既に組み上がったhboxを|\unpackage|により解体したときに発生する.
一度和文処理グルーの挿入処理が行われているため,二重の処理を防ぐためにこうして1つの塊を構成させている.

なお,icflagが15であるnodeは,処理中に発見されしだい削除される
(hboxの先頭や末尾に挿入されたglue/kern/penaltyであるので,
本来の「段落/hboxの中身に適宜グルーを挿入する」という目的を考えると存在すべきでない).
\enum 数式開始を表す\IT{math\_node}から始まる文中数式を表すnodeのリスト:
$$
 \node{\hbox{数式境界(開始)}}
 \longrightarrow\hbox{(この間,行中数式が続く)}
 \node{\hbox{数式境界(終了)}}
$$
\enum \IT{glyph\_node}~$p$と,それと切り離すことが望ましくないと考えられるnode達:
$$
 \Bigl[\node{\nk\hskip-.5em}^2
 \node{\hbox{アクセント文字}}
 \node{\nk\hskip-.5em}^2\Bigr]
 \node{p}
 \Bigl[\node{\nk\hskip-.5em}_1\Bigr]
$$
但し,これには$p$が${\rm icflag}=2$のhboxである場合も含む.%
{\small ←この場合の処理は実はおこらない?}
\enum {\bf 以下のどれにもあてはまらない}node~$p$:
\itemitem 組版結果からは消えてしまう,\IT{ins\_node}, 
\IT{mark\_node}, \IT{adjust\_node}, \IT{whatsit\_node}.
\itemitem penalty(←但し,挿入処理の過程で値が変更されることはある)

\enditem

\medskip\noindent{\bf 記号}\quad
\IT{Bp}で,塊\IT{Nq}と塊\IT{Np}の間にある\IT{penalty\_node}達の配列を表す.


\beginsection 挿入処理の大枠

\beginparagraph 「塊」の保持するデータ

「塊」\IT{Np}は,内部では少なくとも次の要素を持ったテーブルとして表される:

\item \IT{.first}: \IT{Np}の先頭のnode.
\item \IT{.nuc}: \IT{Np}の「核」となるnode.
\itemitem 1., 2.によるものである場合,$\mibox{Np.nuc}=\mibox{Np.first}$.
\item \IT{.last}: \IT{Np}の最後のnode.
\item \IT{.id}: \IT{Np}の種類を表す値.
\itemitem 1.によるものである場合,\IT{id\_pbox}(Pseudo BOXのつもり).
\itemitem 3.によるものであり,$p$が和文文字だった場合,\IT{id\_jglyph}.
\itemitem 4.によるものであり,$p$が垂直変位がnon-zeroなhbox,
あるいはvbox, ruleだった場合,\IT{id\_box\_like}.
\itemitem それ以外の場合,node~$p$の種別を表す数値$p.\mibox{id}$そのもの.
(数値そのものだと使い勝手が悪いので,\IT{id\_glyph}, \IT{id\_glue}, \IT{id\_kern}などと
別名を定義している)
\enditem

\medskip\noindent{\bf 定義}\quad
「\IT{Np}の中身の先頭」を意味する$\mibox{head}(\mibox{Np})$は,以下で定義される:

\noindent{\small (説明の都合上作った記法で,Luaソース中にはこのような書き方はない)}

\item \IT{Np.id}が\IT{id\_hlist}の場合:後に述べる{\tt check\_box}関数を用いて,
hbox~\IT{Np.nuc}中の「最初のnode」「最後のnode」を求める.
\item \IT{Np.id}が\IT{id\_pbox}の場合:\IT{id\_hlist}の場合とほぼ同様.
\item $\mibox{Np}.\mibox{id}=\mibox{id\_glyph}$(欧文文字)の場合:
\itemitem \IT{glyph\_node}~\IT{Np.nuc}が単一の文字を格納している(合字でない)場合は,\IT{Np.nuc}自身.
\itemitem そうでない場合は,合字の構成要素の先頭→構成要素の先頭→……
と再帰的に探索し,最後にたどり着いた\IT{glyph\_node}.
\item $\mibox{Np}.\mibox{id}=\mibox{id\_disc}$ (discretionary break) の場合:
\IT{disc\_node}は,
\item $\mibox{Np}.\mibox{id}=\mibox{id\_jglyph}$(和文文字)の場合:\IT{Np.nuc}自身.
\item $\mibox{Np}.\mibox{id}=\mibox{id\_math}$(数式境界)の場合:
「文字コード$-1$の欧文文字」を仮想的に考え,それを$\mibox{head}(\mibox{Np})$とする.
\item それ以外の場合:未定義.敢えて書けば$\mibox{head}(\mibox{Np}):=\emptyset$.
\enditem

\noindent 同様にして,「\IT{Np}の中身の先頭」を意味する$\mibox{last}(\mibox{Np})$も定義され,
「\IT{Np}は,先頭が$\mibox{head}(\mibox{Np})$,
末尾が$\mibox{tail}(\mibox{Np})$であるような単語」のように
考えることができる.

\medskip\noindent{\bf 定義}\quad
「\IT{glyph\_node}~$h$の情報を算出する」とは,
$h\neq\emptyset$の時に,テーブル\IT{Np}に
以下のような要素を追加することである:

\item \IT{.pre}: $h$の文字コードに対する{\sf prebreakpenalty}パラメタの値
\item \IT{.post}: $h$の文字コードに対する{\sf postbreakpenalty}パラメタの値
\item \IT{.xspc\_before}, \IT{.xspc\_after}: $h$の前後に|\xkanjiskip|が挿入可能であるかの
指定値(パラメタ{\sf jaxspmode}, {\sf alxspmode}由来)
\item \IT{.auto\_xspc}: $h$での{\sf autoxspacing}パラメタの値
\enditem

$h$が和文文字を格納している場合は,さらに次の要素の追加作業も含む:

\item \IT{.size}: $h$で使われている和文フォントのフォントサイズ.
\item \IT{.met}, \IT{.var}: 使われているJFMの情報.
\item \IT{.auto\_kspc}: {\sf autospacing}パラメタの値.
\enditem

\beginparagraph 全体図

\enum 変数類の初期化
\itemitem 処理対象が段落の中身(後で行分割される)の場合:$\mibox{mode}\leftarrow\top$
\itemT \IT{lp}(node走査用カーソル)の初期位置は,
リスト先頭部にある|\parindent|由来のhboxや
local paragraph($\Omega$由来)等の情報を格納するwhatsit nodeたちが終わった所
(つまり,段落本来の先頭部分)となる.
\itemT \IT{last}(リスト末尾のnode)も,リストの最後部に挿入される|\parfillskip|由来のglue%
を指す.
\itemitem 
処理対象がhboxの中身の場合:$\mibox{mode}\leftarrow\bot$
\itemT \IT{lp}はリスト先頭.
\itemT 番人として,リスト末尾にkernを挿入.\IT{last}はこのkernとなる.

\enum 先頭が\IT{lp}以降にある塊で,一番早いものを\IT{Np}にセットする.
\itemitem 作業の途中で$\mibox{lp}=\mibox{last}$となったら,処理対象のリストに塊はないので,8.へ.
\itemitem そうでなければ,$\mibox{head}(\mibox{Np})$の情報を算出しておく.
\itemitem 本段階終了後,\IT{lp}は\IT{Np.last}の次のnodeとなる.


\enum ({\tt handle\_list\_head}) 
リストに最初に出てくる塊\IT{Np}が求まったので,リスト「先頭」とこの塊との間に和文処理グルーを挿入.

\enum 今の塊\IT{Np}と,その次の塊の間に入る和文処理グルーを求めるため,
一旦$\mibox{Nq}\leftarrow \mibox{Np}$として待避させ,次の塊\IT{Np}を探索する.
\itemitem 作業の途中で$\mibox{lp}=\mibox{last}$となったら,\IT{Nq}がリスト中最後の塊であるので,
7.へ.
\itemitem そうでなければ,$\mibox{head}(\mibox{Np})$の情報を算出しておく.
\itemitem 本段階終了後,\IT{lp}は\IT{Np.last}の次のnodeとなる.

\enum \IT{Nq}と\IT{Np}の間に和文処理グルーを挿入する.\IT{Np.id}による場合分けを行う.
「main loop その\nobreak1,~2」を参照のこと.

\enum \IT{Np}が単一の文字ではない{\small (合字など)}可能性がある以下の場合において,
$\mibox{tail}(\mibox{Np})$の情報を算出する.終わったら,再びループに入るため,4.へ.


\itemitem \IT{id\_glyph}(欧文文字)のとき
\itemitem \IT{id\_disc} (discretionary break) のとき
\itemitem \IT{id\_hlist}のとき
\itemitem \IT{id\_pbox}のとき

\enum ({\tt handle\_list\_tail}) 
リストの最後にある塊\IT{Nq}が求まったので,この塊とリスト「末尾」の間に和文処理グルーを挿入.

\enum $\mibox{mode}=\bot$の場合,番人となるkernを1.において挿入したので,その番人を削除する.

\enditem

\beginsection リスト先頭・末尾の処理と「boxの内容」

\beginparagraph リスト先頭の処理 ({\tt handle\_list\_head})

次の場合に,
\IT{Np}で使われているのと同じJFMを使った「文字コードが{\tt 'boxbdd'}の文字」と
\IT{Np}との間に入るglue/kernを,\IT{Np.first}の直前に挿入する:
\item $\mibox{Np.id}=\mibox{id\_jglyph}$(和文文字)
\item $\mibox{Np.id}=\mibox{id\_pbox}$であり,$\mibox{head}(\mibox{Np})$が和文文字であるとき.
\enditem
$$
\vcenter{\halign{$\mibox{mode}=#$:\qquad\hfil&$#$\hfil\cr
\bot&\node{\kern-1em}\,\hbox{(リスト先頭)}\longrightarrow\cdots\node{g}_{15}
\node{\mibox{Np}}\cr
\top&\node{\hbox{{\tt\char92 parindent}由来hbox}}\longrightarrow\cdots
\Bigl[\node{\np 10000}_{15}\Bigr]\node{g}_{15}\node{\mibox{Np}}\cr
}}
$$
ここで,$g$がglueかつ$\mibox{mode}=\top$かつ$\#\mibox{Bp}=0$のときのみ,|\parindent|由来のhboxの直後で改行されることを防ぐために
$g$の直前にpenaltyを挿入する.{\small 
($\#\mibox{Bp}$が1以上の場合は,|\parindent|と\IT{Np}の間にある
penaltyのため,\IT{Np}の直前での改行が起こり得る状態となっているので,
特にそれを抑制することもしない)\inhibitglue}.


\beginparagraph リスト末尾の処理 ({\tt handle\_list\_tail})

この場合,\IT{mode}の値により処理が全く異なる.

\noindent{\bf A: \IT{mode}が偽である場合.}

この場合はリストはhboxの中身だから,行分割はおこり得ない.
リスト先頭の処理と同様に,
次の場合に
\IT{Nq}と「文字コードが{\tt 'boxbdd'}の文字」と
の間に入るglue/kernを,\IT{Nq.last}の直後に挿入する:
\item $\mibox{Nq.id}=\mibox{id\_jglyph}$(和文文字)
\item $\mibox{Nq.id}=\mibox{id\_pbox}$であり,$\mibox{tail}(\mibox{Nq})$が和文文字であるとき.
\enditem
$$
\node{\mibox{Nq}}\node{g}_{15}\longrightarrow\cdots\node{\nk\hbox{(番人)}}
$$
上の番人は,次のstepで除去されるのだった.

\medskip
\noindent{\bf B: \IT{mode}が真である場合.}

この場合,段落の末尾には常に|\penalty 10000|と|\parfillskip|由来のグルーが存在する.
そのため,上のように「文字コードが{\tt 'boxbdd'}の文字」との空白を考えるのではなく,
まず,\IT{Nq}が行末にきたときに行末との間に入る空白$w$を代わりに挿入する.
\item $\mibox{Nq.id}=\mibox{id\_jglyph}$(和文文字)
\item $\mibox{Nq.id}=\mibox{id\_pbox}$であり,$\mibox{tail}(\mibox{Nq})$が和文文字であるとき.
\enditem
$$
\node{\mibox{Nq}}\node{\nk w}_{15}\node{\np10000}\longrightarrow\cdots
\node{\ng (\hbox{\tt\char92 parfillskip})}
$$
次に,|\jcharwidowpenalty|の挿入処理を行う→省略.

\beginparagraph box内の「最初/最後の文字」の検索 ({\tt check\_box})

「hboxの中の文字と外の文字の間に」|\kanjiskip|, |\xkanjiskip|の挿入を行えるようにするため,
{\tt check\_box}関数ではhbox内の「最初のnode」「最後のnode」の検索を行う.

\item 以下のnodeは検索から除外される:
\itemitem 組版結果からは消えてしまう,\IT{ins\_node}, 
\IT{mark\_node}, \IT{adjust\_node}, \IT{whatsit\_node}, penalty.
\itemitem (box中身の先頭/末尾に入っている)icflagが7のglue/kern/penalty.
\itemitem アクセント部とイタリック補正.
\item \IT{hlist\_node}~$q$に出会ったら,$q$の垂直変位量が0である限り,検索は$q$の内部も進む.以下同文.
\item 検索して得られた「最初のnode」「最後のnode」がそれぞれ\IT{glyph\_node}でなければ,
実際には$\emptyset$を返す.
\enditem

\vfill\eject
\beginsection main loop

\beginparagraph 一覧表

\IT{Nq}, \IT{Np}の種類別に挿入されるglue/kernの種別を表にすると次のようになる.

\def\;{\hskip0.25em}\ltjsetparameter{jacharrange={+1}}
\def\gkf#1#2#3#4#5{$\vcenter{\small\rm\halign{\hbox to 1em{\hss##\hss}\;\vrule&%
\hbox to 3em{\hss##\hss}&\vrule\;\hbox to 1em{\hss##\hss}\cr
#1\mathstrut&\omit\hfil #2\span\omit\cr\noalign{\hrule}#3&#4\strut&#5\cr}}$}
\setbox1=\hbox{\gkf{E}{M→K}{○}{nor}{○}}
\setbox2=\hbox to 0.4pt{\vrule height\dimexpr \ht1+0.5em\relax depth \dimexpr \dp1\relax}
$$\def\:{\hskip0.5em}\lineskiplimit=\maxdimen\lineskip=0pt
\vcenter{\halign{\hfil#\hfil\hskip1em\copy2%
\:&\:\hfil#\hfil\:&\:\hfil#\hfil\:&\:\hfil#\hfil\:&\:\hfil#\hfil%
\:&\:\hfil#\hfil\:&\:\hfil#\hfil\:&\:\hfil#\hfil\:&\:\hfil#\hfil\cr
\raise0.5em\hbox{$\vcenter{\hbox{\IT{Nq}→}\smallskip\hbox{\IT{Np}↓}}$}%
&和文1&和文2&欧文&箱&id\_glue&id\_kern\cr
\noalign{\hrule}
和文1&
\gkf{E}{M→K}{○}{nor}{○}&
\gkf{}{$\rm O_A$→K}{×}{nor}{○}&
\gkf{}{$\rm O_A$→X}{○}{nor}{○}&
\gkf{}{$\rm O_A$}{---}{all}{○}&
\gkf{}{$\rm O_A$}{---}{nor}{○}&
\gkf{}{$\rm O_A$}{---}{sup}{○}\cr
和文2&
\gkf{E}{$\rm O_B$→K}{○}{nor}{×}&
\gkf{}{K}{×}{sup}{×}&
\gkf{}{X}{○}{sup}{×}\cr
欧文&
\gkf{E}{$\rm O_B$→X}{○}{nor}{○}&
\gkf{}{X}{○}{sup}{×}\cr
箱&\gkf{E}{$\rm O_B$}{○}{alw}{---}\cr
\IT{id\_glue}&\gkf{E}{$\rm O_B$}{○}{nor}{---}\cr
\IT{id\_kern}&\gkf{E}{$\rm O_B$}{○}{sup}{---}\cr
}}$$

\item {\bf 項目名}\quad 表1行目の\IT{Nq}の種類について説明する.\IT{Np}についても同様.
\itemitem 「和文1」:リスト中に直接出現している和文文字.
\itemT $\mibox{Nq.id}=\mibox{id\_jglyph}$であったとき.
\itemT $\mibox{Nq.id}=\mibox{id\_pbox}$かつ$\mibox{last}(\mibox{Nq})$が和文文字であったとき.
\itemitem 「和文2」:リスト内にあるhboxの中身として出現した和文文字.すなわち,
$\mibox{Nq.id}=\mibox{id\_hlist}$かつ
$\mibox{last}(\mibox{Nq})$が和文文字であったとき.
\itemitem 「欧文」:$\mibox{last}(\mibox{Nq})$が欧文文字であったとき.即ち,
\itemT リスト中に直接出現しているとき($\mibox{Nq.id}=\mibox{id\_jglyph}$ or~%
$\mibox{Nq.id}=\mibox{id\_pbox}$かつ$\mibox{last}(\mibox{Nq})$が欧文文字).
\itemT $\mibox{Nq.id}=\mibox{id\_hlist}$かつ
$\mibox{last}(\mibox{Nq})$が欧文文字であったとき.
\itemT $\mibox{Nq.id}=\mibox{id\_math}$であったとき.
\itemitem 「箱」:前後に和文処理グルーが挿入されない用なbox状のnode.
\itemT $\mibox{Nq.id}=\mibox{id\_list}$かつ
$\mibox{last}(\mibox{Nq})$が文字でなかった(未定義)だったとき.
\itemT $\mibox{Nq.id}=\mibox{id\_box\_like}$のとき.
\itemitem 「\IT{id\_glue}」:そのまま,$\mibox{Nq.id}=\mibox{id\_glue}$であったとき.
\itemitem 「\IT{id\_kern}」:そのまま,$\mibox{Nq.id}=\mibox{id\_kern}$であったとき.

\item 表中の各セルは,それぞれ次のような内容を表している:
$$\vcenter{\rm\halign{\hbox to 3em{\hss#\hss}\;\vrule&%
\hbox to 3.5em{\hss#\hss}&\vrule\;\hbox to 3em{\hss#\hss}\cr
左空白\mathstrut&\omit\hfil 右空白\span\omit\cr\noalign{\hrule}L&P取扱\strut&R\cr}}$$

\itemitem 「左空白」:\IT{Nq}の直後に挿入される空白の種類.空欄は,何も入らないことを表す.
\itemitem 「右空白」:\IT{Np}の直前に挿入される空白の種類.

なお,「A→B」は,まずAの種類のglue/kernを調べ,それが未定義ならば,
Bの種類のglue/kernを採用することを示している.このとき,矢印の右側に入る空白%
(K, X)はいつでも定義されていることに注意.

\itemitem 「P取扱」:\IT{Nq}と\IT{Np}の間に入る禁則用ペナルティの取扱の方法を表す.
\IT{Nq}と\IT{Np}の間で常に行分割を許すかに伴い,
{\bf nor}mal, {\bf alw}ays, {\bf sup}pressの3種類がある.
\itemitem 「L」「R」:禁則用ペナルティの挿入処理において,
\IT{Nq.post}~(L)や\IT{Np.pre}~(R)の値を実際に活用するかどうかを示す.値は次の3種類:
$$
\hbox{○(利用する),×(利用せず,0として扱う),---(未定義のため0扱い)}$$
\enditem


\beginparagraph 挿入されるglue/kernの種類

前節の表にある空白の種類についての解説を行う.

\item E: \IT{Nq}が行末にきたとき,
\IT{Nq}と行末の間に入る空白 (kern).挿入位置は\IT{Nq.last}の直後.
\itemitem JFMでは「文字コード|'lineend'|の文字」との間に入るkern量として設定できる.
\itemitem 右空白がkernであるときは挿入されない.
\itemitem この種類のkernが挿入される時,右空白は自然長がEの分だけ引かれる.


\item M: \IT{Nq}と\IT{Np}の間に入るJFM由来のglue/kern.
\itemitem \IT{Nq}, \IT{Np}の間で|\inhibitglue|を発行した場合,挿入は抑止される.
\itemitem 両方の塊で使われているJFMが(サイズもこめて)等しい場合は,両者で使われている
JFMの情報をそのまま利用できるので,量の決定は容易い.
\itemitem そうでなければ,まず
$$
\vcenter{\halign{\hfil$#:={}$&(\inhibitglue#\inhibitglue)\cr
gb&\IT{Nq}と「文字コードが|'diffmet'|の文字」との間に入るglue/kern\cr
ga&「文字コードが|'diffmet'|の文字」と\IT{Np}との間に入るglue/kern\cr
}}
$$
として2つの量を計算.少なくとも片方が未定義の場合は,もう片方の値を用いる.
そうでなければ,両者の値から自然長,伸び量,縮み量ごとに計算
(方法として,平均,和,大きい方,小さい方)を行い,それによって得られたglue/kernを採用する.
\item K: |\kanjiskip|を表すglueを挿入($\emptyset$にはならない).
\itemitem 両方の塊において「|\kanjiskip|の自動挿入が無効」
 ($\mibox{Nq.auto\_kspc}\vee \mibox{Np.auto\_kspc}=\bot$) ならば,長さ0のglueを挿入する.
\itemitem {\sf kanjiskip}パラメタの自然長が$\hbox{\tt\char92maxdimen}=(2^{30}-1)\,{\rm sp}$で
あれば,
JFMに指定されている|\kanjiskip|の量を用いる.\IT{Nq}, \IT{Np}で使われているJFMが異なった時の処理は,
Mの場合と同じである.
\itemitem 上のどれにも当てはまらなければ,{\sf kanjiskip}パラメタで表される量のglueを挿入する.
\item X: |\xkanjiskip|を表すglueを挿入($\emptyset$にはならない).
\itemitem 次のいずれかの場合には,|\xkanjiskip|は長さ0のglueとなる:
\itemT  両方の塊において,「|\xkanjiskip|の自動挿入が無効」という指定
($\mibox{Nq}.\mibox{auto\_xspc}\vee \mibox{Np}.\mibox{auto\_xspc}=\bot$)
がされていた場合.
\itemT \IT{Nq}内の文字について「直後への|\xkanjiskip|挿入が無効」であった場合,即ち
$\hbox{\sf alxspmode}\ge 2$(欧文)か$\hbox{\sf jaxspmode}\equiv0\pmod2$(和文).
\itemT \IT{Np}内の文字について「直前への|\xkanjiskip|挿入が無効」であった場合,即ち
$\hbox{\sf alxspmode}\equiv0\pmod2$(欧文)か$\hbox{\sf jaxspmode}\ge2$(和文).
\itemitem {\sf xkanjiskip}パラメタの自然長が|\maxdimen|であれば,
$\mibox{last}(\mibox{Nq})$, $\mibox{head}(\mibox{Np})$の片方が和文文字であるので,
そこで使われているJFMで指定されている|\xkanjiskip|の量を用いる
(JFMで指定されていなければ長さ0のglueと見なされる).
\itemitem 上のどれにも当てはまらなければ,{\sf xkanjiskip}パラメタで表される量のglueを挿入する.
\item $\rm O_B$: \IT{Nq}と「文字コードが|'jcharbdd'|の文字」との間に入るglue.
Mのバリエーションと考えればよく,同じように|\inhibitglue|の指定で抑止される.
\item $\rm O_A$: 「文字コードが|'jcharbdd'|の文字」と\IT{Np}との間に入るglue.
Mのバリエーションと考えればよく,同じように|\inhibitglue|の指定で抑止される.
\enditem


\beginparagraph penaltyまわりの処理

隣り合った塊\IT{Nq}, \IT{Np}の間には,集合\IT{Bp}で表される0個以上のpenaltyがあるのだった:
$$
\node{\mibox{Nq}}
\Bigl[\node{\hbox{E}}_4\Bigr]
\longrightarrow\cdots 
\hbox{(penaltyある可能性あり)}\cdots
\Bigl[\node{\hbox{M, K, X他}}_{3,\,5,\,6}\Bigr]
\node{\mibox{Np}}
$$
禁則処理に関係するpenaltyの挿入処理は,以下に述べるところ部分は共通の動作である.

\medskip 
$\#\mibox{Bp}\ge 1$の場合には,{\bf 全ての}\IT{Bp}の元$p$~(penalty)に対して次を行う:
$$p.\mibox{penalty}\mathrel{+}=a,\qquad a:=\mibox{Nq.post}+\mibox{Np.pre}.$$
\item
全ての\IT{Bp}の元に対して行うのは,
実際にはどのpenaltyの位置で行分割が行われるかがわからないからである.
\item 数ページ前の表で,左下が「×」 or 「---」となっていた場合は,上の計算式において
\IT{Nq.post}は0と扱われる.右下が「×」 or 「---」なら,\IT{Np.pre}が0と扱われる.
\item penalty値の計算では,$+10000$は正の無限大,$-10000$は負の無限大として扱っている.
そのため,$a$の計算や$p.\mibox{penalty}$への加算代入のところでは,
通常の加減算で絶対値が10000を越えたら分はカットし,さらに$(10000)+(-10000)=0$としている.
\enditem

$\#\mibox{Bp}=0$の場合が,penalty挿入の3種類の方法「normal」「always」「suppress」で
異なる部分である:

\item {\bf「normal」の場合:}
次の場合に,$p.\mibox{penalty}=a$であるpenalty~$p$を作成し,
それを(M, K他のglue挿入前に)\IT{Np.first}の直前に挿入する:
$$
\hbox{左空白(E)が存在しているか,$a\neq 0$かつ右空白がkernである.}
$$
\item {\bf「always」の場合:}
この場合は,\IT{Nq}, \IT{Np}の間で常に行分割可能にしたいので,挿入する条件は以下のようになる:
$$
\hbox{左空白(E)が存在しているか,右空白がglueでない(つまり,kernか未定義のとき).}
$$

\item {\bf「suppress」の場合:}このとき,\IT{Nq}と\IT{Np}の間での行分割は元々不可能である.
Lua\TeX-ja では,そのような場合を「わざわざ行分割可能に」することはしない.
つまり,右空白がglueであるとき,その直前に|\penalty 10000|を挿入する.
\enditem

\beginparagraph いくつかの例:未完

\beginsection main loop その2: その他の場合

\setbox1=\hbox{hp}\setbox2=\hbox to 0.4pt{\vrule height\dimexpr \ht1+0.25em\relax depth \dimexpr \dp1+0.25em\relax}
$$\def\:{\hskip0.5em}\lineskiplimit=\maxdimen\lineskip=0pt
\vcenter{\halign{\hfil\IT{#}\hfil\hskip1em\copy2%
\:&\:\hfil\IT{#}\hfil\:&\:\hfil\IT{#}\hfil\:&\:\hfil\IT{#}\hfil\:&\:\hfil\IT{#}\hfil%
\:&\:\hfil\IT{#}\hfil\:&\:\hfil\IT{#}\hfil\:&\:\hfil\IT{#}\hfil\:&\:\hfil\IT{#}\hfil\cr
\IT{Np}→&&id\_hlist 非文字\cr
\IT{Nq}↓&$\mibox{head}(\mibox{Nq})$\rm: 欧文&id\_box\_like&id\_glue&id\_kern\cr
\noalign{\vskip.25em\hrule\vskip.25em}
id\_jglyph&\rm E${}+{}$({\rm $\rm O_B$→X})&\rm E${}+\rm O_B^*$&\rm E${}+\rm O_B$&\rm E${}+\rm O_B^+$\cr
id\_pbox 和&\rm E${}+{}$({\rm $\rm O_B$→X})&\rm E${}+\rm O_B^*$&\rm E${}+\rm O_B$&\rm E${}+\rm O_B^+$\cr
id\_hlist 和&\rm X${}^+$&---&---&---\cr
他&---&---&---&---\cr
}}$$

\end

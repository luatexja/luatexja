%#! time luatex jfmglue
\input s1sty.tex % style file
\newcount\ncount
\def\node#1{
  \ifnum\ncount=0 \ncount=1\else\longrightarrow\fi
  \setbox0=\hbox{\kern.5em$\mathstrut#1$\kern.5em}\dp0=0pt
  \hbox{\vrule
    $\vcenter{\hsize=\wd0\hrule\kern.5ex\copy0\kern.5ex\hrule}$%
    \vrule\kern.1em}}
\def\nk{{\rm kern}\ }
\def\ng{{\rm glue}\ }
\def\np{{\rm penalty}\ }
\def\z{\,{\rm zw}}
\jfont\tenmini={file:ipam.ttf:slant=0.5;jfm=ujis} at 13\jQ
\def\mibox#1{\hbox{\it #1\/}}\def\IT#1{{\it #1\/}}

\centerline{\big Lua\TeX-ja 和文処理グルーについて}\bigskip
\centerline{\large\the\year/\the\month/\the\day}\medskip

本文書では,Lua\TeX-jaが(現時点において)和文処理に関わるglue/kernを
どのように挿入するかの内部処理について説明する.

\beginsection 予備知識

説明に入る前に,段落やhboxの中身は,\TeX の内部ではnode達による
リストとして表現されていることに注意する.nodeの種類については,
\mibox{The\ Lua\TeX\ Reference}の第8章を参照して欲しい.代表的なものを挙げると,

\item \IT{glyph\_node}: 文字(合字も含む)を表現する.和文処理グルーを挿入する際には,
既に各\IT{glyph\_node}が欧文文字のものか和文文字のものか区別がついている.また,
しばしば\IT{glyph\_node} $p$と,それの表す文字の文字コード$p.\mibox{char}$とを同一視する.
\item \IT{glue\_node}: glueを表す.
\item \IT{kern\_node}: kernを表す.各|kern_node|には|subtype|という値があり,
次の3種類を区別できるようになっている.
\itemitem 0: 欧文用TFM由来
\itemitem 1: 明示的な|\kern|か,イタリック補正 (|\/|) によるもの
\itemitem 2: 非数式アクセント用文字の左右位置調整のためのもの
\item \IT{penalty\_node}: penaltyを表す.
\item \IT{hlist\_node}: hbox(水平ボックス)を表す.
\enditem


\item 次のように,nodeがどのように連続しているかを表すことにする.
$$
\node{a}\node{b}_{\rm I}\node{c}
$$
右下についている添字は,Lua\TeX-jaにおいてそのnodeの役割を区別するためにつけられた
値(jtypeと呼ぼう)であり,次のようになっている.
$$
\vbox{\halign{#:\ \hfil&#\hfil\quad&#:\ \hfil&#\hfil\cr
I&イタリック補正由来のkern&
T&|\[x]kanjiskip|に置換されうるkern\cr
J&JFM由来のglue/kern&
K&禁則処理用penalty\cr
E&「行末」との間に入るkern&
KS&|\kanjiskip|用glue\cr
XS&|\xkanjiskip|用glue\cr
}}
$$
\item {\sf jaxspmode}のようなサンセリフ体で,|\ltjsetparameter|で設定可能なパラメタ値を表す.
\item タイプライタ体の|\kanjiskip|, |\xkanjiskip|は,それぞれ「和文間空白」「和欧文間空白」の意味で
抽象的に用いている.
\item nil値は$\emptyset$と書く.
\enditem

\beginsection JFM由来グルーの挿入 ({\tt luatexja-jfmglue.lua})

JFM由来グルーの処理は,「連続する2つのnodeの間に何を入れるか」という単位で行われる.
そのため,
\item node生成を伴わないもの(グループ境界,|\relax|等)は全て無視される.
\item 一方,node生成を伴うものは全て「透過しない」.
例えば,次のソースにおいて,閉じ括弧と開き括弧の間に入る物は,
左と右とで異なる:
\begintt
)(  )\hbox{}(
\endtt
\item JFM由来グルーの挿入禁止を行う|\inhibitglue|は,内部では専用のnodeを作ること
によって実装している.この|\inhibitglue|用nodeは透過する.
\enditem
以下,$q$, $p$を連続するnodeとする.


\beginparagraph 2つの和文文字の間

この場合,グルー挿入に関係する量は次の通りである.これら3つの量の値によって,
$q$と$p$の間に何が挿入されるかが決定される.これらの記号は他の場合にも用いる.
\item $g$: JFMで指定された,$q$と$p$の間に入るglue/kern.
JFMで規定されていないときは$\emptyset$と書こう.
両ノードで使われているJFMが異なる時の$g$の決定方法は,後に記述する.
\item $w$: JFMで指定された,「$q$の直後で改行が行われた場合,
$q$と行末の間に入るカーン量」の値.

$g-w$で,$g$の自然長を$w$だけ減算したglue/kernを表すことにする.

\item $P$: $q$に対する行末禁則用ペナルティ (post-break penalty) と,
$p$に対する行頭禁則用ペナルティ (pre-break penalty) との和.どちらも
設定されていないときは0となる.
\enditem

設計方針としては,
\item JFM由来で入るものがkernの場合,この場所では行分割は許さない.
\item そうでない場合,(penaltyの値$P$があるが)この場所での行分割は可能である.
\enditem
である.さて,次が実際の場合わけである:

\enum $w\neq 0$, $g=\emptyset$のとき
$$\ncount=0
\node{q}
\node{\nk w}_{\rm E}
\node{\np P}_{\rm K}
\node{\nk{-w}}_{\rm T}
\node{p}
$$
この$\node{\nk{-w}}_{\rm T}$は,
「$q$と$p$の間で行分割されないときは間に何のglue/kernもないように見える」ために
挿入されたものである.次のステップで|\[x]kanjiskip|の挿入が行われる時に,このnodeは
|\[x]kanjiskip|用のglueに置換される.

\enum $w\neq 0$, $g\neq\emptyset$のとき:
$$\ncount=0
\node{q}
\node{\nk w}_{\rm E}
\node{\np P}_{\rm K}
\node{g-w}_{\rm J}
\node{p}
$$

\enum $w=0$, $g$: kernのとき
$$\ncount=0
\node{q}
\node{g}_{\rm J}
\node{p}
$$

\enum $w=0$, $g$: glueのとき
$$\ncount=0
\node{q}
\node{\np P}_{\rm K}
\node{g}_{\rm J}
\node{p}
$$

\enum $w=0$, $g=\emptyset$, $P\neq 0$のとき
$$\ncount=0
\node{q}
\node{\np P}_{\rm K}
\node{p}
$$

\enum $w=0$, $g=\emptyset$, $P=0$のとき
$$\ncount=0
\node{q}
\node{p}
$$
\enditem

なお,両ノードで使われているJFMが異なる時の$g$の決定方法であるが,
\enum $g_{\rm L}$を,$q$に使用されているJFMにおける,「$q$と文字|'diffmet'|」の間に
入るglue/kernの値とする.
\enum $g_{\rm R}$を,$p$に使用されているJFMにおける,「文字|'diffmet'|と$p$」の間に
入るglue/kernの値とする.
\enum 両方から,実際に入る$g$の値を計算する.
\itemitem $g_L$, $g_R$の少なくとも片方が$\emptyset$のときは,$\emptyset$でない方を
そのまま採用する.
\itemitem 両方とも$\emptyset$でない場合は,|differentjfm|の値にそって$g$の値を計算する.
\enditem

\beginparagraph 和文文字と(和文文字,kern以外のnode)の間

「和文文字の間」の場合に対して,以下が異なる:
\item $g$は,$q$に使用されているJFMにおける,「$q$と文字|'jcharbdd'|」の間に
入るglue/kernの値である.
\item $p$が{\bf penaltyでない場合}は,いつもこの位置で行分割できるようにするため,
case~6 ($w$, $P=0$, $g=\emptyset$) の場合にも,$q$と$p$の間には
0という値のpenaltyが入る.即ち,次のようになる.
$$\ncount=0
\node{q}
\node{\np 0}_{\rm K}
\node{p}
$$
\enditem

\beginparagraph (和文文字,kern以外のnode)と和文文字の間

この場合も,基本的には「和文文字の間」と似ているが,以下が異なる:
\item $g$は,$p$のJFMにおける,「文字|'jcharbdd'|と$p$」の間に
入るglue/kernの値である.
\item 常に$w=0$である.
\item いつもこの位置で行分割できるようにするため,
case~6 ($w$, $P=0$, $g=\emptyset$) の場合にも,$q$と$p$の間には
0という値のpenaltyが入る.
\enditem

即ち,次の3通りになる.
\enum $g$: kernのとき
$$\ncount=0
\node{q}
\node{g}_{\rm J}
\node{p}
$$

\enum $g$: glueのとき
$$\ncount=0
\node{q}
\node{\np P}_{\rm K}
\node{g}_{\rm J}
\node{p}
$$

\enum $g=\emptyset$のとき
$$\ncount=0
\node{q}
\node{\np P}_{\rm K}
\node{p}
$$
\enditem

\beginparagraph 和文文字とkernの間,kernと和文文字の間

和文文字の後にkernが続いた場合,あるいはkernの後に和文文字が続いた場合,
この間で行分割はできないものとしている.そのため,
以下の3ケースに限られる:
\enum $g$: kernのとき
$$\ncount=0
\node{q}
\node{g}_{\rm J}
\node{p}
$$

\enum $g$: glueのとき
$$\ncount=0
\node{q}
\node{\np 10000}_{\rm K}
\node{g}_{\rm J}
\node{p}
$$

\enum $g=\emptyset$のとき
$$\ncount=0
\node{q}
\node{p}
$$
\enditem

なお,ここでの$g$は,
\item kernが前だった場合は,
$q$のJFMにおける,「$q$と|'jcharbdd'|」の間に
入るglue/kernの値.
\item kernが後だった場合は,
$p$のJFMにおける,「|'jcharbdd'|と$p$」の間に
入るglue/kernの値.
\enditem

\beginparagraph 要検討の箇所

私が推測するに,欧文では,
\item 単語内ではフォントは変わらない.
\item 単語内では,明示的に/ハイフネーションにより挿入されたdiscretionary break以外では
行分割がおきない.
\enditem
という事情があるため,TFM由来のkernや合字処理は(nodeを生成しないもの以外は)
何も透過しないという状態になっているものと思われます.

そのため,JFMグルー等の仕様を考える場合,欧文でいう「単語」に対応するようなものは
何か,というのを考える必要があります.現実装では,素直に欧文の合字処理と同様のものであると考え,
透過するnodeはない,という仕様にしています.
しかし,|\[x]kanjiskip|の処理と共通にしてしまうというのも
考え方によってはありかもしれません.

\bigskip
\item {\bf イタリック補正のkernの周囲}

例えば,|jfm-ujis.lua|では,|'jcharbdd'|は文字クラス
0であるため,今の実装では,\hfil\break
「|)\/(|」という入力からは,次のnodeの並びを得る:
\setbox1=\hbox{|\/|}%
$$
\node{\hbox{)}}
\node{\np 10000}_{\rm K}
\node{\ng 0.5\z_{-0.5}}_{\rm J}
\node{\nk \copy1}
\node{\ng 0.5\z_{-0.5}}_{\rm J}
\node{\hbox{(}}
$$
一方,イタリック補正をJFM由来グルーが透過するとしたならば,当然
$$
\node{\hbox{)}}
\node{\nk \copy1}
\node{\ng 0.5\z_{-0.5}}
\node{\hbox{(}}
$$
となる(実際の組版イメージでは,
「{\tenmini 斜め)}\hbox{}(」「{\tenmini 斜め)}(」).どちらにするか?



\item {\bf penaltyの周囲}

これも,例えば次の設定の下では,「|)\penalty1701(|」からは以下を得る:
\itemitem 「)」と行末の間に$-0.5\z$だけkernを入れる.
\itemitem 「)」「(」の行頭/行末禁則用penaltyの値はどれも1000.
$$
\vbox{\halign{$#$\hfil&$#$\hfil&$#$\hfil&$#$\hfil&$#$\hfil\cr
\node{\hbox{)}}&\ncount=1
\node{\nk {-0.5}\z}_{\rm E}&\ncount=1
\node{\np 1000}_{\rm K}&\ncount=1
\node{\ng 1\z_{-0.5}}_{\rm J}\cr
&\ncount=1\node{\np 1701}&\ncount=1
\node{\np 1000}_{\rm K}&\ncount=1
\node{\ng 0.5\z_{-0.5}}_{\rm J}&\ncount=1
\node{\hbox{(}}\cr}}
$$
\leftskip2\zw
例えばpenaltyを合算することとした場合,上の入力例では本来「)」「(」の間に
1701のpenaltyがあるのだから,
$$
\vbox{\halign{$#$\hfil&$#$\hfil&$#$\hfil&$#$\hfil&$#$\hfil&$#$\hfil\cr
\node{\hbox{)}}&\ncount=1
\node{\nk {-0.5}\z}_{\rm E}&\ncount=1
\node{\np 3701}_{\rm K}&\ncount=1
\node{\ng 0.5\z_{-0.5}}_{\rm J}&\ncount=1
\node{\hbox{(}}\cr
\node{\hbox{)}}&\ncount=1
\node{\nk {-0.5}\z}_{\rm E}&\ncount=1
\node{\np 3701}_{\rm K}&\ncount=1
\node{\ng 1\z_{-0.5}}_{\rm J}&\ncount=1
\node{\ng 0.5\z_{-0.5}}_{\rm J}&\ncount=1
\node{\hbox{(}}\cr
}}
$$
のどちらか(上はpenaltyを透過する場合,下は透過しない場合)にするのが
良いと思われます.


\item {\bf discretionary breakの取り扱い}

discretionary break (\IT{disc\_node})は,行分割時の行末の内容<pre>,
行頭の内容<post>,それに行分割しないときの内容<no_break>の3つをリストの形で
持っている.|linebreak.w|を見る限り,Lua\TeX でも<pre>, <post>, <no_break>の
中身にはglueやpenaltyを許容していないようだ.

現行の実装では,<pre>, <post>, <no_break>のどれも,和文フォントへの置換の
段階からして行われていない(だから中身は全部欧文扱いとなる).
{\small 単純にサボっていました|^^;|}
<pre>, <post>, <no_break>の中身にglueやpenaltyが許容されないことから,
これらに対する和文用処理の方法として,次の2種類が挙げられる.
私は前者で良いのではないかと思っているのだが…….
\itemitem (現行のまま)discretionary breakの中身に和文文字はないものと想定する.
例えば<pre>の中身に和文文字を入れたい場合は,<pre>の中身全体を
必ずhboxで括ることとする.
\itemitem 「glueを挿入」を全部「自然長だけを取り出したkernを挿入」に置き換え,
普段の和文処理グルー挿入処理を流用する.


\item {\bf 数式の取り扱い}

まだ数式中に和文文字が(hboxでカプセル化されることなく)出現することは想定していないが,
数式用コードを書いた後は,「数式中は和文処理グルーの挿入処理を無効とする」ようにしないといけないだろう.

\enditem

\beginsection {\tt \char"5C[x]kanjiskip}の挿入%"

現実装の|\[x]kanjiskip|の挿入の方針として,
\item JFMグルーが挿入されていないところに「標準の空き量」として挿入する.
\item 実際の段落/hboxの内容に即して,組版イメージの見た目に関係のないところは透過する.
\enditem

\beginparagraph 処理の概要

|\[x]kanjiskip|挿入処理では,次の3つのnodeを用いている.
$$\ncount=0
\node{\IT{nr}}\longrightarrow\cdots
\node{\IT{nq}}\longrightarrow\cdots
\node{\IT{np}}
$$
\item \IT{nr}と\IT{np}の間に|\[x]kanjiskip|を挿入しようとする.
\item 実際にnodeの形で挿入しようとする場所は$\it nq$の直後である.
\item \IT{nr}, \IT{nq}は異なるnodeとは限らない.
\item {\bf \IT{np}はリストの先頭から末尾までループで渡る.}その過程で
\IT{nr}, \IT{nq}を適宜更新し,実際のnode挿入処理を行っている.
\item 厳密には,コード中では\IT{nr}という変数は使っていない.代わりに使われているのは,
\itemitem $\mibox{insert\_skip}\in \{\mibox{no\_skip}, 
  \mibox{after\_schar}, \mibox{after\_wchar}\}$: 
「node \IT{nr}」の種類を表す:
\itemT \IT{no\_skip}: 「node \IT{nr}」の後ろ(\IT{nr}と\IT{np}の間)に
|\[x]kanjiskip|が入ることはない.
\itemT \IT{after\_schar}: 「node \IT{nr}」を,欧文文字(の入った\IT{glyph\_node})であり,
かつ{\sf alxspmode}パラメータの指定により「\IT{nr}の後ろに|\xkanjiskip|の挿入を許可する」ようなものとみなす.
\itemT \IT{after\_wchar}: 「node \IT{nr}」を和文文字(の入った\IT{glyph\_node})とみなす.
\itemitem \IT{nrc}: 「node \IT{nr}の文字コード」を表す.
\itemitem \IT{nrf\/}: 「\IT{nr}のフォント」を表す.
\itemitem $\mibox{nr\_spc}[1]$: 「node \IT{nr}」における
{\sf autospacing}(|\kanjiskip|の自動挿入を行うか否か)の設定値.
\itemitem $\mibox{nr\_spc}[2]$: 「node \IT{nr}」における
{\sf autoxspacing}(|\xkanjiskip|の自動挿入を行うか否か)の設定値.

\medskip\leftskip=2\zw\noindent
\IT{nrc}, \IT{nrf}の値は,$\mibox{insert\_skip}=\mibox{after\_wchar}$のときのみ用いられる.
$\mibox{insert\_skip}=\mibox{no\_skip}$のときには,それだけで情報は十分であるから,
\IT{nr\_spc}, \IT{nq}の値も用いられない.
\enditem

ループの中で,以下の場合には\IT{nr}は変化せず,$\IT{nq}\leftarrow \IT{np}$となる.
つまり,これらのnodeに対して|\[x]kanjiskip|は透過する:
\item \IT{np}がpenaltyの場合
\item \IT{np}が|subtype|が0のkern(TFM由来)の場合.
\item \IT{np}が|subtype|が1のkern(つまり,明示的kernかイタリック補正由来)であって,jtypeが
$$
\hbox{I (イタリック補正),E(行末との間),T(一時的)}
$$
であった場合.後者2つはJFMグルーの挿入で入るものなので,
ユーザは「イタリック補正は透過」と考えればよい.

\item \IT{np}がinsertion, mark, |\vadjust|, whatsitのnodeである場合.
これらは水平リストからは消え去る運命にある.
\enditem


\beginparagraph \IT{np}が文字 (\IT{glyph\_node}) の場合

この場合がやはり一番基本となる.
\enum $\mibox{insert\_skip}=\mibox{after\_schar}$, \IT{np}: 和文文字の場合

前に書いたように,「node \IT{nr}」は(直後に|\xkanjiskip|の挿入が許可されている)欧文文字と
みなされている.
そのため,「node \IT{nr}」と\IT{np}の間に|\xkanjiskip|の入る条件は以下である.
\itemitem 文字\IT{np}に対する{\sf jaxspmode}パラメータの指定において
「直前への|\xkanjiskip|の挿入が許可」されている.
\itemitem 「node \IT{nr}における」{\sf autoxspacing}パラメタの値 ($\mibox{nr\_spc}[2]$) か,
\IT{np}における{\sf autoxspacing}の値の少なくとも一方が真である.

\medskip\leftskip=2\zw\noindent
まず,実際に入る|\xkanjiskip|の量$g$を次の方法で決定する:
\itemitem {\sf xkanjiskip}パラメータの自然長が|\maxdimen|でない場合,
{\sf xkanjiskip}パラメータの値をそのまま採用する.
\itemitem {\sf xkanjiskip}パラメータの自然長が|\maxdimen|の場合は,
\IT{np}で使われているJFMに設定されている|\xkanjiskip|の量を用いる.
\itemitem 上の2つのどれでもない場合,fallbackとして0を用いる.

{\bf 要検討}:既にJFMグルー挿入処理で和欧文間の行分割は可能としているので
0を挿入する意味はない?

\medskip\leftskip=2\zw\noindent
次にこのようにして決定された$g$を実際に挿入する:
\itemitem ほとんどの場合,$g$の値をもつglueを\IT{nq}の直後に挿入する.
$$\ncount=0
\node{\mibox{nr}}\longrightarrow\cdots
\node{\mibox{nq}}\node{\ng g}_{\rm XS}\longrightarrow\cdots
\node{\mibox{np}}
$$
\itemitem \IT{np}の直前が${\rm jtype}={\rm T}$なnodeの場合,そのnodeに
$g$の分だけ自然長/伸び/縮み量を加算する.
$$\ncount=0
\vbox{\halign{$#$\hfil\cr
\node{\mibox{nr}}\longrightarrow\cdots
\node{\mibox{nq}}\longrightarrow\cdots
\node{\ng h}_{\rm T}
\node{\mibox{np}}\cr
\hfil\Downarrow\cr
\vadjust{\bigskip}
\node{\mibox{nr}}\longrightarrow\cdots
\node{\mibox{nq}}\longrightarrow\cdots
\node{\ng g+h}_{\rm XS}
\node{\mibox{np}}\cr
\node{\mibox{nr}}\longrightarrow\cdots
\node{\mibox{nq}}\longrightarrow\cdots
\node{\nk k}_{\rm T}
\node{\mibox{np}}\cr
\hfil\Downarrow\cr
\node{\mibox{nr}}\longrightarrow\cdots
\node{\mibox{nq}}\longrightarrow\cdots
\node{\ng g+k}_{\rm XS}
\node{\mibox{np}}\cr
}}
$$

\medskip\leftskip=2\zw\noindent
最後に,次のループに移るために,次の処理を行う:
\itemitem 
$\mibox{nq}\leftarrow \mibox{np}$, 
$\mibox{nr\_spcの設定}$, 
$\mibox{nrf}\leftarrow \mibox{np}.\mibox{font}$, 
$\mibox{nrc}\leftarrow \mibox{np}.\mibox{char}$
\itemitem $\mibox{np}\leftarrow \mibox{next\/}(\mibox{np})$
\itemitem $\mibox{insert\_skip}\leftarrow \mibox{after\_wchar}$

\enum $\mibox{insert\_skip}=\mibox{after\_wchar}$, \IT{np}: 欧文文字の場合

前に書いたように,「node \IT{nr}」は和文文字とみなされている.
そのため,「node \IT{nr}」と\IT{np}の間に|\xkanjiskip|の挿入が起こるための条件は
次の3条件が満たされていることである:
\itemitem \IT{nrc}番の和文文字に対する{\sf jaxspmode}パラメータの設定で,
「直後への|\xkanjiskip|挿入」が許可されている.
\itemitem \IT{np}の文字{\small(もし\IT{np}が合字であれば,合字の構成要素の最初の文字)}%
に対する{\sf alxspmode}パラメータの設定で,
「直前への|\xkanjiskip|挿入」が許可されている.
\itemitem \IT{nr}における{\sf autoxspacing}パラメタの値 ($\mibox{nr\_spc}[2]$) か,
「node \IT{nr}」における{\sf autoxspacing}の値の少なくとも一方が真である.

\medskip\leftskip=2\zw\noindent
この後,実際に|\xkanjiskip|の量を計算し,nodeの形で実際に挿入するところは,
量の決定のところで\IT{np}の代わりに\IT{nrf}を用いる以外は同じである.
最後の,次のループに移るための処理では,次が行われる.
\itemitem 
$\mibox{nq}\leftarrow \mibox{np}$, 
$\mibox{nrc}\leftarrow \mibox{np}.\mibox{char}$, 
$\mibox{nr\_spcの設定}$
\itemitem $\mibox{np}\leftarrow \mibox{next\/}(\mibox{np})$
\itemitem \IT{insert\_skip}の設定.

$\mibox{insert\_skip}\leftarrow \mibox{after\_schar}$となるのは,
\IT{np}の文字{\small(もし\IT{np}が合字であれば,合字の構成要素の末尾の文字)}%
における{\sf alxspmode}パラメータの設定で,
「直後への|\xkanjiskip|挿入」が許可されている場合である.
そうでないときは,$\mibox{insert\_skip}\leftarrow \mibox{no\_skip}$となる.

\enum $\mibox{insert\_skip}=\mibox{after\_wchar}$, \IT{np}: 和文文字の場合

この場合は|\xkanjiskip|の代わりに|\kanjiskip|を挿入することとなる.
{\sf jaxspmode}, {\sf alxspmode}のように「直前/直後への|\kanjiskip|挿入許可の制御」
を行うパラメータは存在しない.「{\sf autospacing}で自動挿入が禁止される」とマニュアルでは言っているが,
それは{\bf 挿入する|\kanjiskip|の量を一時的に0にしているだけで,
「node \IT{nr}」と\IT{np}の間には
常に|\kanjiskip|が入ることには変わりはない}ことに注意.

\medskip
実際に入る|\kanjiskip|の量$g$は次の方法で決定される:
\itemitem \IT{nr}における{\sf autospacing}の値 ($\mibox{nr\_spc}[1]$) か,
\IT{np}における{\sf autospacing}の値が共に偽なら,$g=0$.
\itemitem {\sf kanjiskip}パラメータの自然長が|\maxdimen|でない場合,
{\sf kanjiskip}パラメータの値をそのまま採用する.
\itemitem {\sf xkanjiskip}パラメータの自然長が|\maxdimen|の場合は,
\IT{np}で使われているJFMに設定されている|\kanjiskip|の量を用いる:
\itemT 「node \IT{nr}」で使用されているJFMと,\IT{np}で使用されているJFMそれぞれに
|\kanjiskip|の値が設定されている場合,…….

\itemitem 上の3つのどれでもない場合,fallbackとして0を用いる.

\enditem

\medskip
$g$を実際に挿入するところは,今の場合も1.の場合と変わらない.
次のループに移るために\IT{nq}等の設定処理も,1.と同じである.

\beginparagraph \IT{np}がhboxの場合

|\[x]kanjiskip|は,垂直変位が0である(即ち,|\raise|, |\lower|により上下に移動されていない)
hboxの境界を跨ぐ.

\enum hbox内の「最初のnode」\IT{first\_char}と「最後のnode」\IT{last\_char}を探索する.
この探索は,次を透過する:
\itemitem 垂直変位が0であるhboxの境界(但し,空hboxは透過しない).
\itemitem insertion, mark, |\vadjust|, whatsit, penalty用のnode.
\itemitem 「最初のnode」「最後のnode」それぞれにいえることだが,文字 (\IT{glyph\_node}) でない場合は
\IT{first\_char},~\IT{last\_char}はそれぞれ$\emptyset$となる.

\medskip\leftskip=2\zw\noindent
前者の具体例として,例えば,次の入力を考える.
\begintt
あ\hbox{a}い\hbox{\hbox{}b\hbox{}}う\hbox{}cえ\hbox{\hbox{d}}お
\endtt
すると,
$$
\vbox{\halign{#の間のhbox\hfil\quad $\longrightarrow$\quad
&$\mibox{first\_char}=\hbox{#}$,\ \hfil
&$\mibox{last\_char}=\hbox{#}$\hfil\cr
1.\ 「あ」「い」&「a」&「a」\cr
2.\ 「い」「う」&$\emptyset$&$\emptyset$\cr
3.\ 「う」「c」&$\emptyset$&$\emptyset$\cr
4.\ 「え」「お」&「d」&「d」\cr
}}
$$%$
となる.

\enum \IT{np}の前に|\[x]kanjiskip|を挿入するか否か,あるいは実際に挿入する量の決定は,
\IT{first\_char}に対しての処理をそのまま適用する.つまり,上の例では
\itemitem 「あ」と「a」の間に|\xkanjiskip|が挿入されることから,「あ」とhbox 1.の間には|\xkanjiskip|が挿入される.
\itemitem 「い」とhbox 2.の間には|\xkanjiskip|が挿入されない.

\enum 同様に,\IT{np}の後ろに|\[x]kanjiskip|を挿入するか否かは,
文字\IT{last\_char}の後ろに対してどうなるかの値を用いる.上の例では,
\itemitem 「a」と「い」の間に|\xkanjiskip|が挿入されることから,hbox 1.と「い」の間には|\xkanjiskip|が挿入される.
\itemitem hbox 2.と「う」の間には|\xkanjiskip|が挿入されない.
\medskip\leftskip=2\zw\noindent
次のループに進むための設定も,node~\IT{last\_char}における値をもとに行う.
\enditem

以上より,項目1.で与えられた入力では,次の出力が得られる:
$$
\hbox{あ\hbox{a}い\hbox{\hbox{}b\hbox{}}う\hbox{}cえ\hbox{\hbox{d}}お}
$$
{\bf 要検討:}空のhboxは跨がないようにしているのは,次の2つを使い分けられるようにするため:
\item JFMグルー挿入の抑止:|\inhibitglue|
\item 全ての和文処理グルー挿入の抑止:
|\inhibitglue\hbox{}\inhibitglue|
\enditem

なお,\IT{np}の垂直変位が0でない場合は,\IT{np}の前後への|\[x]kanjiskip|の挿入は行われない
(即ち,$\mibox{insert\_skip}\leftarrow\mibox{no\_skip}$となる).


\beginparagraph \IT{np}がkernの場合

前に書いたように,|subtype|が0のkern(TFM由来)や,|subtype|が1であってもjtypeがI, E,~Tのkernは
挿入処理を透過してしまうので,今問題にしているのはそうでない場合である.

\item $\hbox{\tt subtype}=1$の場合.

挿入処理で透過されないのは
jtypeがない(明示的kern)か,jtypeがJ(JFM由来グルー)という2つの場合であるが,
いずれの場合も,\IT{np}の周囲には|\[x]kanjiskip|の挿入は行われない.そのため,次ループのために
行われる処理は,$\mibox{insert\_skip}\leftarrow\mibox{no\_skip}$, 
$\mibox{np}\leftarrow\mibox{next}(\mibox{np})$だけである.

\item $\hbox{\tt subtype}=2$(アクセント由来)の場合.

\IT{np}はリストの先頭から走査されていることから,\IT{np}に続くnodeの並びは
$$\ncount=0
\node{\mibox{np}=\nk\!\!}
\node{\hbox{アクセント文字}}
\node{\nk\!\!}
\node{\hbox{アクセントのつく文字}}
$$
となっている.p\TeX-3.2と同様に,Lua\TeX-jaでは|\[x]kanjiskip|挿入処理で
アクセント文字は無視することにしている.そのため,
\IT{nq}は変化せず,次回のループで処理対象となる\IT{np}は
$\mibox{np}\leftarrow 
\mibox{next\/}(\mibox{next\/}(\mibox{next\/}(\mibox{np})))$となる.
\enditem

\beginparagraph \IT{np}が数式境界 (\IT{math\_node})の場合

数式境界は,便宜的に「文字コードが$-1$である文字」とみなして内部で処理している.

\beginparagraph その他の場合

以上に出てきていないnode(vbox, rule, discretionary break, glue, margin\_kern)の
周囲には|\[x]kanjiskip|の挿入は行われない.

\beginsection JFM由来グルーとの関係の例

最後に,今まで説明した,JFM由来グルーと|\[x]kanjiskip|の処理によって,実際にどのようにnodeの
並びが変わるかをいくつかの例で示す.上付きで$*$がついているnodeは,値が0だと挿入されないことを示す.

\item {\bf 例1: 2つの連続する和文文字の間}
\setbox1=\hbox{|\kanjiskip|}
$$\ncount=0
\vbox{\halign{$#$\hfil\cr
\node{\hbox{和文文字}}\node{\hbox{和文文字}}
\cr
\hfil\Downarrow\cr
\vadjust{\bigskip}
\node{\hbox{和文文字}}
\node{\nk w}_{\rm E}^*
\node{\np P}^*\longrightarrow
\left\{\vcenter{%
\halign{$#$\hfil\cr
\ncount=0\node{{\rm glue/kern}\ g+w}_{\rm J}\cr
\ncount=0\node{\ng \copy1}_{\rm KS}\cr
}}\right\}
\node{\hbox{和文文字}}\cr
}}
$$

\item {\bf 例2: 和文文字と欧文文字の間}
\setbox1=\hbox{|\xkanjiskip|}
$$\ncount=0
\vbox{\halign{$#$\hfil\cr
\node{\hbox{和文文字}}\node{\hbox{欧文文字}}
\cr
\hfil\Downarrow\cr
\vadjust{\bigskip}
\node{\hbox{和文文字}}
\node{\nk w}_{\rm E}^*
\node{\np P}\longrightarrow
\left\{\vcenter{%
\halign{$#$\hfil\cr
\ncount=0\node{{\rm glue/kern}\ g+w}_{\rm J}\cr
\ncount=0\node{\ng \copy1}_{\rm XS}\cr
}}\right\}^*
\node{\hbox{欧文文字}}\cr
}}
$$

\par\vfill\eject
\item {\bf 例3: 2つの和文文字の間にいくつかの「無視される」node達}
\setbox1=\hbox{|\kanjiskip|}
$$\ncount=0
\vbox{\halign{$\relax#$\hfil&$\relax#$\hfil\cr
\node{\hbox{和字}\ A}&
\ncount=1\node{\nk i_A}_{\rm I}
\longrightarrow \cdots
\node{\np p_B}
\node{\hbox{和字}\ B}
\cr
\span\Downarrow\cr
\vadjust{\bigskip}
% T
\node{\hbox{和字}\ A}&
\ncount=1\node{\nk w_A}_{\rm E}^*
\node{\np P_A}_{\rm K}^*
\node{\nk {-w_A}}_{\rm T}^*
\node{\nk i_A}_{\rm I}
\longrightarrow \cdots\cr
&\ncount=1\node{\np p_B}
\node{\np P_B}^*\longrightarrow 
\left\{\vcenter{%
\halign{$#$\hfil\cr
\ncount=0\node{{\rm glue/kern}\ g_B}_{\rm J}\cr
\ncount=0\node{\copy1}_{\rm KS}\cr
}}\right\}
\node{\hbox{和字}\ B}\cr
\span\hbox{or}\cr
% T
\node{\hbox{和字}\ A}&
\ncount=1\node{\nk w_A}_{\rm E}^*
\node{\np P_A}_{\rm K}^*
\node{{\rm glue/kern}\ g_A-w_A}_{\rm J}
\node{\nk i_A}_{\rm I}\cr
&\ncount=1
\longrightarrow \cdots
\node{\np p_B}
\node{\np P_B}
\node{{\rm glue/kern}\ g_B}_{\rm J}^*
\node{\hbox{和字}\ B}\cr
}}
$$
ここで,
\itemitem $w_A$: 和文文字$A$と行末の間に入るkern量.
\itemitem $P_A$: 和文文字$A$の行末禁則用penalty.
\itemitem $g_A$: 和文文字$A$と|'jcharbdd'|との間に入るglue/kern.
\itemitem $P_B$: 和文文字$B$の行頭禁則用penalty.
\itemitem $g_B$: |'jcharbdd'|と和文文字$B$の間に入るglue/kern.
\enditem


\end

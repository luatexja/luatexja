\documentclass{ltjarticle}
\usepackage[twoside,left=23mm,width=170mm,right=17mm,top=25mm,height=231mm,bottom=32mm]{geometry}
\DeclareYokoKanjiEncoding{ZH}{}{}
\DeclareKanjiEncodingDefaults{}{}
\DeclareErrorKanjiFont{ZH}{song}{m}{n}{10}
\DeclareKanjiSubstitution{ZH}{song}{m}{n}
\newcommand\songdefault{song}
\newcommand\heidefault{hei}
\renewcommand\kanjiencodingdefault{ZH}
\renewcommand\kanjifamilydefault{\songdefault}
\renewcommand\kanjiseriesdefault{\mddefault}
\renewcommand\kanjishapedefault{\updefault}
\DeclareKanjiFamily{ZH}{song}{}
\DeclareFontShape{ZH}{song}{m}{n}{<->name:AdobeSongStd-Light:jfm=jis}{}
\DeclareFontShape{ZH}{song}{bx}{n}{<->ssub*hei/m/n}{}
\DeclareKanjiFamily{ZH}{hei}{}
\DeclareFontShape{ZH}{hei}{m}{n}{<->name:AdobeHeitiStd-Regular:jfm=jis}{}
\fontencoding{ZH}\selectfont
\DeclareTextFontCommand{\textsong}{\songfamily}
\DeclareTextFontCommand{\texthei}{\heifamily}
\DeclareOldFontCommand{\song}{\normalfont\songfamily}{}
\DeclareOldFontCommand{\hei}{\normalfont\heifamily}{}
\DeclareSymbolFont{songti}{ZH}{song}{m}{n}
\jfam\symsongti
\SetSymbolFont{songti}{bold}{ZH}{hei}{m}{n}
\DeclareSymbolFontAlphabet{\mathsong}{songti}
\DeclareMathAlphabet{\mathhei}{ZH}{hei}{m}{n}
\makeatletter
\DeclareRobustCommand\songfamily{\not@math@alphabet\songfamily\mathsong\kanjifamily\songdefault\selectfont}
\DeclareRobustCommand\heifamily{\not@math@alphabet\heifamily\mathhei\kanjifamily\heidefault\selectfont}
\DeclareRobustCommand\rmfamily{\not@math@alphabet\rmfamily\mathrm\romanfamily\rmdefault\kanjifamily\songdefault\selectfont}
\DeclareRobustCommand\sffamily{\not@math@alphabet\sffamily\mathsf\romanfamily\sfdefault\kanjifamily\heidefault\selectfont}
\makeatother
\def\LuaTeX{Lua\TeX}
\def\pTeX{p\TeX}
\def\pLaTeX{p\LaTeX}
\begin{document}
\section{\LuaTeX-ja的开发背景}
\LuaTeX-ja是日本开发者北川宏典首倡的一个\LuaTeX的日文支持项目,该项目将\pTeX移植到了\LuaTeX,并做了一定的扩展,删除了一些不正常的特性。

\LuaTeX-ja的项目开发者如下:北川宏典,前田一贵,八登崇之,黑木裕介,阿部纪行,山本宗宏,本田知亮,斋藤修三郎和马起园等。

\LuaTeX-ja现在只能用于plain \TeX格式和\LaTeXe格式,尚不支持\TeX info和Con\TeX t格式。

\LuaTeX-ja现在的代码实现依赖于\verb!luatexbase!以及\verb!luaotfload!等宏包,并且现在仅仅实现了横排,直排组版尚未实现。
\section{一些约定}
在本文档中,有下面一些约定:
\begin{itemize}
\item 所有的CJK字符为\textbf{JAchar},所有的其他字符为\textbf{ALchar}
\item primitive,该词在本文档中不仅表示\LuaTeX的基本控制命令,也包括\LuaTeX-ja的相关的基本控制命令
\item 所有的自然数从0开始
\end{itemize}
\section{安装使用}
\subsection{\LuaTeX版本需求}
\LuaTeX-ja需要使用版本号为大于0.65的\LuaTeX。如果用户使用的是\TeX Live2011或者Mac\TeX 2011以及最新版本的W32\TeX,都可以正常使用\LuaTeX-ja。
\subsection{\LuaTeX-ja宏包安装}
\begin{enumerate}
\item \LuaTeX-ja宏包的获取:
	\begin{itemize}
	\item 如果用户使用git,那么可以使用如下命令获取:\\
		\verb!$ git clone git://git.sourceforge.jp/gitroot/luatex-ja/luatexja.git!
	\item 另外用户还可以直接下载master版本:\\
		\verb!$ wget http://git.sourceforge.jp/view?p=luatex-ja/luatexja.git;a=snapshot;h=HEAD;sf=tgz!
	\end{itemize}
\item 用户需要将获取的\LuaTeX-ja宏包的\verb!src/!放置到你所用的发行版中的\verb!TEXMFLOACL!树下\verb!tex/!文件夹(如果不存在,请建立)下即可,查询该环境变量需要使用:\\
	\verb!$ kpeswhich -expand-var "$TEXMFLOCAL"!
\item 刷新字体数据库:\\
	\verb!$ mkluatexfontdb!
\end{enumerate}
\section{\LuaTeX-ja在plain \TeX格式下的使用}
对于日本用户,日文的排版可以直接使用:
\begin{verbatim}
\input luatexja.sty
\end{verbatim}

这一行会默认调用\verb!luatexja.sty!文件,该文件作用相当于\pTeX中的\verb!ptex.tex!。该文件中设定了两款非嵌入字体:Ryumin-Light(明朝体)和GothicBBB-Medium(哥特体)。
这些字体分别有三种大小:\verb!\tenmin!和\verb!\tengt!(10pt);\verb!\sevenmin!和\verb!\sevengt!(7pt);\verb!\fivemin!和\verb!\fivegt!(5pt)。

有几点请用户注意:
\begin{itemize}
\item 上述的两款字体需要你的pdf阅读器带有良好的非嵌入字体的支持,这里建议使用Adobe Reader,该阅读器使用了Kozuka Mincho字体。
\item 在定义字体的时候,可以使用“级”(Q)单位,$1 Q = 0.25mm$,在\LuaTeX-ja中可以使用\verb!\jQ!命令调用该长度单位
\item 一般情况下,相同大小日文字体比西文字体要大一下(中文字体也是如此),所以定义日文字体的时候需要一个缩放率:0.962216
\item 在\textbf{JAchar}和\textbf{ALchar}之间插入的胶大小为:$(0.25\times0.962216\times\mathrm{10pt})^{\mathrm{+1pt}}_{\mathrm{-1pt}} = 2.40554\mathrm{pt}^{\mathrm{+1pt}}_{\mathrm{-1pt}}$
\end{itemize}
\section{\LuaTeX-ja在\LaTeXe格式的使用}
\subsection{日文排版设置}
在\LaTeXe下使用\LuaTeX-ja比较便利:
\begin{verbatim}
\usepackage{luatexja}
\end{verbatim}

该宏包的作用相当于\pLaTeX中的\verb!plfonts.dtx!和\verb!pldefs.ltx!。
\begin{itemize}
\item 在该宏包中设定了\verb!JY3!编码,这个编码用来调用日文字体
\item 该宏包定义了两个字体族:\verb!mc!和\verb!gt!。\\
\begin{center}
	\begin{tabular}{ccccc}
	\hline
	字体&字体族&\verb!\mdseries!&\verb!\bfseries!&缩放率\\
	\hline
	\textit{mincho}&\verb!mc!&Ryumin-Light&GothicBBB-Medium&0.962216\\
	\textit{gothic}&\verb!gt!&GothicBBB-Medium&GothicBBB-Medium&0.962216\\
	\hline
	\end{tabular}
\end{center}
\item 在数学模式下,所有的字符使用\verb!mc!字体族来排印
%中文的C\TeX和xeCJK都没有进行此项设定,导致不能在数学模式下输入中文
\end{itemize}

上述使用的宏包,只满足了最小的日文环境设定需求,并为满足所有的日文排版需求。
\LuaTeX-ja宏包提供了两个文档类:\verb!jclasses!(\pLaTeX标准文档类)和\verb!jsclasses!(奥村晴彦)。用户可自行选择两种文档类。

\subsection{中文排版设定}
请先下载\verb!zh-classes!:
\begin{verbatim}
$ wget http://fandol-doc.googlecode.com/files/zh-classes.tar.xz
\end{verbatim}

该文档类是基于\verb!jclasses!文档的中文设定版。
这个文档类需要用户在本地安装两款字体:AdobeSongStd-Light.otf和AdobeHeitiStd-Regular.otf。
这两款字体在你安装的中文版Adobe Reader的文件夹下。这两款字体设定如下:
\begin{center}
	\begin{tabular}{cccc}
	\hline
	字体&字体族&\verb!\mdseries!&\verb!\bfseries!\\
	\hline
	\textit{song}&\verb!song!&AdobeSongStd-Light&AdobeHeitiStd-Regular\\
	\textit{hei}&\verb!hei!&AdobeHeitiStd-Regular&AdobeHeitiStd-Regular\\
	\hline
	\end{tabular}
\end{center}

例如:
\begin{verbatim}
\documentclass{zh-article}
\title{Lua\TeX测试}
\author{某人甲}
\begin{document}
\maketitle
Lua\TeX-ja中文测试。{\hei 测试}
\end{document}
\end{verbatim}
\section{\LuaTeX-ja参数设定}
\LuaTeX-ja包含大量的参数,以控制排版细节。
设定这些参数需要使用命令:\verb!\ltjsetparameter!和\verb!\ltjgetparameter!命令。
\subsection{JAchar范围的设定}
在设定\textbf{JAchar}之前,需要分配一个小于217的自然数。如:
\begin{verbatim}
\ltjdefcharrange{100}{"10000-"1FFFF,`漢}
\end{verbatim}

请注意这个设定是全局性的,不建议在文档正文中进行设定。

在范围设定好了之后,需要进行\verb!jacharrange!的设定:
\begin{verbatim}
\ltjsetparameter{jacharrange={-1, +2, +3, -4, -5, +6, +7, +8}}
\end{verbatim}

这里定义了8个范围,在每个范围之前使用“+”或“-”进行设定,其中如果为$-$,则代表该范围为\textbf{ALchar},如果为$+$,则该范围视作\textbf{JAchar}。

\LuaTeX-ja默认设定了8个范围,这些范围来源于下列数据:
\begin{itemize}
\item Unicode 6.0
\item Adobe-Japan1-6与Unicode之间的映射\verb!Adobe-Japan1-UCS2!
\item 八登崇之的up\TeX宏包:\verb!PXbase!
\end{itemize}

\begin{description}
\item[范围 $\mathbf{8^J}$]
\begin{itemize}
\def\ch#1#2{\item \char"#1\ (\texttt{U+00#1}, #2)}
\ch{A7}{Section Sign}
\ch{A8}{Diaeresis}
\ch{B0}{Degree sign}
\ch{B1}{Plus-minus sign}
\ch{B4}{Spacing acute}
\ch{B6}{Paragraph sign}
\ch{D7}{Multiplication sign}
\ch{F7}{Division Sign}
\end{itemize}
\item[范围 $\mathbf{1^A}$]
	\begin{itemize}
	\item \texttt{U+0080}--\texttt{U+00FF}: Latin-1 Supplement
	\item \texttt{U+0100}--\texttt{U+017F}: Latin Extended-A
	\item \texttt{U+0180}--\texttt{U+024F}: Latin Extended-B
	\item \texttt{U+0250}--\texttt{U+02AF}: IPA Extensions
	\item \texttt{U+02B0}--\texttt{U+02FF}: Spacing Modifier Letters
	\item \texttt{U+0300}--\texttt{U+036F}: Combining Diacritical Marks
	\item \texttt{U+1E00}--\texttt{U+1EFF}: Latin Extended Additional
\end{itemize}
\item[范围 $\mathbf{2^J}$]
	\begin{itemize}
	\item \texttt{U+0370}--\texttt{U+03FF}: Greek and Coptic
	\item \texttt{U+0400}--\texttt{U+04FF}: Cyrillic	
	\item \texttt{U+1F00}--\texttt{U+1FFF}: Greek Extended
	\end{itemize}
\end{description}
\section{Lua\TeX-ja与Lua\TeX相关阅读材料}
\begin{itemize}
\item Lua\TeX官方主页:\verb!http://www.luatex.org!
\item Lua\TeX\ SVN:\verb!http://foundry.supelec.fr/gf/project/luatex/!
\item Lua\TeX:\verb!http://ja.wikipedia.org/wiki/LuaTeX!
\item Lua\TeX-ja官方主页:\verb!http://en.sourceforge.jp/projects/luatex-ja/!
\item p\TeX官方主页:\verb!http://ascii.asciimw.jp/pb/ptex/!
\item Publishing \TeX:\verb!http://ja.wikipedia.org/wiki/PTeX!
\item Vertical typesetting in \TeX:\verb!http://tug.org//TUGboat/Articles/tb11-3/tb29hamano.pdf!
\item up\TeX官方主页:\verb!http://homepage3.nifty.com/ttk/comp/tex/uptex.html!
\end{itemize}
\end{document}

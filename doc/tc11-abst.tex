%#!lualatex
\documentclass[a4paper]{bxjsarticle}
\setpagelayout*{left=20truemm, right=20truemm,top=15truemm,bottom=20truemm}
\pagestyle{empty}
\usepackage{luatextra,booktabs,amsmath,lmodern}
\usepackage[unicode=true]{hyperref}
\usepackage{luatexja}

\def\title{\LuaTeX-jaの開発}
\def\author{北川 弘典}
\def\mail{h\_kitagawa2001@yahoo.co.jp}
\hypersetup{pdftitle=\title, pdfauthor=\author}

\makeatletter
\baselineskip=15pt
\DeclareFontShape{JY3}{mc}{m}{n}{<-> s*[0.92489] psft:Ryumin-Light:jfm=ujis}{}
\DeclareFontShape{JY3}{gt}{m}{n}{<-> s*[0.92489] psft:GothicBBB-Medium:jfm=ujis}{}

  \renewcommand{\section}{%
    \@startsection{section}{1}{\z@}%
    {\Cvs}% 前アキ
    {.25\Cvs}% 後アキ
    {\normalfont\large\headfont\raggedright}}
  \renewcommand{\paragraph}{\@startsection{paragraph}{4}{\z@}%
    {0.5\Cvs}{-1\zw}% 改行せず 1zw のアキ
    {\normalfont\normalsize\headfont ■}}
\let\subsection=\paragraph

\def\@listI{\leftmargin\leftmargini
  \parsep \z@\topsep\z@\itemsep\z@}
\@listI
\def\headfont{\normalfont\bfseries\mathversion{bold}}
\def\emph#1{\textbf{\mathversion{bold}#1}}
\def\centerbaseline#1#2{%
  \setbox0=\hbox{#1\ltjsetparameter{yjabaselineshift=\z@}あ}%
  \@tempdima=\dimexpr\cht-\cdp-\ht0+\dp0\relax{#1%
  \ltjsetparameter{yjabaselineshift=-0.5\@tempdima, yalbaselineshift=-0.5\@tempdima}#2%
  }}
\def\.#1{{\normalfont\it$\langle$#1$\rangle$}}
\def\:#1{{\normalfont\tt\char92 #1}}
\begin{document}
\begin{center}
\LARGE\bfseries\title
\end{center}
\smallskip
\begin{flushright}
\author~({\tt\mail})
\end{flushright}


\section{開発目標}
Lua\TeX-ja は,p\TeX と同等あるいはそれ以上の水準の日本語組版を,
次世代標準\TeX エンジンであるLua\TeX で可能にすることを目的としたマクロパッケージである.
\begin{itemize}
\item \emph{最低でもp\TeX と同等の組版の自由度を確保する.}
\item \emph{p\TeX との100\%互換は目的としない.}
p\TeX において不自然/不都合な実装があれば,積極的に改める.
\end{itemize}


\section{p\TeX からの主な違い}
\subsection{縦書きは未実装}
落ち着いてきたら縦書きも開発したいが,現状は左横書きのみサポート.

\subsection{命令名称の変更}
大半のパラメタへの代入は\:{ltjsetparameter}へ
{\tt \.{key}=\.{value}}の形で渡す.

例:行頭禁則用ペナルティ({\sf prebreakpenalty})
\begin{center}\medskip\small%
\begin{tabular}{lll}
\toprule
&p\TeX&Lua\TeX-ja\\\midrule
代入&\:{prebreakpenalty}\.{chr}{\tt =}\.{pena}&
\verb+\ltjsetparameter{prebreakpenalty={+\.{chr}{\tt,}\.{pena}\verb+}}+\\
取得&\:{prebreakpenalty}\.{chr}(count)&
\verb+\ltjgetparameter{prebreakpenalty}{+\.{chr}\verb+}+(string)\\
\bottomrule
\end{tabular}
\end{center}
\ 

\subsection{行末が和文文字の場合の改行の扱い}
Lua\TeX の仕様上,「前行行末時のcatcode」で判定するad hocな仕様.
%1行を外部から読み込むとき,前行行末のcatcodeの状態で,行の末尾が次の形のときに改行を無視する:
%\[
% \hbox{(catcodeが11 or 12の和文文字)}\hbox{(catcodeが1 or 2の文字)}^*
%\]

\subsection{和文間・和欧文間の空白挿入処理}
\begin{enumerate}
\item 空白挿入処理をノードベースに変更(Lua\TeX の合字・カーニング処理に合わせた).
\item 「和文フォント」はメトリックと実際の字形との組:
\begin{center}
$\tt\mathcode`\:="703A\mathcode`\*="7020
\mathcode`\=="703D\mathcode`\-="702D\mathcode`\|="705C%"
\mathcode`\<="707B\mathcode`\>="707D
|jfont|tenmin=\underbrace{\tt psft:Ryumin-Light}_{\text{PostScriptフォント(非埋込)}}:%
\underbrace{\tt jfm=ujis}_{\text{メトリック}}*at*13.5|jQ
$\smallskip
\end{center}
空白挿入処理では,メトリックとサイズの同じ和文フォントは同一視される.

\item 異なるメトリック・サイズの2つの和文文字の間には,両メトリックから決まる空白の平均値が入る.
\end{enumerate}

\begin{center}\medskip\small%
\begin{tabular}{ccc}
\toprule
\textbf{入力}&\textbf{p\TeX}&\textbf{Lua\TeX-ja}\\\midrule
$
\tt\mathcode`\:="703A\mathcode`\*="7020
\mathcode`\=="703D\mathcode`\-="702D\mathcode`\|="705C%"
\mathcode`\<="707B\mathcode`\>="707D
あ\overbrace{\tt)<|gt(}^{\text{2.}}い
\overbrace{\tt)<>(}^{\text{1.}}>う\overbrace{\tt)<|Large(}^{\text{3.}}え>
$
&\large あ)\hbox{}{\gt(い)\hbox{}(}う)\hbox{}{\Large(え}&
\large あ){\gt(い){}(}う){\Large(え}\\
\bottomrule
\end{tabular}
\end{center}
\

\section{現況}
\subsection{\inhibitglue 「エンジン拡張部分」%
\centerbaseline{\scriptsize\normalfont}{(\TeX →p\TeX のエンジン拡張部分
に対応)}}\ \par概ね実装済みであるが,細かい仕様変更の可能性あり.また,
従来ではprimitiveとして実装していた機能をLuaコードと\TeX マクロで実装しな
いといけないので,バグが残っている可能性がある.

\subsection{plain \TeX~formatに対するマクロ%
\centerbaseline{\scriptsize\normalfont}{({\tt ptex.tex}に相当)}}
ほぼ翻訳完了.

\subsection{\LaTeXe 用マクロ%
\centerbaseline{\scriptsize\normalfont}{(p\LaTeXe 実装に相当)}%
及び,{\tt fontspec}, {\tt otf}パッケージ対応}\ \par
試験的に一部の機能が実装され,ある程度は使える.
日本語用クラスファイルとしては,八登さんによるBXjsclsを使用するのが現状では手っ取り早い.



\vfill
\subsection{Lua\TeX-ja プロジェクトについて}\ 

プロジェクト Wiki:\url{http://sourceforge.jp/projects/luatex-ja/wiki/}

開発メンバー:北川 弘典,前田 一貴,八登 崇之,黒木 裕介,阿部 紀行,本田 知亮,山本 宗宏
\end{document}
%#!lualatex ajt-devel-ltja
\documentclass{ajt}

%%% Packages used in this paper

%%% Font setting for \LuaTeX; this is extract from ajt.cls
\makeatletter
  \if@print
    \RequirePackage{fontspec,xunicode}
    \RequirePackage{luatextra}
    \setmainfont[Mapping=tex-text]{Palatino LT Std}
    \setsansfont[Mapping=tex-text]{Optima LT Std}
  \else
    \RequirePackage{fontspec,luatextra}
    \setmainfont[Mapping=tex-text]{TeX Gyre Pagella} % \simeq Palatino
  \fi

%%% LuaTeX-ja
\usepackage{luatexja,luatexja-fontspec}
\ltjsetparameter{jacharrange={-3,-8}}
\DeclareFontShape{JY3}{mc}{m}{n}{<-> s*[0.92489] file:ipam.ttf:jfm=ujis}{}
\DeclareFontShape{JY3}{gt}{m}{n}{<-> s*[0.92489] file:ipag.ttf:jfm=ujis}{}
% quick hack: monospaced Japanese font by \ttfamily
\DeclareKanjiFamily{JY3}{\ttdefault}{}{}
\DeclareFontShape{JY3}{\ttdefault}{m}{n}{<-> s*[0.92489] file:ipag.ttf:jfm=mono}{}


%%% LTXexample environment
\usepackage{showexpl,lltjlisting}
\lstset{basicstyle=\ttfamily\small, width=0.3\textwidth,  basewidth=.5em}

%%% Verbatim environment
\usepackage{fancyvrb}
\CustomVerbatimEnvironment{code}{Verbatim}%
{numbers=left,xleftmargin=1.5em,baselinestretch=1.069,fontsize=\small}
\CustomVerbatimEnvironment{codewithoutnum}{Verbatim}%
{xleftmargin=1.5em,baselinestretch=1.069,fontsize=\small}
\CustomVerbatimEnvironment{codewithoutnumsmall}{Verbatim}%
{xleftmargin=1.5em,baselinestretch=1.0,fontsize=\footnotesize}
\DefineShortVerb{\|}

%%% Others
\usepackage{mflogo,booktabs}
\definecolor{grayx}{gray}{0.85}

%%% Mandatory article metadata %%%
\title{Development of the \LuaTeX-ja package}
\author{Hironori Kitagawa {\normalsize 北川 弘典}}
\address{The \LuaTeX-ja project team}
\email{h\_kitagawa2001@yahoo.co.jp}

\keywords{\TeX, p\TeX, \LuaTeX, \LuaTeX-ja, Japanese}
\abstract{%
The \LuaTeX-ja package is a macro package for typesetting Japanese
documents under \LuaTeX.  This packages has more flexibility of
typesetting than p\TeX, and corrected some unwanted features of p\TeX.
In this paper, we describe specifications, the current status and some
internal processing methods of \LuaTeX-ja.
}

\newcommand{\parname}[1]{\textsf{#1}}
\newcommand{\jstrut}{\vrule width0pt height\cht depth\cdp}
\newcommand{\imagfm}[1]{\ifvmode\leavevmode\fi%
  \hbox{\fboxsep=0pt\fbox{\setbox0=\hbox{#1}\copy0\kern-\wd0
  \smash{\vrule width \wd0 height 0.4pt depth0.4pt}}}}
\begin{document}

%%% Do not forget to start with \maketitle!
\maketitle

\section{Introduction}
\subsection{History}
To typeset Japanese documents with \TeX, ASCII p\TeX~\cite{ptex} has
been widely used in Japan.  There are other methods---for example, using
Omega and OTP~\cite{omega}, or with the CJK package---to do so, however,
these alternative methods did not become a majority.  The author thinks
that this is because p\TeX\ enables us to produce high-quality documents
(e.g.,~supporting vertical typesetting), and the appearance of p\TeX\ is
earlier than alternatives described above.

However, p\TeX\ has been left behind from the extensions of \TeX\
such as \eTeX\ and \pdfTeX, and the diffusion of UTF-8 encoding.  In
recent years, the situation become better, because of development
of |ptexenc|~\cite{ptexenc} by Nobuyuki Tsuchimura (\hbox{土村展之}),
$\varepsilon$-p\TeX~\cite{eptex} by the author,~and up\TeX~\cite{uptex}
by Takuji Tanaka (田中琢爾). However, continuing this approach, namely, to develop
an engine extension localized for Japanese, is not wise. This approach
needs lots of work for \emph{each} engine, and since \LuaTeX\ has an ability
to hook \TeX's internal process by using Lua callbacks, the necessity of
an engine extension is getting smaller.


There were several experimental attempts to typeset
Japanese documents with \LuaTeX\ before. Here we cite three examples:
\begin{itemize}
\item |luaums.sty|~\cite{luaums} developed by the author. This
      experimental package is for creating a certain Japanese-based presentation
      with \LuaTeX.
\item the \emph{luajalayout} package~\cite{luajalayout}, formerly known as the
      \emph{jafontspec} package, by Kazuki Maeda (前田一貴). This package is based on
      \LaTeXe\ and \emph{fontspec} package.
\item the \emph{luajp-test} package~\cite{luajp-test}, a test package made by
      Atsuhito Kohda (香田温人), based on articles on the web page~\cite{joylua}.
\end{itemize}
However, these packages are based on \LaTeXe, and do not have much
ability to control the typesetting rule. And it is inefficient that more
than one people separately develop similar packages.  Development of the
\LuaTeX-ja package is started initially by the author and Kazuki Maeda, because of
these situations.

\subsection{Development Policy of \LuaTeX-ja}
\label{ssec-pol} 
The first aim of the \LuaTeX-ja project is to implement features (from the
'primitive' level) of p\TeX\ as macros under \LuaTeX, so \LuaTeX-ja is
much affected by p\TeX.  However, as development proceeds, some
technical/conceptual difficulties are arisen. Hence we changed the aim
of the project as follows:
\begin{itemize}
\item\emph{\LuaTeX-ja offers at least the same flexibility of
     typesetting that p\TeX\ has.}

     We think that the ability of producing outputs conformed to
     JIS~X~4051~\cite{jisx4051}, the Japanese Industrial Standard for
     typesetting, or to a technical note~\cite{w3c} by W3C is not enough;
     if one wants to produce very incoherent outputs for some reason, it
     should be possible.
In this point, previous attempts of Japanese typesetting with \LuaTeX\
     which we cited in the previous subsection are inadequate.

p\TeX\ has some flexibility of typesetting, by changing internal
     parameters such as |\kanjiskip| or |\prebreakpenalty|, and by using
     custom JFM (Japanese TFM). Therefore we decided to include these
     functionality to \LuaTeX-ja.

\item\emph{\LuaTeX-ja isn't mere re-implementation or porting of p\TeX;
     some (technically and/or conceptually) inconvenient features of
     p\TeX\ are modified.} 

     We describe this point in more detail at the next section.
\end{itemize}


\subsection{Overview of the Processes}
\label{ssec-over}
We describe  an outline of the \LuaTeX-ja's process in order.
\begin{itemize}
\item In the |process_input_buffer| callback: treatment of breaking
      lines after a Japanese character (in Subsection~\ref{ssec-line}).

\item In the |hyphenate| callback: font replacement.

\LuaTeX-ja looks into for each \textit{glyph\_node}~$p$ in the list. If
	   the character represented by $p$ is considered as a Japanese
	   character, the font used in $p$ is replaced by the value of
	   |\ltj@curjfnt|, an attribute for `the current Japanese font'
	   at~$p$.

Furthermore the subtype of $p$ is subtracted by 1 to suppress
	   hyphenation around it by \LuaTeX, because later processes of
	   \LuaTeX-ja take care of all things about Japanese characters.

\item In |pre_linebreak_filter| and |hpack_filter| callbacks:

\begin{enumerate}
\item \LuaTeX-ja has its own stack system, and the current horizontal
      list is traversed in this stage to determine what is the level of
      \LuaTeX-ja's internal stack at the end of the list (in
      Subsection~\ref{ssec-stack}).

\item In this stage, \LuaTeX-ja inserts glues/kerns for Japanese
      typesetting in the list. This is the core of \LuaTeX-ja (in
      Subsection~\ref{ssec-jglue}).

\item To make a match between a metric and a real font, sometimes
      adjustument of the position of (Japanese) glyphs are performed
      (Subsection~\ref{ssec-width}).
\end{enumerate}
\item In the |mlist_to_hlist| callback: replacement of Japanese characters in math formulas.
This stage is similar to adjustument of the position of glyphs (see
      above), so we omit it from this paper.
\end{itemize}

\subsection{Contents of this Paper}
Here we describe the contents of the rest of this paper briefly.  In
Section~2, we describe major differences between p\TeX\ and \LuaTeX-ja.
The next section, Section~3, is concentrated on a problem `how we
distinguish between Japanese characters and alphabetic characters'. In
Section~4, we show rest of features of the \LuaTeX-ja package, and
current status of the package.  Finally, in Section~5, we describe some
internal routines of \LuaTeX-ja.

\subsection*{About the Project}
This \LuaTeX-ja project is hosted by SourceForge.jp. The official wiki
is located on
\url{http://sourceforge.jp/projects/luatex-ja/wiki/FrontPage}.  There is
no stable version at Oct.\ 15, 2011, however the latest developer sources can be
obtained from the git repository.  Members of the project team are as follows
(in random order): Hironori Kitagawa, Kazuki Maeda, Takayuki Yato,
Yusuke Kuroki, Noriyuki Abe, Munehiro Yamamoto, Tomoaki Honda,
and~Shuzaburo Saito.


\section{Major differences with \pTeX}
In this section, we explain several major differences between p\TeX\
and our \LuaTeX-ja.  For general information of Japanese typesetting and the
overview of p\TeX, please see Okumura~\cite{ptexjp}.


\subsection{Names of Control Sequences}
\label{ssec-csname} Because p\TeX\ is an engine modification of Knuth's
original \TeX82 engine, some primitives added by it take a form that is
very difficult to be simulated by a macro.  For example, an additional
primitive |\prebreakpenalty|$\langle\hbox{\it
char\_code}\rangle$|[=]|$\langle\hbox{\it penalty}\rangle$ in p\TeX\
sets the amount of penalty inserted before a character whose code is
$\langle\hbox{\it char\_code}\rangle$ to $\langle\hbox{\it
penalty}\rangle$, and this form |\prebreakpenalty|$\langle\hbox{\it
char\_code}\rangle$ can be also used for retrieving the value.

Moreover, there are some parameters which values of them at the end of a
horizontal box or that of a paragraph are effective in whole box or
paragraph.  These parameters were implemented as additional internal
parameters in \pTeX. However, the implementation of these parameters in
\LuaTeX-ja is not so easy; we will discuss on it in
Subsection~\ref{ssec-stack}.

From above 2~problems we discussed above, the assignment and retrieval
of most parameters in \LuaTeX-ja are summarized into the following
3~control sequences:
\begin{itemize}
\item |\ltjsetparameter{|$\langle\hbox{\it
      name}\rangle$|=|$\langle\hbox{\it value}\rangle$|,...}|: for local
      assignment.
\item |\ltjglobalsetparameter|: for global assignment. These two control
      sequences obey the value of |\globaldefs| primitive.
\item |\ltjgetparameter{|$\langle\hbox{\it
      name}\rangle$|}[{|$\langle\hbox{\it optional
      argument}\rangle$|}]|: for retrieval. The returned value is always
      a string.
\end{itemize}

\subsection{Line-break after a Japanese Character}
\label{ssec-line} 

Japanese texts can break lines almost everywhere, in contrast with
alphabetic texts can break lines only between words (or use
hyphenation). Hence, p\TeX's input processor is modified so that a
line-break after a Japanese character doesn't emit a space. However,
there is no way to customize the input processor of \LuaTeX, other than
to hack its CWEB-source. All a macro package can do is to modify an input line before
when \LuaTeX\ begin to process it, inside the |process_input_buffer|
callback.

Hence, in \LuaTeX-ja, a comment letter (we reserve U+FFFFF for this
purpose) will be appended to an input line, if this line ends with a Japanese
character\footnote{Strictly speaking, it also requires that the catcode
of the end-line character is 5~(\emph{end-of-line}). This condition is
useful under the verbatim environment.}. One might jump to a conclusion
that the treatment of a line break by p\TeX\ and that of \LuaTeX-ja are
totally same, however they are different in the respect that \LuaTeX-ja's
judgement whether a comment letter will be appended the line is done
\emph{before} the line is actually processed by \LuaTeX.

Figure~\ref{fig-linebreak} shows an example of this situation; the
command at the first line marks most of Japanese characters as
`non-Japanese characters'. In other words, from that command onward, the
letter `あ' will be treated as an alphabetic character by
\LuaTeX-ja. Then, it is natural to have a space between `あ' and `y' in
the output, where the actual output in the figure does not so.  This is
because `あ' is considered a Japanese character by \LuaTeX-ja,
when \LuaTeX-ja does a decision whether U+FFFFF will be added to the
input line~2.

\begin{figure}
\begin{LTXexample}
\font\x=IPAMincho \x
\ltjsetparameter{jacharrange={-6}}xあ
y
\end{LTXexample}
\caption{A notable sample showing the treatment of a line break after a
Japanese character.}\label{fig-linebreak}
\end{figure}

\subsection{Separation between `real' fonts and Metrics}
\label{ssec-sepmet}

Traditionally, most Japanese fonts used in typesetting are not
proportional, that is, most glyphs have same size (in most cases,
square-shaped). Hence, it is not rare that the contents of different
JFMs are essentially same, and only differ in their names. For example,
|min10.tfm| and |goth10.tfm|, which are JFMs shipped with p\TeX\ for
seriffed \emph{mincho} family and sans-seriffed \emph{gothic} family,
differ their |FAMILY| and |FACE| only. Moreover, |jis.tfm| and
|jisg.tfm|, which consists a parts of \emph{jis} font metric, which is
used in \emph{jsclasses}~\cite{jsclasses} by Haruhiko Okumura (奥村晴彦),
are totally same as binary files.  Considering this situation, we
decided to separate `real' fonts and metrics used for them in
\LuaTeX-ja. Typical declarations of Japanese fonts in the style of plain
\TeX\ are shown in Figure~\ref{fig-jfdef}. We would like to add several
remarks:
\begin{itemize}
\item A control sequence |\jfont| must be used for Japanese fonts, instead of |\font|.
\item \LuaTeX-ja automatically loads the \emph{luaotfload} package, so
      |file:| and |name:| prefixes, and various font features can be
      used as the line~1 in Figure~\ref{fig-jfdef}.
\item The |jfm| key specifies the metric for the font. In
      Figure~\ref{fig-jfdef}, both fonts will use a metric stored in a
      Lua script named |jfm-ujis.lua|. This metric is the standard
      metric in \LuaTeX-ja, and is based on JFMs used in the \emph{otf}
      package~\cite{otf}.
\item The |psft:| prefix can be used to specify name-only, non-embedded
      fonts. When one display a pdf with these fonts, actual fonts which
      will be used for them depend on a pdf reader. 
\end{itemize}
The specification of a metric for \LuaTeX-ja is similar to that of a JFM
(see \cite{ptexjp}); characters are grouped into several classes, the
size information of characters are specified for each class, and
glue/kern insertions are specified for each pair of classes. Although
the author have not tried, it may be possible to develop a program that
`converts' a JFM to a metric for \LuaTeX-ja.  \LuaTeX-ja offers three
metrics by default; |jfm-ujis.lua|, |jfm-jis.lua| based on the
\emph{jis} font metric, and |jfm-min.lua| based on old |min10.tfm|.

 Note that |-kern| in features
is important, because kerning information from real font itself will
clash with glue/kern informations from the metric.

\begin{figure}
\begin{verbatim}
\jfont\foo=file:ipam.ttf:jfm=ujis;script=latn;-kern;+jp04 at 12pt
\jfont\bar=psft:Ryumin-Light:jfm=ujis at 10pt
\end{verbatim}
\caption{Typical declarations of Japanese fonts.}
\label{fig-jfdef}
\end{figure}

\subsection{Insertion of Kerns and/or Glues for Japanese Typesetting: the Timing}
\label{ssec-jglue}

As described in \cite{luatexref}, \LuaTeX's kerning and ligaturing
processes are totally different from those of \TeX82.  \TeX82's process is
done just when a (sequence of) character is appended to the current
list. Thus we can interrupt this process by writing as
|f{}irm|. However, \LuaTeX's process is \emph{node-based}, that is, the
process will be done when a horizontal box or a paragraph is ended, so
|f{}irm| and |firm| yield  same outputs under \LuaTeX.

The situation for Japanese characters is more complicated.
Glues (and kerns) which are needed for Japanese
typesetting will be divided into the following three categories:
\begin{itemize}
\item Glue (or kern) from the metric of Japanese fonts (\emph{JFM glue},
      for short). 

\item Default glue between a Japanese character and an alphabetic
      character (\emph{xkanjiskip}, for short), usually 1/4 of
      full-width (\emph{shibuaki}) with some stretch and shrink for
      justifying each line.
\item Default glue between two consecutive Japanese characters
      (\emph{kanjiskip}, for short). The main reason of this glue is to
      enable breaking lines almost everywhere in Japanese texts. In most
      cases, its natural width is zero, and some stretch/shrink for
      justifying each line.
\end{itemize}
In p\TeX, these three kinds of glues are treated differently. A JFM glue
is inserted when a (sequence of) Japanese character is appended to the
current list, same as the case of alphabetic characters in \TeX82. This
means that one can interrupt the insertion process by saying |{}|.  A
\emph{xkanjiskip} is inserted just before `hpack' or line-breaking of a
paragraph; this timing is somewhat similar to that of \LuaTeX's kerning
process. Finally, A \emph{kanjiskip} is not appeared as a node anywhere;
only appears implicitly in calculation of the width of a horizontal box,
that of breaking lines, and the actual output process to a DVI
file. These specifications made p\TeX's behavior very hard to
understand.

\LuaTeX-ja inserts glues in all three categories simultaneously inside
|hpack_filter| and |pre_linebreak_filter| callbacks.  The reasons of
this specification are to behave like alphabetic characters in \LuaTeX\
(as described in the first paragraph), and to clarify the specification
for \LuaTeX-ja's process.

\subsection{Insertion of Kerns and/or Glues for Japanese Typesetting: the Spec}
\begin{table}
\caption{Examples of differences between  p\TeX\ and \LuaTeX-ja,}
\label{tab-jfmglue}
\begin{center}
\begin{tabular}{llllllll}
\toprule
&\multicolumn{1}{c}{(1)}&\multicolumn{1}{c}{(2)}&\multicolumn{1}{c}{(3)}&\multicolumn{1}{c}{(4)}\\
Input      &|あ】{}【〙\/〘|        &|い』\/a| &|う)\hbox{}(| &|え]\special{}[|\\\midrule
p\TeX      &あ】\hbox{}【〙\hbox{}〘&い』\/a   &う)\hbox{}(   &え]\hbox{}[\\
\LuaTeX-ja &あ】{}【〙\/〘          &い』\/a   &う)\hbox{}(   &え]\special{}[\\
\bottomrule
\end{tabular}
\end{center}
\end{table}

\begin{figure}
\begin{center}
\fontsize{40}{40}\selectfont
\imagfm{\jstrut あ}%
\imagfm{\jstrut 】\inhibitglue}%
\imagfm{\jstrut\kern.5\zw}%
\imagfm{\jstrut\kern.5\zw}%
\imagfm{\jstrut\inhibitglue【}%
\imagfm{\jstrut 〙\inhibitglue}%
\imagfm{\jstrut\kern.5\zw}%
\imagfm{\jstrut\kern.5\zw}%
\imagfm{\jstrut\inhibitglue〘}%
\end{center}
\caption{Detail of (1) in Table~\ref{tab-jfmglue}.}
\label{fig-ptexjfm}
\end{figure}

Now we will take a look inside the insertion process itself, and describe 4~points.

\begin{description}
\item[Ignored Nodes]
As noted in the previous subsection, the insertion process in p\TeX\ can
	   be interrupted by saying |{}| or anything else\footnote{This
	   is why some tricks like \texttt{ちょ\char`\{\char`\}っと} for
	   \texttt{min10.tfm} and other `old' JFMs work.}. This leads
	   the second row in Table~\ref{tab-jfmglue}, or
	   Figure~\ref{fig-ptexjfm}. `The process is interrupted' means
	   that p\TeX\ does not think the letter `】\inhibitglue' is
	   followed by `\inhibitglue【', hence two half-width glues are
	   inserted between between `】\inhibitglue' and `\inhibitglue【',
	   where one is from `】\inhibitglue' and another is from
	   `\inhibitglue【'.

	   On the other hand, in \LuaTeX-ja, the process is done inside
	   |hpack_filter| and |pre_linebreak_filter| callbacks. Hence,
	   \emph{anything that does not make any node will be
	   ignored}\ in \LuaTeX-ja, as shown in (1) in
	   Table~\ref{tab-jfmglue}. \LuaTeX-ja also ignores any nodes
	   which does not make any contribution to current horizontal
	   list---\emph{ins\_node}, \emph{adjust\_node},
	   \emph{mark\_node}, \emph{whatsit\_node} and
	   \emph{penalty\_node}---, as shown in (4).


By the way, around a \emph{glyph\_node} $p$ there may be some nld odes
	   attached to $p$. These are an accent and kerns for
	   positioning it, and a kern from the italic
	   correction\footnote{\TeX82 (and \LuaTeX) does not distinguish
	   between explicit kern and a kern for italic correction. To
	   distinguish them, an additional subtype for kern is introduced
	   in p\TeX. On the other hand, \LuaTeX-ja uses an additional attribute and
	   redefines \texttt{\char`\\/}.} for $p$. It is natural that
	   these attachments should be ignored inside the process. Hence
	   \LuaTeX-ja takes this approach, as the latest version of
	   p\TeX\ (p3.2). This explains (2) in the figure.

Summerizing above, one should put an empty horizontal box |\hbox{}| to
	   where he wants to interrupt the insertion process in
	   \LuaTeX-ja as (3) in the figure.

\item[Fonts with the Same Metric]
Recall that \LuaTeX-ja separated `real' fonts and metrics, as in Subsection~\ref{ssec-sepmet}. 
Consider the following input, where all Japanese fonts use same metric
	   (in \LuaTeX-ja), and |\gt| selects \emph{gothic} family for
	   the current Japanese font family:
\begin{quote}
\begin{verbatim}
明朝)\gt (ゴシック
\end{verbatim}
\end{quote}
If the above input is processed by p\TeX, because the insertion process is
	   interrupt by |\gt|, the result looks like
\begin{quote}
\mc 明朝)\hbox{}\gt (ゴシック
\end{quote}
However this seems to be unnatural, since two Japanese fonts in the
	   output use the same metric, i.e.,~the same
	   typesetting rule.  Hence, we decided that Japanese fonts with
	   the same metric are treated as one font in the insertion
	   process of \LuaTeX-ja. Thus, the output from the above input
	   in \LuaTeX-ja looks like:
\begin{quote}
\mc 明朝)\gt (ゴシック
\end{quote}
One might have the situation that this default behavior is not
	   suitable. \LuaTeX-ja offers a way to cope with this case, but
	   we leave it to the manual~\cite{man}.

\item[Fonts with Different Metrics] 
In the case where two consecutive Japanese characters use different metrics and/or
	   different size is similar. Consider the following input where
	   the \emph{mincho} family and the \emph{gothic} family use
	   different metrics:
\begin{quote}
\begin{verbatim}
漢)\gt (漢)\large (大
\end{verbatim}
\end{quote}
As the previous paragraph, this input yields the following, by p\TeX:
\begin{quote}
\mc 漢)\hbox{}\gt (漢)\hbox{}\large (大
\end{quote}
We thought that amounts of spaces between parentheses in above output
	   are too much. So we changed the default behavior of
	   \LuaTeX-ja so that the amount of a glue between two Japanese
	   characters with different metrics is the average of a glue
	   from the left character and that from the right
	   character. For example, Figure~\ref{fig-diffmet} shows the
	   output from above input. The width of glue indicated `(1)' is
	   $(a/2 + a/2)/2 = 0.5a$, and the width of glue indicated `(2)'
	   is $(a/2 + 1.2a/2)/2 = 0.55a$. This default behavior can be
	   changed by \textsf{diffrentmet} parameter of \LuaTeX-ja.

\begin{figure}
\begin{center}
\fontsize{40}{40}\selectfont
\imagfm{\jstrut\smash{%
  \vtop{\lineskiplimit=\maxdimen\lineskip2pt\halign{#\cr漢\cr
    \small\vrule height .5ex depth .5ex\hrulefill\ \lower.5ex\hbox{$a$}\ 
    \hrulefill\vrule height .5ex depth .5ex\cr}}}}%
\imagfm{\jstrut )\inhibitglue}%
\hbox to .5\zw{\hss\normalsize (1)\hss}%
\imagfm{\jstrut\inhibitglue\gt (}%
\imagfm{\jstrut\gt 漢}%
\imagfm{\jstrut\gt )\inhibitglue}%
\hbox to .55\zw{\hss\normalsize (2)\hss}%
\imagfm{\fontsize{48}{48}\selectfont\jstrut\gt\inhibitglue (}%
\imagfm{\fontsize{48}{48}\selectfont\jstrut\smash{%
  \vtop{\lineskiplimit=\maxdimen\lineskip2pt\halign{#\cr\gt 大\cr
    \small\vrule height .5ex depth .5ex\hrulefill\ \lower.5ex\hbox{$1.2a$}\ 
    \hrulefill\vrule height .5ex depth .5ex\cr}}}}
\end{center}
\caption{Fonts with different metrics.}
\label{fig-diffmet}
\end{figure}

\item[\emph{kanjiskip} and \emph{xkanjiskip}]
In p\TeX, the value of \emph{xkanjiskip} is controlled by a skip named
	   |\xkanjiskip|. A defect of this implementation is that the
	   value of \emph{xkanjiskip} is not connected with the size of
	   the currnt Japanese font. It seems that |EXTRASPACE|,
	   |EXTRASTRETCH|, |EXTRASHRINK| parameters in a JFM are
	   reserved for specifying the default value of
	   \emph{xkanjiskip} in a unit of the design size, but p\TeX\
	   did not use these parameters. 

Considering this situation of p\TeX, \LuaTeX-ja can use the value of
	   \emph{xkanjiskip} that specified in a metric. If the value of
	   \emph{xkanjiskip} on user side (this is the
	   \textsf{xkanjiskip} parameter in |\ltjsetparameter|) is
	   |\maxdimen|, then the \LuaTeX-ja use the specification from
	   the current used metric as the actual value of
	   \emph{xkanjiskip}.
This description also applies for \emph{kanjiskip}.
\end{description}

\section{Distinction of Characters}
Since \LuaTeX\ can handle Unicode characters natively, it is a major
problem that how we distinguish Japanese characters and alphabetic
characters. For example, the multiplication sign (U+00D7) exists both in
ISO-8859-1 (hence in Latin-1 Supplement in Unicode) and in the basic
Japanese character set JIS~X~0208. It is not desirable that this
character is treated as an alphabetic char, because this symbol is often
used in the sense of `negative' in Japan. 

\subsection{Character Ranges}
Before we describe the approach taken is \LuaTeX-ja, we review the
approach taken by up\TeX.  up\TeX\ extends the |\kcatcode| primitive in
p\TeX, to use this primitive for setting how a character is treated
among alphabetic characters~(15), \emph{kanji}~(16), \emph{kana}~(17),
\emph{kanji}, \emph{Hangul}~(17), or~\emph{other CJK characters}~(18).
The assignment to |\kcatcode| can be done by a Unicode
block\footnote{There are some exceptions. For example, U+FF00--FFEF
(Halfwidth and Fullwidth Forms) are divided into three blocks in recent
up\TeX.}.

\LuaTeX-ja adopted a different approach. There are many Unicode blocks
	   in Basic Multilingual Plane which are not included in
	   Japanese fonts, it is inconvenient if we treat by a Unicode
	   block.  Furthermore, JIS~X~0208 are not just union of Unicode
	   blocks; for example, the intersection of JIS~X~0208 and
	   Latin-1 Supplement is shown in
	   Table~\ref{tab-inter}. Considering these two points, to
	   customize the range of Japanese characters in \LuaTeX-ja, one
	   has to define ranges of character codes in his source in advance.


\begin{table}
\caption{Intersection of JIS~X~0208 and Latin-1 Supplement.}
\label{tab-inter}
\begin{center}
\begin{tabular}{llll}
\ltjjachar"A7 (U+00A7),&
\ltjjachar"A8 (U+00A8),&
\ltjjachar"B0 (U+00B0),&
\ltjjachar"B1 (U+00B1),\\
\ltjjachar"B4 (U+00B4),&
\ltjjachar"B6 (U+00B6),&
\ltjjachar"D7 (U+00D7),&
\ltjjachar"F7 (U+00F7)
\end{tabular}
\end{center}
\end{table}

%%Example...

We note that \LuaTeX-ja offers two additional control sequence,
      |\ltjjachar| and |\ltjalchar|. They are similar to |\char|
      primitive, but |\ltjjachar| always yields a Japanese character (if
      the argument is more than or equal to 128) and |\ltjalchar| always
      yields an alphabetic character, regardless of the argument. 

\subsection{Default Setting of Ranges}
Patches for plain \TeX\ and \LaTeXe of \LuaTeX-ja predefines 8~character
ranges, as shown in Table~\ref{tab-chrrng}.  Almost of these ranges are
just the union of Unicode blocks, and determined from the Adobe-Japan1-6
character collection~\cite{aj16}, and JIS~X~0208. Among these 8~ranges,
the ranges~2, 3, 6, 7, and~8 are considered ranges of Japanese
characters, and others are considered ranges of alphabetic
characters\footnote{Note that ranges 3~and~8 are considered ranges of
alphabetic characters in this paper.}. We remark on ranges 2~and~8:
\begin{description}
\item[The range~2]
JIS~X~0208 includes Greek letters and Cyrillic letters, however, these
	   letters cannot be used for typesetting Greek or Russian, of
	   course. Hence it is reasonable that Greek letters and
	   Cyrillic consist another character range.
\item[The range~8] 
If one want to use 8-bit TFMs, such as T1 or TS1 encodings, he should
	   mark this range~8 as a range of alphabetic characters by
\begin{quote}
|\ltjsetparameter{jacharrange={-8}}|
\end{quote}
This is because some 8-bit TFMs have a glyph in this range; for example,
	   the character `\OE' is located at |"D7| in the T1 encoding. %"
\end{description}


\begin{table}
\caption{Predefined ranges in \LuaTeX-ja}
\label{tab-chrrng}
\begin{center}
\begin{tabular}{@{\bf}rl}
1&(Additional) Latin characters which are not belonged in the range~8.\\
2&Greek and Cyrillic letters.\\
3&Punctuations and miscellaneous symbols.\\
4&Unicode blocks which does not intersect with Adobe-Japan1-6.\\
5&Surrogates and supplementary private use Areas.\\
6&Characters used in Japanese typesetting.\\
7&Characters possibly used in CJK typesetting, but not in Japanese.\\
8&Characters in Table~\ref{tab-inter}.
\end{tabular}
\end{center}
\end{table}

\subsection{Control Sequences Producing Unicode Characters}
\label{ssec-unichar}

The \emph{fontspec} package\footnote{Preciously
saying, it is the \emph{xunicode} package, originally a package for
\XeTeX and automatically loaded by the \emph{fontspec} package.} offer
various control sequences that produce Unicode characters.  However, they as
it stands cannot work with the default range setting of \LuaTeX-ja.  For
example, |\textquotedblleft| is just an abbreviation of
|\char"201C\relax| %"
and the character U+201C (LEFT DOUBLE QUOTATION
MARK) is treated as an Japanese character, because it belongs to the
range~3. 
This problem is resolved by using |\ltjalchar| instead of the |\char| primitive. 
It is included in an optional package named \texttt{luatexja-\penalty0fontspec.sty}.
Figure~\ref{fig-unitxt} ...

\begin{figure}
\begin{LTXexample}
×, \char`×,   % depend on range setting 
\ltjalchar`×, % alphabetic char
\ltjjachar`×, % Japanese char
\texttimes     % alph. char (by fontspec)
\end{LTXexample}
\caption{Control sequences producing a Unicode character}
\label{fig-unitxt}
\end{figure}

The situation looks similar in math formulas, but in fact it differs.
Control sequences that represents ordinary symbols defined by the
\emph{unicode-math} package is just synonym of a character. For example,
the meaning of |\otimes| is just the character U+2297 (CIRCLED TIMES),
which is included in the range~3.  However, it is difficult to define a
control sequence like |\ltjalUmathchar| as a counterpart of
|\Umathchar|, since an input like `|\sum^\ltjalUmathchar ...|' has to be
permitted.

However, we couldn't include a solution to this problem in time for this
paper, due to a lack of time. We are just testing a solution that we
will explain it below:
\begin{itemize}
\item \LuaTeX-ja has a list of character codes which will be treated as
      alphabetic characters in math mode. Considering 8-bit TFMs for
      math symbols, this list includes natural numbers between |"80| and
      |"FF| by default.
\item Redefine internal commands defined in the \emph{unicode-math}
      package so that
codes of characters which are mentioned in the \emph{unicode-math}
      package will be included in the list.
\end{itemize}


We would like to extend treatments described in this section to 8-bit
font encodings, but we leave it to further development too.

\section{Current Status of Development}
At the moment, \LuaTeX-ja can be used under plain \TeX, and under
\LaTeXe. Generally speaking, one only has to read |luatexja.sty|, by
|\input| command or |\usepackage| (in~\LaTeXe), if you merely want to
typeset Japanese characters.  We look more detail by parts. 

\subsection{`Engine Extension'}
The lowest part of \LuaTeX-ja corresponds the p\TeX\ extension as
\emph{an engine extension of \TeX}. We, the project menbers, think that
this part is almost done. There is one more feature of \LuaTeX-ja which
we are going to explain:

\begin{description}
\item[Shifting Baseline]
In order to make a match between Japanese fonts and alphabetic fonts,
	   sometimes shifting the baseline of alphabetic characters may
	   be needed. p\TeX\ has a dimension |\ybaselineshift|, which
	   corresponds the amount of shifting down the baseline of alphabetic
	   characters. This is useful for Japanese-based documents, but
	   not for documents mainly in languages with alphabetic
	   characters.

Hence, \LuaTeX-ja extends p\TeX's |\ybaselineshift| to Japanese
	   characters. Namely, \LuaTeX-ja offers two parameters,
	   \textsf{yjabaselineshift} and \textsf{yalbaselineshift}, for the
	   amount of shifting the baseline of Japanese characters and
	   that of alphabetic characters, respectively. 
\begin{figure}
\begin{center}
\fontsize{40}{40}\selectfont\fboxsep0mm
\vrule width 0.9\textwidth height0.4pt depth0.4pt\kern-0.9\textwidth
\hbox to 0.9\linewidth{%
\hfil
\raise-10pt\imagfm{\jstrut 漢}%
\raise-10pt\imagfm{\jstrut 字}\hskip.25\zw%
\imagfm{p}%
\imagfm{h}%
\hfil\hfil
\imagfm{\jstrut 漢}%
\imagfm{\jstrut 字}\hskip.25\zw%
\raise-10pt\imagfm{p}%
\raise-10pt\imagfm{h}%
\hfil
}
\end{center}

\caption{First example of shifting baseline.}
\label{fig-bls}
\end{figure}

\begin{figure}
\begin{center}
\fontsize{30}{30}\selectfont\fboxsep0mm
\vrule width 0.9\textwidth height0.4pt depth0.4pt\kern-0.9\textwidth
\hbox to 0.9\linewidth{%
\hfil
\imagfm{a}%
\imagfm{b}\hskip.25\zw%
\imagfm{\jstrut 本}%
\imagfm{\jstrut 文}\hskip.33333\zw%
\raise3.514582pt\imagfm{\fontsize{20}{20}\selectfont\jstrut\inhibitglue (}%
\raise3.514582pt\imagfm{\fontsize{20}{20}\selectfont\jstrut 注}%
\raise3.514582pt\imagfm{\fontsize{20}{20}\selectfont\jstrut 釈}\hskip.1666667\zw%
\raise3.514582pt\imagfm{\fontsize{20}{20}\selectfont c}%
\raise3.514582pt\imagfm{\fontsize{20}{20}\selectfont o}%
\raise3.514582pt\imagfm{\fontsize{20}{20}\selectfont m}%
\raise3.514582pt\imagfm{\fontsize{20}{20}\selectfont m}%
\raise3.514582pt\imagfm{\fontsize{20}{20}\selectfont e}%
\raise3.514582pt\imagfm{\fontsize{20}{20}\selectfont n}%
\raise3.514582pt\imagfm{\fontsize{20}{20}\selectfont t}%
\raise3.514582pt\imagfm{\fontsize{20}{20}\selectfont\jstrut )\inhibitglue}%
\hskip.33333\zw%
\imagfm{\jstrut 本}%
\imagfm{\jstrut 文}%
\hfil
}
\end{center}

\caption{Second example of shifting baseline.}
\label{fig-small}
\end{figure}

An example output is shown in Figure~\ref{fig-bls}. The left half is the
	   output when \textsf{yjabaselineshift} is positive, hence the
	   baseline of Japanese characters is shifted down. On the other
	   hand, the right half is the output when
	   \textsf{yalbaselineshift} is positive, hence the baseline of
	   alphabetic characters is shifted. Figure~\ref{fig-small}
	   shows an intresting use of these parameters.

\end{description}
Note that \LuaTeX-ja doesn't support vertical typesetting, \emph{tategaki}, for now. 

\subsection{Patches for plain \TeX\ and \LaTeXe}
p\TeX\ has a patch for plain \TeX, namely |ptex.tex|, that for \LaTeXe\
macro (this patch and \LaTeXe\ consist \emph{p\LaTeXe}), and
|kinsoku.tex| which includes the default setting of \emph{kinsoku
shori}, the Japanese hyphenation.  We ported them to \LuaTeX-ja, except
the codes related to vertical typesetting, because \LuaTeX-ja doesn't
support vertical typesetting yet. We remark one point related to the
porting:
\begin{description}

\item[Behavior of\/ {\tt\char92fontfamily\/}]
The control sequence |\fontfamily| in p\LaTeXe\ changes the current alphabetic
	   font family and/or the current Japanese font family,
	   depending the argument. More concretely,
	   |\fontfamily{|$\langle\hbox{\it arg\/}\rangle$|}| changes the
	   current alphabetic font family to $\langle\hbox{\it
	   arg\/}\rangle$, if and only if one of the following
	   conditions are satisfied:
\begin{itemize}
\item An alphabetic font family named $\langle\hbox{\it arg\/}\rangle$ in
      \emph{some} alphabetic encoding already defined in the document.
\item There exists an alphabetic encoding $\langle\hbox{\it
      enc\/}\rangle$ already defined in the document such that a font
      definition file $\langle\hbox{\it enc\/}\rangle\langle\hbox{\it
      arg\/}\rangle$|.fd| (all lowercase) exists.
\end{itemize}
The same criterion is used for changing Japanese font family.

To work this behavior well, a list of all (alphabetic) encodings defined
	   already in the document is needed. However, since \LuaTeX-ja
	   is loaded as a package, \LuaTeX-ja cannot have this list.
	   Hence \LuaTeX-ja adopted a different approach, namely
	   |\fontfamily{|$\langle\hbox{\it arg\/}\rangle$|}| changes the
	   current alphabetic font family to $\langle\hbox{\it
	   arg\/}\rangle$, if and only if:
\begin{itemize}
\item An alphabetic font family named $\langle\hbox{\it arg\/}\rangle$
      in the current alphabetic encoding $\langle\hbox{\it
      enc\/}\rangle$ already defined in the document.
\item A  font definition file $\langle\hbox{\it enc\/}\rangle\langle\hbox{\it
      arg\/}\rangle$|.fd| (all lowercase) exists.
\end{itemize}


\end{description}



\subsection{Classes for Japanese Documents}
To produce `high-quality' Japanese documents, we need not only that
Japanese characters are correctly placed, but also class files for
Japanese documents. In p\TeX, there are two major families of classes:
\emph{jclasses} which is distributed with the official p\LaTeXe\ macros,
and \emph{jsclasses}.  At the present, \LuaTeX-ja
simply contains their counterparts: \emph{ltjclasses} and
\emph{ltjsclasses}. However, the policy on classess is not determined
now, and we hope to have another family of classes which are useful in
commercial printing.  In the author's opinion, \emph{ltjclasses} is
better to stay as an example of porting of class files for \pTeX\ to
\LuaTeX-ja.

\subsection{Patches for Packages}
Apart from patches for the \LaTeXe~kernel and classes for Japanese
documents, we need to make patches for several packages. At the present,
we considered the following packages, and made patches or porting for
the former two packages.

\begin{description}
\item[The \emph{fontspec} package] The \emph{fontspec} package is built
	   on NFSS2, hence control sequences offered by the
	   \emph{fontspec} package, such as |\setmainfont|, are only
	   effective for alphabetic fonts if \LuaTeX-ja is loaded.
	   \texttt{luatexja-\penalty0fontspec.sty} (not automatically
	   loaded) offers these counterparts for Japanese fonts, with
	   additional `j' in the name of control sequences, such as
	   |\setmainjfont|. As described in
	   Subsection~\ref{ssec-unichar}, it also includes a patch for
	   control sequences producing Unicode characters.

\item[The \emph{otf} package]
This package is widely used in p\TeX\ for characters which is
not in JIS~X~0208, and for using more than one weight in \emph{mincho}
and \emph{gothic} font families. Therefore \LuaTeX-ja supports features
in the \emph{otf} package, by loading \texttt{luatexja-\penalty0otf.sty}
	   manually. Note that characters by |\UTF{xxxx}| and
	   |\CID{xxxx}| are not appended to the current list as a
	   \emph{glyph\_node}, so they are not affected by callbacks by
	   the \emph{luaotfload} package. We have another remark; |\CID|
	   does not work with TrueType fonts.

\item[The \emph{listings} package]
It is well-known that there is a patch |jlisting.sty| of the
	   \emph{listings} package for p\LaTeXe. Generally speaking, it
	   also can be used in \LuaTeX-ja. However, it seems to be that
	   a Japanese character after a space does not recieve any
	   process of the \emph{listings} package; this is inconvinient
	   when we use the \emph{showexpl} package.
\end{description}



\section{Implementation}
\subsection{Handling of Japanese Fonts}
In p\TeX, there are three slots for maintaining current fonts, namely
|\font| for alphabetic fonts, |\jfont| for Japanese font (in horizontal
direction) and |\tfont| for Japanese font (in vertical direction). With
these slots, we can manage the current font for alphabetic characters
and that for Japanese characters separately in p\TeX.  However, \LuaTeX\
has only one slot for maintaining the current font, as \TeX82.  This
situation leads a problem: how can we maintain the `current Japanese
font'?

There are three approaches for this problem. One approach is to make a
mapping table from alphabetic fonts to corresponding Japanese fonts
(here we don't assume that NFSS2 is available).  Another approach is
that we always use composite fonts with alphabetic fonts and Japanese
fonts. The third approach is that the information of the current
Japanese font is stored in an attribute. We adopted the third approach,
since \LuaTeX-ja is much affected by p\TeX\ as we noted in
Subsection~\ref{ssec-pol}.

As in Figure~\ref{fig-jfdef}, \LuaTeX-ja uses |\jfont| for defining
Japanese font, as p\TeX.  However, because the information of the current
Japanese font is stored into an attribute, control sequences defined by
|\jfont| (e.g.,~|\foo| and |\bar| in Figure~\ref{fig-jfdef}) is
not representing a font by the means of \TeX82. In other words, each of
these control sequences is just an assignment to an attribute, therefore
they cannot be an argument of |\the|, |\fontname|, nor |\textfont|.


Callbacks by the \emph{luaotfload} package, e.g.,~replacement of glyphs
according to font features, are executed just after `Examination of
Stack Level' (see Subsection~\ref{ssec-over}). Note that calculation of
character classes for each Japanese character is done \emph{after} the
these callbacks for now. 

\subsection{Stack Management}
\label{ssec-stack}

As we noted in Subsection~\ref{ssec-csname}, parameters that the values
at the end of a horizontal box or that of a paragraph are effective in
whole box or paragraph, such as \emph{kanjiskip}, cannot be implemented
by internal integers or registers of other types in \TeX. We explain it
in this section.

\begin{figure}
\begin{lstlisting}
void package(int c)
{
    ...
    d = box_max_depth;
    unsave();
    save_ptr -= 4;
    if (cur_list.mode_field == -hmode) {
        cur_box = filtered_hpack(cur_list.head_field,
                                 cur_list.tail_field, saved_value(1),
                                 saved_level(1), grp, saved_level(2));
        subtype(cur_box) = HLIST_SUBTYPE_HBOX;
    } else {
\end{lstlisting}
\caption{An extract of a CWEB-source \texttt{tex/packaging.w} of \LuaTeX}
\label{fig-ltsrc}
\end{figure}

Figure~\ref{fig-ltsrc} is an extract of a CWEB-source
\texttt{tex/packaging.w} of \LuaTeX\ (SVN revision 4358). This function
is called just when an explicit |\hbox{...}| or |\vbox{...}| is ended, and
the function |filtered_hpack()| is where the |hpack_filter| and then the
actual `hpack' process are performed. Notice that the |unsave()|
function is called before |filtered_hpack()|. This is the problem;
because of |unsave()|, we can retrive only the values of registers
\emph{outside} the box, even in the |hpack_filter| callback.

To cope with this problem, \LuaTeX-ja has its own stack system, based on
Lua codes in \cite{stack-mail}. Furthermore, \emph{whatsit} nodes whose
\emph{user\_id} is 30112 (\emph{stack\_node}, for short) will be
appended to the current horizontal list each time the current stack
level is incremented, and their values are the values of
|\currentgrouplevel| at that time. In the beginning of the |hpack_filter|
callback, the list in question is traversed to determine whether the
stack level at the end of the list and that outside the box coincides.

Let $x$ be the value of |\currentgrouplevel|, and $y$ be the current
stack level, both inside the |hpack_filter| callback, i.e.,~outside a
horizontal box. Consider a list which represents the content of the box,
then we have:
\begin{itemize}
\item A \emph{stack\_node} whose value is $x+1$ (because all materials in
      the box are included in a group |\hbox{...}|, the value is at
      least $x+1$) in the list represents an assignment related to the
      stack system in just top-level of the list, like
\begin{quote}
\begin{verbatim}
\hbox{...(assignment)...}
\end{verbatim}
\end{quote}
In this case, the current stack level is incremented to $y+1$ after the assignment.
\item A \emph{stack\_node} whose value is more than  $x+1$ in the list represents
an assignment inside another group contained in the box. For example,
      the following input creates
a \emph{stack\_node} whose value is $x+3=(x+1)+2$:
\begin{quote}
\begin{verbatim}
\hbox{...{...{...(assignment)}...}...}
\end{verbatim}
\end{quote}
\end{itemize}
Thus, we can conclude that the stack level at the end of the list is
$y+1$, if and only if there is a \emph{stack\_node} whose value is
$x+1$. Otherwise, the stack level is just $y$.

\subsection{Adjustment of the Position of Japanese Characters}
\label{ssec-width}

The size of a glyph specified in a metric and that of a real font
usually differ. For example, the letter `\inhibitglue【' is half-width
in |jfm-ujis.lua| or |jis.tfm|, while this letter is full-width like `【'
in most TrueType fonts used in Japanese typesetting, such as
IPA~Mincho. Hence the adjustment of position of such glyphs is
needed. In the context of p\TeX, this process was performed using virtual fonts.

On the other hand, Lua\TeX-ja does the adjustment by encapsuling a glyph
into a horizontal box. There are two main reasons why we adopted this
method; one is that we feared Lua codes for coexisting with callbacks by
|luaotfload| package would be large if we use virtual fonts, and the
other is to cope with shifting of the baseline of characters at the
same time. 

\begin{figure}
\begin{center}\unitlength=9pt\small
\begin{picture}(15,12)(-1,-3)

\color{grayx}% real glyph
\put(-1,-1.5){\vrule width 6\unitlength height 7\unitlength depth 2.5\unitlength}

\color{black}% real glyph :step1
\thicklines
\put(-1,-1.5){\line(0,1){7}\line(0,-1){2.5}}
\put(5,-1.5){\line(0,1){7}\line(0,-1){2.5}}
\put(-1,5.5){\line(1,0){6}}
\put(-1,-4){\line(1,0){6}}
\put(-1,0){\makebox(0,0)[r]{\strut$R$\,}}

\thicklines
\put(0,0){\vector(0,1){9}\line(0,-1){3}\vector(1,0){12}}
\put(12,9){\makebox(0,0)[rt]{\strut$M$\,}}
\put(12,0){\line(0,1){9}\vector(0,-1){3}}
\put(0,9){\line(1,0){12}}
\put(0,-3){\line(1,0){12}}
\put(0.2,4.5){\makebox(0,0)[l]{\texttt{height}}}
\put(12.2,-1.5){\makebox(0,0)[l]{\texttt{depth}}}
\put(6,0.2){\makebox(0,0)[b]{\texttt{width}}}

\thicklines
\put(3,0){\line(0,1){7}\line(0,-1){2.5}\line(1,0){6}}
\put(9,0){\line(0,1){7}\line(0,-1){2.5}}
\put(3,7){\line(1,0){6}}
\put(3,-2.5){\line(1,0){6}}
\newsavebox{\eqdist}
\savebox{\eqdist}(0,0)[c]{%
  \thinlines
  \put(-0.08,0.2){\line(0,-1){0.4}}%
  \put(0.08,0.2){\line(0,-1){0.4}}}
\put(1.5,0){\usebox{\eqdist}}
\put(10.5,0){\usebox{\eqdist}}

\thicklines
\put(3,-1.5){\vector(-1,0){4}}
\put(1,-1.7){\makebox(0,0)[t]{\texttt{left}}}
\put(3,0){\vector(0,-1){1.5}}
\put(3.2,-0.75){\makebox(0,0)[l]{\texttt{down}}}
\end{picture} 
\end{center}
\caption{The position of the `real' glyph.}
\label{fig-pos}
\end{figure}

Figure~\ref{fig-pos} shows the adjustment process. A large square $M$ is
the imaginary body which is specified in the metric, and a vertical
rectangle is the imaginary body of a real glyph. First, the real glyph
is aligned with respect to the width of $M$. In the figure, the real
glyph is aligned `middle'; this setting is useful for the full-width
middle dot `・'. We have other settings, namely, `left' and `right'.
After that, it is shifted according to the value of |left| and |down|,
which are specified in the metric. The final position of the real glyph
is shown by the gray rectangle~$R$. If the amount of shifting the baseline is
not zero, $M$ (and hence the real glyph) is shifted by that amount.

We would like to remark briefly about the vertical position of a glyph.
A JFM (or the metric used in \LuaTeX-ja) and the real font used for it
may have different height or depth.  In that case, it may look better if
the real glyph is shifted vertically to match the height-depth ratio
specified in the metric. This situation is carefully studied by
Otobe~\cite{min10}. Here the policy on this problem is not determined
now, however we would like to offer several solutions in future development.

\section{Conclusion}
We have discussed about our \LuaTeX-ja package, which is much affected
by p\TeX. For now, it can be used for experimental use, however there
are much refinements which are needed for regular use. The author hopes
that this paper and this project contribute the typesetting Japanese,
and possibly other Asian languages, under \LuaTeX.

\section*{Acknowledgements}
The author would like to thank Ken Nakano and Hideaki Togashi for their
development of ASCII p\TeX.  The author is very grateful to Haruhiko
Okumura for his leadership in the Japanese \TeX\ community. The author
is also very grateful to members of the \LuaTeX-ja project team for their
valuable cooperation in development.

%%% The style of the bibiliogrphy is `amsplain'.
\providecommand{\bysame}{\leavevmode\hbox to3em{\hrulefill}\thinspace}
\providecommand{\href}[2]{#2}
\begin{thebibliography}{99}

\bibitem{aj16}
Adobe Systems Incorporated, \emph{Adobe-Japan1-6 Character Collection
	for CID-Keyed Fonts}, Technical Note~\#5078, 2004.
\url{http://partners.adobe.com/public/developer/en/font/5078.Adobe-Japan1-6.pdf}

\bibitem{ptex}
ASCII MEDIA WORKS,アスキー日本語\TeX\ (p\TeX).\url{http://ascii.asciimw.jp/pb/ptex/}

\bibitem{omega}
Jin-Hwan~Cho and Haruhiko Okumura, \emph{Typesetting CJK Languages with Omega},
\TeX, XML, and Digital Typography, Lecture Notes in Computer Science, vol.~3130,
Springer, 2004, 139--148.

\bibitem{joylua}
Yannis Haralambous. \emph{The Joy of \LuaTeX}. \url{http://luatex.bluwiki.com/}

\bibitem{jisx4051}
Japanese Industrial Standards Committee. \emph{JIS~X~4051: Formatting
	rules for Japanese documents}, 1993, 1995, 2004.

\bibitem{eptex}
北川弘典,$\varepsilon$-p\TeX についてのwiki.
\url{http://sourceforge.jp/projects/eptex/wiki/FrontPage}

\bibitem{luaums}
北川弘典,\LuaTeX で日本語.
\url{http://oku.edu.mie-u.ac.jp/tex/mod/forum/discuss.php?d=378}

\bibitem{luatexref}
\LuaTeX\ development team, \emph{The \LuaTeX\ reference}. 
\url{http://www.luatex.org/svn/trunk/manual/luatexref-t.pdf} (snapshot of SVN trunk)

\bibitem{man}
The \LuaTeX-ja project team, \emph{The \LuaTeX-ja package}. 
Not completed for now. Available at |doc/man-en.pdf| (in English) or
	|doc/man-ja.pdf| (in Japanese)
in the Git repository.

\bibitem{luajp-test}
香田温人,\LuaTeX と日本語.
\url{http://www1.pm.tokushima-u.ac.jp/~kohda/tex/luatex-old.html}

\bibitem{luajalayout}
前田一貴,luajalayout パッケージ---Lua\LaTeX によ
	る日本語組版---.
\url{http://www-is.amp.i.kyoto-u.ac.jp/lab/kmaeda/lualatex/luajalayout/}

\bibitem{jsclasses}
奥村晴彦,p\LaTeXe 新ドキュメントクラス.
\url{http://oku.edu.mie-u.ac.jp/~okumura/jsclasses/}

\bibitem{ptexjp}
Haruhiko Okumura, \emph{p\TeX\ and Japanese Typesetting},
	The Asian Journal of \TeX\ \textbf{2}~(2008), 43--51.

\bibitem{min10}
乙部厳己,min10フォントについて.
\url{http://argent.shinshu-u.ac.jp/~otobe/tex/files/min10.pdf}

\bibitem{otf}
齋藤修三郎,Open Type Font用VF.
\url{http://psitau.kitunebi.com/otf.html}

\bibitem{stack-mail}
Jonathan Sauer, \emph{[Dev-luatex] tex.currentgrouplevel}. 
\url{http://www.ntg.nl/pipermail/dev-luatex/2008-August/001765.html}

\bibitem{uptex}
Takuji Tanaka, \emph{up\TeX, up\LaTeX---unicode version of p\TeX, p\LaTeX}.
\url{http://homepage3.nifty.com/ttk/comp/tex/uptex_en.html}

\bibitem{ptexenc}
Nobuyuki Tsuchimura, \emph{Development of a Japanese \TeX\ Distribution~`ptetex3'},
Computer Software\ \textbf{24} (2007), no.~4, 40--50, (in Japanese).

\bibitem{w3c}
W3C Working Group, \emph{Requirements for Japanese Text Layout}. 
\url{http://www.w3.org/TR/jlreq/}
\end{thebibliography}

\end{document}

%#!lualatex ajt-devel-ltja
\documentclass{ajt}

%%% Packages used in this paper
%%% LuaTeX-ja

\usepackage{luatexja}
\DeclareFontShape{JY3}{mc}{m}{n}{<-> s*[0.92489] file:ipaexm.ttf:jfm=ujis}{}
\DeclareFontShape{JY3}{gt}{m}{n}{<-> s*[0.92489] file:ipaexg.ttf:jfm=ujis}{}
% quick hack: monospaced Japanese font by \ttfamily
\DeclareKanjiFamily{JY3}{\ttdefault}{}{}
\DeclareFontShape{JY3}{\ttdefault}{m}{n}{<-> s*[0.92489] file:ipaexg.ttf:jfm=mono}{}

%%% for LTXexample environment
\usepackage{showexpl,lltjlisting}
\lstset{basicstyle=\ttfamily\small, width=0.3\textwidth, basewidth=.5em}


\usepackage{mflogo,booktabs}

%%% Verbatim environment
\usepackage{fancyvrb}
\CustomVerbatimEnvironment{code}{Verbatim}%
{numbers=left,xleftmargin=1.5em,baselinestretch=1.069,fontsize=\small}
\CustomVerbatimEnvironment{codewithoutnum}{Verbatim}%
{xleftmargin=1.5em,baselinestretch=1.069,fontsize=\small}
\CustomVerbatimEnvironment{codewithoutnumsmall}{Verbatim}%
{xleftmargin=1.5em,baselinestretch=1.0,fontsize=\footnotesize}
\DefineShortVerb{\|}

%%% Mandatory article metadata %%%
\title{Development of the \LuaTeX-ja package}
\author{Hironori Kitagawa {\normalsize 北川 弘典}}
\address{The \LuaTeX-ja project team}
\email{h\_kitagawa2001@yahoo.co.jp}

\keywords{\TeX, p\TeX, \LuaTeX, \LuaTeX-ja, Japanese}
\abstract{%
The \LuaTeX-ja package is a macro package for typesetting Japanese
documents under \LuaTeX.  This packages has much flexibility of
typesetting than p\TeX, and corrected some unwanted features of p\TeX.
In this paper, we describe specifications, the current status and some
internal processing codes of \LuaTeX-ja.
}

\newcommand{\parname}[1]{\textsf{#1}}
\newcommand{\jstrut}{\vrule width0pt height\cht depth\cdp}
\newcommand{\imagfm}[1]{\ifvmode\leavevmode\fi%
  \hbox{\fboxsep=0pt\fbox{\setbox0=\hbox{#1}\copy0\kern-\wd0
  \vrule width \wd0 height 0.4pt depth0.4pt}}}
\begin{document}

%%% Do not forget to start with \maketitle!
\maketitle

\section{Introduction}
\subsection{History}
To typeset Japanese documents with \TeX, ASCII p\TeX~\cite{ptex} has
been widely used.  There are other methods---for example, using Omega
and OTP~\cite{omegaj}, or with the CJK package---to do so, however,
these alternative methods did not became a majority. On the one hand,
p\TeX\ enables us to produce high-quality documents, but on the other
hand, p\TeX\ is left behind from the extensions of \TeX\ such as \eTeX\
and \pdfTeX, and the diffusion of UTF-8 encoding.  In recent years, the
situation become better, because of the developments of
|ptexenc|~\cite{ptexenc} by Nobuyuki~Tsuchimura,
$\varepsilon$-p\TeX~\cite{eptex} by the author,~and up\TeX~\cite{uptex}
by Takuji~Tanaka.

However, there are still lag now. 


Before this \LuaTeX-ja package, there were several attempts to typeset
Japanese documents under \LuaTeX.  Here we cite three examples:
\begin{itemize}
\item |luaums.sty|~\cite{luaums} developed by the author. This
      experimental package is for creating a Japanese-based presentation
      under \LuaTeX.
\item |luajalayout| package\cite{luajalayout}, formerly known as the
      |jafontspec| package, by Kazuki Maeda. This package is based on
      \LaTeXe\ and |fontspec| package.
\item |luajp-test| package\cite{luajp-test}, a test package made by
      Atsuhito Kohda, based on articles on the web page~\cite{joylua}.
\end{itemize}


\subsection{Development Policy of \LuaTeX-ja}
\label{ssec-pol} 
The first aim of the project is to implement features (from the
''primitive'' level) of p\TeX as macros under \LuaTeX, so \LuaTeX-ja is
much affected by p\TeX.  However, as the development proceeds, some
technical/conceptual difficulties are arisen. Hence we changed the aim
of the project.
\begin{itemize}
\item\emph{\LuaTeX-ja offers more flexibility of typesetting than that by
     p\TeX.}

     We think that the ability of producing outputs conformed to
     JIS~X~4051~\cite{jisx4051}, the Japanese Industrial Standard for
     typesetting, is not enough; if one wants to produce very
     incoherent outputs for some reason, it should be possible. 
In this point, previous attempts of Japanese typesetting with \LuaTeX\
     which we cited in the previous subsection are inadequate.

p\TeX has some flexibility of typesetting, by changing internal
     parameters such as |\kanjiskip| or |\prebreakpenalty|, and by using
     custom JFM (Japanese TFM). ...

\item\emph{\LuaTeX-ja isn't mere re-implementation or porting of p\TeX;
     some (technically and/or conceptually) inconvenient features of
     p\TeX\ are modified.} 

     We describe this point in more detail at the next section.
\end{itemize}


\subsection{Contents of this Paper}
Here we describe the contents of the rest of this paper briefly.  In
Section~2, we describe major differences between p\TeX\ and \LuaTeX-ja,
which is introduced. Some of them are due to specifications of callbacks
in \LuaTeX\ (\emph{i.e.}, technical reason), and others are which we
thought which are better to be changed, for ``natural''
specifications. In Section~3, we show the current status of the
\LuaTeX-ja project.

For implementing features into \LuaTeX-ja, we had to use some tricks in
Lua scripts.  In Section~4, we describe several these tricks and
internal processing methods.  We hope that the materials in this section
have good applications.

\subsection*{About the Project}
This \LuaTeX-ja project is hosted by SourceForge.jp. The official wiki
is located on
\url{http://sourceforge.jp/projects/luatex-ja/wiki/FrontPage}.  There is
no stable version at Oct.\ 6, 2011, but the development source can be
obtained from the git repository.
Members of the project are as follows (in random order):
Hironori Kitagawa, Kazuki Maeda, Takayuki Yato,
Yusuke Kuroki, Noriyuki Abe, Munehiro Yamamoto, Tomoaki Honda, and~Shuzaburo Saito.


\section{Major differences with \pTeX}
In this section, we briefly look at ** major differences between p\TeX\
and \LuaTeX-ja.  For general information of Japanese typesetting and the
facts about p\TeX, please see Okumara~\cite{ptexjp}.


\subsection{Names of Control Sequences}
\label{ssec-csname}
Since p\TeX\ is a engine modification of Knuth's original \TeX82 engine,
some primitives added in it takes a form that cannot be simulated by a
macro.  For example, an additional primitive
|\prebreakpenalty|$\langle\hbox{\it
char\_code}\rangle$|[=]|$\langle\hbox{\it penalty}\rangle$ in p\TeX\
sets the amount of penalty inserted before $\langle\hbox{\it
char\_code}\rangle$ to $\langle\hbox{\it penalty}\rangle$, and
|\prebreakpenalty|$\langle\hbox{\it char\_code}\rangle$ can be also used
for retrieving the value.

Moreover, there are some parameters for Japanese typesetting which were
mere internal integers, dimensions, or~skips in p\TeX\ that cannot be
implemented by same approaches in \LuaTeX-ja.  These parameters have a
common point; the values at the end of a horizontal box or that of a
paragraph are effective in whole box or paragraph.  A good example of
them is |\kanjiskip|, the default amount of a skip which will be
inserted between two consecutive Japanese characters by default.  The
reason of this is the place of |hpack_filter| in the \LuaTeX's
CWEB-source code, and we will discuss on it in
Subsection~\ref{ssec-stack}.

From above 2~problems we discussed above, the assignment and retrieval
of most parameters in \LuaTeX-ja are summarized into 3~control sequences:
\begin{itemize}
\item |\ltjsetparameter{|$\langle\hbox{\it
      name}\rangle$|=|$\langle\hbox{\it value}\rangle$|,...}|: for local
      assignment.
\item |\ltjglobalsetparameter|: for global assignment. These two control
      sequences obey the value of |\globaldefs| primitive.
\item |\ltjgetparameter{|$\langle\hbox{\it
      name}\rangle$|}[{|$\langle\hbox{\it optional
      argument}\rangle$|}]|: for retrieval. The returned value is always
      a string.
\end{itemize}

\subsection{Line break after a Japanese Character}
\label{ssec-line} 

Japanese texts can break lines almost everywhere, in contrast with
alphabetic texts can break lines only between words (or use
hyphenation). Hence, p\TeX's input processor is modified so that a
line break after a Japanese character doesn't emit a space. However,
there is no way to customize the input processor of \LuaTeX, other than
hack its CWEB-source. All we can do is to modify an input line before
when \LuaTeX\ begin to process it, inside the |process_input_buffer|
callback.

Hence, in \LuaTeX-ja, a comment letter (we reserve U+FFFFF for this
purpose) will be appended to an input line, if this ends with a Japanese
character\footnote{Strictly speaking, it also requires that the catcode
of the end-line character is 5~(\emph{end-of-line}). This condition is
useful under the verbatim environment.}. One might jump to a conclusion
that the treatment of a line break by p\TeX\ and that of \LuaTeX-ja is
totally same, but they are different in the respect that \LuaTeX-ja's
judgement whether a comment letter will be appended the line is done
\emph{before} the line is actually processed by \LuaTeX.

Figure~\ref{fig-linebreak} shows an example; the command at the first
line marks most of Japanese characters as ``non-Japanese character''. In
other words, from this command onward, the letter `あ' will be treated
as an alphabetic character by \LuaTeX-ja. Then, it is natural to occur a
space between `あ' and `y' in the output, where the actual output in the
figure does not so.  This is because `あ' is considered to be a Japanese
character by \LuaTeX-ja, when \LuaTeX-ja does a decision whether U+FFFFF
will be added to the input line~2.
\begin{figure}
\begin{LTXexample}
\font\x=IPAMincho \x
\ltjsetparameter{jacharrange={-6}}xあ
y
\end{LTXexample}
\caption{A notable sample showing the treatment of a line break after a
Japanese character.}\label{fig-linebreak}
\end{figure}

\subsection{Separation between ``real'' fonts and Metrics}
\label{ssec-sepmet}

Traditionally, most Japanese fonts used in typesetting are not proportional,
that is, most glyphs have same size (in most cases,
square-shaped). Hence, it is not rare that the contents of different
JFMs are totally same, and only differ in their names. For example, the
difference between |min10.tfm| and |goth10.tfm|, which are JFMs shipped
with p\TeX\ for seriffed \emph{mincho} family and sans-seriffed
\emph{gothic} family, are their |FAMILY| and |FACE| only. Moreover,
|jis.tfm| and |jisg.tfm|, which consists a parts of \emph{jis} font
metric which is used in Haruhiko Okumura's
\emph{jsclasses}~\cite{jsclasses}, are totally same as binary files.
Another example is: if one want to use many fonts which are not
installed in his \TeX\ distribution, of course he needs to prepare TFMs
for them. But, as long as he wants to use Japanese fonts with p\TeX, he
has to only copy and rename some JFM (\emph{e.g.},~copy |jis.tfm| to
|hoge.tfm|).

Considering this situation, we decided to separate ``real'' fonts and
metrics in \LuaTeX-ja, as shown in Figure~\ref{fig-jfdef};
\begin{itemize}
\item a control sequence |\jfont| must be used for Japanese fonts, instead of |\font|.
\item \LuaTeX-ja automatically loads the |luaotfload| package, so
      |file:| prefix and features can be used as the line~1 in
      Figure~\ref{fig-jfdef}.
\item The |jfm| key specifies the metric for the font. In
      Figure~\ref{fig-jfdef}, both fonts will use a metric stored in a
      Lua script named |jfm-ujis.lua|. This metric is the standard
      metric in \LuaTeX-ja, and is based on JFMs used in the \emph{otf}
      package~\cite{otf}.
\item The |psft:| prefix can be used to specify name-only, non-embedded
      fonts. 
\end{itemize}
We note that |-kern| in features is important, since if kerning
information from real font itself will clash with spacing from the
metric.

\begin{figure}
\begin{verbatim}
\jfont\foo=file:ipaexm.ttf:jfm=ujis;script=latn;-kern;+jp04 at 12pt
\jfont\bar=psft:Ryumin-Light:jfm=ujis at 10pt
\end{verbatim}
\caption{Typical declarations of Japanese fonts.}
\label{fig-jfdef}
\end{figure}

\subsection{Insertion of Kerns and/or Glues for Japanese Typesetting: the Timing}
\label{ssec-jglue}

As described in \cite{luatexref}, \LuaTeX's kerning and ligaturing
process is totally different from that of \TeX82.
\TeX82's process is done just when a (sequence of) character is appended
to current list. Thus we can interrupt this process by writing |f{}irm|
(this gives `f\hbox{}irm' in \TeX82). However, \LuaTeX's process is
\emph{node-based}, that is, the process will be done when a horizontal
box of a paragraph is ended, so |f{}irm| and |firm| yield the same
output under \LuaTeX.

The situation for Japanese characters is basically same, but not
entirely.  Glues (and kerns) those will be needed for Japanese
typesetting will be divided into the following three categories:
\begin{description}
\item[Glue (or Kern) from the Metric of Japanese Fonts] 
\item[Default Glue Between a Japanese Character and an Alphabetic Character] 
Usually 1/4 of full-width with some stretch and shrink for justifying
	   each line.
\item[Default Glue Between Two Consecutive Japanese Characters] 
The main reason of this glue is to enable line-breaking almost
	   everywhere in Japanese texts. In most cases, its natural
	   width is zero, and
some stretch/shrink for justifying each line. 
\end{description}
In p\TeX, these three kinds of glues are treated differently. The first
 category (\emph{JFM glue}, for short) is inserted when a (sequence of)
 Japanese character is appended to current list, same as alphabetic
 characters in \TeX82. This means that one can interrupt the insertion
 process by saying |{}|.  The second category (\emph{xkanjiskip}, for
 short) is inserted just before `hpack' or line-breaking of a paragraph;
 this timing is somewhat similar to that of \LuaTeX's kerning
 process. The third category (\emph{kanjiskip}, for short) is not
 appeared as a node anywhere; only appears implicitly in calculation of
 the width of a horizontal box or that of breaking lines. These
 specifications made p\TeX's behavior very hard to understand.

\LuaTeX-ja inserts glues in all three categories simultaneously inside
|hpack_filter| and |pre_linebreak_filter| callbacks.  The reasons of
this specification are to behave like alphabetic characters in \LuaTeX\
(as described in the first paragraph), and to clarify the specification
for \LuaTeX-ja's process.

\subsection{Insertion of Kerns and/or Glues for Japanese Typesetting: the Spec}
\begin{table}
\caption{Examples of differences between  p\TeX\ and \LuaTeX-ja,}
\label{tab-jfmglue}
\begin{center}
\begin{tabular}{llllllll}
\toprule
&\multicolumn{1}{c}{(1)}&\multicolumn{1}{c}{(2)}&\multicolumn{1}{c}{(3)}&\multicolumn{1}{c}{(4)}\\
Input      &|あ】{}【〙\/〘|        &|い』\/a| &|う)\hbox{}(| &|え]\special{}[|\\\midrule
p\TeX      &あ】\hbox{}【〙\hbox{}〘&い』\/a   &う)\hbox{}(   &え]\hbox{}[\\
\LuaTeX-ja &あ】{}【〙\/〘          &い』\/a   &う)\hbox{}(   &え]\special{}[\\
\bottomrule
\end{tabular}
\end{center}
\end{table}

\begin{figure}
\begin{center}
\fontsize{40}{40}\selectfont
\imagfm{\jstrut あ}%
\imagfm{\jstrut 】\inhibitglue}%
\imagfm{\jstrut\kern.5\zw}%
\imagfm{\jstrut\kern.5\zw}%
\imagfm{\jstrut\hbox{}\inhibitglue【}%
\imagfm{\jstrut 〙\inhibitglue}%
\imagfm{\jstrut\kern.5\zw}%
\imagfm{\jstrut\kern.5\zw}%
\imagfm{\jstrut \hbox{}\inhibitglue〘}%
\end{center}
\caption{Detail of (1) in Table~\ref{tab-jfmglue}.}
\label{fig-ptexjfm}
\end{figure}

Now we will take a look inside the insertion process itself, and describe three points.

\begin{description}
\item[Ignored Nodes]
As noted in the previous subsection, the insertion process in p\TeX\ is
	   interrupted by saying |{}| or anything else. This leads the
	   second row in Table~\ref{tab-jfmglue}, or
	   Figure~\ref{fig-ptexjfm}. ``The process is interrupted''
	   means that p\TeX\ does not think the letter `】\inhibitglue'
	   is followed by `\inhibitglue【', hence two half-width glues
	   are inserted between between `】\inhibitglue' and
	   `\inhibitglue【', where one is from `】\inhibitglue' and
	   another is from `\inhibitglue【'.

	   On the other hand, in \LuaTeX-ja, the process is done inside
	   |hpack_filter| and |pre_linebreak_filter| callbacks. Hence,
	   \emph{anything that does not make any nodes will be
	   ignored,}\ in \LuaTeX-ja, as shown in (1) in
	   Table~\ref{tab-jfmglue}. \LuaTeX-ja also ignores any nodes
	   which does not make any contribution to current horizontal
	   list---\emph{ins\_node}, \emph{adjust\_node},
	   \emph{mark\_node}, \emph{whatsit\_node} and
	   \emph{penalty\_node}---, as shown in (4).

By the way, around a \emph{glyph\_node} $p$ there may be some nodes
	   attached to $p$. These are an accent and kerns for
	   positioning it, and kerns from italic correction for $p$, and
	   it is natural that these attachments should be ignored in the
	   process. Hence \LuaTeX-ja takes this approach, as the latest
	   version of p\TeX\ (p3.2). This explains (2) in the figure.

Summerizing, to 

\item[Fonts with the Same Metric]
Recall that \LuaTeX-ja separated ``real'' fonts and metrics, as in Subsection~\ref{ssec-sepmet}. 
Consider the following input, where all Japanese fonts
	   use same metric (in \LuaTeX-ja), and |\gt| selects \emph{gothic} family:
\begin{quote}
\begin{verbatim}
明朝)\gt (ゴシック
\end{verbatim}
\end{quote}
If the above input is processed by p\TeX, since the insertion process is
	   interrupt by |\gt|, the result looks like
\begin{quote}
\mc 明朝)\hbox{}\gt (ゴシック
\end{quote}
But this is unnatural, since two Japanese fonts in the output uses the
same metric, \emph{i.e.}, the same typesetting rule.  Hence, we decided
that Japanese fonts with the same metric are treated as one font in the
insertion process of \LuaTeX-ja. Thus, the output from the above input
in \LuaTeX-ja is:
\begin{quote}
\mc 明朝)\gt (ゴシック
\end{quote}
One might have the situation that this specification is not
	   suitable. \LuaTeX-ja offers a way to cope with this case, but
	   we leave it to the manual~\cite{man} of \LuaTeX-ja.

\item[Fonts with Different Metrics] 
In the case where two Japanese characters with different metrics and/or
	   different size is similar. Consider the following input where
	   the \emph{mincho} fmaily and the \emph{gothic} family use
	   different metrics:
\begin{quote}
\begin{verbatim}
漢)\gt (漢)\large (大
\end{verbatim}
\end{quote}
As he previous point, this input yields an output like the following by p\TeX:
\begin{quote}
\mc 漢)\hbox{}\gt (漢)\hbox{}\large (大
\end{quote}
We thought that amounts of spaces between parentheses in above
	   output. So we changed the default behavior of \LuaTeX-ja that
	   the amount of a glue between two Japanese characters with
	   different metrics is the average of a glue from the left
	   character and that from the right character. For example,
	   Figure~\ref{fig-diffmet} shows the output from above
	   input. The width of glue indicated `①' is half-width , and
	   the width of glue indicated `②' is about 0.55 times of
	   fullwidth. This default behavior can be changed by
	   |diffrentmet| parameter of \LuaTeX-ja.

\begin{figure}
\begin{center}
\fontsize{40}{40}\selectfont
\imagfm{\jstrut 漢}%
\imagfm{\jstrut )\inhibitglue}%
\imagfm{\jstrut\hbox to .5\zw{\hss\Large ①\hss}}%
\imagfm{\jstrut\hbox{}\inhibitglue\gt (}%
\imagfm{\jstrut\gt 漢}%
\imagfm{\jstrut\gt )\inhibitglue}%
\imagfm{\jstrut\hbox to .55\zw{\hss\Large ②\hss}}%
\imagfm{\fontsize{48}{48}\selectfont\jstrut\gt\hbox{}\inhibitglue (}%
\imagfm{\fontsize{48}{48}\selectfont\jstrut\gt 漢}%
\end{center}
\caption{Fonts with Different Metrics.}
\label{fig-diffmet}
\end{figure}
\end{description}


\section{Current Status of the Development}
At the moment, \LuaTeX-ja can be used under plain \TeX, and under
\LaTeXe. Generally speaking, one has to read |luatexja.sty|, by |\input|
command or |\usepackage|~(\LaTeXe) if you merely want to typeset
Japanese character.  We look more detail by parts.

\subsection{``Engine Extension''}
The lowest part of \LuaTeX-ja corresponds the p\TeX\ extension as
\emph{\TeX\ engine}. We, the project menbers, think that this part is almost
done. Other features of \LuaTeX-ja which we have not described are the
followings:
\begin{description}
\item[Setting the range of ``Japanese characters''] This feature is
	   inspired by up\TeX. up\TeX\ has an additional primitive named
	   |\kcatcode| for setting a character is treated as alphabetic
	   character, \emph{kana}, \emph{kanji}, \emph{Hangul},
	   or~\emph{other CJK character}, and the assignment of
	   |\kcatcode| can be done by a block of Unicode\footnote{There
	   are some exceptions. For example, U+FF00--FFEF (Halfwidth and
	   Fullwidth Forms) are divided into three blocks in up\TeX.}.

\LuaTeX-ja uses a slightly different approach. Because there are many
	   Unicode blocks in Basic Multilingual Plane which are not
	   included in most Japanese fonts, ...  Furthermore, the basic
	   Japanese character set JIS~X~0208 are not just union of
	   Unicode blocks. For example, the intersection of JIS~X~0208
	   and Latin-1 Supplement is shown in Table~\ref{tab-inter}.
	   Considering these two points, to customize the range of
	   Japanese characters in \LuaTeX-ja, one must follow the
	   following steps:
\begin{enumerate}
\item Assign a range number to character codes. For example, the following
      input assigns the number~10 to a unicode block ``Halfwidth and
      Fullwidth Forms'' and ``\char"A7'' (the Section Sign):
\begin{quote}
\begin{verbatim}
\ltjdefcharrange{10}{"FF00-"FFEF,"A7}
\end{verbatim}
\end{quote}
\item Assigning to \textsf{jacharrange} ...
\end{enumerate}

\item[Baseline Shifting]
In order to make a match between Japanese fonts and alphabetic fonts,
	   sometimes shifting the baseline of alphabetic characters is
	   needed. p\TeX\ has a dimension |\ybaselineshift|, which
	   corresponds the amount of shifting the baseline of alphabetic
	   characters. 

\LuaTeX-ja extends p\TeX's |\ybaselineshift| to Japanese
	   characters. Namely, \LuaTeX-ja offers two parameters,
	   \emph{yjabaselineshift} and \emph{yalbaselineshift} for the
	   amount of shifting the baseline of Japanese characters and
	   that of alphabetic characters, respectively. The example
	   output is shown in Figure~\ref{fig-bls}. The left half is the
	   output when \emph{yjabaselineshift} is positive, hence the
	   baseline of Japanese characters is shifted down. On the other
	   hand, the right half is the output when
	   \emph{yalbaselineshift} is positive, hence the baseline of
	   alphabetic characters is shifted.

\begin{figure}
\begin{center}
\fontsize{40}{40}\selectfont\fboxsep0mm
\vrule width 0.9\textwidth height0.4pt depth0.4pt\kern-0.9\textwidth
\hbox to 0.9\linewidth{%
\hfil
\raise-10pt\imagfm{\jstrut 漢}%
\raise-10pt\imagfm{\jstrut 字}\hskip.25\zw%
\imagfm{p}%
\imagfm{h}%
\hfil\hfil
\imagfm{\jstrut 漢}%
\imagfm{\jstrut 字}\hskip.25\zw%
\raise-10pt\imagfm{p}%
\raise-10pt\imagfm{h}%
\hfil
}
\end{center}

\caption{Baseline shifting.}
\label{fig-bls}
\end{figure}

\end{description}
Note that \LuaTeX-ja doesn't support for vertical typesetting, \emph{tategaki}, for now. 

\begin{table}
\caption{Intersection of JIS~X~0208 and Latin-1 Supplement.}
\label{tab-inter}
\begin{center}
\begin{tabular}{llll}
\char"A7 (U+00A7),&
\char"A8 (U+00A8),&
\char"B0 (U+00B0),&
\char"B1 (U+00B1),\\
\char"B4 (U+00B4),&
\char"B6 (U+00B6),&
\char"D7 (U+00B7),&
\char"F7 (U+00D7)
\end{tabular}
\end{center}
\end{table}

\subsection{Patches for plain \TeX\ and \LaTeXe}
p\TeX\ has patches for plain \TeX, namely |ptex.tex|, that for \LaTeXe\
macro (this patch and \LaTeXe\ consist \emph{p\LaTeXe}), and
|kinsoku.tex| which includes the default setting of \emph{kinsoku
shori}, the Japanese hyphenation.  We ported them to \LuaTeX-ja, except
the codes related to vertical typesetting. We remark two points related to the porting:
\begin{description}
\item[Default Range of Japanese Characters] 
As described in the previos subsection, \LuaTeX-ja can customize the
range of Japanese characters.  \LuaTeX-ja predefines 8~character ranges,
as shown in Table~\ref{tab-chrrng}.  Almost of these ranges are just the
union of Unicode blocks, and determined from the Adobe-Japan1 character
set, and JIS~X~0208.  And, among these 8~ranges, the ranges~2, 3, 6, 7,
and~8 are considered ranges of Japanese characters, and others are
considered ranges of alphabetic characters.

This default setting is suitable for Japanese-based documents, but it
	   causes that other packages with Unicode fonts do not work
	   correctly. For example, |\times| provided by the
	   |unicode-math| package is the character U+00D7, which belongs
	   to the range~8, and ...
, the |fontspec| package, ... 
...

\begin{table}
\caption{Predefined Ranges in \LuaTeX-ja}
\label{tab-chrrng}
\begin{center}
\begin{tabular}{@{\bf}rl}
1&(Additional) Latin characters which is not belonged in the range~8.\\
2&Greek and Cyrillic letters.\\
3&Punctuations and miscellaneous symbols.\\
4&Unicode blocks which does not intersect with Adobe-Japan1.\\
5&Surrogates and supplementary private use Areas.\\
6&Characters used in Japanese typesetting.\\
7&Characters possibly used in CJK typesetting, but not in Japanese.\\
8&Characters in Table~\ref{tab-inter}.
\end{tabular}
\end{center}
\end{table}


\item[The behavior of\/ {\tt\char92fontfamily\/} command]
The |\fontfamily| command in p\LaTeXe\ changes the current alphabetic
	   font family and/or the current Japanese font family,
	   depending the argument. More concretely,
	   |\fontfamily{|$\langle\hbox{\it arg\/}\rangle$|}| changes the
	   current alphabetic font family to $\langle\hbox{\it
	   arg\/}\rangle$, if and only if one of the following
	   conditions are satisfied:
\begin{itemize}
\item Alphabetic font family named $\langle\hbox{\it arg\/}\rangle$ in
      \emph{some} alphabetic encoding already defined in the document.
\item There exists an alphabetic encoding $\langle\hbox{\it
      enc\/}\rangle$ already defined in the document such that a font
      definition file $\langle\hbox{\it enc\/}\rangle\langle\hbox{\it
      arg\/}\rangle$|.fd| exists.
\end{itemize}
The same criterion is used for changing Japanese font family.

To make possible this behavior, one has to create a list of already
	   defined alphabetic encodings. Hence it works in p\LaTeXe, ...
However, since \LuaTeX-ja is loaded as a package, it will not
	   work. Hence \LuaTeX-ja adopted different approach, namely
	   |\fontfamily{|$\langle\hbox{\it arg\/}\rangle$|}| changes the
	   current alphabetic font family to $\langle\hbox{\it
	   arg\/}\rangle$, if and only if:
\begin{itemize}
\item Alphabetic font family named $\langle\hbox{\it arg\/}\rangle$ in
      the current alphabetic encoding $\langle\hbox{\it enc\/}\rangle$.
\item A  font definition file $\langle\hbox{\it enc\/}\rangle\langle\hbox{\it
      arg\/}\rangle$|.fd| exists.
\end{itemize}


\end{description}



\subsection{Classes for Japanese Documents}
To produce ``high-quality'' Japanese documents, we need not only that
Japanese characters are correctly placed, but also class files for
Japanese documents. In p\TeX, there are two major families of classes:
\emph{jclasses} which is distributed with the official p\LaTeXe\ macros,
and \emph{jsclasses}~\cite{jsclasses} which has developed by Haruhiko
Okumura and now widely used in Japanese \TeX\ users.  At the present,
\LuaTeX-ja simply contains their counterparts: \emph{ltjclasses} and
\emph{ltjsclasses}. However, the policy on classess is not determined
now, and we hope to have another family of classes which are useful in
commercial printing.  In the author's opinion, \emph{ltjclasses} is
better to stay as an example of porting of class files for \pTeX\ to
\LuaTeX-ja.


\section{Implementation}
\subsection{Handling of Japanese Fonts}
In p\TeX, there are three slots for maintaining current fonts, namely
|\font| for alphabetic fonts, |\jfont| for Japanese font (in horizontal
direction) and |\tfont| for Japanese font (in vertical direction). With
these slots, we can select the current font for alphabetic characters
and that for Japanese characters separately in p\TeX.  However, \LuaTeX\
has only one slot for maintaining the current font, as \TeX82.  This
situation leads a problem: how can we maintain the ``current Japanese
font''?

There are three approaches for this problem. One approach is to make a
mapping table from alphabetic fonts to corresponding Japanese fonts
(here we don't assume that NFSS2 is available), and when current
alphabetic font is changed, the current Japanese font also changes
according to the table.  Another approach is that we always use
composite fonts with alphabetic fonts and Japanese fonts. The third
approach is that the information of the current Japanese font is stored
in an attribute. We adopted the third approach, since \LuaTeX-ja is much
affected by p\TeX\ as we noted in Subsection~\ref{ssec-pol}.

As in Figure~\ref{fig-jfdef}, \LuaTeX-ja uses |\jfont| for defining
Japanese font, as p\TeX.  However, since the information of the current
Japanese font is stored into an attribute, control sequences defined by
|\jfont| (\emph{e.g.},~|\foo| and |\bar| in Figure~\ref{fig-jfdef}) is
not representing a font by the means of original \TeX. In other words,
these control sequence cannot be an argument of |\the| or |\textfont|,
and they are just an assignments to an attribute, in fact.


\subsection{Overview of the Processes}
Now we describe an outline of the \LuaTeX-ja's process briefly.
\begin{description}
\item[Treatment of Linebreaks after Japanese Characters] This part is
	   described already at Subsection~\ref{ssec-line}. Done in the
	   |process_input_buffer| callback.
\item[Font Replacement] In the |hyphenate| callback, we looks into for
	   each \textit{glyph\_node}~$p$. If its character is considered
	   to be a Japanese character, the font used in $p$ is replaced
	   by the value of |\ltj@curjfnt| that is associated
	   with~$p$. Also we subtract the subtype of $p$ by 1 to
	   suppress hyphenation around it by \LuaTeX, since later
	   processes of \LuaTeX-ja take care of all things about
	   Japanese charaters.
\end{description}
%
Following processes are all executed in |pre_linebreak_filter| and
|hpack_filter| callback. These are main routines of \LuaTeX-ja:

\begin{description}
\item[Examination of Stack Level] We traverse the horizontal list which
	   is the content of a horizontal box
to determine what is the level of \LuaTeX-ja's internal stack in the end
	   of the list. This is needed because of the place of
	   |hpack_filter| in the source of \LuaTeX. We will discuss more
	   detail in Subsection~\ref{ssec-stack}.

\item[Insertion of Glues/Kerns for Japanese Typesetting]
This part is already described at Subsection~\ref{ssec-jglue}. 

\item[Adjustument of the Places of (Japanese) Characters]
Under \LuaTeX-ja, the size of the virtual body of a Japanese character
	   and its position (\emph{i.e.}, offset) are determined by the
	   metric, since the optimal width of a character in
	   typesetting---in most cases, this is specified width in the
	   metric---and the actual width in TrueType/Opentype fonts
	   often differ. For example, the width the fullwidth open brace
	   `\inhibitglue {' is considered to be half-width in
	   typesetting, although this character is full-width in
	   TrueType fonts like IPA~Mincho.

To adjust size/places of Japanese characters, \LuaTeX-ja encapsules a
	   \textit{glyph\_node} which containing a Japanese character
	   into a horizontal box which size is specified in the metric.
We will discuss more detail in Subsection~\ref{ssec-width}.
\end{description}

\subsection{Stack Management}
\label{ssec-stack}

As we noted on Subsection~\ref{ssec-csname}, parameters that the values
at the end of a horizontal box or that of a paragraph are effective in
whole box or paragraph, such as \emph{kanjiskip}, cannot be implemented by internal integers or
registers of other types in \TeX. We explain it in this section.

\begin{figure}
\begin{lstlisting}
void package(int c)
{
    ...
    d = box_max_depth;
    unsave();
    save_ptr -= 4;
    if (cur_list.mode_field == -hmode) {
        cur_box = filtered_hpack(cur_list.head_field,
                                 cur_list.tail_field, saved_value(1),
                                 saved_level(1), grp, saved_level(2));
        subtype(cur_box) = HLIST_SUBTYPE_HBOX;
    } else {
\end{lstlisting}
\caption{An extract of a CWEB-source \texttt{tex/packaging.w} of \LuaTeX}
\label{fig-ltsrc}
\end{figure}

Figure~\ref{fig-ltsrc} is an expert of a CWEB-source
\texttt{tex/packaging.w} of \LuaTeX\ (version?). This function is called
just when explicit |\hbox{...}| or |\vbox{...}| is ended, and the
function |filtered_hpack()| is where the |hpack_filter| and then the
`hpack' process is performed. Notice that the |unsave()| function is
called before |filtered_hpack()|. This is the problem; because of
|unsave()|, we can only the values of registers outside the box, even in
the |hpack_filter| callback.

To cope with this problem, \LuaTeX-ja has its own stack system, based on
Lua codes in \cite{stack-mail}. Furthermore, \emph{whatsit} nodes whose
\emph{user\_id} is 30112 (\emph{stack\_node}, for short) will be
appended to the current horizontal list each time the current stack
level is incremented, and their values are the values of
|\currentgrouplevel| at that time. In the beginning of |hpack_filter|
callback, the list in question is traversed to determine whether the
stack level at the end of the list and that outside the box coincides.

Let $x$ be the value of |\currentgrouplevel|, and $y$ be the current
stack level, both inside the |hpack_filter| callback. Then we have:
\begin{itemize}
\item A \emph{stack\_node} whose value is $x+1$ (since all materials in
      the box are included in a group |\hbox{...}|) in the list
      represents an assignment related to the stack system in just
      top-level of the list, like
\begin{quote}
\begin{verbatim}
\hbox{...(assignment)...}
\end{verbatim}
\end{quote}
In this case, the current stack level is incremented to $y+1$ after the assignment.
\item A \emph{stack\_node} whose value is more than  $x+1$ in the list represents
an assignment inside another group contained in the box. For example,
      the following input creates
a \emph{stack\_node} whose value is more than  $x+3=(x+1)+2$:
\begin{quote}
\begin{verbatim}
\hbox{...{...{...(assignment)}...}...}
\end{verbatim}
\end{quote}
\end{itemize}
Thus, we can conclude that the stack
level at the end of the list is $y+1$, if and only if there is a
\emph{whatsit} node whose \emph{user\_id} is 30112 and whose value is
$x+1$. Otherwise, the stack level is just $y$.

\subsection{Adjustment Of the Place of Japanese Characters}
\label{ssec-width}


\section*{Acknowledgements}


%%% The style of the bibiliogrphy is `amsplain'.
\providecommand{\bysame}{\leavevmode\hbox to3em{\hrulefill}\thinspace}
\providecommand{\href}[2]{#2}
\begin{thebibliography}{99}

%\bibitem{Knuth}
%Donald E.~Knuth, \emph{The \TeX book}, Addison-Wesley, 1986.

\bibitem{ptex}
ASCII MEDIA WORKS, \textbf{アスキー日本語\TeX\ (p\TeX)}\ (in
	Japanese). \url{http://ascii.asciimw.jp/pb/ptex/}

%\bibitem{Eijkhout}
%Victor Eijkhout, \emph{\TeX\ by Topic, A \TeX nician's Reference}, Addison-Wesley, 1992. \url{http://www.cs.utk.edu/~eijkhout/texbytopic-a4.pdf}

\bibitem{luaums}
Hironori Kitagawa, \textbf{LuaTeXで日本語}\ (in
	Japanese). \url{http://oku.edu.mie-u.ac.jp/tex/mod/forum/discuss.php?d=378}

\bibitem{luajalayout}
Kazuki Maeda\ (前田一貴), \textbf{luajalayout パッケージ —LuaLaTeX によ
	る日本語組版—}\ (in Japanese).
\url{http://www-is.amp.i.kyoto-u.ac.jp/lab/kmaeda/lualatex/luajalayout/}

\bibitem{luajp-test}
Atsuhito Kohda, \textbf{LuaTeXと日本語}\ (in
	Japanese). \url{http://www1.pm.tokushima-u.ac.jp/~kohda/tex/luatex-old.html}

\bibitem{joylua}
Yannis Haralambous. \textbf{The Joy of LuaTeX}. \url{http://luatex.bluwiki.com/}

\bibitem{otf}
Shuzaburo Saito\ (齋藤修三郎), \textbf{Open Type Font用VF}\ (in Japanese).
\url{http://psitau.kitunebi.com/otf.html}

\bibitem{luatexref}
\textbf{The \LuaTeX reference}

\bibitem{jsclasses}
Haruhiko Okumura\ (奥村晴彦), \textbf{pLaTeX2e 新ドキュメントクラス}\
	(in
	Japanese). \url{http://oku.edu.mie-u.ac.jp/~okumura/jsclasses/}

\bibitem{ptexjp}
Haruhiko Okumura\ (奥村晴彦), \textbf{p\TeX\ and Japanese Typesetting},
	The Asian Journal of \TeX\ \textbf{2}~(2008), 43--51.

\bibitem{stack-mail}
Jonathan Sauer, \textbf{[Dev-luatex] tex.currentgrouplevel}. 
\url{http://www.ntg.nl/pipermail/dev-luatex/2008-August/001765.html}

\bibitem{min10}
Yoshiki Otobe\ (乙部厳己), \textbf{min10フォントについて}\ (in japanese).
\url{http://argent.shinshu-u.ac.jp/~otobe/tex/files/min10.pdf}
\end{thebibliography}

\end{document}

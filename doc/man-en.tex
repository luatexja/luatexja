%#!lualatex
\documentclass[a4paper,titlepage]{article}
\usepackage{booktabs,amsmath}
\usepackage{luatexja}
%\usepackage{luatexja-fontspec}
\usepackage[margin=20mm]{geometry}
\usepackage[unicode=true]{hyperref}

\title{The Lua\TeX-ja package}
\author{The Lua\TeX-ja project team}
\begin{document}
\maketitle
\part{User's manual}
{\Large\bf This documentation is far from complete. It may have many grammatical errors.}

\section{Introduction}

The Lua\TeX-ja package is a macro package for typesetting high-quality Japanese documents in Lua\TeX.

\subsection{Backgrounds}
Traditionally, ASCII p\TeX, an extension of \TeX, and its derivatives are used to typeset Japanese documents in \TeX.
p\TeX is an engine extension of \TeX: so it can produce high-quality Japanese documents without using very complicated macros.
But this point is a mixed blessing: p\TeX\ is left behind from other extensions of \TeX, especially $\varepsilon$-\TeX\ and pdf\TeX,
and from changes about Japanese processing in computers (\textit{e.g.}, the UTF-8 encoding).
Recently the extensions of p\TeX, namely up\TeX\ (Unicode-implementation of p\TeX) and 
$\varepsilon$-p\TeX\ (Merging of p\TeX and $\varepsilon$-\TeX\ extension), have developed to fill those gap to some extent,
but gaps are still exist.

However, the appearance of Lua\TeX\ changed the whole situation. With using Lua ``callbacks'', 
users can customize the internal processing of Lua\TeX. So there is no need to modify sources of the \TeX\ engine to
support Japanese typesetting: to do this, we only have to write Lua script for appropriate callbacks.

\subsection{Functionality relation with p\TeX}
The Lua\TeX-ja package is much influenced by p\TeX\ engine. The initial target of development was to implement features of p\TeX.
However, \emph{Lua\TeX-ja is not a just porting of p\TeX: Unnatural specifications/behaviors of p\TeX\ were not adopted}.
The followings are major changes from p\TeX:
\begin{itemize}
\item At the present, vertical typesetting, or \textit{tategaki}, is not supported in Lua\TeX-ja.
\end{itemize} 
For detailed information, see Part~\ref{part-imp}.

\subsection{Notations}
In this document, the following terms and notations are used:
\begin{itemize}
\item Characters are divided into two types: 
\begin{itemize}
\item \textbf{JAchar}: standing for Japanese characters such as Hiragana, Katakana, Kanji and other punctuation marks for Japanese.
\item \textbf{ALchar}: standing for all other characters like alphabets. 
\end{itemize}
\item A word in sans-serif font (like \textsf{prebreakpenalty}) represents an internal parameter for Japanese typesetting, and
it is used as a key in \verb+\ltjsetparameter+ command.
\item The word ``primitive'' is used not only for primitives in Lua\TeX, but also for control sequences that defined 
in the core module of Lua\TeX-ja.
\end{itemize}

\newpage
\section{Getting Started}
\subsection{Installation}
To install the Lua\TeX-ja\ package, you will need:
\begin{itemize}
\item Lua\TeX, version 0.65.0-beta or later.\\
If you are using \TeX~Live\ 2011 or W32\TeX, you don't have to worry.
\item The source archive or Lua\TeX-ja, of course{\tt:)}
\end{itemize}

The installation methods are as follows:
\begin{enumerate}
\item Download the source archive.

At the present, Lua\TeX-ja has no official release, so you have to retrieve
the archive from the repository.
You can retrieve the Git repository via
\begin{verbatim}
$ git clone git://git.sourceforge.jp/gitroot/luatex-ja/luatexja.git
\end{verbatim} 
or download the archive of HEAD in the master branch from
\begin{flushleft}
\url{http://git.sourceforge.jp/view?p=luatex-ja/luatexja.git;a=snapshot;h=HEAD;sf=tgz}.
\end{flushleft}
\item Extract the archive. You will see {\tt src/} and several other sub-directories.
\item Copy all the contents of {\tt src/} into your TEXMF trees.
\item If {\tt mktexlsr} is needed to update the filename database, make it so.
\end{enumerate}

\subsection{Cautions}
\begin{itemize}
\item UTF-8
\item conflicts with unicode-math
\end{itemize}

\subsection{Using in plain \TeX}
To use Lua\TeX-ja in plain \TeX, simply put the following  at the beginning of the document:
\begin{verbatim}
\input luatexja.sty
\end{verbatim}

This does the minimal setting (like {\tt ptex.tex}) for typesetting Japanese documents:
\begin{itemize}
\item The following 6 Japanese fonts are preloaded.
\begin{center}
\begin{tabular}{ccccc}
\toprule
\textbf{classification}&\textbf{font name}&\textbf{13.5\,Q}&\textbf{9.5\,Q}&\textbf{7\,Q}\\\midrule
\textit{mincho}&Ryumin-Light    &\verb+\tenmin+&\verb+\sevenmin+&\verb+\fivemin+\\
\textit{gothic}&GothicBBB-Medium&\verb+\tengt+ &\verb+\sevengt+ &\verb+\fivegt+\\
\bottomrule
\end{tabular}
\end{center}
\begin{itemize}
\item The `Q' is an unit used in Japanese phototypesetting, and $1\,\textrm{Q}=0.25\,\textrm{mm}$.
\item It is widely accepted that the font `Ryumin-Light' and `GothicBBB-Medium' aren't embedded into PDF files, and
the PDF reader substitutes them by some external Japanese font. We adopt this custom to the default setting.
\item size
\end{itemize}
\item A character in Unicode is treated as \textbf{JAchar} if and only if its code-point has more than or equal to U+0100.
\item The amount of glue that are inserted between \textbf{JAchar} and \textbf{ALchar} (the parameter {\sf xkanjiskip}) is
set to
\[
 0.25\,\hbox{\verb+\zw+}^{+1\,\text{pt}}_{-1\,\text{pt}} = \frac{27}{32}\,\mathrm{mm}^{+1\,\text{pt}}_{-1\,\text{pt}}.
\]
Here \verb+\zw+ is the virtual width of `current' Japanese font.
\end{itemize}


\subsection{Using in \LaTeX}
\paragraph{\LaTeXe}
Using in \LaTeXe\ is basically same. To set up the minimal environment for Japanese, you only have to load {\tt luatexja.sty}:
\begin{verbatim}
\usepackage{luatexja}
\end{verbatim}
It also does the minimal setting (the counterpart in p\LaTeX\ is  {\tt plfonts.dtx} and {\tt pldefs.ltx}):

\begin{itemize}
\item {\tt JY3} is used as the font encoding for Japanese fonts (in horizontal direction).\\
If vertical typesetting is supported by Lua\TeX-ja, {\tt JT3} will be used for vertical fonts.
\item Two font families {\tt mc} and {\tt gt} are defined: 
\begin{center}
\begin{tabular}{ccccc}
\toprule
\textbf{classification}&\textbf{family}&\verb+\mdseries+&\verb+\bfseries+&\textbf{scale}\\\midrule
\textit{mincho}&\tt mc&Ryumin-Light    &GothicBBB-Medium&0.960444\\
\textit{gothic}&\tt gt&GothicBBB-Medium&GothicBBB-Medium&0.960444\\
\bottomrule
\end{tabular}
\end{center}
\item Japanese characters in math mode are typeset by font family {\tt mc}.
\end{itemize}

However, the above setting is not sufficient for Japanese-based documents. To do this, 
You are better to use class files other than {\tt article.cls}, {\tt book.cls}, ...
The better alternatives are:
\begin{itemize}
\item BXjscls
\item ltjarticle, ltjbook?
\item ltjsarticle, ltjsbook?
\end{itemize}

\subsection{Changing Fonts}
\paragraph{Remark: Japanese Characters in Math Mode}


\paragraph{plain \TeX}
\paragraph{NFSS2}
\paragraph{fontspec}

\subsection{Changing parameters}


\part{Reference}
\section{Font Metric and Japanese Font}
\section{Parameters}
\section{Other Primitives}
\part{Implementations}\label{part-imp}
\end{document}
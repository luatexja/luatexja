%#! lualatex -shell-escape tc11-slide.tex
\documentclass[notheorems,12pt,hyperref={unicode=true}]{beamer}
\usepackage{luatexja,luatexja-otf,showexpl,lltjlisting}
\usepackage{lmodern,mathtools,graphicx,pict2e}
\usepackage{amsmath,bm,mflogo,booktabs}
\usepackage[all]{xy}

\SelectTips{cm}{}

%%% theme
\useinnertheme{rectangles}
\useoutertheme{split}
\usecolortheme{rose}
\usecolortheme{dolphin}
\setbeamertemplate{navigation symbols}{}
\setbeamertemplate{headline}{}

\makeatletter
\definecolor{purple}{rgb}{0.60, 0.0, 1.0}
\definecolor{green}{rgb}{0.0, 0.60, 0.0}
\definecolor{lblue}{rgb}{0.30, 0.0, 1.00}
\definecolor{gray}{rgb}{0.60, 0.60, 0.60}
\definecolor{linenavy}{rgb}{0.0 ,0.0 ,0.50}
\definecolor{linesky}{rgb} {0.50,0.75,1.00}
\newdimen\lineheight \lineheight=6pt
\def\lines#1{%
  \noindent\null\hskip-30pt\setbox0=\hbox{%
  {\color{linenavy}\vrule height \lineheight width #1\lineheight}%
  \kern 0.2\lineheight{\color{linenavy!80!linesky}\vrule height\lineheight width 0.6\lineheight}%
  \kern 0.2\lineheight{\color{linenavy!60!linesky}\vrule height\lineheight width 0.52\lineheight}%
  \kern 0.2\lineheight{\color{linenavy!40!linesky}\vrule height\lineheight width 0.44\lineheight}%
  \kern 0.2\lineheight{\color{linenavy!20!linesky}\vrule height\lineheight width 0.36\lineheight}%
  \kern 0.2\lineheight}\copy0\nobreak
  \@tempdima=\textwidth\advance\@tempdima64pt\advance\@tempdima-\wd0
  \hbox to 0pt{\color{linesky}\vrule height \lineheight width \@tempdima \hss}\par
}
\setbeamertemplate{frametitle}{
  \noindent\insertframetitle\par\vskip-8pt\lines{12}\vskip-16pt
}
\setbeamertemplate{footline}{
  \leavevmode%
  \hbox{\begin{beamercolorbox}[wd=.5\paperwidth,ht=2.5ex,dp=1.125ex,
leftskip=.3cm,rightskip=.3cm]{author in head/foot}%
    \usebeamerfont{author in head/foot}\ \hfill \insertshortauthor
  \end{beamercolorbox}%
  \begin{beamercolorbox}[wd=.5\paperwidth,ht=2.5ex,dp=1.125ex,
leftskip=.3cm,rightskip=.3cm plus1fil]{title in head/foot}%
    \usebeamerfont{title in head/foot}\insertshorttitle
  \end{beamercolorbox}}%
  \vskip0pt%
}
\expandafter\let\csname beamer@@tmpop@title page@default\endcsname=\relax
\defbeamertemplate*{title page}{default}[1][]
{
  \vbox{}
  \vfill
  \begin{centering}
    \begin{beamercolorbox}[sep=8pt,center,#1]{title}
      \usebeamerfont{title}\inserttitle
    \end{beamercolorbox}
  \end{centering}

\lines{29.5}\vskip8pt
  \begin{centering}
    \begin{beamercolorbox}[sep=8pt,center,#1]{author}
      \usebeamerfont{author}\insertauthor
    \end{beamercolorbox}
    \begin{beamercolorbox}[sep=8pt,center,#1]{date}
      \usebeamerfont{date}\insertdate
    \end{beamercolorbox}
  \end{centering}
  \vskip-8pt\vfill
}

\AtBeginSection[]{
  \begin{frame}
    \frametitle{Outline}
    \tableofcontents[currentsection,hideothersubsections]
  \end{frame}
}

\setbeamercolor{structure}{fg=linenavy}
\setbeamercolor{author in head/foot}{fg=white,bg=linenavy}
\setbeamercolor{title in head/foot}{fg=black,bg=linesky}
\setbeamercolor{block title}{bg=linenavy!50!linesky,fg=white}
\setbeamercolor{block body}{fg=black,bg=linesky!20!white}
\setbeamertemplate{section in toc}{■\null\inserttocsection\par\smallskip}
\setbeamerfont{section in toc}{size=\normalsize}
\setbeamerfont{title}{size=\Large, series=\bfseries}
\setbeamerfont{frametitle}{size=\Large, series=\bfseries, shape=\upshape}

\def\mcdefault{gt}
\DeclareFontShape{JY3}{gt}{bx}{n}{ <-> s*[0.924872] psft:FutoGoB101-Bold:jfm=ujis}{}
\DeclareFontShape{JY3}{gt}{m}{n}{ <-> s*[0.924872] psft:GothicBBB-Medium:jfm=ujis}{}
\DeclareFontShape{JY3}{mc}{m}{n}{ <-> s*[0.924872] psft:Ryumin-Light:jfm=ujis}{}
\def\notice#1{{\color{blue!50!black}#1}}
\def\alert#1{{\color{red}\bfseries#1}}
\def\pTeX{p\TeX}
\parindent=0pt
\catcode`\|=13\def|{\char92 }
\def\{{{\ttfamily\char`\{}}
\def\}{{\ttfamily\char`\}}}
\ltjsetparameter{alxspmode={`\\,allow}}
\lstset{numbers=left, basicstyle=\ttfamily}
\makeatother

\title{Lua\TeX-jaの開発}
\author[北川 弘典]{北川 弘典\\\footnotesize(Lua\TeX-jaプロジェクトチーム)}
\date{2011年10月22日}

\begin{document}

\begin{frame}
  \titlepage
\end{frame}

\section{導入}


\subsection{背景}
\begin{frame}
  \frametitle{Lua\TeX-ja}
  {\large\[
    \xymatrix{
      \text{\TeX}\ar[r]\ar[d]&\text{$\varepsilon$-\TeX}\ar[d]\ar[r]
      &\text{pdf\TeX}\ar[r]&\text{Lua\TeX}\ar@{-->}[d]\\
      \text{p\TeX}\ar[r]&\text{$\varepsilon$-p\TeX}
      \ar@{-->}[rr]&&\text{\alert{Lua\TeX-ja}}
    }
  \]}%
  \begin{center}
    \bfseries\Large 
    Lua\TeX-jaは,Lua\TeX 上で日本語組版を実現させるマクロパッケージである.
  \end{center}
\end{frame}

\begin{frame}[fragile]
  \frametitle{Lua\TeX}
  {\large\begin{align*}
    \text{Lua\TeX}&\simeq \text{pdf\TeX}+ \text{Lua}+\text{\MP}\\
    &\qquad+\text{Omega}+\text{OpenType}+ \cdots
  \end{align*}}
  \textbf{主な特徴:}
  \begin{itemize}
    \item pdfへの直接出力(pdf\TeX の後継)
    \item Unicodeへの対応{\small (SIP内の文字も余裕)}
    \item TrueType/OpenTypeフォントの直接利用
    \item Luaスクリプトで\TeX 内部処理のカスタマイズが可能\\
      \quad →もはやエンジンレベルで拡張する必要はない
  \end{itemize}

  \footnotesize 詳しくは,昨年度の八登さんの講演を参照.
\end{frame}

\subsection{開発方針}
\begin{frame}
  \frametitle{開発方針1}
  \begin{flushleft}
   \alert{\bfseries\large 
    p\TeX のプリミティブに対応する機能から実装.}%
  \end{flushleft}\vskip-\medskipamount
  \begin{itemize}
  \item 最低でもp\TeX と同等の組版の自由度を確保する.\medskip
  \item 以前から存在した,Lua\TeX で日本語組版を行う実験:
  \begin{itemize}
    \item \texttt{luaums.sty}(北川)\\\quad
      {\small 適当にでっち上げた最低限の実装.}
    \item \texttt{luajalayout}パッケージ(作者:前田一貴)\\\quad
      {\small フォント合成・fontspecパッケージを用いた実装.}
  \end{itemize}
  は「\LaTeX ベース」で,組版の調整機能が不足していた.
  \end{itemize}
\end{frame}

\begin{frame}
  \frametitle{開発方針2}
  \begin{flushleft}
    \alert{\bfseries\large 
    p\TeX と100\%の互換性は目指さない.}%
  \end{flushleft}\vskip-\medskipamount
  \begin{itemize}
    \item p\TeX の不都合・不可解な点があれば,積極的に改める.
    \item p\TeX と全く同じ文法・動作の実装は事実上不可能.
  \end{itemize}\medskip
  \begin{flushleft}
    \alert{\bfseries\large 
    最終的に,(空白挿入等の)仕様を文書化する.}
  \end{flushleft}
\end{frame}

\subsection{使い方: plain \TeX}
\begin{frame}
  \frametitle{\only<1>{plain p\TeX 用ソース}\only<2>{``plain Lua\TeX-ja''用ソース}}
\begin{flushleft}\ttfamily
\only<2>{\alert{|input luatexja.sty} \notice{\% \normalfont Lua\TeX-ja本体}}\ \\
|hsize=20\alert{\only<2>{|zw}}\only<1>{zw}\\
\only<1>{|font|bigmc=jis at 14.4pt}
\alert{\only<2>{|jfont|bigmc=psft:Ryumin-Light:jfm=ujis at14.4pt}}\\
\notice{\ \ \% 和文フォント定義}\\
こんにちは,|TeX の世界へ!\\
\{|bigmc 大きい文字だよ.\}\\
|end
\end{flushleft}
\uncover<2>{
\begin{itemize}
\item {\tt luatexja.sty}を読み込む{\footnotesize(これがないと話が始まらない)}
\item {\tt zw}, {\tt zh}は{\tt |zw}, {\tt |zh}に.
\item 和文フォント定義は{\tt |jfont}のみ可.書式も変化.
\end{itemize}}
\end{frame}


\subsection{使い方: \LaTeX}
\begin{frame}
  \frametitle{\only<1>{p\LaTeX}\only<2>{``Lua\LaTeX-ja''}用ソース}
\begin{flushleft}\ttfamily
|documentclass[a4paper,10pt]\{\only<1>{jsarticle}\alert{\only<2>{ltjsarticle}}\}\\
|usepackage\{lmodern\}|begin\{document\}\\
「これはまったく意味がない日本語の文だ.」\\
(あいう)\{|large|gtfamily(abcゴシック)\}\\
\ \\
何かalphabet(欧文文字)も打ってみるか.\\
\only<1>{\notice{\%}}\only<2>{\$|zeta(2) |simeq}\\
\only<2>{\ \ |directlua\{tex.print(math.pi\char`\^2/6)\}\$.}\only<1>{\notice{\%}}\\
|end\{document\}
\end{flushleft}
\uncover<2>{
\begin{itemize}
\item クラスを\alert{Lua\TeX-ja 同梱のもの}に変える.
\item 欧文用クラスに対しても,{\ttfamily|usepackage\{luatexja\}}で\\
最低限の設定がされる.
\end{itemize}
}
\end{frame}

\begin{frame}
  \frametitle{組版結果}
\begin{center}
\fboxsep=1\zw
\fbox{\large\parbox{20\zw}{\parindent=1\zw%
\fontfamily{mc}\fontfamily{rm}\selectfont
\baselineskip=1.6em
「これはまったく意味がない日本語の文だ.」
(あいう){\Large\gtfamily(abcゴシック)}

何かalphabet(欧文文字)も打ってみるか.
$\zeta(2) \simeq
  \directlua{tex.print(math.pi^2/6)}$.
}}
\end{center}
\end{frame}



\section{現在の状況}

\begin{frame}
  \frametitle{実装の模式図}
  \begin{center}\unitlength=10mm
  \begin{picture}(9.2,6.5)
    \linethickness{1pt}
    % primitive:
    \only<1>{%
    \color{lblue!20!white}
    \put(0,0){\vrule width 9.2\unitlength height 3.2\unitlength depth 0pt}
    \put(7.2,3.2){\vrule width 2\unitlength height 3.3\unitlength depth 0pt}
    \color{lblue}
    \put(0,0){\line(1,0){9.2}}
    \put(0,0){\line(0,1){3.2}}
    \put(9.2,0){\line(0,1){6.5}}
    \put(0,3.2){\line(1,0){7.2}}
    \put(7.2,3.2){\line(0,1){3.3}}
    \put(7.2,6.5){\line(1,0){2}}
    \color{lblue!50!black}
    \put(4.6,1.6){\makebox(0,0)[c]{「エンジン拡張」}}
    }%
    % primitive (detailed)
    \only<2>{%
    %% typesetting
    \color{linenavy!20!white}
    \put(0,0){\vrule width 9.2\unitlength height \unitlength depth 0pt}
    \color{linenavy}
    \put(0,0){\line(1,0){9.2}}
    \put(0,0){\line(0,1){1}}
    \put(0,1){\line(1,0){9.2}}
    \put(9.2,0){\line(0,1){1}}
    \color{linenavy!50!black}
    \put(4.6,0.5){\makebox(0,0)[c]{実際の組版処理用Luaコード}}
    \color{blue!20!white}
    \put(0,1.1){\vrule width 9.2\unitlength height \unitlength depth 0pt}
    \color{blue}
    \put(0,1.1){\line(1,0){9.2}}
    \put(0,1.1){\line(0,1){1}}
    \put(0,2.1){\line(1,0){9.2}}
    \put(9.2,1.1){\line(0,1){1}}
    \color{blue!50!black}
    \put(4.6,1.6){\makebox(0,0)[c]{パラメタ設定用Luaコード}}
    \color{lblue!20!white}
    \put(0,2.2){\vrule width 9.2\unitlength height \unitlength depth 0pt}
    \put(7.2,3.2){\vrule width 2\unitlength height 3.3\unitlength depth 0pt}
    \color{lblue}
    \put(0,2.2){\line(1,0){9.2}}
    \put(0,2.2){\line(0,1){1}}
    \put(9.2,2.2){\line(0,1){4.3}}
    \put(0,3.2){\line(1,0){7.2}}
    \put(7.2,3.2){\line(0,1){3.3}}
    \put(7.2,6.5){\line(1,0){2}}
    \color{lblue!50!black}
    \put(4.6,2.7){\makebox(0,0)[c]{\TeX インターフェース}}
    }%
    % plain
    \color{green!20!white}
    \put(0,3.3){\vrule width 2\unitlength height 3.2\unitlength depth 0pt}
    \color{green}
    \put(0,3.3){\line(1,0){2}}
    \put(0,3.3){\line(0,1){3.2}}
    \put(2,3.3){\line(0,1){3.2}}
    \put(0,6.5){\line(1,0){2}}
    \color{green!50!black}
    \put(1,4.9){\makebox(0,0)[c]{\parbox[c]{7\zw}{\centering plain \TeX\\対応}}}
    % LaTeX kernel
    \color{purple!20!white}
    \put(2.1,3.3){\vrule width 5\unitlength height \unitlength depth 0pt}
    \color{purple}
    \put(2.1,3.3){\line(1,0){5}}
    \put(2.1,3.3){\line(0,1){1}}
    \put(7.1,3.3){\line(0,1){1}}
    \put(2.1,4.3){\line(1,0){5}}
    \color{purple!50!black}
    \put(4.6,3.8){\makebox(0,0)[c]{\LaTeX 対応}}
    % class file
    \color{gray!20!white}
    \put(2.1,4.4){\vrule width 5\unitlength height \unitlength depth 0pt}
    \color{gray}
    \put(2.1,4.4){\line(1,0){5}}
    \put(2.1,4.4){\line(0,1){1}}
    \put(7.1,4.4){\line(0,1){1}}
    \put(2.1,5.4){\line(1,0){5}}
    \color{gray!50!black}
    \put(4.6,4.9){\makebox(0,0)[c]{日本語用クラスファイル}}
    % patches for packages
    \color{red!20!white}
    \put(2.1,5.5){\vrule width 5\unitlength height \unitlength depth 0pt}
    \color{red}
    \put(2.1,5.5){\line(1,0){5}}
    \put(2.1,5.5){\line(0,1){1}}
    \put(7.1,5.5){\line(0,1){1}}
    \put(2.1,6.5){\line(1,0){5}}
    \color{red!50!black}
    \put(4.6,6.0){\makebox(0,0)[c]{各種パッケージへの対応}}
  \end{picture}
  \end{center}
\end{frame}

\subsection{「エンジン拡張」部分・plain \TeX 対応}
\begin{frame}
  \frametitle{「エンジン拡張」部分・plain \TeX 対応}
  \begin{flushleft}
    \large\bfseries
    \color{linenavy!50!black}概ね実装完了,テスト段階.
  \end{flushleft}\vskip-\medskipamount
  \begin{itemize}
    \item 和文フォントの(欧文フォントとの)独立管理
    \item 和文文字間・和欧文間の空白挿入
    \item 禁則処理用のペナルティ挿入
    \item 欧文・和文のベースライン上下移動
    \item 和文文字直後の改行での空白挿入抑制\alert{(限定的)}
    \item \alert{縦書き関連はまだ}.また,\alert{速度が非常に遅い}.
  \end{itemize}
  しかし,細かい仕様変更はまだ行う可能性はある.
\end{frame}

\subsection{\LaTeX 対応}
\begin{frame}
  \frametitle{\LaTeX 対応}
  \begin{flushleft}
    \large\bfseries
    \color{purple!50!black}(横組みに関する)大半のp\LaTeXe 拡張を実装.
  \end{flushleft}\vskip-\medskipamount
  \begin{itemize}
    \item 和文フォントの管理(\texttt{plfonts.dtx}相当)
    \item \LaTeX カーネルへのパッチ(\texttt{plcore.dtx}相当)\\
    但し,次の変更はomitしている:
    \begin{itemize}
      \item ボトムフロートの出力順序
      \item 脚注マクロ
    \end{itemize}
    \item 日本語用クラスファイルを試験的に作成.\\
    {\small(しかし,最終的にどうなるかは未決定)}
    \begin{itemize}
     \def\ {\setbox0=\hbox{M}\hskip\wd0}
      \item \alert{\texttt{ltjclasses\ }}: \texttt{jclasses\ }%
	    \hskip\ltjgetparameter{xkanjiskip}のLua\TeX-ja対応版
      \item \alert{\texttt{ltjsclasses}}: \texttt{jsclasses}のLua\TeX-ja対応版
    \end{itemize}
  \end{itemize}
\end{frame}


\subsection{fontspec等への対応}

\begin{frame}[fragile]
  \frametitle{fontspec等への対応}
\noindent\textbf{fontspec対応}
\begin{itemize}
\item \verb+luatexja-fontspec+パッケージを使用する.
\item 和文フォント用命令は\verb+\setmainjfont+, \verb+\setsansjfont+のように「j」がつく.
\end{itemize}

\noindent\textbf{OTFパッケージの機能}
\begin{itemize}
\item \verb+luatexja-otf+パッケージを使用する.
\item 例:「\verb+\CID{8705}と高+」→\quad\CID{8705}と高
\item \verb+\CID+, \verb+\UTF+と,\texttt{ajmacros.sty}の一部機能が実装.
\end{itemize}

\end{frame}

\section{p\TeX との主要な変更点}

\subsection{命令名称・書式}
\def\.#1{{\rm\fontshape{it}\selectfont$\langle$#1$\rangle$}}
\begin{frame}
\frametitle{命令名称の変更}
{\large 殆どの組版パラメタは\texttt{|ltjsetparameter}にkey-valueリストを渡すことで設定.}

\medskip
\begin{tabular}{cll}
\toprule
\multicolumn{2}{l}{\bf 和欧文間空白}\\
\hskip1\zw代入&\tt
|ltjsetparameter\{xkanjiskip=\.{length}\}\\
\hskip1\zw取得&\tt
|ltjgetparameter\{xkanjiskip\}\sf\ (as string)\\\midrule
\multicolumn{2}{l}{\bf 禁則用ペナルティ\hss}\\
\hskip1\zw代入&\tt
|ltjsetparameter\{\\
&\tt\hskip2emprebreakpenalty=\{\.{chr\_code},\.{penalty}\}\}\\
\hskip1\zw取得&\tt
|ltjgetparameter\\
&\tt\hskip2em\{prebreakpenalty\}\{\.{chr\_code}\}\sf\ (as string)\\
\bottomrule
\end{tabular}

\end{frame}

\begin{frame}[fragile]
\frametitle{和文フォントの指定方法}

\begin{flushleft}
\tt\Large |jfont|piyo=\textcolor{green}{psft:GothicBBB-Medium}\\
\hskip5em:\textcolor{blue}{jfm=ujis};...\ \textcolor{red}{at 20pt}
\end{flushleft}

和文フォントは,次の3要素の組である:
\begin{description}[metric]\def\makelabel#1{\hbox to \labelwidth{\bf#1}}
\item[\textcolor{green}{字形}]
  OpenType/TrueTypeフォントも可.\\
  prefix \texttt{psft:}で非埋込フォントを指定可能.
\item[\textcolor{blue}{metric}] p\TeX のJFMに相当.\verb+jfm-ujis.lua+に格納.
\item[\textcolor{red}{サイズ}]\ 
\end{description}

\end{frame}

\subsection{空白挿入処理}

\newdimen\bx\bx=3.2pt
\newdimen\by
\makeatletter
\def\fw#1#2#3#4#5{\fboxsep0pt\vtop{\centering\by=#3\bx\hsize=\by
  \leavevmode\fcolorbox{#4}{#4!25!white}{%
  \hbox to\by{\fontsize{\by}{\by}\selectfont\color{black}%
  \inhibitglue#1\inhibitglue}}\par\vskip5pt#2\par#5}\ignorespaces}
\def\hw#1#2#3#4#5{\fboxsep0pt\vtop{\centering\by=#3\bx\hsize=0.5\by
  \leavevmode\fcolorbox{#4}{#4!25!white}{%
  \hbox to0.5\by{\fontsize{\by}{\by}\selectfont\color{black}%
  \inhibitglue#1\inhibitglue}}\par\vskip5pt#2\par#5}\ignorespaces}
\def\spc#1#2{\fboxsep0pt\vtop{\centering\by=#1\bx\hsize=0.5\by
  \leavevmode\color{#2}\vrule width 0.5\by height 2pt depth 2pt}\ignorespaces}
\def\bar{\vrule width 0.4pt height 5pt depth 5pt}

\begin{frame}
\frametitle{空白挿入の単位:\only<1>{p\TeX}\only<2>{\alert{Lua\TeX-ja}}の場合}

\textbf{入力例:\quad}{\large\tt
\textcolor{green}{$\underbracket{\mathstrut\hbox{)\{\}(}}_{\scriptstyle 1}$}%
\textcolor{blue}{$\underbracket{\mathstrut\hbox{)|typeout\{\}(}}_{\scriptstyle 2}$}%
\textcolor{red}{$\underbracket{\mathstrut\hbox{)\{|gt (}}_{\scriptstyle 3}$}\}}

\begin{center}
\vskip-\bigskipamount
\leavevmode\bx=3.6pt
\only<1>{%
\hw{\fontfamily{mc}\selectfont )}{jis}{10}{green}{\ }
\spc{10}{green}\bar
\spc{10}{green}
\hw{\fontfamily{mc}\selectfont (}{jis}{10}{green}{}
\hw{\fontfamily{mc}\selectfont )}{jis}{10}{blue}{}
\spc{10}{blue}
\hw{\hss\setbox0=\hbox{\rotatebox{270}{\small whatsit}}\raise.5\dp0\copy0\hss}{}{10}{blue}{}
\spc{10}{blue}
\hw{\fontfamily{mc}\selectfont (}{jis}{10}{blue}{}
\hw{\fontfamily{mc}\selectfont )}{jis}{10}{red}{}
\spc{10}{red}\bar
\spc{10}{red}
\hw{(}{jisg}{10}{red}{}
}%
\only<2>{%
\hw{\fontfamily{mc}\selectfont )}{ujis}{10}{green}{R}
\spc{10}{green}
\hw{\fontfamily{mc}\selectfont (}{ujis}{10}{green}{R}
\hw{\fontfamily{mc}\selectfont )}{ujis}{10}{blue}{R}
\hw{\hss\setbox0=\hbox{\rotatebox{270}{\small whatsit}}\raise.5\dp0\copy0\hss}{}{10}{blue}{}
\spc{10}{blue}
\hw{\fontfamily{mc}\selectfont (}{ujis}{10}{blue}{R}
\hw{\fontfamily{mc}\selectfont )}{ujis}{10}{red}{R}
\spc{10}{red}
\hw{(}{ujis}{10}{red}{G}
}
\end{center}

\vskip-\medskipamount
\vbox to 5\baselineskip{%
\only<1>{
\alert{入力ソース中で連続していなければ,空白挿入処理は分断.}\\
元来の\TeX でも,{\tt of\{\}fice}では合字は抑制される.
}%
\only<2>{
\begin{enumerate}
\item \alert{水平リスト内に寄与しないものは無視}\\
 Lua\TeX でも,{\tt of\{\}fice}では合字は抑制されない
\item 行分割に影響しないものも無視
\item 例え\textcolor{green}{字形}が異なっても,
\textcolor{blue}{metric}と\textcolor{red}{サイズ}が同じならば,
空白挿入処理では同じフォントとして扱われる\\
(違うフォントとして扱うことも設定により可能)
\end{enumerate}
}}
\end{frame}

\begin{frame}
\frametitle{異フォントの文字:p\TeX の場合}

\pTeX では,異なるフォントの文字間には,\\
両者のJFM由来の空白が(両方別々に)入る:

\medskip

\leavevmode
\vtop{\parindent=0pt\hsize=40pt\ \par\vskip5pt JFM\par size}
\fw{\fontfamily{mc}\selectfont あ}{jis}{10}{green}{10}
\hw{\fontfamily{mc}\selectfont 〗}{\textcolor{green}{jis}}{10}{green}{10}
\spc{10}{green}\bar\spc{10}{blue}
\hw{〖}{\textcolor{blue}{jisg}}{10}{blue}{10}
\fw{い}{jisg}{10}{blue}{10}
\hw{】}{jisg}{10}{blue}{\textcolor{blue}{10}}
\spc{10}{blue}\bar\spc{14.4}{red}
\hw{【}{jisg}{14.4}{red}{\textcolor{red}{14.4}}
\fw{う}{jisg}{14.4}{red}{14.4}

\begin{itemize}
\item \leavevmode
\smash{\spc{10}{green}\bar\spc{10}{blue}\hskip7.04pt}%
:$\textcolor{green}{5\,\text{pt\footnotesize (左側由来)}}
+\textcolor{blue}{5\,\text{pt\footnotesize (右側由来)}}=10\,\text{pt}$
\item \leavevmode
\smash{\spc{10}{blue}\bar\spc{14.4}{red}}%
:$\textcolor{blue}{5\,\text{pt\footnotesize (左側由来)}}
+\textcolor{red}{7.2\,\text{pt\footnotesize (右側由来)}}=12.2\,\text{pt}$
\end{itemize}
\end{frame}
\begin{frame}
\frametitle{異フォントの文字:\alert{Lua\TeX-ja} の場合}

Lua\TeX-jaにおいて,異なるフォントの文字間には,\\
両者のmetric由来の空白の\alert{平均}値が入る{\small(設定で変更可)\inhibitglue\hbox{}}:

\medskip


\leavevmode
\vtop{\parindent=0pt\hsize=40pt\ \par\vskip5pt metric\par size}
\fw{\fontfamily{mc}\selectfont あ}{ujis$'$}{10}{green}{10}
\hw{\fontfamily{mc}\selectfont〗}{\textcolor{green}{ujis$'$}}{10}{green}{10}
\spc{10}{green!50!blue}
\hw{〖}{\textcolor{blue}{ujis}}{10}{blue}{10}
\fw{い}{ujis}{10}{blue}{10}
\hw{】}{ujis}{10}{blue}{\textcolor{blue}{10}}
\spc{12.2}{purple}
\hw{【}{ujis}{14.4}{red}{\textcolor{red}{14.4}}
\fw{う}{ujis}{14.4}{red}{14.4}

\begin{itemize}
\item \leavevmode
\smash{\spc{10}{blue!50!green}\hskip3.52pt}%
:$(\textcolor{green}{5\,\text{pt\footnotesize (左側由来)}}
+\textcolor{blue}{5\,\text{pt\footnotesize (右側由来)}})/2=5\,\text{pt}$
\item \leavevmode
\smash{\spc{12.2}{purple}}%
:$(\textcolor{blue}{5\,\text{pt\footnotesize (左側由来)}}
+\textcolor{red}{7.2\,\text{pt\footnotesize (右側由来)}})/2=6.1\,\text{pt}$
\end{itemize}
\end{frame}

\subsection{注意}
\begin{frame}[fragile]
\frametitle{和文文字直後の改行}
{\bf\large Lua\TeX の仕様により,ad hocな実装}

\medskip
改行による空白が抑制されるかは,\\
\alert{その行を入力から読み込む前}の内部状態で決まる.

\bigskip
\textbf{入力例}:\hskip2\zw\unitlength=1\zw
{\color{green}\begin{picture}(0,0)\thicklines\put(0,0.38){\vector(0,-1){1.1}}\end{picture}%
「ひらがな他を欧文扱いにする」}
\fbox{\vbox{\tt
{\color{green}|ltjsetparameter\{jacharrange=\{-6\}\}}xあ\\
y
}}

→出力は「xy」となる(\alert{行末空白は入らない}).

∵1行目を入力から読み込む時点で,「あ」は和文文字扱い.
\end{frame}
\section*{まとめ}
\begin{frame}
\frametitle{まとめ}
\large Lua\TeX-ja は,
\begin{itemize}
\item 日本語組版をLua\TeX 上で行うパッケージ.
\item p\TeX をかなり意識しているが,\\ 100\% 互換とはならない.
\item \LaTeX, fontspec用コードが試験的に整備され,ある程度は使える.
しかし,バグが埋まっている可能性ありなので,使用には注意.
\end{itemize}
\end{frame}


\begin{frame}
\frametitle{Lua\TeX-ja プロジェクトについて}
\begin{itemize}
\item \textbf{公式ページ}\\
\url{http://sourceforge.jp/projects/luatex-ja/wiki/FrontPage}
\item まだ安定版のリリースはない.\\スナップショットがダウンロード可能.
\item \textbf{開発メンバー}
\begin{itemize}
\item 北川 弘典
\item 前田 一貴
\item 八登 崇之
\item 黒木 裕介
\item 阿部 紀行
\item 本田 知亮
\item 山本 宗宏
\end{itemize}
\end{itemize}
\end{frame}
\end{document}
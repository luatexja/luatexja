\documentclass{ltjarticle}
\pdfgentounicode=1
\pdfglyphtounicode{ff}{0066 0066}
\pdfglyphtounicode{fi}{0066 0069}
\pdfglyphtounicode{fl}{0066 006C}
\pdfglyphtounicode{ffi}{0066 0066 0069}
\pdfglyphtounicode{ffl}{0066 0066 006C}
\DeclareYokoKanjiEncoding{ZH}{}{}
\DeclareKanjiEncodingDefaults{}{}
\DeclareErrorKanjiFont{ZH}{song}{m}{n}{10}
\DeclareKanjiSubstitution{ZH}{song}{m}{n}
\newcommand\songdefault{song}
\newcommand\heidefault{hei}
\newcommand\minchodefault{mincho}
\renewcommand\kanjiencodingdefault{ZH}
\renewcommand\kanjifamilydefault{\songdefault}
\renewcommand\kanjiseriesdefault{\mddefault}
\renewcommand\kanjishapedefault{\updefault}
\DeclareKanjiFamily{ZH}{song}{}
\DeclareFontShape{ZH}{song}{m}{n}{<->name:AdobeSongStd-Light:jfm=jis}{}
\DeclareFontShape{ZH}{song}{m}{it}{<->name:AdobeSongStd-Light:jfm=jis}{}
\DeclareFontShape{ZH}{song}{bx}{n}{<->ssub*hei/m/n}{}
\DeclareKanjiFamily{ZH}{hei}{}
\DeclareFontShape{ZH}{hei}{m}{n}{<->name:AdobeHeitiStd-Regular:jfm=jis}{}
\DeclareFontShape{ZH}{hei}{bx}{n}{<->name:AdobeHeitiStd-Regular:jfm=jis}{}
\DeclareKanjiFamily{ZH}{mincho}{}
\DeclareFontShape{ZH}{mincho}{m}{n}{<->name:KozMinPr6N-Regular:jfm=jis}{}
\fontencoding{ZH}\selectfont
\DeclareTextFontCommand{\textsong}{\songfamily}
\DeclareTextFontCommand{\texthei}{\heifamily}
\DeclareTextFontCommand{\textmincho}{\minchofamily}
\DeclareOldFontCommand{\song}{\normalfont\songfamily}{}
\DeclareOldFontCommand{\hei}{\normalfont\heifamily}{}
\DeclareOldFontCommand{\mincho}{\normalfont\minchofamily}{}
\DeclareSymbolFont{songti}{ZH}{song}{m}{n}
\jfam\symsongti
\SetSymbolFont{songti}{bold}{ZH}{hei}{m}{n}
\DeclareSymbolFontAlphabet{\mathsong}{songti}
\DeclareMathAlphabet{\mathhei}{ZH}{hei}{m}{n}
\makeatletter
\DeclareRobustCommand\songfamily{\not@math@alphabet\songfamily\mathsong\kanjifamily\songdefault\selectfont}
\DeclareRobustCommand\heifamily{\not@math@alphabet\heifamily\mathhei\kanjifamily\heidefault\selectfont}
\DeclareRobustCommand\minchofamily{\not@math@alphabet\minchofamily\mathhei\kanjifamily\minchodefault\selectfont}
\DeclareRobustCommand\rmfamily{\not@math@alphabet\rmfamily\mathrm\romanfamily\rmdefault\kanjifamily\songdefault\selectfont}
\DeclareRobustCommand\sffamily{\not@math@alphabet\sffamily\mathsf\romanfamily\sfdefault\kanjifamily\heidefault\selectfont}
\makeatother
\protected\def\Pkg#1{\underline{\smash{\texttt{#1}}}}
\def\postpartname{编}
\DeclareRobustCommand\eTeX{\ensuremath{\varepsilon}-\kern-.125em\TeX}
\DeclareRobustCommand\LuaTeX{Lua\TeX}
\DeclareRobustCommand\pdfTeX{pdf\TeX}
\DeclareRobustCommand\pTeX{p\kern-.05em\TeX}
\DeclareRobustCommand\upTeX{up\kern-.05em\TeX}
\DeclareRobustCommand\pLaTeX{p\kern-.05em\LaTeX}
\DeclareRobustCommand\pLaTeXe{p\kern-.05em\LaTeXe}
\DeclareRobustCommand\epTeX{\ensuremath{\varepsilon}-\kern-.125em\pTeX}
\usepackage[twoside,left=23mm,width=170mm,right=17mm,top=25mm,height=231mm,bottom=32mm]{geometry}
\usepackage{amsmath,amssymb,xcolor,pict2e,multienum,amsthm,float,makecell,booktabs,multicol,showexpl}
\usepackage{luatexja-otf}
\usepackage{luatexja-fontspec}
\usepackage[all]{xy}
\usepackage{hyperref}
\hypersetup{%
	unicode,
	bookmarksnumbered,
	bookmarksopen,
	colorlinks,
	allbordercolors=1 1 1,
	allcolors=blue,
	pdfauthor={LuaTeX-ja团队},
	pdftitle={LuaTeX-ja宏包}
}

\lstset{
  basicstyle=\ttfamily\small, pos=r, breaklines=true,
  numbers=none, rframe={}, basewidth=0.5em
}
\title{\LuaTeX-ja宏包}
\author{\LuaTeX-ja项目团队}
\date{}
\begin{document}
\maketitle
\tableofcontents
\part{用户手册}
\section{引言}
\LuaTeX-ja宏包是应用于\LuaTeX引擎上的高质量日语文档排版宏包。
\subsection{开发背景}
一般情况下,\TeX下的日语文档输出,是ASCII \pTeX(\TeX的一个扩展)及其衍生软件来完成的。
\pTeX作为\TeX的一个扩展引擎,在生成高质量的日语文档时,规避了繁杂的宏编写。
但是在和同时期的引擎相比之下,\pTeX的处境未免有些尴尬:\pTeX已经远远落后于\eTeX和pdf\TeX,此外也没有跟上计算机上对日文处理的演进(比如,UTF-8编码,TrueType字体,OpenType字体)。

最近开发的\pTeX扩展,即\upTeX(Unicode下的\pTeX实现)和\epTeX(\pTeX和\eTeX的融合版本),虽然在部分情况上弥补了上述的差距,但是差距依然存在。

不过,\LuaTeX的出现改变了整个状况。用户可以通过使用Lua语言的“callback”来调整\LuaTeX的内部处理机制。所以,没有必要去通过修改引擎的源代码来支持日文排版,相反,我们需要做的仅仅是编写其当处理callback的Lua脚本。
\subsection{与\pTeX的差异所在}
\LuaTeX-ja宏包在设计上,受\pTeX影响很大。最初开发的主要议题是实现\pTeX的特性。
不过,{\bf\LuaTeX-ja不是简简单单的移植\pTeX,很多不自然的特征和现象都被移出了。}

下面列举出了一些和\pTeX的差异:
\begin{itemize}
\item 一个日文字体是由三部分构成的元组:实际的字体(如小塚明朝,IPA明朝),日文字体测度(\textbf{JFM})和变体字串。
\item \pTeX中,日文字符之后的断行并不允许(也不产生空格),其他在源码中的断行是可以随处允许的。不过,因为\LuaTeX的特殊关系,\LuaTeX-ja并没有这个功能。
\item 插在日文字符和其他字符言之间的胶/出格(我们将此称为\textbf{JAglue})是重现实现的。
	\begin{itemize}
	\item 在\LuaTeX中,内部的字符处理是“基于node的”(例如:\verb!of{}fice!不会避免合字),
		\textbf{JAglue}的插入处理,现在也是“基于node的”。
	\item 此外,两个字符之间的node在断行时不起作用的(例如,\verb!\special!node),还有
		意大利体校正带来的出格在插入处理中也是被忽略的。
	\item \textbf{警告:鉴于以上两点,在\pTeX中分割\textbf{JAglue}处理的多种方法不再生效。}
		明确地说,下列两种方法不再生效:
		\begin{verbatim}
		  ちょ{}っと  ちょ\/っと
		\end{verbatim}
		如果想得到此种结果,请使用空盒子替代:
		\begin{verbatim}
		  ちょ\hbox{}っと
		\end{verbatim}
	\item 处理过程中,两个在“真实”字体上具区别的日文字体可以被识别出来。
	\end{itemize}
\item 当下,\LuaTeX-ja并不支持直行排版。
\end{itemize}
\subsection{一些约定} 
在本文档中,有下面一些约定:
\begin{itemize}
\item 所有的日文字符为\textbf{JAchar},所有的其他字符为\textbf{ALchar}
\item primitive,该词在本文档中不仅表示\LuaTeX的基本控制命令,也包括\LuaTeX-ja的相关的基本控制命令
\item 所有的自然数从0开始
\item 本文档中文文档是根据日文文档翻译并添加部分材料组织而成的,所以在大部分篇幅上还是主要依赖于日文原文档
\end{itemize}
\subsection{关于本项目}
\paragraph{项目wiki} 本项目wiki正在不断编写中。
\begin{itemize}
\item \url{http://sourceforge.jp/projects/luatex-ja/wiki/FrontPage%28en%29}(英文)
\item \url{http://sourceforge.jp/projects/luatex-ja/wiki/FrontPage}(日文)
\item \url{http://sourceforge.jp/projects/luatex-ja/wiki/FrontPage%28zh%29}(中文)
\end{itemize}
\paragraph{开发者}
\begin{multienumerate}
\def\labelenumi{$\bullet$}
\mitemxxx{北川弘典}{前田一贵}{八登崇之}
\mitemxxx{黒木裕介}{阿部纪行}{山本宗宏}
\mitemxxx{本田知亮}{斋藤修三郎}{马起园}
\end{multienumerate}
\section{初次使用}
\subsection{安装}
在安装\LuaTeX-ja之前,请确认:
\begin{itemize}
\item \LuaTeX(版本号为大于0.65)和相关支持宏包。\\
如果用户使用的是\TeX Live2011以及最新版本的W32\TeX,都可以不考虑此项。
\item \LuaTeX-ja的源码:)
\item \Pkg{xunicode}宏包,当前版本必须为\textit{v0.981(2011/09/09)}。\\
如果你使用\Pkg{fontspec}宏包,\Pkg{xunicode}必须存在。但是请注意该包版本,其他版本可能不会正常工作。
\end{itemize}
安装方法如下:
\begin{enumerate}
\item 用下列诸多方法下载源码归档。当前,\LuaTeX-ja并无\textbf{稳定}版本。
	\begin{itemize}
	\item 从git中复制\\
		\verb!$ git clone git://git.sourceforge.jp/gitroot/luatex-ja/luatexja.git!
	\item 下载\texttt{master}分支HEAD的\texttt{tar.gz}归档\\
		\url{http://git.sourceforge.jp/view?p=luatex-ja/luatexja.git;a=snapshot;h=HEAD;sf=tgz}
	\item 现在\LuaTeX-ja已经包含于CTAN中(在\texttt{macros/luatex/generic/luatexja}文件夹)和W32\TeX中(\texttt{luatexja.tar.gz})。这些版本基于\texttt{master}版本。
	\end{itemize}
\item 解压归档。你将看到\texttt{src/}和其他文件夹。但是只有\texttt{src/}文件夹的内容是\LuaTeX-ja宏包运行所需。
\item 将\texttt{src/}文件夹内容复制到\texttt{TEXMF}树下某个位置。\texttt{TEXMF/tex/luatex/luatexja/}是一个例子。如果你克隆了全部的Git仓库,将\texttt{src/}做成符号链接来代替复制也不错。
\item 如果需要升级文件名数据库,请执行\texttt{mktexlsr}。
\end{enumerate}
\subsection{警告}
\begin{itemize}
\item 你的源文件编码必须是UTF-8。其他编码不行,例如EUC-JP或者Shift-JIS。
\end{itemize}
\subsection{plain \TeX格式下的使用}
在plain \TeX下使用\LuaTeX-ja相当简易,在文档开头放置一行:
\begin{verbatim}
\input luatexja.sty
\end{verbatim}

这里做出了做小的日文文档排版设定(如\texttt{ptex.tex}):
\begin{itemize}
\item 提前加载了六种日文字体,如下:
\begin{center}
\begin{tabular}{ccccc}
\toprule
\textbf{字体}&\textbf{字体名}&\bf `10\,pt'&\bf`7\,pt'&\bf`5\,pt'\\\midrule
明朝体&Ryumin-Light    &\verb+\tenmin+&\verb+\sevenmin+&\verb+\fivemin+\\
哥特体&GothicBBB-Medium&\verb+\tengt+ &\verb+\sevengt+ &\verb+\fivegt+\\
\bottomrule
\end{tabular}
\end{center}
	\begin{itemize}
	\item “Q(级)”是日本照排中使用的尺寸单位,$1 \mathrm{Q} = 0.25\mathrm{mm}$。该长度保存在长度\verb!\jQ!中。
	\item 广为接受的“Ryumin-Light”和“GothicBBB-Medium”字体不嵌入PDF文件,而PDF阅读器则会使用外部日文字体替代(例如,在Adobe Reader中使用Kozuka Mincho字体替代Ryumin-Light)。我们使用默认设定。
	\item 一般情况下,相同大小日文字体比西文字体要大一下。所以实际的日文字体尺寸需哟小于西文字体,即使用一个缩放率:0.962216。
	\end{itemize}
\item 在\textbf{JAchar}和\textbf{ALchar}之间插入的胶(\textsf{xkanjiskip}参数)大小为:
$$(0.25\times0.962216\times\mathrm{10pt})^{\mathrm{+1pt}}_{\mathrm{-1pt}} = 2.40554\mathrm{pt}^{\mathrm{+1pt}}_{\mathrm{-1pt}}$$
\end{itemize}
\subsection{在\LaTeX下使用}
\paragraph{\LaTeXe} 在\LaTeXe下使用基本相同。设定日文的最小环境,你只需加载\texttt{luatexja.sty}:
\begin{verbatim}
\usepackage{luatexja}
\end{verbatim}

这些做了最小的设定(作用相当于\pLaTeX中的\verb!plfonts.dtx!和\verb!pldefs.ltx!):
\begin{itemize}
\item \texttt{JY3}是日文字体编码(在水平方向)。\\
	在将来\LuaTeX-ja要支持直行排版的时候,\texttt{JT3}会用于直行字体。
\item 定义了两个字体族:\verb!mc!和\verb!gt!:\\
\begin{center}
	\begin{tabular}{ccccc}
	\hline
	\textbf{字体}&\textbf{字体族}&\verb!\mdseries!&\verb!\bfseries!&缩放率\\
	\hline
	\textit{mincho}&\verb!mc!&Ryumin-Light&GothicBBB-Medium&0.962216\\
	\textit{gothic}&\verb!gt!&GothicBBB-Medium&GothicBBB-Medium&0.962216\\
	\hline
	\end{tabular}
\end{center}
注意的是两个字体族的粗体系列同为中等系列的\textbf{哥特}族。这\pLaTeX中的规定。在近些年中的DTP实务中有仅使用2个字体的趋向(是为Ryumin-Light和GothicBBB-Medium)。
\item 在数学模式下,日文字符使用\verb!mc!字体族来排印
\end{itemize}

不过,上述设定并不能满足排版基于日文的文档。为了排印基于日文的文档,你最好不要使用\texttt{article.cls},\texttt{book.cls}等文档类文件。
现在,我们有相当于\Pkg{jclasses}(\pLaTeX标准文档类)和\Pkg{jsclasses}(奥村晴彦)的文档类,即\Pkg{ltjclasses}和\Pkg{ltjsclasses}。

\paragraph{OTF包中的\texttt{\char92CID},\texttt{\char92UTF}及其他宏} \pLaTeX下,\underline{\texttt{otf}}宏包(斋藤修三郎开发)是用来排印存在于Adobe-Japan1-6但不存在于JIS X 0208中的字符。
该包已经广泛使用,\LuaTeX-ja支持部分\Pkg{otf}包中的部分功能。
如果你想使用这些功能,加载\Pkg{luatexja-otf}宏包。
\bgroup
\mincho
\begin{LTXexample}
森\UTF{9DD7}外と内田百\UTF{9592}とが\UTF{9AD9}島屋に行く。

\CID{7652}飾区の\CID{13706}野家,
葛飾区の吉野家
\end{LTXexample}
\egroup
\subsection{改变字体}
\paragraph{注记:数学模式下的日文字符} \pTeX支持在数学模式下的日文字符,如以下源码:
\begin{LTXexample}
$f_{高温}$~($f_{\text{high temperature}}$).
\[ y=(x-1)^2+2\quad よって\quad y>0 \]
$5\in 素:=\{\,p\in\mathbb N:\text{$p$ is a prime}\,\}$.
\end{LTXexample}

我们(\LuaTeX-ja项目成员)认为在数学模式下使用日文字符,只有在这些字符充当标识符时才是正确的。在这点下,
\begin{itemize}
\item 第1行和第2行是不正确的,因为“高温”的作用为文本标签,“よって”用作为连词。
\item 不过,第3行是正确的,因为“素”是作为标识符的。
\end{itemize}

那么,根据我们的观点,上述输入应当校正为:
\begin{LTXexample}
$f_{\text{高温}}$~%
($f_{\text{high temperature}}$).
\[ y=(x-1)^2+2\quad
  \mathrel{\text{よって}}\quad y>0 \]
$5\in 素:=\{\,p\in\mathbb N:\text{$p$ is a prime}\,\}$.
\end{LTXexample}

我们也认为使用日文字符作为标识符的情况极为少见,所以我们不在此章节描述如何在数学模式下改变日文字体。
关于此方法,请参见。
\paragraph{plain \TeX} 在plain \TeX下改变日文字体,你必须使用基本语句\verb!\jfont!。请参见。

\paragraph{NFSS2} 对于\LaTeXe,\LuaTeX-ja采用了\pLaTeXe中(即\texttt{plfonts.dtx})大部分字体选择系统。
\begin{itemize}
\item \verb!\mcdefault!和\verb!\gtdefault!控制语句用来分别控制默认的\textit{mincho}和\textit{gothic}字体族。
	默认值:\texttt{mc}用于\verb!\mcdefault!,\texttt{gt}用于\verb!\gtdefault!。
\item 命令\verb!\fontfamily!,\verb!\fontseries!,\verb!\fontshape!个\verb!\selectfont!用来改变日文字体属性。
\begin{center}
\begin{tabular}{cccccc}
\toprule
&\textbf{编码}&\textbf{族}&\textbf{系列}&\textbf{形状}&\textbf{选择}\\\midrule
西文字体
&\verb+\romanencoding+&\verb+\romanfamily+&\verb+\romanseries+&\verb+\romanshape+
&\verb+\useroman+\\
日文字体
&\verb+\kanjiencoding+&\verb+\kanjifamily+&\verb+\kanjiseries+&\verb+\kanjishape+
&\verb+\usekanji+\\
两者&---&--&\verb+\fontseries+&\verb+\fontshape+&---\\
自动选择&\verb+\fontencoding+&\verb+\fontfamily+&---&---&\verb+\usefont+\\
\bottomrule
\end{tabular}
\end{center}
\verb!\fontencoding{<encoding>}!依赖于参数以改变西文字体或者日文字体。
例如,\verb!\fontencoding{JY3}!改变当前日文字体至\texttt{JY3},\verb!\fontencoding{T1}!改变西文字体至\texttt{T1}。
\verb!\fontfamily!也会改变日文字体或西文字体的族,抑或二者。
细节详见。
\item 对于定义日文字体族,使用\verb!\DeclareKanjiFamily!代替\verb!\DeclareFontFamily!。
不过,在现在的实现中,使用\verb!\DeclareFontFamily!不会引起任何问题。
\end{itemize}

\subsection{fontspec}
为与\Pkg{fontspec}宏包共存,需要在导言区中使用\Pkg{luatexja-fontspec}宏包。
这个附加宏包会自动加载\Pkg{luatexja}和\Pkg{fontspec}。

在\texttt{luatexja-fontspec},定义了如下七条命令,这些命令和\texttt{fontspec}的相关命令对比如下:
\begin{center}
	\begin{tabular}{ccccc}
	\hline
	日文字体&\verb+\jfontspec+&\verb+\setmainjfont+&\verb+\setsansjfont+&\verb+\newjfontfamily+\\
	西文字体&\verb+\fontspec+&\verb+\setmainfont+&\verb+\setsansfont+&\verb+\newfontfamily+\\
	\hline
	日文字体&\verb+\newjfontface+&\verb+\defaultjfontfeatures+&\verb+\addjfontfeatures+&\\
	西文字体&\verb+\newfontface+&\verb+\defaultfontfeatures+&\verb+\addfontfeatures+&\\
	\hline
	\end{tabular}
\end{center}
\bgroup
\begin{LTXexample}
\fontspec[Numbers=OldStyle]{TeX Gyre Termes}
\jfontspec{IPAexMincho}
JIS~X~0213:2004→辻

\addjfontfeatures{CJKShape=JIS1990}
JIS~X~0208:1990→辻
\end{LTXexample}
\egroup

请注意并没有\verb!\setmonofont!命令,因为流行的日文字体几乎全部是等宽的。
另注意,出格特性在这7个命令中默认关闭,因为此特性会与\textbf{JAglue}冲突(参见)。
\section{参数设定}
\LuaTeX-ja包含大量的参数,以控制排版细节。
设定这些参数需要使用命令:\verb!\ltjsetparameter!和\verb!\ltjgetparameter!命令。
\subsection{JAchar范围的设定}
在设定\textbf{JAchar}之前,需要分配一个小于217的自然数。如:
\begin{verbatim}
\ltjdefcharrange{100}{"10000-"1FFFF,`漢}
\end{verbatim}

请注意这个设定是全局性的,不建议在文档正文中进行设定。

在范围设定好了之后,需要进行\verb!jacharrange!的设定:
\begin{verbatim}
\ltjsetparameter{jacharrange={-1, +2, +3, -4, -5, +6, +7, +8}}
\end{verbatim}

这里定义了8个范围,在每个范围之前使用“+”或“-”进行设定,其中如果为$-$,则代表该范围为\textbf{ALchar},如果为$+$,则该范围视作\textbf{JAchar}。

\LuaTeX-ja默认设定了8个范围,这些范围来源于下列数据:
\begin{itemize}
\item Unicode 6.0
\item Adobe-Japan1-6与Unicode之间的映射\verb!Adobe-Japan1-UCS2!
\item 八登崇之的up\TeX宏包:\verb!PXbase!
\end{itemize}

\begin{description}
\item[范围 $\mathbf{8^J}$] ISO 8859-1(Latin-1补充)的上半部和JIS X 0208(日文基本字符集)的重叠部分,包含下列字符:
\begin{multicols}{2}
	\begin{itemize}
	\def\ch#1#2{\item \char"#1\ (\texttt{U+00#1}, #2)}
	\ch{A7}{Section Sign}
	\ch{A8}{Diaeresis}
	\ch{B0}{Degree sign}
	\ch{B1}{Plus-minus sign}
	\ch{B4}{Spacing acute}
	\ch{B6}{Paragraph sign}
	\ch{D7}{Multiplication sign}
	\ch{F7}{Division Sign}
	\end{itemize}
\end{multicols}

\item[范围 $\mathbf{1^A}$] 包含于Adobe-Japan1-6中的拉丁字符,此范围包含下列Unicode区域,但不包括上述提到过的范围8:
\begin{multicols}{2}
	\begin{itemize}
	\item \texttt{U+0080}--\texttt{U+00FF}: 拉丁字母补充-1
	\item \texttt{U+0100}--\texttt{U+017F}: 拉丁字母扩充-A
	\item \texttt{U+0180}--\texttt{U+024F}: 拉丁字母扩充-B
	\item \texttt{U+0250}--\texttt{U+02AF}: 国际音标扩充
	\item \texttt{U+02B0}--\texttt{U+02FF}: 进格修饰符元
	\item \texttt{U+0300}--\texttt{U+036F}: 组合音标附加符号
	\item \texttt{U+1E00}--\texttt{U+1EFF}: 拉丁字母扩充附加
	\end{itemize}
\end{multicols}

\item[范围 $\mathbf{2^J}$] 希腊文和西里尔字母,使用JIS X 0208的大部分日文字体包含这些字符:
\begin{multicols}{2}
	\begin{itemize}
	\item \texttt{U+0370}--\texttt{U+03FF}: 希腊字母
	\item \texttt{U+0400}--\texttt{U+04FF}: 西里尔字母	
	\item \texttt{U+1F00}--\texttt{U+1FFF}: 希腊文扩充
	\end{itemize}
\end{multicols}

\item[范围 $\mathbf{3^J}$] 标点以及杂项符号:
\begin{multicols}{2}
	\begin{itemize}
	\item \texttt{U+2000}--\texttt{U+206F}: 一般标点符号
	\item \texttt{U+2070}--\texttt{U+209F}: 上标及下标
	\item \texttt{U+20A0}--\texttt{U+20CF}: 货币符号
	\item \texttt{U+20D0}--\texttt{U+20FF}: 符号用组合附加符号
	\item \texttt{U+2100}--\texttt{U+214F}: 类字母符号
	\item \texttt{U+2150}--\texttt{U+218F}: 数字形式
	\item \texttt{U+2190}--\texttt{U+21FF}: 箭头符号
	\item \texttt{U+2200}--\texttt{U+22FF}: 数学运算符号
	\item \texttt{U+2300}--\texttt{U+23FF}: 杂项技术符号
	\item \texttt{U+2400}--\texttt{U+243F}: 控制图像
	\item \texttt{U+2500}--\texttt{U+257F}: 制表符
	\item \texttt{U+2580}--\texttt{U+259F}: 区块元素
	\item \texttt{U+25A0}--\texttt{U+25FF}: 几何形状
	\item \texttt{U+2600}--\texttt{U+26FF}: 杂项符号
	\item \texttt{U+2700}--\texttt{U+27BF}: 什锦符号
	\item \texttt{U+2900}--\texttt{U+297F}: 补充性箭头-B
	\item \texttt{U+2980}--\texttt{U+29FF}: 混合数学符号-B
	\item \texttt{U+2B00}--\texttt{U+2BFF}: 杂项符号和箭头符号
	\item \texttt{U+E000}--\texttt{U+F8FF}: 私用区域
	\end{itemize}
\end{multicols}

\item[范围 $\mathbf{4^A}$] 通常情况下不包含于日文字体的部分。本范围包含有其他范围尚未涵盖部分。故,我们直接给出定义:
\begin{verbatim}
\ltjdefcharrange{4}{%
   "500-"10FF, "1200-"1DFF, "2440-"245F, "27C0-"28FF, "2A00-"2AFF, 
  "2C00-"2E7F, "4DC0-"4DFF, "A4D0-"A82F, "A840-"ABFF, "FB50-"FE0F, 
  "FE20-"FE2F, "FE70-"FEFF, "FB00-"FB4F, "10000-"1FFFF} % non-Japanese
\end{verbatim}

\item[范围 $\mathbf{5^A}$] 代替以及补充私有使用区域。
\item[范围 $\mathbf{6^J}$] 日文字符。
\begin{multicols}{2}
	\begin{itemize}
	\item \texttt{U+2460}--\texttt{U+24FF}: 圈状字母数字
	\item \texttt{U+2E80}--\texttt{U+2EFF}: CJK部首补充
	\item \texttt{U+3000}--\texttt{U+303F}: CJK标点符号
	\item \texttt{U+3040}--\texttt{U+309F}: 平假名
	\item \texttt{U+30A0}--\texttt{U+30FF}: 片假名
	\item \texttt{U+3190}--\texttt{U+319F}: 汉文标注号
	\item \texttt{U+31F0}--\texttt{U+31FF}: 片假名音标补充
	\item \texttt{U+3200}--\texttt{U+32FF}: 圈状CJK字母及月份
	\item \texttt{U+3300}--\texttt{U+33FF}: CJK兼容
	\item \texttt{U+3400}--\texttt{U+4DBF}: CJK统一表意文字扩充A		\item \texttt{U+4E00}--\texttt{U+9FFF}: CJK统一表意文字
	\item \texttt{U+F900}--\texttt{U+FAFF}: CJK兼容表意文字
	\item \texttt{U+FE10}--\texttt{U+FE1F}: 直行标点
	\item \texttt{U+FE30}--\texttt{U+FE4F}: CJK兼容形式
	\item \texttt{U+FE50}--\texttt{U+FE6F}: 小写变体
	\item \texttt{U+20000}--\texttt{U+2FFFF}: (补充字符)
	\end{itemize}
\end{multicols}

\item[范围 $\mathbf{7^J}$] 不包含于Adobe-Japan1-6的CJK字符。
\begin{multicols}{2}
	\begin{itemize}
	\item \texttt{U+1100}--\texttt{U+11FF}: 谚文字母
	\item \texttt{U+2F00}--\texttt{U+2FDF}: 康熙部首
	\item \texttt{U+2FF0}--\texttt{U+2FFF}: 汉字结构描述字符
	\item \texttt{U+3100}--\texttt{U+312F}: 注音字母
	\item \texttt{U+3130}--\texttt{U+318F}: 谚文兼容字母
	\item \texttt{U+31A0}--\texttt{U+31BF}: 注音字母扩充
	\item \texttt{U+31C0}--\texttt{U+31EF}: CJK笔划
	\item \texttt{U+A000}--\texttt{U+A48F}: 彝文音节
	\item \texttt{U+A490}--\texttt{U+A4CF}: 彝文字母
	\item \texttt{U+A830}--\texttt{U+A83F}: 一般印度数字
	\item \texttt{U+AC00}--\texttt{U+D7AF}: 谚文音节
	\item \texttt{U+D7B0}--\texttt{U+D7FF}: 谚文字母扩充-B
	\end{itemize}
\end{multicols}
\end{description}
\subsection{\textsf{kanjiskip}和\textsf{xkanjiskip}}
\textbf{JAglue}分为下列三类范畴:
\begin{itemize}
\item JFM设定的胶或出格值。如果在一个日文字符附近使用\verb!\inhibitglue!,则胶便不会插入。
\item 两个\textbf{JAchar}之间默认插入的胶(\textsf{kanjiskip})
\item \textbf{JAchar}和\textbf{ALchar}之间默认插入的胶(\textsf{xkanjiskip})
\end{itemize}

\textsf{kanjiskip}和\textsf{xkanjiskip}的设定如下所示:
\begin{verbatim}
\ltjsetparameter{kanjiskip={0pt plus 0.4pt minus 0.4pt},
                 xkanjiskip={0.25\zw plus 1pt minus 1pt}}
\end{verbatim}

当JFM包含“\textsf{kanjiskip}理想宽度”和/或“\textsf{xkanjiskip}理想宽度”数据时,上述设定产生作用。如果想用JFM中的数据,请设定\textsf{kanjiskip}或\textsf{xkanjiskip}为\verb!\maxdimen!。
\subsection{\textsf{xkanjiskip}插入设定}
并不是在所有的\textbf{JAchar}和\textbf{ALchar}周围插入\textsf{xkanjiskip}都是合适的。
比如,在开标点之后插入\textsf{xkanjiskip}并不合适[如,比较“(あ”和“(\hskip\ltjgetparameter{xkanjiskip}あ”]。
\LuaTeX-ja可以通过设定\textbf{JAchar}的\textsf{jaxspmode}以及\textbf{ALchar}的\textsf{alxspmode}来控制\textsf{xkanjiskip}在字符前后的插入。

\begin{LTXexample}
\ltjsetparameter{jaxspmode={`あ,preonly}, alxspmode={`\!,postonly}}
pあq い!う
\end{LTXexample}

第二个参数\textsf{preonly}表示的含义为“允许在该字符前插入\textsf{xkanjiskip},但不允许在该字符之后插入”。
其他参数还有\textsf{postonly},\textsf{allow}和\textsf{inhibit}。[TODO]

用户如果想开启/关闭\textsf{kanjiskip}和\textsf{xkanjiskip}的插入,设定\textsf{autospacing}和\textsf{autoxspacing}参数为\textsf{true}/\textsf{false}即可。
\subsection{基线浮动}
为了确保日文字体和西文字体能够对其,有时需要浮动其中一者的基线。
在\pTeX中,此项设定由设定\verb!\yabaselineshift!为非零长度(西文字体基线应向下浮动)。
不过,如果文档的中主要语言不是日文,那么最好上浮日文字体的基线,西文字体不变。
如上所述,\LuaTeX-ja可以独立设定西文字体的基线(\textsf{yabaselineshift}参数)和日文字体的基线(\textsf{yjabaselineshift}参数)。

\begin{LTXexample}
\vrule width 150pt height 0.4pt depth 0pt\hskip-120pt
\ltjsetparameter{yjabaselineshift=0pt, yalbaselineshift=0pt}abcあいう
\ltjsetparameter{yjabaselineshift=5pt, yalbaselineshift=2pt}abcあいう
\end{LTXexample}

上述水平线为此行基线。

这里还有一个有趣的副作用:不同大小的字符可以通过适当调整这两个参数而在一行中垂直居中。下面是一个例子(注意,参数值并没有精心调整):

\begin{LTXexample}
xyz漢字
{\scriptsize
  \ltjsetparameter{yjabaselineshift=-1pt,
    yalbaselineshift=-1pt}
  XYZひらがな
}abcかな
\end{LTXexample}
\subsection{裁剪框标记}
裁剪框标记是在一页的四角和水平/垂直中央放置的标记。在日文中,裁剪框被称为“トンボ”。
\pLaTeX和\LuaTeX-ja均在底层支持裁剪框标记。需要下列步骤来实现:
\begin{enumerate}
\item 首先,首先定义页面左上角将会出现的注记。这由向\verb!@bannertoken!分配一个token列完成。\\
例如,下列所示将会设定注记为“\textsf{filename (YYYY-MM-DD hh:mm)}”:

\begin{verbatim}
\makeatletter

\hour\time \divide\hour by 60 \@tempcnta\hour \multiply\@tempcnta 60\relax
\minute\time \advance\minute-\@tempcnta
\@bannertoken{%
   \jobname\space(\number\year-\two@digits\month-\two@digits\day
   \space\two@digits\hour:\two@digits\minute)}%
\end{verbatim}

\item {[TODO]}
\end{enumerate}
\part{参考指南}
\section{字体测度和日文字体}
\subsection{\texttt{\char92jfont}基本语句}
为了加载日文字体,需要使用\verb!\jfont!基本语句替代\verb!\font!,前者支持后者所有相同句法。
\LuaTeX-ja自动加载\Pkg{luaotfload}宏包,故TrueType/OpenType字体的特性可以使用于日文字体:
\begin{LTXexample}
\jfont\tradgt={file:ipaexg.ttf:script=latn;%
  +trad;-kern;jfm=ujis} at 14pt
\tradgt{}当/体/医/区
\end{LTXexample}

注意定义的控制序列(上例中的\verb!\tradgt!)使用的\verb!\jfont!并不是一个\textit{font\_def}标记,故类似\verb!\fontname\tradgt!输入会引起错误。
我们指出定义于\verb!\jfont!控制序列>
\paragraph{JFM}
\paragraph{注:kern特性}
\subsection{\texttt{\char92psft}前缀}
除使用\texttt{file:}和\texttt{name:}外,我们还可以在\verb!\jfont!(以及\verb!\font!)中使用\texttt{psft:}来设定一个“名义上”的并不嵌入PDF中的日文字体。
此前缀的典型使用是定义“标准”日文字体,即“Ryumin-Light”和“GothicBBB-Medium”。
\paragraph{\texttt{cid}键} 默认使用\texttt{psft:}前缀定义的字体是为Adobe-Japan1-6 CID字体。
也可以使用\texttt{cid}键来使用其他的CID字体,如中文和韩文。
\begin{lstlisting}[numbers=left]
\jfont\testJ={psft:Ryumin-Light:cid=Adobe-Japan1-6;jfm=jis}     % 日语
\jfont\testD={psft:Ryumin-Light:jfm=jis}                        % 默认值为Adobe-Japan1-6
\jfont\testC={psft:AdobeMingStd-Light:cid=Adobe-CNS1-5;jfm=jis} % 繁体中文
\jfont\testG={psft:SimSun:cid=Adobe-GB1-5;jfm=jis}              % 简体中文
\jfont\testK={psft:Batang:cid=Adobe-Korea1-2;jfm=jis}           % 韩文
\end{lstlisting}
注意上述代码使用了\texttt{jfm-jis.lua},这可以用于日文字体,也可以用于中文和韩文字体。

当下,\LuaTeX-ja只支持如上例中的4个值,改为其他值,例如:
\begin{lstlisting}
\jfont\test={psft:Ryumin-Light:cid=Adobe-Japan2;jfm=jis}
\end{lstlisting}
会发生下列错误:
\begin{lstlisting}[numbers=left]
! Package luatexja Error: bad cid key `Adobe-Japan2'.

See the luatexja package documentation for explanation.
Type  H <return>  for immediate help.
<to be read again>
                   \par
l.78

? h
I couldn't find any non-embedded font information for the CID
`Adobe-Japan2'. For now, I'll use `Adobe-Japan1-6'.
Please contact the LuaTeX-ja project team.
?
\end{lstlisting}

%\subsection{JFM文件结构}
%\subsection{数学字体族}
%\subsection{Callbacks}
%\part{实现细节}
%\section{Lua\TeX-ja与Lua\TeX相关阅读材料}
%\begin{itemize}
%\item Lua\TeX官方主页:\verb!http://www.luatex.org!
%\item Lua\TeX\ SVN:\verb!http://foundry.supelec.fr/gf/project/luatex/!
%\item Lua\TeX:\verb!http://ja.wikipedia.org/wiki/LuaTeX!
%\item Lua\TeX-ja官方主页:\verb!http://en.sourceforge.jp/projects/luatex-ja/!
%\item p\TeX官方主页:\verb!http://ascii.asciimw.jp/pb/ptex/!
%\item Publishing \TeX:\verb!http://ja.wikipedia.org/wiki/PTeX!
%\item Vertical typesetting in \TeX:\verb!http://tug.org//TUGboat/Articles/tb11-3/tb29hamano.pdf!
%\item up\TeX官方主页:\verb!http://homepage3.nifty.com/ttk/comp/tex/uptex.html!
%\item “LuaTeX で日本語”:\verb!http://oku.edu.mie-u.ac.jp/tex/mod/forum/discuss.php?d=627!
%\item luajalayout宏包:\verb!http://www-is.amp.i.kyoto-u.ac.jp/lab/kmaeda/lualatex/luajalayout/!
%\item luafontcomp宏包:\verb!http://www-is.amp.i.kyoto-u.ac.jp/lab/kmaeda/lualatex/luafontcomp/!
%\item 思わずLuaでLaTeXしてみた:\verb!http://zrbabbler.sp.land.to/lualatexlua.html!
%\item luaums.sty:\verb!http://oku.edu.mie-u.ac.jp/tex/mod/forum/discuss.php?d=378!
%\item koTeXは日本語LuaTeXへの先導役になるか?:\\\verb!http://oku.edu.mie-u.ac.jp/tex/mod/forum/discuss.php?d=485!
%\end{itemize}
\end{document}

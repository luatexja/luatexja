%#! lualatex -shell-escape manual.ins

%<en>\documentclass[a4paper,titlepage]{article}
%<ja>\documentclass[a4paper,titlepage]{ltjsarticle}
\usepackage[margin=20mm,footskip=5mm]{geometry}

\usepackage{amsmath,amssymb,xcolor,pict2e,multienum,amsthm,float}
\usepackage{booktabs,listings,showexpl,multicol}
\usepackage{luatexja-otf}
\usepackage{luatexja-fontspec}
\usepackage[unicode]{hyperref}
\usepackage[all]{xy}
\SelectTips{cm}{}

\DeclareRobustCommand\eTeX{\ensuremath{\varepsilon}-\kern-.125em\TeX}
\DeclareRobustCommand\LuaTeX{Lua\TeX}
\DeclareRobustCommand\pdfTeX{pdf\TeX}
\DeclareRobustCommand\pTeX{p\kern-.05em\TeX}
\DeclareRobustCommand\upTeX{up\kern-.05em\TeX}
\DeclareRobustCommand\pLaTeX{p\kern-.05em\LaTeX}
\DeclareRobustCommand\pLaTeXe{p\kern-.05em\LaTeXe}
\DeclareRobustCommand\epTeX{\ensuremath{\varepsilon}-\kern-.125em\pTeX}

\ltjsetparameter{jacharrange={-3}}
\theoremstyle{definition}
%<en>\newtheorem{defn}{Definition}
%<ja>\newtheorem{defn}{定義}

\newenvironment{cslist}{%
  \leftskip2em\parindent=0pt\def\makelabel##1{{\tt\char92##1}}
  \def\{{\char`\{}\def\}{\char`\}}
  \let\origitem=\item
  \def\item[##1]{\par\smallskip\par\hskip-\leftskip\makelabel{##1}\par}
}{}

\makeatletter
\long\def\@makecaption#1#2{%
  \vskip\abovecaptionskip
  \sbox\@tempboxa{{\small #1. #2}}%
  \ifdim \wd\@tempboxa >\hsize
    {\small #1. #2}\par
  \else
    \global \@minipagefalse
    \hb@xt@\hsize{\hfil\box\@tempboxa\hfil}%
  \fi
  \vskip\belowcaptionskip}
\makeatother

%<*en>
\title{The \LuaTeX-ja package}
\author{The \LuaTeX-ja project team}
%</en>
%<*ja>
\title{\LuaTeX-jaパッケージ}
\author{\LuaTeX-jaプロジェクトチーム}
%</ja>

\lstset{
  basicstyle=\ttfamily\small, pos=o, breaklines=true,
  numbers=none, rframe={}, basewidth=0.5em
}

\parskip=\smallskipamount
\protected\def\Param#1{\textsf{#1}} % parameter name
\protected\def\Pkg#1{\underline{\smash{\texttt{#1}}}} % packages/classes

\begin{document}
\catcode`\<=13
\def<#1>{{\normalfont\rm\itshape$\langle$#1$\rangle$}}
\maketitle

\tableofcontents
\bigskip

%<*en>
\textbf{This documentation is far from complete. It may have many
grammatical (and contextual) errors.}
%</en>
%<*ja>
\textbf{\large 本ドキュメントはまだまだ未完成です.}
%</ja>

\clearpage
%<en>\part{User's manual}
%<ja>\part{ユーザーズマニュアル}

%<en>\section{Introduction}
%<ja>\section{はじめに}

%<*en>
The \LuaTeX-ja package is a macro package for typesetting high-quality
Japanese documents when using \LuaTeX.
%</en>
%<*ja>
\LuaTeX-jaパッケージは,次世代標準\TeX である\LuaTeX の上で,\pTeX と同等
/それ以上の品質の日本語組版を実現させようとするマクロパッケージである.
%</ja>

%<en>\subsection{Backgrounds}
%<ja>\subsection{背景}

%<*en>
Traditionally, ASCII \pTeX, an extension of \TeX, and its derivatives
are used to typeset Japanese documents in \TeX. \pTeX\ is an engine
extension of \TeX: so it can produce high-quality Japanese documents
without using very complicated macros. But this point is a mixed
blessing: \pTeX\ is left behind from other extensions of \TeX,
especially \eTeX\ and \pdfTeX, and from changes about
Japanese processing in computers (\textit{e.g.}, the UTF-8 encoding).
%</en>
%<*ja>
従来,「\TeX を用いて日本語組版を行う」といったとき,エンジンとしては
ASCII \pTeX やそれの拡張物が用いられることが一般的であった.\pTeX は\TeX
のエンジン拡張であり,(少々仕様上不便な点はあるものの)商業印刷の分野に
も用いられるほどの高品質な日本語組版を可能としている.だが,それは弱点に
もなってしまった:\pTeX という(組版的に)満足なものがあったため,海外で
行われている数々の\TeX の拡張──例えば\eTeX や\pdfTeX ──や,TrueType,
OpenType, Unicodeといった計算機で日本語を扱う際の状況の変化に追従すること
を怠ってしまったのだ.
%</ja>

%<*en>
Recently extensions of \pTeX, namely \upTeX\ (Unicode-implementation
of \pTeX) and \epTeX\ (merging of \pTeX\ and
\eTeX\ extension), have developed to fill those gaps to some
extent, but gaps still exist.
%</en>
%<*ja>
ここ数年,若干状況は改善されてきた.現在手に入る大半の\pTeX バイナリでは
外部UTF-8入力が利用可能となり,さらにUnicode化を推進し,\pTeX の内部処理
までUnicode化した\upTeX も開発されている.また,\pTeX に\eTeX 拡張をマー
ジした\epTeX も登場し,\TeX\ Live\ 2011では\pLaTeX が\epTeX の上で動作す
るようになった.だが,\pdfTeX 拡張(PDF 直接出力やmicro-typesetting)を
\pTeX に対応させようという動きはなく,海外とのgapは未だにあるのが現状であ
る.
%</ja>

%<*en>
However, the appearance of \LuaTeX\ changed the whole situation. With
using Lua `callbacks', users can customize the internal processing of
\LuaTeX. So there is no need to modify sources of engines to
support Japanese typesetting: to do this, we only have to write Lua
scripts for appropriate callbacks.
%</en>
%<*ja>
しかし,\LuaTeX の登場で,状況は大きく変わることになった.Luaコードで
`callback'を書くことにより,\LuaTeX の内部処理に割り込みをかけることが可
能となった.これは,エンジン拡張という真似をしなくても,Luaコードとそれに
関する\TeX マクロを書けば,エンジン拡張とほぼ同程度のことができるようになっ
たということを意味する.\LuaTeX-jaは,このアプローチによってLuaコード・
\TeX マクロによって日本語組版を\LuaTeX の上で実現させようという目的で開発
が始まったパッケージである.
%</ja>

%<en>\subsection{Major Changes from \pTeX}
%<ja>\subsection{\pTeX からの主な変更点}

%<*en>
The \LuaTeX-ja package is under much influence of \pTeX\ engine. The initial
target of development was to implement features of \pTeX. However,
\emph{\LuaTeX-ja is not a just porting of \pTeX; unnatural
specifications/behaviors of \pTeX\ were not adopted}.
%</en>
%<*ja>
\LuaTeX-jaは,\pTeX に多大な影響を受けている.初期の開発目標は,\pTeX の機
能をLuaコードにより実装することであった.しかし,開発が進むにつれ,\pTeX
の完全な移植は不可能であり,また\pTeX における実装がいささか不可解になっ
ているような状況も発見された.そのため,\textbf{\LuaTeX-ja は,もはや
\pTeX の完全な移植は目標とはしない.\pTeX における不自然な仕様・挙動があ
れば,そこは積極的に改める.}
%</ja>

%<*en>
The followings are major changes from \pTeX:
%</en>
%<*ja>
以下は \pTeX からの主な変更点である.
%</ja>

\begin{itemize}
%<*en>
\item A Japanese font is a tuple of a `real' font, a Japanese font
      metric (\textbf{JFM}, for short), and an optional string called
      `variation'.
%</en>
%<*ja>
\item 和文フォントは「実際の」フォント,和文フォントメトリック(JFM と呼ぶ),そ
      して `variation' と呼ばれる文字列の組である.
%</ja>

%<*en>
\item In \pTeX, a line break after Japanese character is ignored (and
      doesn't yield a space), since line breaks (in source files) are
      permitted almost everywhere in Japanese texts. However, \LuaTeX-ja
      doesn't have this function completely, because of a specification
      of \LuaTeX.
%</en>
%<*ja>
\item 日本語の文書中では改行はほとんどどこでも許されるので,\pTeX では和文文字
      直後の改行は無視される(スペースが入らない)ようになっていた.しかし,
      \LuaTeX-ja では \LuaTeX の仕様のためにこの機能は完全には実装されていない.
%</ja>

%<*en>
\item The insertion process of glues/kerns between two Japanese
      characters and between a Japanese character and other characters
      (we refer these glues/kerns as \textbf{JAglue}) is rewritten from
      scratch.
%</en>
%<*ja>
\item 2つの和文文字の間,和文文字と欧文文字の間に入るグルー/カーン
      (\textbf{JAglue} と呼ぶ)の挿入処理が0から書き直されている.
%</ja>

\begin{itemize}
%<*en>
\item As \LuaTeX's internal character handling is `node-based'
      (\textit{e.g.}, \verb+of{}fice+ doesn't prevent ligatures), the
      insertion process of \textbf{JAglue} is now `node-based'.
%</en>
%<*ja>
\item \LuaTeX の内部での文字の扱いが「ノードベース」になっているように(例えば,
      \verb+of{}fice+ で合字は抑制されない),\textbf{JAglue} の挿入処理も
      「ノードベース」である.
%</ja>

%<*en>
\item Furthermore, nodes between two characters which have no effects in
      line break (\textit{e.g.}, \verb+\special+ node) and kerns from
      italic correction are ignored in the insertion process.
%</en>
%<*ja>
\item さらに,2つの文字の間にある行末では効果を持たないノード(例えば
      \verb+\special+ ノード)や,イタリック補正に伴い挿入されるカーンは
      挿入処理中では無視される.
%</ja>

%<*en>
\item \emph{Caution: due to above two points, many methods which did the
      dividing the process of the insertion of \textbf{JAglue} in \pTeX\ are not
      effective anymore.} In concrete terms, the following two methods are not effective anymore:
%</en>
%<*ja>
\item \textbf{注意:上の2つの変更により,従来 \textbf{JAglue} の挿入処理を分断するのに
      使われていたいくつかの方法は用いることができない.具体的には,次の方法はもはや無効である:}
%</ja>
\begin{verbatim}
  ちょ{}っと  ちょ\/っと
\end{verbatim}
%<*en>
      If you want to do so, please put an empty hbox between it instead:
%</en>
%<*ja>
      もし同じことをやりたければ,空のhboxを間に挟めばよい:
%</ja>
\begin{verbatim}
  ちょ\hbox{}っと
\end{verbatim}


%<*en>
\item In the process, two Japanese fonts which only differ in their `real'
      fonts are identified.
%</en>
%<*ja>
\item 処理中では,2つの和文フォントは,「実際の」フォントのみが異なる場合に同一視
      される.
%</ja>
\end{itemize}

%<*en>
\item At the present, vertical typesetting (\emph{tategaki}), is not
      supported in \LuaTeX-ja.
%</en>
%<*ja>
\item 現時点では,縦書きは \LuaTeX-ja ではサポートされていない.
%</ja>

\end{itemize} 

%<*en>
For detailed information, see Part~\ref{part-imp}.
%</en>
%<*ja>
詳細については第 \ref{part-imp} 部を見よ.
%</ja>

%<en>\subsection{Notations}
%<ja>\subsection{用語と記法}

%<*en>
In this document, the following terms and notations are used:
%</en>
%<*ja>
本ドキュメントでは,以下の用語と記法を用いる:
%</ja>

\begin{itemize}
%<*en>
\item Characters are divided into two types: 
%</en>
%<*ja>
\item 文字は2種類に分けられる:
%</ja>
\begin{itemize}
%<*en>
\item \textbf{JAchar}: standing for Japanese characters such as
      Hiragana, Katakana, Kanji and other punctuation marks for
      Japanese.
%</en>
%<*ja>
\item \textbf{JAchar}: ひらがな,カタカナ,漢字,和文用の約物といった和文文字の
      ことを指す.
%</ja>

%<*en>
\item \textbf{ALchar}: standing for all other characters like alphabets. 
%</en>
%<*ja>
\item \textbf{ALchar}: アルファベットを始めとする,その他全ての文字を指す.
%</ja>
\end{itemize}

%<*en>
We say `alphabetic fonts' for fonts used in \textbf{ALchar}, and `Japanese fonts' for fonts used in \textbf{JAchar}.
%</en>
%<*ja>
そして,\textbf{ALchar} の出力に用いられるフォントを「欧文フォント」と呼び,
\textbf{JAchar} の出力に用いられるフォントを「和文フォント」と呼ぶ.
%</ja>

%<*en>
\item A word in a sans-serif font (like \Param{prebreakpenalty})
      means an internal parameter for Japanese typesetting, and it
      is used as a key in \verb+\ltjsetparameter+ command.
%</en>
%<*ja>
\item サンセリフ体で書かれた語(例:\Param{prebreakpenalty})は日本語組版用の
      パラメータを表し,これらは \verb+\ltjsetparameter+ コマンドのキーとして
      用いられる.
%</ja>

%<*en>
\item A word in typewriter font with underline (like \Pkg{fontspec})
      means a package or a class of \LaTeX.
%</en>
%<*ja>
\item 下線付きのタイプライタ体で書かれた語(例:\Pkg{fontspec})は \LaTeX の
      パッケージやクラスを表す.
%</ja>

%<*en>
\item The word `primitive' is used not only for primitives in \LuaTeX,
      but also for control sequences that defined in the core module of
      \LuaTeX-ja.
%</en>
%<*ja>
\item 「プリミティブ」という語を,\LuaTeX のプリミティブだけではなく \LuaTeX-ja
      のコアモジュールで定義されたコントロールシーケンスに対しても用いる.
%</ja>

%<*en>
\item In this document, natural numbers start from~0.
%</en>
%<*ja>
\item 本ドキュメントでは,自然数は0から始まる.
%</ja>
\end{itemize}

%<en>\subsection{About the project}
%<ja>\subsection{プロジェクトについて}

%<*en>
\paragraph{Project Wiki} Project Wiki is under construction.
%</en>
%<*ja>
\paragraph{プロジェクト Wiki} プロジェクト Wiki は構築中である.
%</ja>

%<*en>
\begin{itemize}
\item \url{http://sourceforge.jp/projects/luatex-ja/wiki/FrontPage%28en%29} (English)
\item \url{http://sourceforge.jp/projects/luatex-ja/wiki/FrontPage} (Japanese)
\end{itemize}
%</en>
%<*ja>
\begin{itemize}
\item \url{http://sourceforge.jp/projects/luatex-ja/wiki/FrontPage}(日本語)
\item \url{http://sourceforge.jp/projects/luatex-ja/wiki/FrontPage%28en%29}(英語)
\end{itemize}
%</ja>

%<*en>
This project is hosted by SourceForge.JP.
%</en>
%<*ja>
本プロジェクトは SourceForge.JP のサービスを用いて運営されている.
%</ja>

%<en>\paragraph{Members}\ 
%<ja>\paragraph{開発メンバー}\ 
%<*en>
\begin{multienumerate}
\def\labelenumi{$\bullet$}
\mitemxxx{Hironori KITAGAWA}{Kazuki MAEDA}{Takayuki YATO}
\mitemxxx{Yusuke KUROKI}{Noriyuki ABE}{Munehiro YAMAMOTO}
\mitemxxo{Tomoaki HONDA}{Shuzaburo SAITO}
\end{multienumerate}
%</en>
%<*ja>
\begin{multienumerate}
\def\labelenumi{$\bullet$}
\mitemxxx{北川 弘典}{前田 一貴}{八登 崇之}
\mitemxxx{黒木 裕介}{阿部 紀行}{山本 宗宏}
\mitemxxo{本田 知亮}{齋藤 修三郎}
\end{multienumerate}
%</ja>


% \paragraph{Acknowledgments} -- 挿入するならここ

\clearpage
%<en>\section{Getting Started}
%<ja>\section{使い方}
%<en>\subsection{Installation}
%<ja>\subsection{インストール}

%<*en>
To install the \LuaTeX-ja\ package, you will need:
%</en>
%<*ja>
\LuaTeX-ja パッケージのインストールには,次のものが必要である.
%</ja>
\begin{itemize}
%<*en>
\item \LuaTeX\ (version 0.65.0-beta or later) and its supporting packages.\\
If you are using \TeX~Live~2011 or current W32\TeX, you don't have to worry.
%</en>
%<*ja>
\item \LuaTeX(バージョン 0.65.0-beta 以降)とその支援パッケージ.
      \TeX~Live~2011 や W32\TeX の最新版ならば問題ない.
%</ja>
%<*en>
\item The source archive of \LuaTeX-ja, of course\texttt{:)}
%</en>
%<*ja>
\item \LuaTeX-ja のソースアーカイブ(もちろん \texttt{:)}).
%</ja>
\end{itemize}

%<*en>
The installation methods are as follows:
%</en>
%<*ja>
インストール方法は以下のようになる:
%</ja>
\begin{enumerate}
%<*en>
\item Download the source archive.

      At the present, \LuaTeX-ja has no official release, so you have to retrieve
      the archive from the repository.
      You can retrieve the Git repository via
%</en>
%<*ja>
\item ソースアーカイブをダウンロードする.

      現時点では,\LuaTeX-ja の公式リリースはないので,レポジトリからアーカイブを
      取得しなければならない.
      次のようにすることで,Git レポジトリを取得することができる:
%</ja>
\begin{verbatim}
$ git clone git://git.sourceforge.jp/gitroot/luatex-ja/luatexja.git
\end{verbatim} 
%<*en>
      or download the archive of HEAD in \texttt{master} branch from
%</en>
%<*ja>
      もしくは,\texttt{master} ブランチの HEAD アーカイブを以下からダウンロードすることができる:
%</ja>
\begin{flushleft}
\url{http://git.sourceforge.jp/view?p=luatex-ja/luatexja.git;a=snapshot;h=HEAD;sf=tgz}.
\end{flushleft}

%<*en>
Note that the forefront of development may not be in \texttt{master} branch.
%</en>
%<*ja>
開発中の最新の成果は \texttt{master} ブランチには含まれていないかもしれないことに注意.
%</ja>

%<*en>
\item Extract the archive. You will see \texttt{src/} and several other sub-directories.
%</en>
%<*ja>
\item アーカイブを展開する.\texttt{src/} をはじめとしたいくつかのディレクトリができる.
%</ja>

%<*en>
\item Copy all the contents of \texttt{src/} into one of your \texttt{TEXMF} tree.
%</en>
%<*ja>
\item \texttt{src/} の中身全てを自分の \texttt{TEXMF} ツリーにコピーする.
%</ja>

%<*en>
\item If \texttt{mktexlsr} is needed to update the file name database, make it so.
%</en>
%<*ja>
\item もし \texttt{mktexlsr} を実行する必要があればそうする.
%</ja>
\end{enumerate}

%<en>\subsection{Cautions}
%<ja>\subsection{注意点}

\begin{itemize}
%<*en>
\item The encoding of your source file must be UTF-8. No other
      encodings, such as EUC-JP or Shift-JIS, are not supported.
%</en>
%<*ja>
\item 原稿のソースファイルの文字コードは UTF-8 でなければならない.
      EUC-JP や Shift-JIS は使用できない.
%</ja>

%<*en>
\item May be conflict with other packages. 

      For example, the default setting of \textbf{JAchar} in the present
      version does not coexist with the \Pkg{unicode-math}
      package. Putting the following line in preamble makes that
      mathematical symbols will be typeset correctly, but several
      Japanese characters will be treated as an \textbf{ALchar} as
      side-effect:
%</en>
%<*ja>
\item いくつかのパッケージと衝突する.

      例えば,\textbf{JAchar} の範囲の設定がデフォルトのままだと,現行の
      バージョンでは \Pkg{unicode-math} パッケージと共存できない.
      以下の行をプリアンブルに追加することで数学記号が正しく出るように
      なるが,副作用としていくつかの和文文字が \textbf{ALchar} として
      扱われるようになってしまう:
%</ja>
\begin{verbatim}
\ltjsetparameter{jacharrange={-3, -8}}
\end{verbatim}
\end{itemize}

%<en>\subsection{Using in plain \TeX}
%<ja>\subsection{plain \TeX で使う}
\label{ssec-plain}

%<*en>
To use \LuaTeX-ja in plain \TeX, simply put the following at the beginning of the document:
%</en>
%<*ja>
\LuaTeX-ja を plain \TeX で使うためには,単に次の行をソースファイルの冒頭に追加すればよい:
%</ja>
\begin{verbatim}
\input luatexja.sty
\end{verbatim}
%<*en>
This does minimal settings (like \texttt{ptex.tex}) for typesetting Japanese documents:
%</en>
%<*ja>
これで(\texttt{ptex.tex} のように)日本語組版のための最低限の設定がなされる:
%</ja>

\begin{itemize}
%<*en>
\item The following 6~Japanese fonts are preloaded:
\begin{center}
\begin{tabular}{ccccc}
\toprule
\textbf{classification}&\textbf{font name}&\bf `10\,pt'&\bf`7\,pt'&\bf`5\,pt'\\\midrule
\emph{mincho}&Ryumin-Light    &\verb+\tenmin+&\verb+\sevenmin+&\verb+\fivemin+\\
\emph{gothic}&GothicBBB-Medium&\verb+\tengt+ &\verb+\sevengt+ &\verb+\fivegt+\\
\bottomrule
\end{tabular}
\end{center}
%</en>
%<*ja>
\item 以下の6つの和文フォントが定義される:
\begin{center}
\begin{tabular}{ccccc}
\toprule
\textbf{字体}&\textbf{フォント名}&\bf `10\,pt'&\bf`7\,pt'&\bf`5\,pt'\\\midrule
\emph{明朝体}&Ryumin-Light    &\verb+\tenmin+&\verb+\sevenmin+&\verb+\fivemin+\\
\emph{ゴシック体}&GothicBBB-Medium&\verb+\tengt+ &\verb+\sevengt+ &\verb+\fivegt+\\
\bottomrule
\end{tabular}
\end{center}
%</ja>
\begin{itemize}
%<*en>
\item The `Q(級)' is a unit used in Japanese phototypesetting, and
      $1\,\textrm{Q}=0.25\,\textrm{mm}$. This length is stored in a
      dimension \verb+\jQ+.
%</en>
%<*ja>
\item `Q(級)' は日本の写植で用いられる単位で,$1\,\textrm{Q}=0.25\,\textrm{mm}$
      である.この長さは \verb+\jQ+ に保持されている.
%</ja>

%<*en>
\item It is widely accepted that the font `Ryumin-Light' and
      `GothicBBB-Medium' aren't embedded into PDF files, and PDF reader
      substitute them by some external Japanese fonts (\textit{e.g.},
      Kozuka Mincho is used for Ryumin-Light in Adobe Reader). We adopt this custom to
      the default setting.
%</en>
%<*ja>
\item `Ryumin-Light' と `GothicBBB-Medium' は PDF ファイルに埋め込まずに
      名前参照のみで用いることが広く受け入れられており,この場合 PDF リーダーが
      適切な外部フォントで代用する(例えば,Adobe Reader では Ryumin-Light は
      小塚明朝で代替される).そこで,これらをデフォルトのフォントと
      して採用する.
%</ja>

%<*en>
\item A character in an alphabetic font is generally smaller than a
      Japanese font in the same size. So actual size specification of
      these Japanese fonts is in fact smaller than that of alphabetic
      fonts, namely scaled by 0.962216.
%</en>
%<*ja>
\item 欧文フォントの文字は和文フォントの文字よりも,同じ文字サイズでも
      一般に小さい.そこで,これらの和文フォントの実際のサイズは指定された値よりも
      小さくなるように設定されており,具体的には指定の 0.962216 倍にスケールされる.
%</ja>
\end{itemize}

%<*en>
\item The amount of glue that are inserted between a \textbf{JAchar} and
      an \textbf{ALchar} (the parameter \Param{xkanjiskip}) is set to
%</en>
%<*ja>
\item \textbf{JAchar} と \textbf{ALchar} の間に入るグルー (\Param{xkanjiskip}) の
      量は次のように設定されている:
%</ja>
\[
 (0.25\cdot 0.962216\cdot 10\,\mathrm{pt})^{+1\,\text{pt}}_{-1\,\text{pt}} 
 = 2.40554\,\mathrm{pt}^{+1\,\text{pt}}_{-1\,\text{pt}}.
\]
\end{itemize}

%<en>\subsection{Using in \LaTeX}
%<ja>\subsection{\LaTeX で使う}
\label{ssec-ltx}

\paragraph{\LaTeXe}

%<*en>
Using in \LaTeXe\ is basically same. To set up the minimal environment
for Japanese, you only have to load \texttt{luatexja.sty}:
%</en>
%<*ja>
\LaTeXe を用いる場合も基本的には同じである.日本語組版のための最低限の環境を
設定するためには,\texttt{luatexja.sty} を読み込むだけでよい:
%</ja>
\begin{verbatim}
\usepackage{luatexja}
\end{verbatim}
%<*en>
It also does minimal settings (counterparts in \pLaTeX\ are \texttt{
plfonts.dtx} and \texttt{pldefs.ltx}):
%</en>
%<*ja>
これで \pLaTeX の \texttt{plfonts.dtx} と \texttt{pldefs.ltx} に相当する最低限の設定が
なされる:
%</ja>

\begin{itemize}
%<*en>
\item \texttt{JY3} is the font encoding for Japanese fonts (in horizontal direction).\\
When vertical typesetting is supported by \LuaTeX-ja in the future, \texttt{JT3} will be used for vertical fonts.
%</en>
%<*ja>
\item \texttt{JY3} は和文フォント用のフォントエンコーディングである(横書き用).\\
将来的に,\LuaTeX-ja で縦書きがサポートされる際には,\texttt{JT3} を縦書き用として
用いる予定である.
%</ja>

%<*en>
\item Two font families \texttt{mc} and \texttt{gt} are defined: 
\begin{center}
\begin{tabular}{ccccc}
\toprule
\textbf{classification}&\textbf{family}&\verb+\mdseries+&\verb+\bfseries+&\textbf{scale}\\\midrule
\emph{mincho}&\tt mc&Ryumin-Light    &GothicBBB-Medium&0.962216\\
\emph{gothic}&\tt gt&GothicBBB-Medium&GothicBBB-Medium&0.962216\\
\bottomrule
\end{tabular}
\end{center}
Remark that the bold series in both family are same as the medium series of \emph{gothic} family.
This is a convention in \pLaTeX. This is a trace that there were only 2~fonts (these are Ryumin-Light 
and GothicBBB-Medium) in early years of DTP.
%</en>
%<*ja>
\item 2つのフォントファミリ \texttt{mc} と \texttt{gt} が定義されている:
\begin{center}
\begin{tabular}{ccccc}
\toprule
\textbf{字体}&\textbf{ファミリ}&\verb+\mdseries+&\verb+\bfseries+&\textbf{スケール}\\\midrule
明朝体&\tt mc&Ryumin-Light    &GothicBBB-Medium&0.962216\\
ゴシック体&\tt gt&GothicBBB-Medium&GothicBBB-Medium&0.962216\\
\bottomrule
\end{tabular}
\end{center}
どちらのファミリにおいても,その bold シリーズはゴシック体の medium シリーズで
あることに注意.これは初期の DTP において和文フォントが2つ(それがちょうど\ 
Ryumin-Light, GothicBBB-Medium だった)しか利用できなかった時の名残であり,\pLaTeX{}
での標準設定とも同じである.
%</ja>

%<*en>
\item Japanese characters in math mode are typeset by the font family \texttt{mc}.
%</en>
%<*ja>
\item 数式モード中の和文文字は \texttt{mc} ファミリで出力される.
%</ja>
\end{itemize}

%<*en>
However, above settings are not sufficient for Japanese-based
documents. To typeset Japanese-based documents, you are better to use
class files other than \texttt{article.cls}, \texttt{book.cls}, and so on.  At
the present, we have the counterparts of \Pkg{jclasses} (standard
classes in \pLaTeX) and \Pkg{jsclasses} (classes by Haruhiko
Okumura), namely, \Pkg{ltjclasses} and \Pkg{ltjsclasses}.
%</en>
%<*ja>
しかしながら,上記の設定は日本語の文書にとって十分とは言えない.
日本語文書を組版するためには,\texttt{article.cls}, \texttt{book.cls} といった
欧文用のクラスファイルではなく,和文用のクラスファイルを用いた方がよい.
現時点では,\Pkg{jclasses} (\pLaTeX の標準クラス)と \Pkg{jsclasses}
(奥村晴彦氏によるクラスファイル)に対応するものとして,\Pkg{ltjclasses},
\Pkg{ltjsclasses} がそれぞれ用意されている.
%</ja>

%<*en>
\paragraph{\texttt{\char92 CID, \texttt{\char92 UTF}} and macros in OTF package}
Under \pLaTeX, \Pkg{otf} package (developed by Shuzaburo Saito) is
used for typesetting characters which is in Adobe-Japan1-6 CID but not
in JIS~X~0208. Since this package is widely used, \LuaTeX-ja
supports some of functions in \Pkg{otf} package.
If you want to use these functions, load \Pkg{luatexja-otf} package.
%</en>
%<*ja>
\paragraph{\texttt{\char92 CID, \texttt{\char92 UTF}} と OTF パッケージのマクロ}
\pLaTeX では,JIS~X~0208 にない Adobe-Japan1-6 の文字を出力するために,
齋藤修三郎氏による \Pkg{otf} パッケージが用いられていた.このパッケージは
広く用いられているため,\LuaTeX-ja においても \Pkg{otf} パッケージの機能の
一部をサポートしている.これらの機能を用いるためには \Pkg{luatexja-otf} パッケージ
を読み込めばよい.
%</ja>

\begin{LTXexample}
森\UTF{9DD7}外と内田百\UTF{9592}とが\UTF{9AD9}島屋に行く。

\CID{7652}飾区の\CID{13706}野家,
葛飾区の吉野家
\end{LTXexample}
%lltjp-listings.sty要修正?:↑「森」の直後で改行.


%<en>\subsection{Changing Fonts}
%<ja>\subsection{フォントの変更}
\label{ssub-chgfnt}

%<*en>
\paragraph{Remark: Japanese Characters in Math Mode}
Since \pTeX\ supports Japanese characters in math mode, there are
sources like the following:
%</en>
%<*ja>
\paragraph{意見:数式モード中の和文文字}
\pTeX では数式中での和文文字の使用がサポートされていたので,
以下のようなソースが存在した:
%</ja>
\begin{LTXexample}
$f_{高温}$~($f_{\text{high temperature}}$).
\[ y=(x-1)^2+2\quad{}よって\quad y>0 \]
$5\in{}素:=\{\,p\in\mathbb N:\text{$p$ is a prime}\,\}$.
\end{LTXexample}
%<*en>
We (the project members of \LuaTeX-ja) think that using
Japanese characters in math mode are allowed if and only if these are used as identifiers.
In this point of view, 
%</en>
%<*ja>
\LuaTeX-ja プロジェクトでは,数式モード中での和文文字はそれらが識別子として用いられる
ときのみ許されると考えている.
この観点から,
%</ja>
\begin{itemize}
%<*en>
\item The lines 1~and~2 above are not correct, since `高温' in above is used as a textual label, and
`よって' is used as a conjunction. 
%</en>
%<*ja>
\item 上記数式のうち 1, 2 行目は正しくない.なぜならば `高温' が意味のあるラベルと
      して,`よって' が接続詞として用いられているからである.
%</ja>
%<*en>
\item However, the line~3 is correct, since `素' is used as an identifier.
%</en>
%<*ja>
\item しかしながら,3行目は `素' が識別子として用いられているので正しい.
%</ja>
\end{itemize}
%<*en>
Hence, in our opinion, the above input should be corrected as:
%</en>
%<*ja>
したがって,\LuaTeX-ja プロジェクトの意見としては,上記の入力は次のように直されるべき
である:
%</ja>
\begin{LTXexample}
$f_{\text{高温}}$~%
($f_{\text{high temperature}}$).
\[ y=(x-1)^2+2\quad 
  \mathrel{\text{よって}}\quad y>0 \]
$5\in{}素:=\{\,p\in\mathbb N:\text{$p$ is a prime}\,\}$.
\end{LTXexample}
%BUG?: \{\}がなければ「素」がでない.上の段落の「よって」もでてない.
%<*en>
We also believe that using Japanese characters as identifiers is rare,
hence we don't describe how to change Japanese fonts in math mode in
this chapter. For the method, please see Part~\ref{part-ref}.
%</en>
%<*ja>
また \LuaTeX-ja プロジェクトでは,和文文字が識別子として用いられることはほとんどない
と考えており,したがってこの章では数式モード中の和文フォントを変更する方法については
記述しない.この方法については第 \ref{part-ref} 部を参照のこと.
%</ja>

%<*en>
\paragraph{plain \TeX}
To change Japanese fonts in plain \TeX, you must use the primitive
\verb+\jfont+. So please see Part~\ref{part-ref}.
%</en>
%<*ja>
\paragraph{plain \TeX}
plain \TeX で和文フォントを変更するためには,\verb+\jfont+ プリミティブを用いなけれ
ばならない.第 \ref{part-ref} 部を参照せよ.
%</ja>

%<*en>
\paragraph{NFSS2}
For \LaTeXe, \LuaTeX-ja simply adopted the font selection system from that
of \pLaTeXe\ (in \texttt{plfonts.dtx}).
%</en>
%<*ja>
\paragraph{NFSS2}
\LaTeXe については,\LuaTeX-ja では単にフォント選択システムを \pLaTeXe\ (\texttt{plfonts.dtx})
から採用している.
%</ja>
\begin{itemize}
%<*en>
\item Two control sequences \verb+\mcdefault+ and \verb+\gtdefault+ are
      used to specify the default font families for \emph{mincho} and
      \emph{gothic}, respectively. Default values: \texttt{mc} for
      \verb+\mcdefault+ and \texttt{gt} for \verb+\gtdefault+.
%</en>
%<*ja>
\item 2つのコントロールシーケンス \verb+\mcdefault+ と \verb+\gtdefault+ がそれぞ
      れ明朝体とゴシック体のデフォルトのフォントファミリを指定するために用いられる.
      初期値:\verb+\mcdefault+ は \texttt{mc},\verb+\gtdefault+ は \texttt{gt}.
%</ja>
%<*en>
\item Commands \verb+\fontfamily+, \verb+\fontseries+,
      \verb+\fontshape+ and \verb+\selectfont+ can be used to change
      attributes of Japanese fonts. 
%</en>
%<*ja>
\item \verb+\fontfamily+, \verb+\fontseries+, \verb+\fontshape+, そして 
      \verb+\selectfont+ が和文フォントの属性を変更するために使用できる.
%</ja>

%<*en>
\begin{center}
\begin{tabular}{cccccc}
\toprule
&\textbf{encoding}&\textbf{family}&\textbf{series}&\textbf{shape}&\textbf{selection}\\\midrule
alphabetic fonts
&\verb+\romanencoding+&\verb+\romanfamily+&\verb+\romanseries+&\verb+\romanshape+
&\verb+\useroman+\\
Japanese fonts
&\verb+\kanjiencoding+&\verb+\kanjifamily+&\verb+\kanjiseries+&\verb+\kanjishape+
&\verb+\usekanji+\\
both&---&--&\verb+\fontseries+&\verb+\fontshape+&---\\
auto select&\verb+\fontencoding+&\verb+\fontfamily+&---&---&\verb+\usefont+\\
\bottomrule
\end{tabular}
\end{center}
%</en>
%<*ja>
\begin{center}
\begin{tabular}{cccccc}
\toprule
&\textbf{エンコーディング}&\textbf{ファミリ}&\textbf{シリーズ}&\textbf{シェープ}&\textbf{選択}\\\midrule
欧文フォント
&\verb+\romanencoding+&\verb+\romanfamily+&\verb+\romanseries+&\verb+\romanshape+
&\verb+\useroman+\\
和文フォント
&\verb+\kanjiencoding+&\verb+\kanjifamily+&\verb+\kanjiseries+&\verb+\kanjishape+
&\verb+\usekanji+\\
両方&---&--&\verb+\fontseries+&\verb+\fontshape+&---\\
自動選択&\verb+\fontencoding+&\verb+\fontfamily+&---&---&\verb+\usefont+\\
\bottomrule
\end{tabular}
\end{center}
%</ja>

%<*en>
      \verb+\fontencoding{<encoding>}+ changes the encoding of alphabetic fonts
      or Japanese fonts depending on the argument. For example, 
      \verb+\fontencoding{JY3}+ changes the encoding of Japanese fonts to 
      \texttt{JY3} and \verb+\fontencoding{T1}+ changes the encoding of 
      alphabetic fonts to \texttt{T1}.
      \verb+\fontfamily+ also changes the family of Japanese fonts, alphabetic
      fonts, \emph{or both}. For detail, see Subsection~\ref{ssub-nfsspat}.
%</en>
%<*ja>
      ここで,\verb+\fontencoding{<encoding>}+は,引数により和文側か欧文側かの
      どちらかが切り替わる.例えば,\verb+\fontencoding{JY3}+は和文フォントの
      エンコーディングを\texttt{JY3}に変更し,\verb+\fontencoding{T1}+ は欧文
      フォント側を\texttt{T1}へと変更する.
      \verb+\fontfamily+も引数により和文側,欧文側,\textbf{あるいは両方}のフォ
      ントファミリが切り替わる.詳細は\ref{ssub-nfsspat}節を参照すること.
%</ja>

%<*en>
\item For defining a Japanese font family, use
      \verb+\DeclareKanjiFamily+ instead of
      \verb+\DeclareFontFamily+. However, in the present implementation,
      using \verb+\DeclareFontFamily+ doesn't cause any problem.
%</en>
%<*ja>
\item 和文フォントファミリの定義には \verb+\DeclareFontFamily+ の代わりに
      \verb+\DeclareKanjiFamily+ を用いる.しかし,現在の実装では 
      \verb+\DeclareFontFamily+ を用いても問題は生じない.
%</ja>
\end{itemize}

%<*en>
\paragraph{fontspec}
To coexist with the \Pkg{fontspec} package, it is needed to load
\Pkg{luatexja-fontspec} package in the preamble. This additional
package automatically loads \Pkg{luatexja} and \Pkg{fontspec}
package, if needed.
%</en>
%<*ja>
\Pkg{fontspec} パッケージと同様の機能を和文フォントに対しても用いるためには,
\Pkg{luatexja-fontspec} パッケージをプリアンブルで読み込む必要がある.このパッケージ
は必要ならば自動で \Pkg{luatexja} パッケージと \Pkg{fontspec} パッケージを読み込む.
%</ja>

%<*en>
In \Pkg{luatexja-fontspec} package, the following 7~commands are defined as
counterparts of original commands in the \Pkg{fontspec} package:
%</en>
%<*ja>
\Pkg{luatexja-fontspec} パッケージでは,以下の7つのコマンドを \Pkg{fontspec} 
パッケージの元のコマンドに対応するものとして定義している:
%</ja>
%<*en>
\begin{center}
\begin{tabular}{ccccc}
\toprule
Japanese fonts
&\verb+\jfontspec+&\verb+\setmainjfont+&\verb+\setsansjfont+&\verb+\newjfontfamily+\\
alphabetic fonts
&\verb+\fontspec+&\verb+\setmainfont+&\verb+\setsansfont+&\verb+\newfontfamily+\\
\midrule
Japanese fonts
&\verb+\newjfontface+&\verb+\defaultjfontfeatures+&\verb+\addjfontfeatures+\\
alphabetic fonts
&\verb+\newfontface+&\verb+\defaultfontfeatures+&\verb+\addfontfeatures+\\
\bottomrule
\end{tabular}
\end{center}
%</en>
%<*ja>
\begin{center}
\begin{tabular}{ccccc}
\toprule
和文フォント
&\verb+\jfontspec+&\verb+\setmainjfont+&\verb+\setsansjfont+&\verb+\newjfontfamily+\\
欧文フォント
&\verb+\fontspec+&\verb+\setmainfont+&\verb+\setsansfont+&\verb+\newfontfamily+\\
\midrule
和文フォント
&\verb+\newjfontface+&\verb+\defaultjfontfeatures+&\verb+\addjfontfeatures+\\
欧文フォント
&\verb+\newfontface+&\verb+\defaultfontfeatures+&\verb+\addfontfeatures+\\
\bottomrule
\end{tabular}
\end{center}
%</ja>

\begin{LTXexample}
\fontspec[Numbers=OldStyle]{TeX Gyre Termes}
\jfontspec{IPAexMincho}
JIS~X~0213:2004→辻

\addjfontfeatures{CJKShape=JIS1990}
JIS~X~0208:1990→辻
\end{LTXexample}

%<*en>
Note that there is no command named \verb+\setmonojfont+, since it is
popular for Japanese fonts that nearly all Japanese glyphs have same
widths.  Also note that the kerning feature is set off by default in
these 7~commands, since this feature and \textbf{JAglue} will clash (see
\ref{para-kern}).
%</en>
%<*ja>
和文フォントについては全ての和文文字のグリフがほぼ等幅であるのが普通であるため,
\verb+\setmonojfont+ コマンドは存在しないことに注意.また,これらの和文用の7つのコマ
ンドでは Kerning feature はデフォルトでは off となっている.これはこの
feature が \textbf{JAglue} と衝突するためである(\ref{para-kern} 節を見よ).
%</ja>

%<en>\section{Changing Parameters}
%<ja>\section{パラメータの変更}

%<*en>
There are many parameters in \LuaTeX-ja. And due to the behavior of \LuaTeX,
most of them are not stored as internal register of \TeX, but as an
original storage system in \LuaTeX-ja. Hence, to assign or acquire those
parameters, you have to use commands \verb+\ltjsetparameter+ and
\verb+\ltjgetparameter+.
%</en>
%<*ja>
\LuaTeX-ja には多くのパラメータが存在する.そして \LuaTeX の振る舞いのために,
その多くは \TeX のレジスタにではなく,\LuaTeX-ja 独自の方法で保持されている.
そのため,これらのパラメータを設定・取得するためには \verb+\ltjsetparameter+ と
\verb+\ltjgetparameter+ を用いなければならない.
%</ja>

%<en>\subsection{Editing the range of \textbf{JAchar}s}
%<ja>\subsection{\textbf{JAchar} の範囲の設定}

%<*en>
To edit the range of \textbf{JAchar}s, you have to assign a non-zero
natural number which is less than 217 to the character range first. This
can be done by using \verb+\ltjdefcharrange+ primitive. For example, the
next line assigns whole characters in Supplementary Multilingual Plane
and the character `漢' to the range number~100.
%</en>
%<*ja>
\textbf{JAchar} の範囲を設定するためには,まず文字範囲に0より大きく217より小さい
自然数を割り当てる必要がある.これには \verb+\ltjdefcharrange+ プリミティブを用いる.
例えば,次のように書くことで追加多言語面 (SMP) にある全ての文字と `漢' の範囲番号が
100に設定される.
%</ja>
\begin{lstlisting}
\ltjdefcharrange{100}{"10000-"1FFFF,`漢}
\end{lstlisting}
%<*en>
This assignment of numbers to ranges are always global, so you should
not do this in the middle of a document. 
%</en>
%<*ja>
この文字範囲への番号の割り当てはいつもグローバルであり,したがって文書の途中で
この操作をするべきではない.
%</ja>

%<*en>
If some character has been belonged to some non-zero numbered range,
this will be overwritten by the new setting. For example, whole SMP
belong to the range~4 in the default setting of \LuaTeX-ja, and if you
specify the above line, then SMP will belong to the range~100 and be
removed from the range~4.
%</en>
%<*ja>
もし指定されたある文字がある非零番号の範囲に属していたならば,これは新しい設定で
上書きされる.例えば,SMP は全て \LuaTeX-ja のデフォルトでは4番の文字範囲に
属しているが,上記の指定を行えば SMP は100番に属すようになり,4番からは除かれる.
%</ja>

%<*en>
After assigning numbers to ranges, the \textsf{jacharrange} parameter can
be used to customize which character range will be treated as ranges of
\textbf{JAchar}s, as the following line (this is just the default
setting of \LuaTeX-ja):
%</en>
%<*ja>
文字範囲に番号を割り当てた後は,\textsf{jacharrange} パラメータが \textbf{JAchar} と
して扱われる文字の範囲を設定するために用いられる.例えば,以下は \LuaTeX-ja の
初期設定である:
%</ja>
\begin{verbatim}
\ltjsetparameter{jacharrange={-1, +2, +3, -4, -5, +6, +7, +8}}
\end{verbatim}
%<*en>
The argument to \textsf{jacharrange} parameter is a list of integer. 
Negative integer $-n$ in the list means that `the characters that belong to 
range~$n$ are treated as \textbf{ALchar}', and positive integer $+n$ means
that `the characters that belong to range~$n$ are treated as \textbf{JAchar}'.
%</en>
%<*ja>
\textsf{jacharrange} パラメータには整数のリストを与える.
リスト中の負の整数 $-n$ は「文字範囲 $n$ に属する文字は \textbf{ALchar} として
扱われる」ことを意味し,正の整数 $+n$ は \textbf{JAchar} として扱うことを意味する.
%</ja>

%<*en>
\paragraph{Default Setting}
\LuaTeX-ja predefines eight character ranges for convenience. They are
determined from the following data:
%</en>
%<*ja>
\paragraph{初期設定}
\LuaTeX-ja では8つの文字範囲を設定している.これらは以下のデータに基づいて決定して
いる.
%</ja>
\begin{itemize}
%<*en>
\item Blocks in Unicode~6.0.
%</en>
%<*ja>
\item Unicode~6.0 のブロック.
%</ja>
%<*en>
\item The \texttt{Adobe-Japan1-UCS2} mapping between a CID Adobe-Japan1-6 and Unicode.
%</en>
%<*ja>
\item \texttt{Adobe-Japan1-UCS2} による Adobe-Japan1-6 の CID と Unicode の間の
      マッピング.
%</ja>
%<*en>
\item The \texttt{PXbase} bundle for \upTeX\ by Takayuki Yato.
%</en>
%<*ja>
\item 八登崇之氏による \upTeX 用の \texttt{PXbase} バンドル.
%</ja>
\end{itemize}

%<*en>
Now we describe these eight ranges. The alphabet `J' or `A' after the
number shows whether characters in the range is treated as
\textbf{JAchar}s or not by default. These settings are similar to the 
\texttt{prefercjk} settings defined in \texttt{PXbase} bundle.
%</en>
%<*ja>
以下ではこれら8つの文字範囲について記述する.番号のあとのアルファベット `J' と `A'
はデフォルトで \textbf{JAchar} として扱われるかどうかを示す.これらの設定は
\texttt{PXbase} バンドルで定義されている \texttt{prefercjk} と類似のものである.
%</ja>
\begin{description}
%<*en>
\item[Range~8${}^{\text{J}}$] Symbols in the intersection of the upper half of ISO~8859-1
	 (Latin-1 Supplement) and JIS~X~0208 (a basic character set for Japanese). This character range
	 consists of the following characters:
%</en>
%<*ja>
\item{範囲 8${}^{\text{J}}$} ISO~8869~1 の上半分(ラテン1補助)と 
                             JIS~X~0208 の共通部分にある記号.この文字範囲は
                             以下の文字で構成される:
%</ja>
\begin{multicols}{2}
\begin{itemize}
\def\ch#1#2{\item \char"#1\ (\texttt{U+00#1}, #2)}%"
%<*en>
\ch{A7}{Section Sign}
\ch{A8}{Diaeresis}
\ch{B0}{Degree sign}
\ch{B1}{Plus-minus sign}
\ch{B4}{Spacing acute}
\ch{B6}{Paragraph sign}
\ch{D7}{Multiplication sign}
\ch{F7}{Division Sign}
%</en>
%<*ja>
\ch{A7}{節記号}
\ch{A8}{トレマ}
\ch{B0}{度}
\ch{B1}{正又は負符号}
\ch{B4}{アキュート・アクセント}
\ch{B6}{段落記号}
\ch{D7}{乗算記号}
\ch{F7}{除算記号}
%</ja>
\end{itemize}
\end{multicols}

%<*en>
\item[Range~1${}^{\text{A}}$] Latin characters that some of them are included in Adobe-Japan1-6.
This range consist of the following Unicode ranges, \emph{except characters in the range~8 above}:
%</en>
%<*ja>
\item{範囲 1${}^{\text{A}}$} ラテン文字.一部はAdobe-Japan1-6 にも含まれている.
                             この範囲は以下の Unicode のブロックから構成されている.
                             \textbf{ただし,範囲 8 は除く.}
%</ja>
\begin{multicols}{2}
\begin{itemize}
%<*en>
\item \texttt{U+0080}--\texttt{U+00FF}: Latin-1 Supplement
\item \texttt{U+0100}--\texttt{U+017F}: Latin Extended-A
\item \texttt{U+0180}--\texttt{U+024F}: Latin Extended-B
\item \texttt{U+0250}--\texttt{U+02AF}: IPA Extensions
\item \texttt{U+02B0}--\texttt{U+02FF}: Spacing Modifier Letters
\item \texttt{U+0300}--\texttt{U+036F}: Combining Diacritical Marks
\item \texttt{U+1E00}--\texttt{U+1EFF}: Latin Extended Additional
%</en>
%<*ja>
\item \texttt{U+0080}--\texttt{U+00FF}: ラテン1補助
\item \texttt{U+0100}--\texttt{U+017F}: ラテン文字拡張A
\item \texttt{U+0180}--\texttt{U+024F}: ラテン文字拡張B
\item \texttt{U+0250}--\texttt{U+02AF}: IPA拡張(国際音声記号)
\item \texttt{U+02B0}--\texttt{U+02FF}: 前進を伴う修飾文字
\item \texttt{U+0300}--\texttt{U+036F}: ダイアクリティカルマーク(合成可能)
\item \texttt{U+1E00}--\texttt{U+1EFF}: ラテン文字拡張追加
%</ja>
\par\ 
\end{itemize}
\end{multicols}
%<*en>
\item[Range~2${}^{\text{J}}$] Greek and Cyrillic letters. JIS~X~0208 (hence most of Japanese
	   fonts) has some of these characters.
%</en>
%<*ja>
\item[範囲 2${}^{\text{J}}$] ギリシャ文字とキリル文字.JIS~X~0208(したがって多くの
                             和文フォント)はこれらの文字を持つ.
%</ja>
\begin{multicols}{2}
\begin{itemize}
%<*en>
\item \texttt{U+0370}--\texttt{U+03FF}: Greek and Coptic
\item \texttt{U+0400}--\texttt{U+04FF}: Cyrillic
\item \texttt{U+1F00}--\texttt{U+1FFF}: Greek Extended
%</en>
%<*ja>
\item \texttt{U+0370}--\texttt{U+03FF}: ギリシア文字及びコプト文字
\item \texttt{U+0400}--\texttt{U+04FF}: キリル文字
\item \texttt{U+1F00}--\texttt{U+1FFF}: キリル文字補助
%</ja>
\\\ 
\end{itemize}
\end{multicols}
%<*en>
\item[Range~3${}^{\text{J}}$] Punctuations and Miscellaneous symbols. The block list is
	   indicated in Table~\ref{table-rng3}.
%</en>
%<*ja>
\item[範囲 3${}^{\text{J}}$] 句読点と記号類.ブロックのリストは表~\ref{table-rng3}
                             に示してある.
%</ja>
\begin{table}[!tb]
%<*en>
\caption{Unicode blocks in predefined character range~3.}
%</en>
%<*ja>
\caption{文字範囲 3 に指定されている Unicode ブロック.}
%</ja>
\label{table-rng3}
\catcode`\"=13\def"#1#2#3#4{\texttt{U+#1#2#3#4}}%"
\begin{center}\small
\begin{tabular}{llll}
%<*en>
"2000--"206F&General Punctuation&
"2070--"209F&Superscripts and Subscripts\\
"20A0--"20CF&Currency Symbols&
"20D0--"20FF&Comb.\ Diacritical Marks for Symbols\\
"2100--"214F&Letterlike Symbols&
"2150--"218F&Number Forms\\
"2190--"21FF&Arrows&
"2200--"22FF&Mathematical Operators\\
"2300--"23FF&Miscellaneous Technical&
"2400--"243F&Control Pictures\\
"2500--"257F&Box Drawing&
"2580--"259F&Block Elements\\
"25A0--"25FF&Geometric Shapes&
"2600--"26FF&Miscellaneous Symbols\\
"2700--"27BF&Dingbats&
"2900--"297F&Supplemental Arrows-B\\
"2980--"29FF&Misc.\ Mathematical Symbols-B&
"2B00--"2BFF&Miscellaneous Symbols and Arrows\\
"E000--"F8FF&Private Use Area&
%</en>
%<*ja>
"2000--"206F&一般句読点&
"2070--"209F&上付き・下付き\\
"20A0--"20CF&通貨記号&
"20D0--"20FF&記号用ダイアクリティカルマーク(合成可能)\\
"2100--"214F&文字様記号&
"2150--"218F&数字に準じるもの\\
"2190--"21FF&矢印&
"2200--"22FF&数学記号(演算子)\\
"2300--"23FF&その他の技術用記号&
"2400--"243F&制御機能用記号\\
"2500--"257F&罫線素片&
"2580--"259F&ブロック要素\\
"25A0--"25FF&幾何学模様&
"2600--"26FF&その他の記号\\
"2700--"27BF&装飾記号&
"2900--"297F&補助矢印B\\
"2980--"29FF&その他の数学記号B&
"2B00--"2BFF&その他の記号及び矢印\\
"E000--"F8FF&私用領域(外字領域)&
%</ja>
\end{tabular}
\end{center}
\end{table}
%<*en>
\item[Range~4${}^{\text{A}}$] Characters usually not in Japanese fonts. This range consists
	   of almost all Unicode blocks which are not in other
	   predefined ranges. Hence, instead of showing the block list,
	   we put the definition of this range itself:
%</en>
%<*ja>
\item[範囲 4${}^{\text{A}}$] 通常和文フォントには含まれていない文字.
                             この範囲は他の範囲にないほとんど全ての Unicode ブロック
                             で構成されている.したがって,ブロックのリストを示す
                             代わりに,範囲の定義そのものを示す:
%</ja>
\begin{lstlisting}
\ltjdefcharrange{4}{%
   "500-"10FF, "1200-"1DFF, "2440-"245F, "27C0-"28FF, "2A00-"2AFF, 
  "2C00-"2E7F, "4DC0-"4DFF, "A4D0-"A82F, "A840-"ABFF, "FB50-"FE0F, 
  "FE20-"FE2F, "FE70-"FEFF, "FB00-"FB4F, "10000-"1FFFF} % non-Japanese
\end{lstlisting}
%<*en>
\item[Range~5${}^{\text{A}}$] Surrogates and Supplementary Private Use Areas.
%</en>
%<*ja>
\item[範囲 5${}^{\text{A}}$] 代用符号と補助私用領域.
%</ja>
%<*en>
\item[Range~6${}^{\text{J}}$] Characters used in Japanese. The block list is indicated in Table~\ref{table-rng6}.
%</en>
%<*ja>
\item[範囲 6${}^{\text{J}}$] 日本語で用いられる文字.ブロックのリストは
                             表~\ref{table-rng6} に示す.
%</ja>
\begin{table}[!tb]
%<*en>
\caption{Unicode blocks in predefined character range~6.}
%</en>
%<*ja>
\caption{文字範囲 6 に指定されている Unicode ブロック.}
%</ja>
\label{table-rng6}
\catcode`\"=13\def"#1#2#3#4{\texttt{U+#1#2#3#4}}%"
\begin{center}\small
\begin{tabular}{llll}
%<*en>
"2460--"24FF&Enclosed Alphanumerics&
"2E80--"2EFF&CJK Radicals Supplement\\
"3000--"303F&CJK Symbols and Punctuation&
"3040--"309F&Hiragana\\
"30A0--"30FF&Katakana&
"3190--"319F&Kanbun\\
"31F0--"31FF&Katakana Phonetic Extensions&
"3200--"32FF&Enclosed CJK Letters and Months\\
"3300--"33FF&CJK Compatibility&
"3400--"4DBF&CJK Unified Ideographs Extension A\\
"4E00--"9FFF&CJK Unified Ideographs&
"F900--"FAFF&CJK Compatibility Ideographs\\
"FE10--"FE1F&Vertical Forms&
"FE30--"FE4F&CJK Compatibility Forms\\
"FE50--"FE6F&Small Form Variants&
"{20}000--"{2F}FFF&(Supplementary Ideographic Plane)
%</en>
%<*ja>
"2460--"24FF&囲み英数字&
"2E80--"2EFF&CJK部首補助\\
"3000--"303F&CJKの記号及び句読点&
"3040--"309F&平仮名\\
"30A0--"30FF&片仮名&
"3190--"319F&漢文用記号(返り点)\\
"31F0--"31FF&片仮名拡張&
"3200--"32FF&囲みCJK文字・月\\
"3300--"33FF&CJK互換用文字&
"3400--"4DBF&CJK統合漢字拡張A\\
"4E00--"9FFF&CJK統合漢字&
"F900--"FAFF&CJK互換漢字\\
"FE10--"FE1F&縦書き形&
"FE30--"FE4F&CJK互換形\\
"FE50--"FE6F&小字形&
"{20}000--"{2F}FFF&(追加多言語面)
%</ja>
\end{tabular}
\end{center}
\end{table}
%<*en>
\item[Range~7${}^{\text{J}}$] Characters used in CJK languages, but not included in  Adobe-Japan1-6.
The block list is indicated in Table~\ref{table-rng7}.
%</en>
%<*ja>
\item[範囲 7${}^{\text{J}}$] CJK 言語で用いられる文字のうち,Adobe-Japan1-6 に
                            含まれていないもの.
                            ブロックのリストは表~\ref{table-rng7} に示す.
%</ja>
\begin{table}[!tb]
%<*en>
\caption{Unicode blocks in predefined character range~7.}
%</en>
%<*ja>
\caption{文字範囲 7 に指定されている Unicode ブロック.}
%</ja>
\label{table-rng7}
\catcode`\"=13\def"#1#2#3#4{\texttt{U+#1#2#3#4}}%"
\begin{center}\small
\begin{tabular}{llll}
%<*en>
"1100--"11FF&Hangul Jamo&
"2F00--"2FDF&Kangxi Radicals\\
"2FF0--"2FFF&Ideographic Description Characters&
"3100--"312F&Bopomofo\\
"3130--"318F&Hangul Compatibility Jamo&
"31A0--"31BF&Bopomofo Extended\\
"31C0--"31EF&CJK Strokes&
"A000--"A48F&Yi Syllables\\
"A490--"A4CF&Yi Radicals&
"A830--"A83F&Common Indic Number Forms\\
"AC00--"D7AF&Hangul Syllables&
"D7B0--"D7FF&Hangul Jamo Extended-B
%</en>
%<*ja>
"1100--"11FF&ハングル字母&
"2F00--"2FDF&康熙部首\\
"2FF0--"2FFF&漢字構成記述文字&
"3100--"312F&注音字母 (注音符号)\\
"3130--"318F&ハングル互換字母&
"31A0--"31BF&注音字母拡張\\
"31C0--"31EF&CJKの筆画&
"A000--"A48F&イ文字\\
"A490--"A4CF&イ文字部首&
"A830--"A83F&共通インド数字に準じるもの\\
"AC00--"D7AF&ハングル音節文字&
"D7B0--"D7FF&ハングル字母拡張B
%</ja>
\end{tabular}
\end{center}
\end{table}
\end{description}


%<en>\subsection{\Param{kanjiskip} and \Param{xkanjiskip}}
%<ja>\subsection{\Param{kanjiskip} と \Param{xkanjiskip}}
\label{subs-kskip}
\textbf{JAglue} is divided into the following three categories:
\begin{itemize}
\item Glues/kerns specified in JFM. If \verb+\inhibitglue+ is issued
      around a Japanese character, this glue will be not inserted at the
      place.
\item The default glue which inserted between two \textbf{JAchar}s (\textsf{
      kanjiskip}).
\item The default glue which inserted between a \textbf{JAchar} and an
      \textbf{ALchar} (\Param{xkanjiskip}).
\end{itemize}
The value (a skip) of \Param{kanjiskip} or \Param{xkanjiskip} can be
changed as the following.
\begin{lstlisting}
\ltjsetparameter{kanjiskip={0pt plus 0.4pt minus 0.4pt}, 
                 xkanjiskip={0.25\zw plus 1pt minus 1pt}}
\end{lstlisting}


It may occur that JFM contains the data of `ideal width of \textsf{kanjiskip}' 
and/or `ideal width of \Param{xkanjiskip}'.
To use these data from JFM, set the value of \Param{kanjiskip} or 
\Param{xkanjiskip} to \verb+\maxdimen+.

%<en>\subsection{Insertion Setting of \Param{xkanjiskip}}
%<ja>\subsection{\Param{xkanjiskip} の設定の挿入}
It is not desirable that \Param{xkanjiskip} is inserted between every
boundary between \textbf{JAchar}s and \textbf{ALchar}s. For example,
\Param{xkanjiskip} should not be inserted after opening parenthesis
(\textit{e.g.}, compare `(あ' and `(\hskip\ltjgetparameter{xkanjiskip}あ').

\LuaTeX-ja can control whether \Param{xkanjiskip} can be inserted
before/after a character, by changing \Param{jaxspmode} for \textbf{JAchar}s and
\Param{alxspmode} parameters \textbf{ALchar}s respectively. 
\begin{LTXexample}
\ltjsetparameter{jaxspmode={`あ,preonly}, alxspmode={`\!,postonly}} 
pあq い!う
\end{LTXexample}

The second argument \texttt{preonly} means `the insertion of
\Param{xkanjiskip} is allowed before this character, but not after'.
the other possible values are \texttt{postonly}, \texttt{allow} and \texttt{
inhibit}. For the compatibility with \pTeX, natural numbers between
0~and~3 are also allowed as the second argument\footnote{But we don't
recommend this: since numbers 1~and~2 have opposite meanings in
\Param{jaxspmode} and \Param{alxspmode}.}.

If you want to enable/disable all insertions of \Param{kanjiskip} and
\Param{xkanjiskip}, set \Param{autospacing} and \Param{autoxspacing}
parameters to \texttt{false}, respectively.


%<en>\subsection{Shifting Baseline}
%<ja>\subsection{ベースラインの移動}
To make a match between a Japanese font and an alphabetic font, sometimes
shifting of the baseline of one of the pair is needed. In \pTeX, this is achieved
by setting \verb+\ybaselineshift+ to a non-zero length (the
baseline of alphabetic fonts is shifted below). However, for documents
whose main language is not Japanese, it is good to shift the baseline of
Japanese fonts, but not that of alphabetic fonts.
Because of this, \LuaTeX-ja can independently set the shifting amount
of the baseline of alphabetic fonts (\Param{yalbaselineshift}
parameter) and that of Japanese fonts (\Param{yjabaselineshift}
parameter). 

\begin{LTXexample}
\vrule width 150pt height 0.4pt depth 0pt\hskip-120pt
\ltjsetparameter{yjabaselineshift=0pt, yalbaselineshift=0pt}abcあいう
\ltjsetparameter{yjabaselineshift=5pt, yalbaselineshift=2pt}abcあいう
\end{LTXexample}
Here the horizontal line in above is the baseline of a line.

There is an interesting side-effect: characters in different size can be
vertically aligned center in a line, by setting two parameters appropriately.
The following is an example (beware the value is not well tuned):
\begin{LTXexample}
xyz漢字 
{\scriptsize
  \ltjsetparameter{yjabaselineshift=-1pt, 
    yalbaselineshift=-1pt}
  XYZひらがな
}abcかな
\end{LTXexample}


%<en>\subsection{Cropmark}
%<ja>\subsection{トンボ}
Cropmark is a mark for indicating 4~corners and horizontal/vertical
center of the paper. In Japanese, we call cropmark as tombo(w).
\pLaTeX\ and this \LuaTeX-ja support `tombow' by their kernel.
The following steps are needed to typeset cropmark:

\begin{enumerate}
\item First, define the banner which will be printed at the upper left
      of the paper. This is done by assigning a token list to
      \verb+\@bannertoken+.

For example, the following sets banner as `\texttt{filename (2012-01-01 17:01)}':
\begin{verbatim}
\makeatletter

\hour\time \divide\hour by 60 \@tempcnta\hour \multiply\@tempcnta 60\relax
\minute\time \advance\minute-\@tempcnta
\@bannertoken{%
   \jobname\space(\number\year-\two@digits\month-\two@digits\day
   \space\two@digits\hour:\two@digits\minute)}%
\end{verbatim}

\item ...
\end{enumerate}


%<en>\part{Reference}
%<ja>\part{リファレンス}
\label{part-ref}
%<en>\section{Font Metric and Japanese Font}
%<ja>\section{フォントメトリックと日本語フォント}
%<en>\subsection{\texttt{\char92jfont} primitive}
%<ja>\subsection{\texttt{\char92jfont} プリミティブ}
To load a font as a Japanese font, you must use the
\verb+\jfont+ primitive instead of~\verb+\font+, while
\verb+\jfont+ admits the same syntax used in~\verb+\font+. 
\LuaTeX-ja automatically loads \Pkg{luaotfload} package,
so TrueType/OpenType fonts with features can be used for Japanese fonts:
\begin{LTXexample}
\jfont\tradgt={file:ipaexg.ttf:script=latn;%
  +trad;-kern;jfm=ujis} at 14pt
\tradgt{}当/体/医/区
\end{LTXexample}

Note that the defined control sequence
(\verb+\tradgt+ in the example above) using \verb+\jfont+ is not a
\textit{font\_def} token, hence the input like \verb+\fontname\tradgt+
causes a error.  We denote control sequences which are defined in
\verb+\jfont+ by <jfont\_cs>.

\paragraph{Prefix \texttt{psft}}
Besides \texttt{file:}\ and \texttt{name:}\ prefixes, \texttt{psft:}\
can be used a prefix in \verb+\jfont+ (and~\verb+\font+) primitive.
Using this prefix, you can specify a `name-only' Japanese font which
will be not embedded to PDF. Typical use of this prefix is to specify
the `standard' Japanese fonts, namely, `Ryumin-Light' and
`GothicBBB-Medium'. For kerning or other information, that of Kozuka
Mincho Pr6N Regular (this is a font by Adobe Inc., and included in
Japanese Font Packs for Adore Reader) will be used.


\paragraph{JFM}
As noted in Introduction, a JFM has measurements of characters and
glues/kerns that are automatically inserted for Japanese
typesetting. The structure of JFM will be described in the next
subsection. At the calling of \verb+\jfont+ primitive, you must specify
which JFM will be used for this font by the following keys:

\begin{list}{}{\def\makelabel{\ttfamily}\def\{{\char`\{}\def\}{\char`\}}}
\item[jfm=<name>]
Specify the name of JFM. A file named \texttt{jfm-<name>.lua} will be searched and/or loaded.

The followings are JFMs shipped with \LuaTeX-ja:
\begin{description}
\item[\tt jfm-ujis.lua] A standard JFM in \LuaTeX-ja. This JFM is
      based on \verb+upnmlminr-h.tfm+, a metric for UTF/OTF package that
      is used in \upTeX. When you use the \Pkg{luatexja-otf} package, please use this JFM.
\item[\tt jfm-jis.lua] A counterpart for \verb+jis.tfm+, `JIS font
	   metric' which is widely used in \pTeX. A major difference of
	   \texttt{jfm-ujis.lua} and this \texttt{jfm-jis.lua} is that
	   most characters under \texttt{jfm-ujis.lua} are square-shaped,
	   while that under \texttt{jfm-jis.lua} are horizontal
	   rectangles.

\item[\tt jfm-min.lua] A counterpart for \verb+min10.tfm+, which is one
	   of the default Japanese font metric shipped with \pTeX. There
	   are notable difference between this JFM and other 2~JFMs, as
	   shown in Table~\ref{tab-difjfm}.
\end{description}

\item[jfmvar=<string>] Sometimes there is a need that 
\end{list}

\begin{table}[t]
\caption{Differences between JFMs shipped with \LuaTeX-ja}
\label{tab-difjfm}
\ltjsetparameter{jacharrange={+3}}
\begin{center}
\def\r#1{{\jfont\g=psft:Ryumin-Light:jfm=#1 at 14.43324pt \g
\setbox0=\vtop{\hsize=7\zw\noindent ◆◆◆◆◆◆◆
ある日モモちゃんがお使いで迷子になって泣きました.}\copy0
\vrule height 0pt depth \dp0}}
\def\s#1{{\jfont\g=psft:Ryumin-Light:jfm=#1 at 14.43324pt \g
\setbox0=\vtop{\hsize=7\zw\noindent ちょっと!何}\copy0}}
\def\t#1{{\jfont\g=psft:Ryumin-Light:jfm=#1 at 19.24432pt \g
\setbox0=\hbox{漢}%
\vrule width 0.4pt height\ht0 depth\dp0\kern-.2pt\copy0
\kern-\wd0\vrule width\wd0height .2pt depth .2pt
\kern-\wd0\raise\ht0\hbox{\vrule width\wd0height .2pt depth .2pt}%
\kern-\wd0\lower\dp0\hbox{\vrule width\wd0height .2pt depth .2pt}%
\kern-.2pt\vrule width 0.4pt height\ht0 depth \dp0}}
\begin{tabular}{rccc}
\toprule
&\tt jfm-ujis.lua&\tt jfm-jis.lua&\tt jfm-min.lua\\
\midrule
Example~1&\r{ujis}&\r{jis}&\r{min}\\
Example~2&\s{ujis}&\s{jis}&\s{min}\\
Bounding Box&\t{ujis}&\t{jis}&\t{min}\\
\bottomrule
\end{tabular}
\end{center}
\ltjsetparameter{jacharrange={-3}}
\end{table}

\paragraph{Note: kern feature}\label{para-kern}
Some fonts have information for inter-glyph spacing. However, this
information is not well-compatible with \LuaTeX-ja.  More concretely,
this kerning space from this information are inserted \emph{before} the
insertion process of \textbf{JAglue}, and this causes incorrect spacing
between two characters when both a glue/kern from the data in the font
and it from JFM are present.

\begin{itemize}
\item You should specify \texttt{-kern} in
\texttt{\char92jfont} primitive, when you want to use other font features,
      such as \texttt{script=...}\,.
\item If you want to use Japanese fonts in proportional width, and use
      information from this font, use \texttt{jfm-prop.lua} for its JFM, and ...

TODO: kanjiskip?
\end{itemize}


%<en>\subsection{Structure of JFM file}
%<ja>\subsection{JFM ファイルの構造}
A JFM file is a Lua script which has only one function call:
\begin{verbatim}
luatexja.jfont.define_jfm { ... }
\end{verbatim}
Real data are stored in the table which indicated above by
\verb+{ ... }+.  So, the rest of this subsection are devoted to describe the
structure of this table.  Note that all lengths in a JFM file are
floating-point numbers in design-size unit.

\begin{list}{}{\def\makelabel{\ttfamily}\def\{{\char`\{}\def\}{\char`\}}}
\item[dir=<direction>] (required)

The direction of JFM. At the present, only \texttt{'yoko'} is supported.

\item[zw=<length>] (required)

The amount of the length of the `full-width'.

\item[zh=<length>] (required)

\item[kanjiskip=\{<natural>, <stretch>, <shrink>\}] (optional)

This field specifies the `ideal' amount of \Param{kanjiskip}. As noted
	     in Subsection~\ref{subs-kskip}, if the parameter
	     \Param{kanjiskip} is \verb+\maxdimen+, the value specified
	     in this field is actually used (if this field is not specified in
	     JFM, it is regarded as 0\,pt). Note that <stretch> and <shrink>
	     fields are in design-size unit too.


\item[xkanjiskip=\{<natural>, <stretch>, <shrink>\}] (optional)

Like the \Param{kanjiskip} field, this field specifies the `ideal'
	     amount of \Param{xkanjiskip}.

\end{list}

Besides from above fields, a JFM file have several sub-tables those
indices are natural numbers.  The table indexed by~$i\in\omega$ stores
information of `character class'~$i$. At least, the character class~0 is
always present, so each JFM file must have a sub-table whose index is
\texttt{[0]}.  Each sub-table (its numerical index is denoted by $i$) has
the following fields:

\begin{list}{}{\def\makelabel{\ttfamily}\def\{{\char`\{}\def\}{\char`\}}}
\item[chars=\{<character>, ...\}] (required except character class~0)

This field is a list of characters which are in this character
	     type~$i$. This field is not required if $i=0$, since all
	     \textbf{JAchar} which are not in any character class other
	     than 0 (hence, the character class~0 contains most of
	     \textbf{JAchar}s). In the list, a character can be
	     specified by its code number, or by the character itself
	     (as a string of length~1). Moreover, there are `imaginary
	     characters' which specified in the list. We will describe these later.

\item[width=<length>, height=<length>, depth=<length>, italic=<length>]\ (required)

Specify width of characters in character class~$i$, height, depth and
the amount of italic correction. All characters in character class~$i$ are regarded that its width, height and depth are
as values of these fields. 
But there is one exception: if \texttt{'prop'} is specified in \texttt{width} field, width of a character becomes that of its `real' glyph 

\item[left=<length>, down=<length>, align=<align>]\ 

These fields are for adjusting the position of the `real' glyph. Legal
	     values of \texttt{align} field are \texttt{'left'},
	     \texttt{'middle'} and \texttt{'right'}. If one of these
	     3~fields are omitted, \texttt{left} and \texttt{down} are
	     treated as~0, and \texttt{align} field is treated as
	     \texttt{'left'}.
The effects of these 3~fields are indicated in Figure~\ref{fig-pos}.

In most cases, \texttt{left} and \texttt{down} fields are~0, while
it is not uncommon that the \texttt{align} field is \texttt{'middle'} or \texttt{'right'}.
For example, setting the \texttt{align} field to \texttt{'right'} is practically needed 
when the current character class is the class for opening delimiters'. 
\begin{figure}[!tb]
\begin{minipage}{0.4\textwidth}%
\begin{center}\unitlength=10pt\small
\begin{picture}(15,12)(-1,-4)
\color{black!10!white}% real glyph :step1
\put(0,0){\vrule width 12\unitlength height 8\unitlength depth 3\unitlength}

\color{red!20!white}% real glyph :step1
\put(-1,-1.5){\vrule width 6\unitlength height 7\unitlength depth 2.5\unitlength}

\color{red}% real glyph
\thicklines
\put(-1,-1.5){\vector(0,1){7}\vector(0,-1){2.5}\vector(1,0){6}}
\put(5,-1.5){\line(0,1){7}\line(0,-1){2.5}}
\put(-1,5.5){\line(1,0){6}}
\put(-1,-4){\line(1,0){6}}

\color{green!20!white}% real glyph :step1
\put(3,0){\vrule width 6\unitlength height 7\unitlength depth 2.5\unitlength}

\color{black}% real glyph :step1
\thicklines
\put(0,0){\vector(0,1){8}\line(0,-1){3}\vector(1,0){12}}
\put(12,0){\line(0,1){8}\vector(0,-1){3}}
\put(0,8){\line(1,0){12}}
\put(0,-3){\line(1,0){12}}
\put(0.2,4){\makebox(0,0)[l]{\texttt{height}}}
\put(12.2,-1.5){\makebox(0,0)[l]{\texttt{depth}}}
\put(6,0.2){\makebox(0,0)[b]{\texttt{width}}}

\color{green!50!black}% real glyph :step1
\thicklines
\put(3,0){\vector(0,1){7}\vector(0,-1){2.5}\vector(1,0){6}}
\put(9,0){\line(0,1){7}\line(0,-1){2.5}}
\put(3,7){\line(1,0){6}}
\put(3,-2.5){\line(1,0){6}}
\newsavebox{\eqdist}
\savebox{\eqdist}(0,0)[b]{%
  \thinlines
  \put(-0.08,0.2){\line(0,-1){0.4}}%
  \put(0.08,0.2){\line(0,-1){0.4}}}
\put(1.5,0){\usebox{\eqdist}}
\put(10.5,0){\usebox{\eqdist}}

\color{blue}% shifted
\thicklines
\put(3,-1.5){\vector(-1,0){4}}
\put(1,-1.7){\makebox(0,0)[t]{\texttt{left}}}
\put(3,0){\vector(0,-1){1.5}}
\put(3.2,-0.75){\makebox(0,0)[l]{\texttt{down}}}
\end{picture} 
\end{center}
\end{minipage}%
\begin{minipage}{0.6\textwidth}%
Consider a node containing Japanese character whose value of the \texttt{align} 
field is \texttt{'middle'}.
\begin{itemize}
\item The black rectangle is a frame of the node.
Its width, height and depth are specified by JFM.
\item Since the \texttt{align} field is \texttt{'middle'}, 
the `real' glyph is centered horizontally (the green rectangle).
\item Furthermore, the glyph is shifted according to values of fields
      \texttt{left} and \texttt{down}. The ultimate position of the real
      glyph is indicated by the red rectangle.
\end{itemize}
\end{minipage}
\caption{The position of the `real' glyph.}
\label{fig-pos}
\end{figure}


\item[kern={\{[$j$]=<kern>, ...\}}]

\item[glue={\{[$j$]=\{<width>, <stretch>, <shrink>\}, ...\}}]
\end{list}

%<*en>
\begin{list}{}{\def\makelabel{\ttfamily}\def\{{\char`\{}\def\}{\char`\}}}
\item['lineend'] An ending of a line. 
\item['diffmet'] Used at a boundary between two \textbf{JAchar}s whose JFM or size is different.
\item['boxbdd'] The beginning/ending of a horizontal box, and the beginning of a noindented paragraph.
\item['parbdd'] The beginning of an (indented) paragraph.
\item['jcharbdd'] A boundary between \textbf{JAchar} and anything else
	     (such as \textbf{ALchar}, kern, glue, ...).
\item[$-1$] The left/right boundary of an inline math formula.
\end{list}
%</en>

%<*ja>
上で説明した通り,\texttt{chars}フィールド中にはいくつかの「特殊文字」も
指定可能である.これらは,大半が\pTeX のJFMグルーの挿入処理ではみな「文字
クラス0の文字」として扱われていた文字であり,その結果として\pTeX より細か
い組版調整ができるようになっている.以下のその一覧を述べる:
\begin{list}{}{\def\makelabel{\ttfamily}\def\{{\char`\{}\def\}{\char`\}}}
\item['lineend'] 行の終端を表す.
\item['diffmet'] 

\item['boxbdd'] hboxの先頭と末尾,及びインデントされていない
	     (\verb+\noindent+で開始された)段落の先頭を表す.
\item['parbdd'] 通常の(\verb+\noindent+で開始されていない)段落の先頭.
\item['jcharbdd'] 和文文字と「その他のもの」(欧文文字,glue,kern等)との境界.
\item[$-1$] 行中数式と地の文との境界.
\end{list}

\paragraph{\pTeX 用和文フォントメトリックの移植}
以下に,\pTeX 用和文フォントメトリックを\LuaTeX-ja用に移植する場合の注意点を挙げておく.
\begin{itemize}
\item 実際に出力される和文フォントのサイズがdesign sizeとなる.
このため,例えば$1\,\textrm{zw}$がdesign sizeの0.962216倍であるJISフォン
      トメトリック等を移植する場合は,
\begin{itemize}
\item JFM中の全ての数値を$1/0.962216$倍しておく.
\item \TeX ソース中で使用するところで,サイズ指定を0.962216倍にする.
      \LaTeX でのフォント宣言なら,例えば次のように:
\begin{verbatim}
\DeclareFontShape{JY3}{mc}{m}{n}{<-> s*[0.962216] psft:Ryumin-Light:jfm=jis}{}
\end{verbatim}
\end{itemize}
\item 上に述べた特殊文字は,\texttt{'boxbdd'}を除き文字クラスを全部0とする
      (JFM中に単に書かなければよい).
\item \texttt{'boxbdd'}については,それのみで一つの文字クラスを形成し,その
      文字クラスに関してはglue/kernの設定はしない.

これは,\pTeX では,
      hboxの先頭・末尾とインデントされていない(\verb+\noindent+で開始さ
      れた)段落の先頭にはJFMグルーは入らないという仕様を実現させるためである.
\item \pTeX の組版を再現させようというのが目的であれば以上の注意を守れば十分である.

ところで,\pTeX では通常の段落の先頭にJFMグルーが残るという仕様があるので,
      段落先頭の開き括弧は全角二分下がりになる.全角下がりを実現させるに
      は,段落の最初に手動で\verb+\inhibitglue+を追加するか,あるいは
      \verb+\everypar+のhackを行い,それを自動化させるしかなかった.

一方,\LuaTeX-jaでは,\texttt{'parbdd'}によって,それがJFM側で調整できるよ
うになった.例えば,\LuaTeX-ja同梱のJFMのように,\texttt{'boxbdd'}と同じ文字クラスに
\texttt{'parbdd'}を入れれば全角下がりとなる.

\ltjsetparameter{jacharrange={+3}}
\begin{LTXexample}
\jfont\g=psft:Ryumin-Light:jfm=test \g
\parindent1\zw\noindent{}◆◆◆◆◆
\par{}「◆◆←二分下がり
\par{}【◆◆←全角下がり
\par{}〔◆◆←全角二分下がり
\end{LTXexample}
\end{itemize}
\ltjsetparameter{jacharrange={-3}}
%</ja>

%<en>\subsection{Math Font Family}
%<ja>\subsection{数式フォントファミリ}
\TeX\ handles fonts in math formulas by 16~font families\footnote{Omega,
Aleph, \LuaTeX~and $\varepsilon$-\kern-.125em(u)\pTeX can handles 256~families, but
an external package is needed to support this in plain \TeX\ and
\LaTeX.}, and each family has three fonts:
\verb+\textfont+, \verb+\scriptfont+ and \verb+\scriptscriptfont+.

\LuaTeX-ja's handling of Japanese fonts in math formulas is similar;
Table~\ref{tab-math} shows counterparts to \TeX's primitives for math
font families. There is no relation between the value of
\verb+\fam+ and that of \verb+\jfam+; with appropriate settings, 
you can set both \verb+\fam+ and \verb+\jfam+ to~the same value.

\begin{table}[!tb]
\caption{Primitives for Japanese math fonts.}
\label{tab-math}
\begin{center}\def\{{\char`\{}\def\}{\char`\}}
\begin{tabular}{lll}
\toprule
&Japanese fonts&alphabetic fonts\\
\midrule
font family&\verb+\jfam+${}\in [0,256)$&\verb+\fam+\\
text size&\tt\Param{jatextfont}\,=\{<jfam>,<jfont\_cs>\}&\tt\verb+\textfont+<fam>=<font\_cs>\\
script size&\tt\Param{jascriptfont}\,=\{<jfam>,<jfont\_cs>\}&\tt\verb+\scriptfont+<fam>=<font\_cs>\\
scriptscript size&\tt\Param{jascriptscriptfont}\,=\{<jfam>,<jfont\_cs>\}&\tt\verb+\scriptscriptfont+<fam>=<font\_cs>\\
\bottomrule
\end{tabular}
\end{center}
\end{table}

%<en>\subsection{Callbacks}
%<ja>\subsection{コールバック}
Like \LuaTeX\ itself, \LuaTeX-ja also has callbacks. These callbacks can
be accessed via \verb+luatexbase.add_to_callback+ function and so on, as other callbacks
\begin{list}{}%
{\def\makelabel#1{\bfseries#1}}
\item[\texttt{luatexja.load\_jfm} callback]
With this callback you can overwrite JFMs. 
This callback is called when a new JFM is loaded.

\begin{verbatim}
function (<table> jfm_info, <string> jfm_name)
  return <table> new_jfm_info
end
\end{verbatim}

The argument \verb+jfm_info+ contains a table similar to the table in a JFM file, except 
this argument has \texttt{chars} field which contains character codes
	     whose character class is not~0.

An example of this callback is the \texttt{ltjarticle} class, with
	     forcefully assigning character class~0 to \texttt{'parbdd'}
	     in the JFM \texttt{jfm-min.lua}. This callback doesn't
	     replace any code of \LuaTeX-ja.

\item[\texttt{luatexja.define\_font} callback]
This callback and the next callback form a pair, and you can assign letters which don't have
	     fixed code points in Unicode to non-zero character classes.
This \texttt{luatexja.define\_font} callback is called just when new Japanese font is loaded.
\begin{verbatim}
function (<table> jfont_info, <number> font_number)
  return <table> new_jfont_info
end
\end{verbatim}

You may assume that \verb+jfont_info+ has the following fields:
\begin{description}
\item[\tt jfm] The index number of JFM.
\item[\tt size] Font size in a scaled point (${}=2^{-16}\,\textrm{pt}$).
\item[\tt var] The value specified in \texttt{jfmvar=...} at a call of \verb+\jfont+.
\end{description}

The returned table \verb+new_jfont_info+ also should include these three fields. 
The \verb+font_number+ is a font number. 

A good example of this and the next callbacks is the \Pkg{luatexja-otf}
	     package, supporting \verb+"AJ1-xxx"+ form for Adobe-Japan1
	     CID characters in a JFM. This callback doesn't replace any
	     code of \LuaTeX-ja.


\item[\texttt{luatexja.find\_char\_class} callback]
This callback is called just when \LuaTeX-ja inready to determine which
	     character class a character \verb+chr_code+ belongs.
A function used in this callback should be in the following form:
\begin{lstlisting}[numbers=left]
function (<number> char_class, <table> jfont_info, <number> chr_code)
  if char_class~=0 then return char_class
  else
    ....
    return (<number> new_char_class or 0)
  end
end
\end{lstlisting}

The argument \verb+char_class+ is the result of \LuaTeX-ja's default
	     routine or previous function calls in this callback, hence
	     this argument may not be 0. Moreover, the returned
	     \verb+new_char_class+ should be as same as \verb+char_class+ when \verb+char_class+
	     is not~0, otherwise you will overwrite the \LuaTeX-ja's
	     default routine.

This callback doesn't replace any code of \LuaTeX-ja.

\item[\texttt{luatexja.set\_width} callback]
This callback is called when \LuaTeX-ja is trying to encapsule a
	     \textbf{JAchar} \textit{glyph\_node}, to adjust its
	     dimension and position.
\begin{lstlisting}[numbers=left]
function (<table> shift_info, <table> jfont_info, <number> char_class)
  return <table> new_shift_info
end
\end{lstlisting}

The argument \verb+shift_info+ and the returned \verb+new_shift_info+ have 
\texttt{down} and \texttt{left} fields, which are the amount of shifting
	     down/left the character in a scaled-point.

%<*ja>
良い例が\Pkg{test/valign.lua}である.このファイルが読み込まれた状態では,
	     JFM内で規定された文字クラス0の文字における$(\hbox{高
	     さ}):(\hbox{深さ})$の比になるように,実際のフォントの出力上
	     下位置が自動調整される.例えば,
\begin{itemize}
\item JFM側の設定:$(\hbox{高さ})=88x$, $(\hbox{深さ})=12x$%
(和文OpenTypeフォントの標準値)
\item 実フォント側の数値:$(\hbox{高さ})=28y$, $(\hbox{深さ})=5y$(和文TrueTypeフォントの標準値)
\end{itemize}
となっていたとする.すると,実際の文字の出力位置は,
\[
 \frac{88x}{88x+12x} (28y+5y) - 28y = \frac{26}{825}y= 0.03\dot1\dot5y
\]
だけ上にずらされることになる,
%</ja>
\end{list}



%<en>\section{Parameters}
%<ja>\section{パラメータ}
%<en>\subsection{\texttt{\char92 ltjsetparameter} primitive}
%<ja>\subsection{\texttt{\char92 ltjsetparameter} プリミティブ}
As noted before, \verb+\ltjsetparameter+ and \verb+\ltjgetparameter+ are
primitives for accessing most parameters of \LuaTeX-ja. One of the main
reason that \LuaTeX-ja didn't adopted the syntax similar to that of \pTeX\ 
(\textit{e.g.},~\verb+\prebreakpenalty`)=10000+) 
is the position of \verb+hpack_filter+ callback in the source
of \LuaTeX, see Section~\ref{sec-para}.

\verb+\ltjsetparameter+ and \verb+\ltjglobalsetparameter+ are primitives
for assigning parameters. These take one argument which is a
\texttt{<key>=<value>} list. Allowed keys are described in the next
subsection.  
The difference between
\verb+\ltjsetparameter+ and \verb+\ltjglobalsetparameter+ is only the
scope of assignment;
\verb+\ltjsetparameter+ does a local assignment and 
\verb+\ltjglobalsetparameter+ does a global one. 
They also obey the value of \verb+\globaldefs+,
like other assignment.

\verb+\ltjgetparameter+ is the primitive for acquiring parameters. It
always takes a parameter name as first argument, and also takes the
additional argument---a character code, for example---in some cases.
\begin{LTXexample}
\ltjgetparameter{differentjfm}, 
\ltjgetparameter{autospacing},
\ltjgetparameter{prebreakpenalty}{`)}.
\end{LTXexample}
\emph{The return value of\/ {\normalfont\tt\char92ltjgetparameter} is
always a string}. This is outputted by \texttt{tex.write()}, so any
character other than space~`\texttt{\char32}'~(U+0020) has the category code
12~(other), while the space has 10~(space).

%<en>\subsection{List of Parameters}
%<ja>\subsection{パラメータ一覧}
The following is the list of parameters which can be specified by the
\verb+\ltjsetparameter+ command. [\verb+\cs+] indicates the counterpart
in \pTeX, and symbols beside each parameter has the following meaning:
\begin{itemize}
\item No mark: values at the end of the paragraph or the hbox are
      adopted in the whole paragraph/hbox.
\item `\ast' : local parameters, which can change everywhere inside a paragraph/hbox.
\item `\dagger': assignments are always global.
\end{itemize}

\begin{list}{}{\def\makelabel{\ttfamily}\def\{{\char`\{}\def\}{\char`\}}}
\item[\Param{jcharwidowpenalty}\,=<penalty>] [\verb+\jcharwidowpenalty+]

Penalty value for suppressing orphans. This penalty is inserted just
	     after the last \textbf{JAchar} which is not regarded as a
	     (Japanese) punctuation mark.

\item[\Param{kcatcode}\,=\{<chr\_code>,<natural number>\}]\

An additional attributes having each character whose character code is <chr\_code>.  
At the present version, the lowermost bit of <natural number> indicates
	     whether the character is considered as a punctuation mark
	     (see the description of \Param{jcharwidowpenalty} above).


\item[\Param{prebreakpenalty}\,=\{<chr\_code>,<penalty>\}] [\verb+\prebreakpenalty+]\ 
%<*ja>
文字コード<chr\_code>の\textbf{JAchar}が行頭にくることを抑止するために,
	     この文字の前に挿入/追加されるペナルティの量を指定する.

例えば閉じ括弧「〗」は絶対に行頭にきてはならないので,標準で読み込まれる
	     \texttt{luatexja-kinsoku.tex}において
\begin{verbatim}
\ltjsetparameter{prebreakpenalty={`〙,10000}}
\end{verbatim}
と,最大値の10000が指定されている.他にも,小書きのカナなど,絶対禁止とい
	     うわけではないができれば行頭にはきて欲しくない場合に,0と
	     10000の間の値を指定するのも有用であろう.
\begin{verbatim}
\ltjsetparameter{prebreakpenalty={`ゕ,150}}
\end{verbatim}
%</ja>

\item[\Param{postbreakpenalty}\,=\{<chr\_code>,<penalty>\}] [\verb+\postbreakpenalty+]
%<*ja>
文字コード<chr\_code>の\textbf{JAchar}が行末にくることを抑止するために,
	     この文字の後に挿入/追加されるペナルティの量を指定する.

\pTeX では,\verb+\prebreakpenalty+, \verb+\postbreakpenalty+において,
\begin{itemize}
\item 一つの文字に対して,pre, postどちらか一つしか指定することができなかっ
      た(後から指定した方で上書きされる).
\item pre, post合わせて256文字分の情報を格納することしかできなかった.
\end{itemize}
という制限があったが,\LuaTeX-ja ではこれらの制限は解消されている.
%</ja>

\item[\Param{jatextfont}\,=\{<jfam>,<jfont\_cs>\}] [\verb+\textfont+ in \TeX]
\item[\Param{jascriptfont}\,=\{<jfam>,<jfont\_cs>\}] [\verb+\scriptfont+ in \TeX]
\item[\Param{jascriptscriptfont}\,=\{<jfam>,<jfont\_cs>\}] [\verb+\scriptscriptfont+ in \TeX]
\item[\Param{yjabaselineshift}\,=<dimen>$^\ast$]\ 
\item[\Param{yalbaselineshift}\,=<dimen>$^\ast$] [\verb+\ybaselineshift+]

\item[\Param{jaxspmode}\,=\{<chr\_code>,<mode>\}] [\verb+\inhibitxspcode+]

Setting whether inserting  \Param{xkanjiskip} is allowed before/after a \textbf{JAchar} whose character code is <chr\_code>.
The followings are allowed for <mode>:
\begin{description}
\item[0, \texttt{inhibit}] Insertion of \Param{xkanjiskip} is inhibited before the character, nor after the character.
\item[2, \texttt{preonly}] Insertion of \Param{xkanjiskip} is allowed before the character, but not after.
\item[1, \texttt{postonly}] Insertion of \Param{xkanjiskip} is allowed after the character, but not before.
\item[3, \texttt{allow}] Insertion of \Param{xkanjiskip} is allowed before the character and after the character.
This is the default value.
\end{description}

\item[\Param{alxspmode}\,=\{<chr\_code>,<mode>\}] [\verb+\xspcode+]

Setting whether inserting \Param{xkanjiskip} is allowed before/after a
	     \textbf{ALchar} whose character code is <chr\_code>.
The followings are allowed for <mode>:
\begin{description}
\item[0, \texttt{inhibit}] Insertion of \Param{xkanjiskip} is inhibited
	   before the character, nor after the character.
\item[1, \texttt{preonly}] Insertion of \Param{xkanjiskip} is allowed
	   before the character, but not after.
\item[2, \texttt{postonly}] Insertion of \Param{xkanjiskip} is allowed
	   after the character, but not before.
\item[3, \texttt{allow}] Insertion of \Param{xkanjiskip} is allowed both
	   before the character and after the character.
This is the default value.
\end{description}
Note that parameters \Param{jaxspmode} and \Param{alxspmode} use a common table.

\item[\Param{autospacing}\,=<bool>$^\ast$] [\verb+\autospacing+]
\item[\Param{autoxspacing}\,=<bool>$^\ast$] [\verb+\autoxspacing+]
\item[\Param{kanjiskip}\,=<skip>] [\verb+\kanjiskip+]
\item[\Param{xkanjiskip}\,=<skip>] [\verb+\xkanjiskip+]

\item[\Param{differentjfm}\,=<mode>$^\dagger$]

Specify how glues/kerns between two \textbf{JAchar}s whose JFM (or size) are different. 
The allowed arguments are the followings:
\begin{description}
\item[\texttt{average}]
\item[\texttt{both}]
\item[\texttt{large}]
\item[\texttt{small}]
\end{description}

\item[\Param{jacharrange}\,=<ranges>$^\ast$]
\item[\Param{kansujichar}\,=\{<digit>, <chr\_code>\}] [\verb+\kansujichar+]
\end{list}


%<en>\section{Other Primitives}
%<ja>\section{その他のプリミティブ}
%<en>\subsection{Primitives for Compatibility}
%<ja>\subsection{互換プリミティブ}
The following primitives are implemented for compatibility with \pTeX:
\begin{list}{}{\def\makelabel{\ttfamily\char92 }}
\item[kuten]
\item[jis]
\item[euc]
\item[sjis]
\item[ucs]
\item[kansuji]
\end{list}
%<en>\subsection{\texttt{\char92 inhibitglue} primitive}
%<ja>\subsection{\texttt{\char92 inhibitglue} プリミティブ}
The primitive \verb+\inhibitglue+ suppresses the insertion of \textbf{JAglue}.
The following is an example, using a special JFM that there will be a glue between 
the beginning of a box and `あ', and also between `あ' and `ウ'.

\begin{LTXexample}
\jfont\g=psft:Ryumin-Light:jfm=test \g
あウあ\inhibitglue{}ウ\inhibitglue\par
あ\par\inhibitglue{}あ
\par\inhibitglue\hrule{}あoff\inhibitglue ice
\end{LTXexample}

With the help of this example, we remark the specification of \verb+\inhibitglue+:
\begin{itemize}
\item The call of \verb+\inhibitglue+ in the (internal) vertical mode is
      effective at the beginning of the next paragraph. This is realized
      by hacking \verb+\everypar+.
\item The call of \verb+\inhibitglue+ in the (restricted) horizontal
      mode is only effective on the spot; does not get over boundary of
      paragraphs. Moreover, \verb+\inhibitglue+ cancels ligatures and
      kernings, as shown in line~4 of above example.
\item The call of \verb+\inhibitglue+ in math mode is just ignored.
\end{itemize}

%<en>\section{Control Sequences for \LaTeXe}
%<ja>\section{\LaTeXe 用のコントロールシーケンス}
%<en>\subsection{Patch for NFSS2}
%<ja>\subsection{NFSS2 へのパッチ}
\label{ssub-nfsspat}
As described in Subsection~\ref{ssec-ltx}, \LuaTeX-ja simply adopted
\texttt{plfonts.dtx} in \pLaTeXe\ for the Japanese patch for NFSS2.
For an convenience, we will describe
commands which are not described in Subsection~\ref{ssub-chgfnt}.

\begin{cslist}%
\item[DeclareYokoKanjiEncoding\{<encoding>\}\{<text-settings>\}\{<math-settings>\}]
In NFSS2 under \LuaTeX-ja, distinction between alphabetic font families
	     and Japanese font families is only made by its
	     encoding. For example, encodings OT1 and T1 are for
	     alphabetic font families, and a Japanese font family cannot
	     have these encodings. This command defines a new encoding
	     scheme for Japanese font family (in horizontal direction).

\item[DeclareKanjiEncodingDefaults\{<text-settings>\}\{<math-settings>\}]
\item[DeclareKanjiSubstitution\{<encoding>\}\{<family>\}\{<series>\}\{<shape>\}]
\item[DeclareErrorKanjiFont\{<encoding>\}\{<family>\}\{<series>\}\{<shape>\}\{<size>\}]

The above 3~commands are just the counterparts for \verb+DeclareFontEncodingDefaults+ and~others.

\item[reDeclareMathAlphabet\{<unified-cmd>\}\{<al-cmd>\}\{<ja-cmd>\}]
和文・欧文の数式用フォントファミリを一度に変更する命令を作成する.
具体的には,欧文数式用フォントファミリ変更の命令<al-cmd>と,和文数式用フォ
	     ントファミリ変更の命令<ja-cmd>の2つを同時に行う命令として
	     <unified-cmd>を(再)定義する.実際の使用では<unified-cmd>と
	     <al-cmd>に同じものを指定する,すなわち,<al-cmd>に和文側も変
	     更させるようにするのが一般的と思われる.

本コマンドの使用については,\pLaTeX 配布中の\texttt{plfonts.dtx}に詳しく
	     注意点が述べられているので,そちらを参照されたい.

\item[DeclareRelationFont\{<ja-encoding>\}\{<ja-family>\}\{<ja-series>\}\{<ja-shape>\}\\
  \hfill\{<al-encoding>\}\{<al-family>\}\{<al-series>\}\{<al-shape>\}]
%<*en>
This command sets the `accompanied' alphabetic font family (given by the latter 4~arguments) 
with respect to a Japanese font family given by the former 4~arguments. 
%</en>
%<*ja>
いわゆる「従属欧文」を設定するための命令である.前半の4引数で表される和文フォントファミリに対して,
そのフォントに対応する「従属欧文」フォントファミリを後半の4引数により与える.
%</ja>
\item[SetRelationFont]
This command is almost same as \verb+\DeclareRelationFont+, except that this command does a local 
assignment, where \verb+\DeclareRelationFont+ does a global assignment.
\item[userelfont]
Change current alphabetic font encoding/family/\dots\ to the `accompanied' alphabetic
	     font family with respect to current Japanese font family,
	     which was set by
	     \verb+\DeclareRelationFont+ or \verb+SetRelationFont+.
Like \verb+\fontfamily+, \verb+\selectfont+ is required to take an effect.

\item[adjustbaseline]
...

\item[fontfamily\{<family>\}]
{\let\item\origitem
As in \LaTeXe, this command changes current font family (alphabetic, Japanese,~\emph{or both}) 
to <family>. Which family will be changed is determined as follows:
\begin{itemize}
\item Let current encoding scheme for Japanese fonts be
      <ja-enc>. Current Japanese font family will be changed to
      <family>, if one of the following two conditions is met:
\begin{itemize}
\item The family <fam> under the encoding <ja-enc> is already defined by
      \verb+\DeclareKanijFamily+.
\item A font definition named \texttt{<enc><ja-enc>.fd} (the file name is
      all lowercase) exists.
\end{itemize}
\item Let current encoding scheme for Japanese fonts be
      <al-enc>. For alphabetic font family, the criterion as above is used.
\item There is a case which none of the above applies, that is, the font
      family named <family> doesn't seem to be defined neither under the
      encoding <ja-enc>, nor under <al-enc>. 

In this case, the default family for font substitution is used for
      alphabetic and Japanese fonts. Note that current encoding will not
      be set to <family>, unlike the original implementation in \LaTeX.
\end{itemize}
}
\end{cslist}

As closing this subsection, we shall introduce an example of
\verb+SetRelationFont+ and \verb+\userelfont+:
\begin{LTXexample}
\gtfamily{}あいうabc
\SetRelationFont{JY3}{gt}{m}{n}{OT1}{pag}{m}{n}
\userelfont\selectfont{}あいうabc
\end{LTXexample}


%<en>\subsection{Cropmark/`tombow'}
%<ja>\subsection{トンボ}

%<en>\section{Extensions}
%<ja>\section{拡張}
\subsection{\texttt{luatexja-fontspec.sty}}

\subsection{\texttt{luatexja-otf.sty}}
This optional package supports typesetting characters in
Adobe-Japan1. \texttt{luatexja-otf.sty} offers the following 2~low-level
commands:
\begin{list}{}{\def\makelabel{\ttfamily}\def\{{\char`\{}\def\}{\char`\}}}
\item[\char92CID\{<number>\}]
Typeset a character whose CID number is <number>.
\item[\char92UTF\{<hex\_number>\}]
Typeset a character whose character code is <hex\_number> (in hexadecimal).
This command is similar to \verb+\char"+<hex\_number>,\ %"
but please remind remarks below.
\end{list}

\paragraph{Remarks}
Characters by \verb+\CID+ and \verb+\UTF+ commands are different from
ordinary characters in the following points:
\begin{itemize}
\item Always treated as \textbf{JAchar}s.
\item Processing codes for supporting OpenType features (\textit{e.g.},
      glyph replacement and kerning) by the \Pkg{luaotfload} package
      is not performed to these characters.
\end{itemize}


\paragraph{Additionally Syntax of JFM}
\texttt{luatexja-otf.sty} extends the syntax of JFM; the entries of \texttt{
chars} table in JFM now allows a string in the form
\verb+'AJ1-xxx'+, which stands for the character 
whose CID number in Adobe-Japan1 is \verb+xxx+. 

%<en>\part{Implementations}
%<ja>\part{実装}
\label{part-imp}
%<en>\section{Storing Parameters}
%<ja>\section{パラメータの保持}
\label{sec-para}
%<en>\subsection{Used Dimensions, Attributes and whatsit nodes}
%<ja>\subsection{用いられる寸法レジスタ,属性レジスタ,whatsit ノード}
Here the following is the list of dimensions and attributes which are used in \LuaTeX-ja.
\begin{list}{}{%
\def\makelabel{\ttfamily}
\def\dim#1{\item[\char92 #1\ \textrm{(dimension)}]}
\def\attr#1{\item[\char92 #1\ \textrm{(attribute)}]}
}

\dim{jQ}
As explained in Subsection~\ref{ssec-plain}, \verb+\jQ+ is equal to
			$1\,\textrm{Q}=0.25\,\textrm{mm}$, where `Q'~(also called `級') is
			a unit used in Japanese phototypesetting. So one should not change the value of this dimension. 
\dim{jH}
There is also a unit called `歯' which equals to $0.25\,\textrm{mm}$ and
			used in Japanese phototypesetting. This
			\verb+\jH+ is a synonym of \verb+\jQ+.
\dim{ltj@zw} A temporal register for the `full-width' of current Japanese font.
\dim{ltj@zh} A temporal register for the `full-height' (usually the sum of height of imaginary body and its depth) of current Japanese font.
\attr{jfam} Current number of Japanese font family for math formulas.
\attr{ltj@curjfnt} The font index of current Japanese font.
\attr{ltj@charclass} The character class of Japanese \textit{glyph\_node}.
\attr{ltj@yablshift} The amount of shifting the baseline of alphabetic
			fonts in scaled point ($2^{-16}\,\textrm{pt}$).
\attr{ltj@ykblshift} The amount of shifting the baseline of Japanese
			fonts in scaled point ($2^{-16}\,\textrm{pt}$).
\attr{ltj@autospc} Whether the auto insertion of \Param{kanjiskip} is allowed at the node.
\attr{ltj@autoxspc} Whether the auto insertion of \Param{xkanjiskip} is allowed at the node.
\attr{ltj@icflag} An attribute for distinguishing `kinds' of a node. One of the following value is
			assigned to this attribute:
\begin{description}
\item[\textit{italic} (1)] Glues from an italic correction
	   (\verb+\/+). This distinction of origins of glues 
	   (from explicit \verb+\kern+, or from \verb+\/+)
	   is needed in the insertion process of \Param{xkanjiskip}.
\item[\textit{packed} (2)] 
\item[\textit{kinsoku} (3)] Penalties inserted for the word-wrapping  process of Japanese characters (\emph{kinsoku}).
\item[\textit{from\_jfm} (4)] Glues/kerns from JFM.
\item[\textit{line\_end} (5)] Kerns for ...
\item[\textit{kanji\_skip} (6)] Glues for \Param{kanjiskip}.
\item[\textit{xkanji\_skip} (7)] Glues for \Param{xkanjiskip}.
\item[\textit{processed} (8)] Nodes which is already processed by ...
\item[\textit{ic\_processed} (9)] Glues from an italic correction, but also already processed.
\item[\textit{boxbdd} (15)] Glues/kerns that inserted just the beginning or the ending of an hbox or a paragraph.
\end{description}
\attr{ltj@kcat$i$} Where $i$~is a natural number which is less than~7.
These 7~attributes store bit~vectors indicating which character block is regarded as a block of \textbf{JAchar}s.
\end{list}

Furthermore, \LuaTeX-ja uses several `user-defined' whatsit nodes for
typesetting. All those nodes store a natural number (hence the node's
\texttt{type} is 100).
\begin{description}
\item[30111] Nodes for indicating that \verb+\inhibitglue+ is
	   specified. The \texttt{value} field of these nodes doesn't matter.
\item[30112] Nodes for \LuaTeX-ja's stack system (see the next
	   subsection). The \texttt{value} field of these nodes is
	   current group.
\item[30113] Nodes for Japanese Characters which the callback process of
	   luaotfload won't be applied, and the character code is
	   stored in the \texttt{value} field. Each node having this
	   \verb+user_id+ is converted to a `glyph\_node' \emph{after}
	   the callback process of luaotfload.
\end{description}
These whatsits will be removed during the process of inserting \textbf{JAglue}s.

%<en>\subsection{Stack System of \LuaTeX-ja}
%<ja>\subsection{\LuaTeX-ja のスタックシステム}
\label{ssec-stack}
\paragraph{Background}
\LuaTeX-ja has its own stack system, and most parameters of \LuaTeX-ja
are stored in it.  To clarify the reason, imagine the parameter
\Param{kanjiskip} is stored by a skip, and consider the following
source:
\begin{LTXexample}
\ltjsetparameter{kanjiskip=0pt}ふがふが.%
\setbox0=\hbox{\ltjsetparameter{kanjiskip=5pt}ほげほげ}
\box0.ぴよぴよ\par
\end{LTXexample}

As described in Part~\ref{part-ref}, the only effective value of
\Param{kanjiskip} in an hbox is the latest value, so the value of
\Param{kanjiskip} which applied in the entire hbox should be 5\,pt.
However, by the implementation method of \LuaTeX, this `5\,pt' cannot be
known from any callbacks.  In the \texttt{tex/packaging.w} (which is a
file in the source of \LuaTeX), there are the following codes:
\begin{lstlisting}
void package(int c)
{
    scaled h;                   /* height of box */
    halfword p;                 /* first node in a box */
    scaled d;                   /* max depth */
    int grp;
    grp = cur_group;
    d = box_max_depth;
    unsave();
    save_ptr -= 4;
    if (cur_list.mode_field == -hmode) {
        cur_box = filtered_hpack(cur_list.head_field,
                                 cur_list.tail_field, saved_value(1),
                                 saved_level(1), grp, saved_level(2));
        subtype(cur_box) = HLIST_SUBTYPE_HBOX;
\end{lstlisting}
Notice that \verb+unsave+ is executed \emph{before}
\verb+filtered_hpack+ (this is where \verb+hpack_filter+ callback is
executed): so `5\,pt' in the above source is orphaned at
\texttt+unsave+, and hence it can't be accessed from \verb+hpack_filter+
callback.

\paragraph{The method}
The code of stack system is based on that in a post of Dev-luatex mailing list\footnote{%
\texttt{[Dev-luatex] tex.currentgrouplevel}, a post at 2008/8/19 by Jonathan Sauer.}.

These are two \TeX\ count registers for maintaining information:
\verb+\ltj@@stack+ for the stack level, and \verb+\ltj@@group@level+ for
the \TeX's group level when the last assignment was done.  Parameters
are stored in one big table named \texttt{charprop\_stack\_table}, where
\texttt{charprop\_stack\_table[$i$]} stores data of stack level~$i$. If
a new stack level is created by \verb+\ltjsetparameter+, all data of the
previous level is copied.

To resolve the problem mentioned in `Background' above, \LuaTeX-ja uses
another thing: When a new stack level is about to be created, a whatsit
node whose type, subtype and value are 44~(\textit{user\_defined}),
30112, and current group level respectively is appended to the current
list (we refer this node by \textit{stack\_flag}). This enables us to
know whether assignment is done just inside a hbox. Suppose that the
stack level is~$s$ and the \TeX's group level is~$t$ just after the hbox
group, then:
\begin{itemize}
\item If there is no \textit{stack\_flag} node in the list of hbox, then
      no assignment was occurred inside the hbox. Hence values of
      parameters at the end of the hbox are stored in the stack
      level~$s$.
\item If there is a \textit{stack\_flag} node whose value is~$t+1$, then
      an assignment was occurred just inside the hbox group. Hence
      values of parameters at the end of the hbox are stored in the
      stack level~$s+1$.
\item If there are \textit{stack\_flag} nodes but all of their values
      are more than~$t+1$, then an assignment was occurred in the box,
      but it is done is `more internal' group. Hence values of
      parameters at the end of the hbox are stored in the stack
      level~$s$.
\end{itemize}

Note that to work this trick correctly, assignments to
\verb+\ltj@@stack+ and \verb+\ltj@@group@level+ have to be local always, 
regardless the value of \verb+\globaldefs+.
This problem is resolved by using
\hbox{\verb+\directlua{tex.globaldefs=0}+} (this assignment is local).


%<en>\section{Linebreak after Japanese Character}
%<ja>\section{和文文字直後の改行}
\label{sec-lbreak}
%<en>\subsection{Reference: Behavior in \pTeX}
%<ja>\subsection{参考: \pTeX の挙動}
%<*en>
In~\pTeX, a line break after a Japanese character doesn't emit a space,
since words are not separated by spaces in Japanese writings. However,
this feature isn't fully implemented in \LuaTeX-ja due to the
specification of callbacks in~\LuaTeX. To clarify the difference between
\pTeX~and~\LuaTeX, We briefly describe the handling of a line break in~\pTeX, in
this subsection. 

\pTeX's input processor can be described in terms of a finite state
automaton, as that of~\TeX\ in~Section~2.5 of~\cite{texbytopic}. The
internal states are as follows:
\begin{itemize}
\item State~$N$: new line
\item State~$S$: skipping spaces
\item State~$M$: middle of line
\item State~$K$: after a Japanese character
\end{itemize}
The first three states---$N$, $S$~and~$M$---are as same as \TeX's input
processor.  State~$K$ is similar to state~$M$, and is entered after
Japanese characters.  The diagram of state transitions are indicated in
Figure~\ref{fig-ptexipro}.  Note that \pTeX\ doesn't leave state~$K$
after `beginning/ending of a group' characters.
%</en>

%<*ja>
欧文では文章の改行は単語間でしか行わない.そのため,\TeX では,(文字の直後の)改行は
空白文字と同じ扱いとして扱われる.一方,和文ではほとんどどどこでも改行が可能なため,
\pTeX では和文文字の直後の改行は単純に無視されるようになっている.

このような動作は,\pTeX が\TeX からエンジンとして拡張されたことによって可能になったことである.
\pTeX の入力処理部は,\TeX におけるそれと同じように,有限オートマトンとして記述することができ,
以下に述べるような4状態を持っている.

\begin{itemize}
\item State~$N$: 行の開始.
\item State~$S$: 空白読み飛ばし.
\item State~$M$: 行中.
\item State~$K$: 行中(和文文字の後).
\end{itemize}
また,状態遷移は,図~\ref{fig-ptexipro} のようになっており,図中の数字は
カテゴリーコードを表している.最初の3状態は\TeX の入力処理部と同じであり,
図中から状態$K$と「$j$」と書かれた矢印を取り除けば,\TeX の入力処理部と同
じものになる.

この図から分かることは,
\begin{quote}
行が和文文字(とグループ境界文字)で終わっていれば,改行は無視される
\end{quote}
ということである.
%</ja>

\begin{figure}[!tb]
\begin{gather*}
 \def\sp{\texttt{\char32}}
 \xymatrix{&&
   {\text{scan a cs}}\ar@(r,ul)[dr]&\\
\ar[r]&
   *++[o][F-]{N}\ar[ur]^0\ar[dd]_{d,\ g}\ar[u]^{5\ (\texttt{\char92par})}
     \ar@{->}@(d,l)[ddrr]_(0.45){j}&&
   *++[o][F-]{S}\ar@(l,dr)[ul]^0\ar@(l,ur)[ddll]_{d,\ g}\ar[u]_{5}
     \ar@{->}@(r,r)[dd]^{j}\\&\\&
   *++[o][F-]{M}\ar[uuur]^0\ar@(r,dl)[uurr]_(0.55){10\ (\sp)}
     \ar[d]_{5\ ({\sp})}\ar@{->}@(dr,dl)[rr]_{j}&&
   *++[o][F-]{K}\ar@{->}@(ul,d)[uuul]^0\ar@{->}[ll]^{d}
     \ar@{->}@(ur,dr)[uu]^{10\ (\sp)}\ar@{->}[d]_5\\
   &&&
 }\\
 d:=\{3,4,6,7,8,11,12,13\},\quad g:=\{1,2\},\quad j:=(\text{Japanese characters})
\end{gather*}
\begin{itemize}
\item Numbers represent category codes.
\item Category codes 9~(ignored), 14~(comment)~and~15~(invalid) are omitted in above diagram.
\end{itemize}
\caption{State transitions of \pTeX's input processor.}
\label{fig-ptexipro}
\end{figure}


%<en>\subsection{Behavior in \LuaTeX-ja}
%<ja>\subsection{\LuaTeX-ja の挙動}
%<*en>
States in the input processor of \LuaTeX\ is the same as that of \TeX,
and they can't be customized by any callbacks. Hence, we can only use
\verb+process_input_buffer+ and \verb+token_filter+ callbacks for to
suppress a space by a line break which is after Japanese characters.

However, \verb+token_filter+ callback cannot be used either, since a
character in category code 5~(end-of-line) is converted into an space
token \emph{in the input processor}.  So we can use only the
\verb+process_input_buffer+ callback.  This means that suppressing a
space must be done \emph{just before} an input line is read.

Considering these situations, handling of an end-of-line in \LuaTeX-ja are as follows:
\begin{quote}
A character U+FFFFF (its category code is set to 14~(comment) by
\LuaTeX-ja) is appended to an input line, \emph{before \LuaTeX\ actually
process it}, if and only if the following two conditions are satisfied:
\begin{enumerate}
\item The category code of the character $\langle${return}$\rangle$
      (whose character code is 13) is 5~(end-of-line).
\item The input line matches the following `regular expression':
\[
  (\text{any char})^*(\textbf{JAchar})
  \bigl(\{\text{catcode}=1\}\cup\{\text{catcode}=2\}\bigr)^*
\]
\end{enumerate}
\end{quote}

\paragraph{Remark}
The following example shows the major difference from the behavior of \pTeX:
\begin{LTXexample}
\ltjsetparameter{autoxspacing=false}
\ltjsetparameter{jacharrange={-6}}xあ
y\ltjsetparameter{jacharrange={+6}}zあ
u
\end{LTXexample}
\begin{itemize}
\item There is no space between `x' and `y', since the line~2 ends with a \textbf{JAchar} `あ'
(this `あ' considered as an \textbf{JAchar} at the ending of line~1).
\item There is no space between `あ' (in the line~3) and `u', since the
      line~3 ends with an \textbf{ALchar}
(the letter `あ' considered as an \textbf{ALchar} at the ending of line~2).
\end{itemize}
%</en>

%<*ja>
\LuaTeX の入力処理部は\TeX のそれと全く同じであり,callbackによりユーザが
カスタマイズすることはできない.このため,改行抑制の目的でユーザが利用で
きそうなcallbackとしては,\verb+process_input_buffer+や
\verb+token_filter+に限られてしまう.しかし,\TeX の入力処理部をよく見る
と,後者も役には経たないことが分かる:改行文字は,入力処理部によってトー
クン化される時に,カテゴリーコード10の32番文字へと置き換えられてしまうた
め,\verb+token_filter+で非標準なトークン読み出しを行おうとしても,空白文
字由来のトークンと,改行文字由来のトークンは区別できないのだ.

すると,我々のとれる道は,\verb+process_input_buffer+を用いて
\LuaTeX の入力処理部に引き渡される前に入力文字列を編集するというものしかない.
以上を踏まえ,\LuaTeX-jaにおける「和文文字直後の改行抑制」の処理は,次のようになっている:

\begin{quote}
各入力行に対し,\textbf{その入力行が読まれる前の内部状態で}
以下の2条件が満たされている場合,\LuaTeX-jaはU+FFFFF番の文字
\footnote{この文字はコメント文字として扱われるように\LuaTeX-ja内部で設定をしている.}
を末尾に追加する.よって,その場合に改行は空白とは見做されないこととなる.
\begin{enumerate}
\item 改行文字(文字コード13番)のカテゴリーコードが5~(end-of-line)である.
\item 入力行は次の「正規表現」にマッチしている:
\[
  (\text{any char})^*(\textbf{JAchar})
  \bigl(\{\text{catcode}=1\}\cup\{\text{catcode}=2\}\bigr)^*
\]
\end{enumerate}
\end{quote}

この仕様は,前節で述べた\pTeX の仕様にできるだけ近づけたものとなっている.最初の条件は,
\texttt{verbatim}系環境などの日本語対応マクロを書かなくてすませるためのものである.
しかしながら,完全に同じ挙動が実現できたわけではない.
差異は,次の例が示すように,和文文字の範囲を変更した行の改行において見られる:
\begin{LTXexample}
\ltjsetparameter{autoxspacing=false}
\ltjsetparameter{jacharrange={-6}}xあ
y\ltjsetparameter{jacharrange={+6}}zあ
u
\end{LTXexample}
もし\pTeX とまったく同じ挙動を示すならば,出力は
「\hbox{\ltjsetparameter{autoxspacing=false}x yzあu}」となるべきである.しかし,実際には
上のように異なる挙動となっている.
\begin{itemize}
\item 2行目は「あ」という和文文字で終わる(2行目を処理する前の時点では,
      「あ」は和文文字扱いである)ため,直後の改行文字は無視される.
\item 3行目は「あ」という欧文文字で終わる(2行目を処理する前の時点では,
      「あ」は欧文文字扱いである)ため,直後の改行文字は空白に置き換わる.
\end{itemize}
このため,トラブルを避けるために,和文文字の範囲を\verb+\ltjsetparameter+で編集した場合,
その行はそこで改行するようにした方がいいだろう.
%</ja>


%<en>\section{Insertion of JFM glues, \Param{kanjiskip} and \Param{xkanjiskip}}
%<ja>\section{JFM グルーの挿入,\Param{kanjiskip} と \Param{xkanjiskip}}
%<en>\subsection{Overview}
%<ja>\subsection{概要}
%<*en>
NOT COMPLETED
%</en>

%<*ja>
\LuaTeX-ja における和文処理グルーの挿入方法は,\pTeX のそれとは全く異なる.
\pTeX では次のような仕様であった:
\begin{itemize}
\item JFMグルーの挿入は,和文文字を表すトークンを元に水平リストに(文字を表す)<char\_node>を
追加する過程で行われる.
\item \Param{xkanjiskip}の挿入は,hboxへのパッケージングや行分割前に行われる.
\item \Param{kanjiskip}はノードとしては挿入されない.パッケージングや行分割の計算時に
「和文文字を表す2つの<char\_node>の間には\Param{kanjiskip}がある」ものとみなされる.
\end{itemize}
しかし,\LuaTeX-jaでは,hboxへのパッケージングや行分割前に全ての
\textbf{JAglue},即ちJFMグルー・\Param{xkanjiskip}・\Param{kanjiskip}の
3種類を一度に挿入することになっている.これは,\LuaTeX において欧文の合字・
カーニング処理がノードベースになったことに対応する変更である.

\LuaTeX-jaにおける\textbf{JAglue}挿入処理では,下の図\ref{fig-clu}のよう
に「塊」を単位にして行われる.大雑把にいうと,「塊」は文字とそれに付随す
るノード達(アクセント位置補正用のkernや,イタリック補正)をまとめたもの
であり,2つの塊の間には,ペナルティ,\verb+\vadjust+,whatsitなど,行組版
には関係しないものがある.そのため,……
%</ja>

% \begin{figure}[!tb]
% \unitlength=10mm
% \end{figure}

%<en>\subsection{Definition of a `cluster'}
%<ja>\subsection{「クラスタ」の定義}

\begin{defn}
A \emph{cluster} is a list of nodes in one of the following forms, with the \textit{id} of it:
\begin{enumerate}
\item Nodes whose value of\ \verb+\ltj@icflag+ is in $[3,15)$.  These
      nodes come from a hbox which is already packaged, by unpackaging
      (\verb+\unhbox+).
      The \textit{id} is \textit{id\_pbox}.
\item A inline math formula, including two \textit{math\_node}s at the boundary of it:
HOGE
      The \textit{id} is \textit{id\_math}.
\item A \textit{glyph\_node} with nodes which relate with it: 
HOGE
The \textit{id} is \textit{id\_jglyph} or
\textit{id\_glyph}, according to whether the \textit{glyph\_node}
represents a Japanese character or not.
\item An box-like node, that is, an hbox, an vbox and an rule (\verb+\vrule+). 
The \textit{id} is \textit{id\_hlist} if the node is an
      hbox which is not shifted vertically, or \textit{id\_box\_like}
      otherwise.
\item A glue, a kern whose subtype is not 2~(\textit{accent}), and a discretionary break.
The \textit{id} is \textit{id\_glue}, \textit{id\_kern}
      and \textit{id\_disc}, respectively.
%Just a node which will \dots, \textit{i.e.}, a node which is \emph{not} one of the following:
%\textit{ins\_node}, \textit{mark\_node}, \textit{adjust\_node}, \textit{whatsit\_node}
%and \textit{penalty\_node}.
\end{enumerate}
We denote a cluster by \textit{Np}, \textit{Nq} and \textit{Nr}.
\end{defn}

Internally, a cluster is represented by a table $\textit{Np}$ with the following fields.

\begin{description}
\def\makelabel#1{\textbf{\textit{#1}}}
\item[first, last] The first/last node of the cluster.
\item[id] The \textit{id} in above definition.
\item[nuc]

% jachar
\item[auto\_kspc, auto\_xspc]
\item[xspc\_before, xspc\_after]

% alchar, jachar
\item[pre, post]
\item[char]
\item[class]
\item[lend]
\item[met, var]
\end{description}
\end{document}

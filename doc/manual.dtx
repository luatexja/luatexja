%#! lualatex -shell-escape manual.ins

%<*Eng>
\documentclass[a4paper,titlepage]{article}
\usepackage[margin=20mm]{geometry}
%</Eng>
%<*Jpn>
\documentclass[a4paper,titlepage]{bxjsarticle}
\setpagelayout*{margin=20mm}
\def\headfont{\normalfont\bfseries}
% \def\headfont{\sffamily\gtfamily} is needed in ordinal documents
%</Jpn>

\usepackage{amsmath,amssymb,xcolor,pict2e}
\usepackage{booktabs,listings,lltjlisting,showexpl,multicol}
\usepackage{luatexja}
\usepackage[unicode=true]{hyperref}
\usepackage[all]{xy}
\SelectTips{cm}{}

%<*Eng>
\title{The Lua\TeX-ja package}
\author{The Lua\TeX-ja project team}
%</Eng>
%<*Jpn>
\title{Lua\TeX-jaパッケージ}
\author{Lua\TeX-jaプロジェクトチーム}
%</Jpn>

\lstset{
  basicstyle=\ttfamily\small, pos=o, breaklines=true,
  numbers=none, rframe={}
}

\parskip=\smallskipamount
\begin{document}
\catcode`\<=13
\def<#1>{{\normalfont\rm\itshape$\langle$#1$\rangle$}}
\maketitle

\tableofcontents
\bigskip

%<*Eng>
{\Large\bf This documentation is far from complete. It may have many
grammatical (and contextual) errors.}
%</Eng>
%<*Jpn>
\textbf{\large 本ドキュメントはまだまだ未完成です.
また,英語版と日本語版をdocstripプログラムを用いることで一緒に生成している都合上,
見出しが英語のままになっています.}
%</Jpn>

\newpage
\part{User's manual}

\section{Introduction}

%<*Eng>
The Lua\TeX-ja package is a macro package for typesetting high-quality
Japanese documents in Lua\TeX.
%</Eng>
%<*Jpn>
Lua\TeX-jaパッケージは,次世代標準\TeX であるLua\TeX の上で,p\TeX と同等
/それ以上の品質の日本語組版を実現させようとするマクロパッケージである.
%</Jpn>

\subsection{Backgrounds}
Traditionally, ASCII p\TeX, an extension of \TeX, and its derivatives
are used to typeset Japanese documents in \TeX. p\TeX\ is an engine
extension of \TeX: so it can produce high-quality Japanese documents
without using very complicated macros. But this point is a mixed
blessing: p\TeX\ is left behind from other extensions of \TeX,
especially $\varepsilon$-\TeX\ and pdf\TeX, and from changes about
Japanese processing in computers (\textit{e.g.}, the UTF-8 encoding).

Recently extensions of p\TeX, namely up\TeX\ (Unicode-implementation
of p\TeX) and $\varepsilon$-p\TeX\ (merging of p\TeX and
$\varepsilon$-\TeX\ extension), have developed to fill those gaps to some
extent, but gaps are still exist.

However, the appearance of Lua\TeX\ changed the whole situation. With
using Lua `callbacks', users can customize the internal processing of
Lua\TeX. So there is no need to modify sources of engines to
support Japanese typesetting: to do this, we only have to write Lua
scripts for appropriate callbacks.


\subsection{Major Changes from p\TeX}
The Lua\TeX-ja package is under much influence of p\TeX\ engine. The initial
target of development was to implement features of p\TeX. However,
\emph{Lua\TeX-ja is not a just porting of p\TeX; unnatural
specifications/behaviors of p\TeX\ were not adopted}.

The followings are major changes from p\TeX:
\begin{itemize}
\item A Japanese font is a tuple of a `real' font, a Japanese font
      metric (\textbf{JFM}, for short), and an optional string called
      `variation'.

\item In p\TeX, a linebreak after Japanese character is ignored (and
      doesn't yield a space), since linebreaks (in source files) are
      permitted almost everywhere in Japanese texts. However, Lua\TeX-ja
      doesn't have this function completely, because of a specification
      of Lua\TeX.
\item The insertion process of glues/kerns between two Japanese
      characters and between a Japanese character and other characters
      (we refer these glues/kerns as \textbf{JAglue}) is rewritten from
      scratch.

\begin{itemize}
\item As Lua\TeX's internal character handling is `node-based'
      (\textit{e.g.}, \verb+of{}fice+ doesn't prevent ligatures), the
      insertion process of \textbf{JAglue} is now `node-based'.
\item Furthermore, nodes between two characters which have no effects in
      linebreak (\textit{e.g.}, \verb+\special+ node) are ignored in the
      insertion process.
\item In the process, two Japanese fonts which differ in their `real'
      fonts only are identified.
\end{itemize}
\item At the present, vertical typesetting (\emph{tategaki}), is not
      supported in Lua\TeX-ja.

\end{itemize} 
For detailed information, see Part~\ref{part-imp}.

\subsection{Notations}
In this document, the following terms and notations are used:
\begin{itemize}
\item Characters are divided into two types: 
\begin{itemize}
\item \textbf{JAchar}: standing for Japanese characters such as
      Hiragana, Katakana, Kanji and other punctuation marks for
      Japanese.'
\item \textbf{ALchar}: standing for all other characters like alphabets. 
\end{itemize}
We say `alphabetic fonts' for fonts used in \textbf{ALchar}, and `Japanese fonts' for fonts used in \textbf{JAchar}.

\item A word in a sans-serif font (like \textsf{prebreakpenalty})
      represents an internal parameter for Japanese typesetting, and it
      is used as a key in \verb+\ltjsetparameter+ command.
\item The word `primitive' is used not only for primitives in Lua\TeX,
      but also for control sequences that defined in the core module of
      Lua\TeX-ja.
\item In this document, natural numbers start from~0.
\end{itemize}

\subsection{About the project}
\paragraph{Project Wiki} Project Wiki is under construction.
\begin{itemize}
\item \url{http://sourceforge.jp/projects/luatex-ja/wiki/FrontPage%28en%29} (English)
\item \url{http://sourceforge.jp/projects/luatex-ja/wiki/FrontPage} (Japanese)
\end{itemize}

This project is hosted by SourceForge.JP.

\paragraph{Members}\ 
% \begin{multicols}{2}
% \begin{itemize}
% \item Hironori KITAGAWA
% \item Kazuki MAEDA
% \item Takayuki YATO
% \item Yusuke KUROKI
% \item Noriyuki ABE
% \item Munehiro YAMAMOTO
% \item Tomoaki HONDA
% \par\ 
% \end{itemize}
% \end{multicols}

\newpage
\section{Getting Started}
\subsection{Installation}
To install the Lua\TeX-ja\ package, you will need:
\begin{itemize}
\item Lua\TeX\ (version 0.65.0-beta or later) and its supporting packages.\\
If you are using \TeX~Live\ 2011 or W32\TeX, you don't have to worry.
\item The source archive of Lua\TeX-ja, of course{\tt:)}
\end{itemize}

The installation methods are as follows:
\begin{enumerate}
\item Download the source archive.

At the present, Lua\TeX-ja has no official release, so you have to retrieve
the archive from the repository.
You can retrieve the Git repository via
\begin{verbatim}
$ git clone git://git.sourceforge.jp/gitroot/luatex-ja/luatexja.git
\end{verbatim} 
or download the archive of HEAD in \texttt{master} branch from
\begin{flushleft}
\url{http://git.sourceforge.jp/view?p=luatex-ja/luatexja.git;a=snapshot;h=HEAD;sf=tgz}.
\end{flushleft}
\item Extract the archive. You will see {\tt src/} and several other sub-directories.
\item Copy all the contents of {\tt src/} into one of your \texttt{TEXMF} tree.
\item If {\tt mktexlsr} is needed to update the filename database, make it so.
\end{enumerate}

\subsection{Cautions}
\begin{itemize}
\item The encoding of your source file must be UTF-8. 
\item Not well-tested. In particular, the default setting of the range
      of \textbf{JAchar} in the present version does not coexist with
      other packages which use Unicode fonts. 
\end{itemize}

\subsection{Using in plain \TeX}\label{ssec-plain}
To use Lua\TeX-ja in plain \TeX, simply put the following  at the beginning of the document:
\begin{verbatim}
\input luatexja.sty
\end{verbatim}

This does minimal settings (like {\tt ptex.tex}) for typesetting Japanese documents:
\begin{itemize}
\item The following 6~Japanese fonts are preloaded:
\begin{center}
\begin{tabular}{ccccc}
\toprule
\textbf{classification}&\textbf{font name}&\textbf{13.5\,Q}&\textbf{9.5\,Q}&\textbf{7\,Q}\\\midrule
\emph{mincho}&Ryumin-Light    &\verb+\tenmin+&\verb+\sevenmin+&\verb+\fivemin+\\
\emph{gothic}&GothicBBB-Medium&\verb+\tengt+ &\verb+\sevengt+ &\verb+\fivegt+\\
\bottomrule
\end{tabular}
\end{center}
\begin{itemize}
\item The `Q' is a unit used in Japanese phototypesetting, and
      $1\,\textrm{Q}=0.25\,\textrm{mm}$. This length is stored in a
      dimension \verb+\jQ+.

\item It is widely accepted that the font `Ryumin-Light' and
      `GothicBBB-Medium' aren't embedded into PDF files, and PDF reader
      substitute them by some external Japanese fonts (\textit{e.g.},
      Kozuka Mincho is used in Adobe Reader). We adopt this custom to
      the default setting.
\item You may notice that size of above fonts is slightly smaller than
      their alphabetic counterparts: for example, the size
      \verb+\texmin+ is $13.5\,\textrm{Q}\simeq 9.60444\,\textrm{pt}$. This is intensional: ...
\end{itemize}
\item A character in Unicode is treated as \textbf{JAchar} if and only
      if its code-point has more than or equal to U+0100.
\item The amount of glue that are inserted between a \textbf{JAchar} and
      an \textbf{ALchar} (the parameter \textsf{xkanjiskip}) is set to
\[
 0.25\,\hbox{\verb+\zw+}^{+1\,\text{pt}}_{-1\,\text{pt}} = \frac{27}{32}\,\mathrm{mm}^{+1\,\text{pt}}_{-1\,\text{pt}}.
\]
Here \verb+\zw+ is the counterpart of \texttt{em} for Japanese fonts, that is, the length of `full-width' in current Japanese font.
\end{itemize}

\subsection{Using in \LaTeX}\label{ssec-ltx}
\paragraph{\LaTeXe}
Using in \LaTeXe\ is basically same. To set up the minimal environment
for Japanese, you only have to load {\tt luatexja.sty}:
\begin{verbatim}
\usepackage{luatexja}
\end{verbatim}
It also does minimal settings (counterparts in p\LaTeX\ are {\tt
plfonts.dtx} and {\tt pldefs.ltx}):

\begin{itemize}
\item {\tt JY3} is the font encoding for Japanese fonts (in horizontal direction).\\
When vertical typesetting is supported by Lua\TeX-ja in the future, {\tt JT3} will be used for vertical fonts.
\item Two font families {\tt mc} and {\tt gt} are defined: 
\begin{center}
\begin{tabular}{ccccc}
\toprule
\textbf{classification}&\textbf{family}&\verb+\mdseries+&\verb+\bfseries+&\textbf{scale}\\\midrule
\emph{mincho}&\tt mc&Ryumin-Light    &GothicBBB-Medium&0.960444\\
\emph{gothic}&\tt gt&GothicBBB-Medium&GothicBBB-Medium&0.960444\\
\bottomrule
\end{tabular}
\end{center}
\textbf{Note on fonts in bold series}

\item Japanese characters in math mode are typeset by the font family {\tt mc}.
\end{itemize}

However, above settings are not sufficient for Japanese-based
documents. To typeset Japanese-based documents, You are better to use
class files other than {\tt article.cls}, {\tt book.cls}, ...  At the
present, BXjscls (\texttt{bxjsarticle.cls} and \texttt{bxjsbook.cls}, by
Takayuki Yato) are better alternative. It is not determined whether
Lua\TeX-ja will develop and contain counterparts of major classes used
in p\TeX\ (including jsclasses by Haruhiko Okumura).

\subsection{Changing Fonts}
\paragraph{Remark: Japanese Characters in Math Mode}
Since p\TeX\ supports Japanese characters in math mode, there are
sources like the following:

\begin{LTXexample}
$f_{高温}$~($f_{\text{high temperature}}$).
\[ y=(x-1)^2+2\quad{}よって\quad y>0 \]
$5\in{}素:=\{\,p\in\mathbb N:\text{$p$ is a prime}\,\}$.
\end{LTXexample}

We (the project members of Lua\TeX-ja) think that using
Japanese characters in math mode are allowed if and only if these are used as identifiers.
In this point of view, 
\begin{itemize}
\item The lines 1~and~2 above are not correct, since `高温' in above is used as a textual label, and
`よって' is used as a conjunction. 
\item However, the line~3 is correct, since `素' is used as an identifier.
\end{itemize}
Hence, in our opinion, the above input should be corrected as:
\begin{LTXexample}
$f_{\text{高温}}$~%
($f_{\text{high temperature}}$).
\[ y=(x-1)^2+2\quad 
  \mathrel{\text{よって}}\quad y>0 \]
$5\in{}素:=\{\,p\in\mathbb N:\text{$p$ is a prime}\,\}$.
\end{LTXexample}
%BUG?: \{\}がなければ「素」がでない.上の段落の「よって」もでてない.
We also believe that using Japanese characters as identifiers is rare,
hence we don't describe how to change Japanese fonts in math mode in
this chapter. For the method, please see Part~\ref{part-ref}.


\paragraph{plain \TeX}
To change Japanese fonts in plain \TeX, you must use the primitive
\verb+\jfont+. So please see Part~\ref{part-ref}.


\paragraph{NFSS2}
For \LaTeXe, Lua\TeX-ja simply adopted the font selection system from that
of p\LaTeXe\ (in {\tt plfonts.dtx}).
\begin{itemize}
\item Two control sequences \verb+\mcdefault+ and \verb+\gtdefault+ are
      used to specify the default font families for \emph{mincho} and
      \emph{gothic}, respectively. Default values: \texttt{mc} for
      \verb+\mcdefault+ and \texttt{gt} for \verb+\gtdefault+.
\item Commands \verb+\fontfamily+, \verb+\fontseries+,
      \verb+\fontshape+ and \verb+\selectfont+ can be used to change
      attributes of Japanese fonts. 
\begin{center}
\begin{tabular}{ccccc}
\toprule
&\textbf{encoding}&\textbf{family}&\textbf{series}&\textbf{shape}\\\midrule
alphabetic fonts
&\verb+\romanencoding+&\verb+\romanfamily+&\verb+\romanseries+&\verb+\romanshape+\\
Japanese fonts
&\verb+\kanjiencoding+&\verb+\kanjifamily+&\verb+\kanjiseries+&\verb+\kanjishape+\\
both&---&--&\verb+\fontseries+&\verb+\fontshape+\\
auto select&\verb+\fontencoding+&\verb+\fontfamily+&---&---\\
\bottomrule
\end{tabular}
\end{center}
\item For defining a Japanese font family, use \verb+\DeclareKanjiFamily+
      instead of \verb+\DeclareFontFamily+.
\end{itemize}

\paragraph{fontspec}
To coexist with \texttt{fontspec} package, it is needed to load
\texttt{luatexja-fontspec} package in the preamble. This additional
package automatically loads \texttt{luatexja} and \texttt{fontspec}
package, if needed.

In \texttt{luatexja-fontspec} package, the following 7~commands are defined as
counterparts of original commands in \texttt{fontspec}:
\begin{center}
\begin{tabular}{ccccc}
\toprule
Japanese fonts
&\verb+\jfontspec+&\verb+\setmainjfont+&\verb+\setsansjfont+&\verb+\newjfontfamily+\\
alphabetic fonts
&\verb+\fontspec+&\verb+\setmainfont+&\verb+\setsansfont+&\verb+\newfontfamily+\\
\midrule
Japanese fonts
&\verb+\newjfontface+&\verb+\defaultjfontfeatures+&\verb+\addjfontfeatures+\\
alphabetic fonts
&\verb+\newfontface+&\verb+\defaultfontfeatures+&\verb+\addfontfeatures+\\
\bottomrule
\end{tabular}
\end{center}
使用例


Note that there is no command named \verb+\setmonojfont+, since it is
popular for Japanese fonts that nearly all Japanese glyphs have same widths.


\section{Changing Parameters}
There are many parameters in Lua\TeX-ja. And due to the behavior of Lua\TeX,
most of them are not stored as internal register of \TeX, but as an
original storage system in Lua\TeX-ja. Hence, to assign or acquire those
parameters, you have to use commands \verb+\ltjsetparameter+ and
\verb+\ltjgetparameter+.

\subsection{Editing the range of \textbf{JAchar}s}
As noted before, the default setting is:
\begin{center}
A character in Unicode is treated as \textbf{JAchar},\\
 if and only if its
 code-point has more than or equal to U+0100.
\end{center}
$\uparrow$ TODO: CHANGE THIS!


To edit the range of \textbf{JAchar}s, You have to assign a non-zero
natural number which is less than 217 to the character range first. This
can be done by using \verb+\ltjdefcharrange+ primitive. For example, the
next line assigns whole characters in Supplementary Multilingual Plane
and the character `漢' to the range number~4.
\begin{lstlisting}
\ltjdefcharrange{4}{"10000-"1FFFF,`漢}
\end{lstlisting}
This assignment of numbers to ranges are always global, so you should
not do this in the middle of a document. 上書き

After assigning numbers to ranges, ...

\subsection{\textsf{kanjiskip} and \textsf{xkanjiskip}}\label{subs-kskip}
\textbf{JAglue} is divided into the following three categories:
\begin{itemize}
\item Glues/kerns specified in JFM. If \verb+\inhibitglue+ is issued around a Japanese character,
      this glue will be not inserted at the place.
\item The default glue which inserted between two \textbf{JAchar}s ({\sf
      kanjiskip}).
\item The default glue which inserted between a \textbf{JAchar} and an
      \textbf{ALchar} (\textsf{xkanjiskip}).
\end{itemize}
The value (a skip) of \textsf{kanjiskip} or \textsf{xkanjiskip} can be
changed as the following.
\begin{verbatim}
\ltjsetparameter{kanjiskip={0pt plus 0.4pt minus 0.4pt}, 
                 xkanjiskip={0.25\zw plus 1pt minus 1pt}}
\end{verbatim}


It may occur that JFM contains the data of `ideal width of {\sf
kanjiskip}' and/or `ideal width of \textsf{xkanjiskip}'.
To use these data from JFM, set the value of \textsf{kanjiskip} or 
\textsf{xkanjiskip} to \verb+\maxdimen+.

\subsection{Insertion Setting of \textsf{xkanjiskip}}
It is not desirable that \textsf{xkanjiskip} is inserted between every
boundary between \textbf{JAchar}s and \textbf{ALchar}s. For example,
\textsf{xkanjiskip} should not be inserted after opening parenthesis
(\textit{e.g.}, compare `(あ' and `(\hskip\ltjgetparameter{xkanjiskip}あ').

Lua\TeX-ja can control whether \textsf{xkanjiskip} can be inserted
before/after a character, by changing \textsf{jaxspmode} for \textbf{JAchar}s and
\textsf{alxspmode} parameters \textbf{ALchar}s respectively. 
\begin{LTXexample}
\ltjsetparameter{jaxspmode={`あ,preonly}, alxspmode={`\!,postonly}} 
pあq い!う
\end{LTXexample}

The second argument {\tt preonly} means `the insertion of
\textsf{xkanjiskip} is allowed before this character, but not after'.
the other possible values are {\tt postonly}, {\tt allow} and {\tt
inhibit}. For the compatibility with p\TeX, natural numbers between
0~and~3 are also allowed as the second argument\footnote{But we don't
recommend this: since numbers 1~and~2 have opposite meanings in
\textsf{jaxspmode} and \textsf{alxspmode}.}.

If you want to enable/disable all insertions of \textsf{kanjiskip} and
\textsf{xkanjiskip}, set \textsf{autospacing} and \textsf{autoxspacing}
parameters to {\tt false}, respectively.


\subsection{Shifting Baseline}
To make a match between a Japanese font and an alphabetic font, sometimes
shifting of the baseline of one of the pair is needed. In p\TeX, this is achieved
by setting \verb+\ybaselineshift+ to a non-zero length (the
baseline of alphabetic fonts is shifted below). However, for documents
whose main language is not Japanese, it is good to shift the baseline of
Japanese fonts, but not that of alphabetic fonts.
Because of this, Lua\TeX-ja can independently set the shifting amount
of the baseline of alphabetic fonts (\textsf{yalbaselineshift}
parameter) and that of Japanese fonts (\textsf{yjabaselineshift}
parameter). 

\begin{LTXexample}
\vrule width 150pt height 0.4pt depth 0pt\hskip-120pt
\ltjsetparameter{yjabaselineshift=0pt, yalbaselineshift=0pt}abcあいう
\ltjsetparameter{yjabaselineshift=5pt, yalbaselineshift=2pt}abcあいう
\end{LTXexample}
Here the horizontal line in above is the baseline of a line.

There is an interesting side-effect: characters in different size can be
vertically aligned center in a line, by setting two parameters appropriately.
The following is an example (beware the value is not well tuned):
\begin{LTXexample}
xyz漢字 
{\scriptsize
  \ltjsetparameter{yjabaselineshift=-1pt, 
    yalbaselineshift=-1pt}
  XYZひらがな
}abcかな
\end{LTXexample}


\subsection{`tombow'}
`tombow' is a mark for indicating 4~corners and horizontal/vertical
center of the paper. p\LaTeX and this Lua\TeX-ja support `tombow' by
their kernel. The following steps are needed to typeset tombow:

\begin{enumerate}
\item First, define the banner which will be printed at the upper left
      of the paper. This is done by assigning a token list to
      \verb+\@bannertoken+.

For example, the following sets banner as `{\tt filename (2012-01-01 17:01)}':
\begin{verbatim}
\makeatletter

\hour\time \divide\hour by 60 \@tempcnta\hour \multiply\@tempcnta 60\relax
\minute\time \advance\minute-\@tempcnta
\@bannertoken{%
   \jobname\space(\number\year-\two@digits\month-\two@digits\day
   \space\two@digits\hour:\two@digits\minute)}%
\end{verbatim}

\item ...
\end{enumerate}


\part{Reference}\label{part-ref}
\section{Font Metric and Japanese Font}
\subsection{\texttt{\char92jfont} primitive}
To load a font as a Japanese font, you must use the
\verb+\jfont+ primitive instead of~\verb+\font+, while
\verb+\jfont+ admits the same syntax used in~\verb+\font+. 
Lua\TeX-ja automatically loads \texttt{luaotfload} package,
so TrueType/OpenType fonts with features can be used for Japanese fonts:
\begin{LTXexample}
\jfont\tradgt={file:ipaexg.ttf:script=latn;%
  +trad;jfm=ujis} at 14pt
\tradgt{}当/体/医/区
\end{LTXexample}

Note that the defined control sequence
(\verb+\tradgt+ in the example above) using \verb+\jfont+ is not a
\textit{font\_def} token, hence the input like
\verb+\fontname\tradgt+ causes a error. We denote control sequences which are defined in \verb+\jfont+
by <jfont\_cs>.

\paragraph{Prefix}
Besides \texttt{file:}\ and \texttt{name:}\ prefixes, \texttt{psft:}\ can
be used a prefix in \verb+\jfont+ (and~\verb+\font+) primitive.  Using
this prefix, you can specify a font that has its name only and is not
related to any real font.

Mainly, use of this \texttt{psft:}\ prefix is for using non-embedding `standard' Japanese fonts (Ryumin-Light and GothicBBB-Medium).
歴史

\paragraph{Features}
jfm, jfmvar



\subsection{Structure of JFM file}
A JFM file is a Lua script which has only one function call:
\begin{verbatim}
luatexja.jfont.define_jfm { ... }
\end{verbatim}
Real data are stored in the table which indicated above by
\verb+{ ... }+.  So, the rest of this subsection are devoted to describe the
structure of this table.  Note that all lengths in a JFM file are
floating-point numbers in design-size unit.

\begin{list}{}{\def\makelabel{\ttfamily}\def\{{\char`\{}\def\}{\char`\}}}
\item[dir=<direction>] (required)

The direction of JFM. At the present, only \texttt{'yoko'} is supported.

\item[zw=<length>] (required)

The amount of the length of the `full-width'.

\item[zh=<length>] (required)

\item[kanjiskip=\{<natural>, <stretch>, <shrink>\}] (optional)

This field specifies the `ideal' amount of \textsf{kanjiskip}. As noted
	     in Subsection~\ref{subs-kskip}, if the parameter
	     \textsf{kanjiskip} is \verb+\maxdimen+, the value specified
	     in this field is actually used (if this field is not specified in
	     JFM, it is regarded as 0\,pt). Note that <stretch> and <shrink>
	     fields are in design-size unit too.


\item[xkanjiskip=\{<natural>, <stretch>, <shrink>\}] (optional)

Like the \texttt{kanjiskip} field, this field specifies the `ideal'
	     amount of \textsf{xkanjiskip}.

\end{list}

Besides from above fields, a JFM file have several sub-tables those
indices are natural numbers.  The table indexed by~$i\in\omega$ stores
informations of `character class'~$i$. At least, the character class~0 is
always present, so each JFM file must have a sub-table whose index is
\texttt{[0]}.  Each sub-table (its numerical index is denoted by $i$) has
the following fields:

\begin{list}{}{\def\makelabel{\ttfamily}\def\{{\char`\{}\def\}{\char`\}}}
\item[chars=\{<character>, ...\}] (required except character class~0)

This field is a list of characters which are in this character
	     type~$i$. This field is not required if $i=0$, since all
	     \textbf{JAchar} which are not in any character class other
	     than 0 (hence, the character class~0 contains most of
	     \textbf{JAchar}s). In the list, a character can be
	     specified by its code number, or by the character itself
	     (as a string of length~1). 

In addition to those `real' characters, the following `imaginary
	     characters' can be specified in the list:

\item[width=<length>, height=<length>, depth=<length>, italic=<length>]\ (required)

Specify width of characters in character class~$i$, height, depth and
the amount of italic correction. All characters in character class~$i$ are regarded that its width, height and depth are
as values of these fields. 
But there is one exception: if \texttt{'prop'} is specified in \texttt{width} field, width of a character becomes that of its `real' glyph 

\item[left=<length>, down=<length>, align=<align>]\ 

These fields are for adjusting the position of the `real' glyph. Legal
	     values of \texttt{align} field are \texttt{'left'},
	     \texttt{'middle'} and \texttt{'right'}. If one of these
	     3~fields are omitted, \texttt{left} and \texttt{down} are
	     treated as~0, and \texttt{align} field is treated as
	     \texttt{'left'}.
The effects of these 3~fields are indicated in Figure~\ref{fig-pos}.

In most cases, \texttt{left} and \texttt{down} fields are~0, while
it is not uncommon that the \texttt{align} field is \texttt{'middle'} or \texttt{'right'}.
For example, setting the \texttt{align} field to \texttt{'right'} is practically needed 
when the current character class is the class for opening delimiters'. 
\begin{figure}[tb]
\begin{minipage}{0.4\textwidth}%
\begin{center}\unitlength=10pt\small
\begin{picture}(15,12)(-1,-4)
\color{black!10!white}% real glyph :step1
\put(0,0){\vrule width 12\unitlength height 8\unitlength depth 3\unitlength}

\color{red!20!white}% real glyph :step1
\put(-1,-1.5){\vrule width 6\unitlength height 7\unitlength depth 2.5\unitlength}

\color{red}% real glyph
\thicklines
\put(-1,-1.5){\vector(0,1){7}\vector(0,-1){2.5}\vector(1,0){6}}
\put(5,-1.5){\line(0,1){7}\line(0,-1){2.5}}
\put(-1,5.5){\line(1,0){6}}
\put(-1,-4){\line(1,0){6}}

\color{green!20!white}% real glyph :step1
\put(3,0){\vrule width 6\unitlength height 7\unitlength depth 2.5\unitlength}

\color{black}% real glyph :step1
\thicklines
\put(0,0){\vector(0,1){8}\line(0,-1){3}\vector(1,0){12}}
\put(12,0){\line(0,1){8}\vector(0,-1){3}}
\put(0,8){\line(1,0){12}}
\put(0,-3){\line(1,0){12}}
\put(0.2,4){\makebox(0,0)[l]{\texttt{height}}}
\put(12.2,-1.5){\makebox(0,0)[l]{\texttt{depth}}}
\put(6,0.2){\makebox(0,0)[b]{\texttt{width}}}

\color{green!50!black}% real glyph :step1
\thicklines
\put(3,0){\vector(0,1){7}\vector(0,-1){2.5}\vector(1,0){6}}
\put(9,0){\line(0,1){7}\line(0,-1){2.5}}
\put(3,7){\line(1,0){6}}
\put(3,-2.5){\line(1,0){6}}
\newsavebox{\eqdist}
\savebox{\eqdist}(0,0)[b]{%
  \thinlines
  \put(-0.08,0.2){\line(0,-1){0.4}}%
  \put(0.08,0.2){\line(0,-1){0.4}}}
\put(1.5,0){\usebox{\eqdist}}
\put(10.5,0){\usebox{\eqdist}}

\color{blue}% shifted
\thicklines
\put(3,-1.5){\vector(-1,0){4}}
\put(1,-1.7){\makebox(0,0)[t]{\texttt{left}}}
\put(3,0){\vector(0,-1){1.5}}
\put(3.2,-0.75){\makebox(0,0)[l]{\texttt{down}}}
\end{picture} 
\end{center}
\end{minipage}%
\begin{minipage}{0.6\textwidth}%
Consider a node containing Japanese character whose value of the \texttt{align} 
field is \texttt{'middle'}.
\begin{itemize}
\item The black rectangle is a frame of the node.
Its width, height and depth are specified by JFM.
\item Since the \texttt{align} field is \texttt{'middle'}, 
the `real' glyph is centered horizontally (the green rectangle).
\item Furthermore, the glyph is shifted according to values of fields
      \texttt{left} and \texttt{down}. The ultimate position of the real
      glyph is indicated by the red rectangle.
\end{itemize}
\end{minipage}
\caption{The position of the `real' glyph}
\label{fig-pos}
\end{figure}


\item[kern={\{[$j$]=<kern>, ...\}}]

\item[glue={\{[$j$]=\{<width>, <stretch>, <shrink>\}, ...\}}]
\end{list}

\subsection{Math Font Family}
\TeX\ handles fonts in math formulas by 16~font families\footnote{Omega,
Aleph, Lua\TeX~and $\varepsilon$-(u)p\TeX can handles 256~families, but
an external package is needed to support this in plain \TeX\ and
\LaTeX.}, and each family has three fonts:
\verb+\textfont+, \verb+\scriptfont+ and \verb+\scriptscriptfont+.

Lua\TeX-ja's handling of Japanese fonts in math formulas is similar;
Table~\ref{tab-math} shows counterparts to \TeX's primitives for math
font families.

\begin{table}[tb]
\label{tab-math}
\caption{Primitives for Japanese math fonts}
\begin{center}\def\{{\char`\{}\def\}{\char`\}}
\begin{tabular}{lll}
\toprule
&Japanese fonts&alphabetic fonts\\
font family&\verb+\jfam+${}\in [0,256)$&\verb+\fam+\\
text size&\tt\textsf{jatextfont}\,=\{<jfam>,<jfont\_cs>\}&\tt\verb+\textfont+<fam>=<font\_cs>\\
script size&\tt\textsf{jascriptfont}\,=\{<jfam>,<jfont\_cs>\}&\tt\verb+\scriptfont+<fam>=<font\_cs>\\
scriptscript size&\tt\textsf{jascriptscriptfont}\,=\{<jfam>,<jfont\_cs>\}&\tt\verb+\scriptscriptfont+<fam>=<font\_cs>\\
\bottomrule
\end{tabular}
\end{center}
\end{table}


\section{Parameters}
\subsection{{\tt\char92 ltjsetparameter} primitive}
As noted before, \verb+\ltjsetparameter+ and \verb+\ltjgetparameter+ are
primitives for accessing most parameters of Lua\TeX-ja. One of the main
reason that Lua\TeX-ja didn't adopted the syntax similar to that of p\TeX\ 
(\textit{e.g.},~\verb+\prebreakpenalty`)=10000+) 
is the position of \verb+hpack_filter+ callback in the source
of Lua\TeX, see Section~\ref{sec-para}.

\verb+\ltjsetparameter+ and \verb+\ltjglobalsetparameter+ are primitives
for assigning parameters. These take one argument which is a
\texttt{<key>=<value>} list. Allowed keys are described in the next
subsection.  
The difference between
\verb+\ltjsetparameter+ and \verb+\ltjglobalsetparameter+ is only the
scope of assignment;
\verb+\ltjsetparameter+ does a local assignment and 
\verb+\ltjglobalsetparameter+ does a global one. 
They also obey the value of \verb+\globaldefs+,
like other assignment.

\verb+\ltjgetparameter+ is the primitive for acquiring parameters. It
always takes a parameter name as first argument, and also takes the
additional argument---a character code, for example---in some cases.
\begin{LTXexample}
\ltjgetparameter{differentjfm}, 
\ltjgetparameter{autospacing},
\ltjgetparameter{prebreakpenalty}{`)}.
\end{LTXexample}
\emph{The return value of\/ {\normalfont\tt\char92ltjgetparameter} is
always a string}. This is outputted by \texttt{tex.write()}, so any
character other than space~`{\tt\char32}'~(U+0020) has the category code
12~(other), while the space has 10~(space).

\subsection{List of Parameters}
In the following list of parameters, [\verb+\cs+] indicates the counterpart in p\TeX, and each symbol has the following meaning:
\begin{itemize}
\item No mark: values at the end of the paragraph or the hbox are
      adopted in the whole paragraph/hbox.
\item `\ast' : local parameters, which can change everywhere inside a paragraph/hbox.
\item `\dagger': assignments are always global.
\end{itemize}

\begin{list}{}{\def\makelabel{\ttfamily}\def\{{\char`\{}\def\}{\char`\}}}
\item[\textsf{jcharwidowpenalty}\,=<penalty>] [\verb+\jcharwidowpenalty+]

Penalty value for supressing orphans. This penalty is inserted just
	     after the last \textbf{JAchar} which is not regarded as a
	     (Japanese) punctuation mark.

\item[\textsf{kcatcode}\,=\{<chr\_code>,<natural number>\}]\

An additional attributes having each character whose character code is <chr\_code>.  
At the present version, the lowermost bit of <natural number> indicates
	     whether the character is considered as a punctuation mark
	     (see the description of \textsf{jcharwidowpenalty} above).


\item[\textsf{prebreakpenalty}\,=\{<chr\_code>,<penalty>\}] [\verb+\prebreakpenalty+]
\item[\textsf{postbreakpenalty}\,=\{<chr\_code>,<penalty>\}] [\verb+\postbreakpenalty+]
\item[\textsf{jatextfont}\,=\{<jfam>,<jfont\_cs>\}] [\verb+\textfont+ in \TeX]
\item[\textsf{jascriptfont}\,=\{<jfam>,<jfont\_cs>\}] [\verb+\scriptfont+ in \TeX]
\item[\textsf{jascriptscriptfont}\,=\{<jfam>,<jfont\_cs>\}] [\verb+\scriptscriptfont+ in \TeX]
\item[\textsf{yjabaselineshift}\,=<dimen>$^\ast$]\ 
\item[\textsf{yalbaselineshift}\,=<dimen>$^\ast$] [\verb+\ybaselineshift+]

\item[\textsf{jaxspmode}\,=\{<chr\_code>,<mode>\}] [\verb+\inhibitxspcode+]

Setting whether inserting  \textsf{xkanjiskip} is allowed before/after a \textbf{JAchar} whose character code is <chr\_code>.
The followings are allowed for <mode>:
\begin{description}
\item[0, \texttt{inhibit}] Insertion of \textsf{xkanjiskip} is inhibited before the charater, nor after the charater.
\item[2, \texttt{preonly}] Insertion of \textsf{xkanjiskip} is allowed before the charater, but not after.
\item[1, \texttt{postonly}] Insertion of \textsf{xkanjiskip} is allowed after the charater, but not before.
\item[3, \texttt{allow}] Insertion of \textsf{xkanjiskip} is allowed before the charater and after the charater.
This is the default value.
\end{description}

\item[\textsf{alxspmode}\,=\{<chr\_code>,<mode>\}] [\verb+\xspcode+]

Setting whether inserting  \textsf{xkanjiskip} is allowed before/after a \textbf{ALchar} whose character code is <chr\_code>.
The followings are allowed for <mode>:
\begin{description}
\item[0, \texttt{inhibit}] Insertion of \textsf{xkanjiskip} is inhibited before the charater, nor after the charater.
\item[1 \texttt{preonly}] Insertion of \textsf{xkanjiskip} is allowed before the charater, but not after.
\item[2 \texttt{postonly}] Insertion of \textsf{xkanjiskip} is allowed after the charater, but not before.
\item[3, \texttt{allow}] Insertion of \textsf{xkanjiskip} is allowed before the charater and after the charater.
This is the default value.
\end{description}
Note that parameters \textsf{jaxspmode} and \textsf{alxspmode} use a common table.

\item[\textsf{autospacing}\,=<bool>$^\ast$] [\verb+\autospacing+]
\item[\textsf{autoxspacing}\,=<bool>$^\ast$] [\verb+\autoxspacing+]
\item[\textsf{kanjiskip}\,=<skip>] [\verb+\kanjiskip+]
\item[\textsf{xkanjiskip}\,=<skip>] [\verb+\xkanjiskip+]

\item[\textsf{differentjfm}\,=<mode>$^\dagger$]

Specify how glues/kerns between two \textbf{JAchar}s whose JFM (or size) are different. 
The allowed arguments are the followings:
\begin{description}
\item[\texttt{average}]
\item[\texttt{both}]
\item[\texttt{large}]
\item[\texttt{small}]
\end{description}

\item[\textsf{jacharrange}\,=<ranges>$^\ast$]
\item[\textsf{kansujichar}\,=\{<digit>, <chr\_code>\}] [\verb+\kansujichar+]
\end{list}


\section{Other Primitives}
\subsection{Compatibility with p\TeX}
\begin{list}{}{\def\makelabel{\ttfamily\char92 }}
\item[kuten]
\item[jis]
\item[euc]
\item[sjis]
\item[ucs]
\item[kansuji]
\end{list}

\section{Control Sequences for \LaTeXe}
\subsection{Patch for NFSS2}
As described in Subsection~\ref{ssec-ltx}, Lua\TeX-ja simply adopted \texttt{plfonts.dtx} in p\LaTeXe for the Japanese patch for NFSS2. 

\subsection{`tombow'}

\part{Implementations}\label{part-imp}
\section{Storing Parameters}\label{sec-para}
\subsection{Used Dimensions and Attributes}
Here the following is the list of dimension and attributes which are used in Lua\TeX-ja.
\begin{list}{}{%
\def\makelabel{\ttfamily}
\def\dim#1{\item[\char92 #1\ \textrm{(dimension)}]}
\def\attr#1{\item[\char92 #1\ \textrm{(attribute)}]}
}

\dim{jQ}
As explained in Subsection~\ref{ssec-plain}, \verb+\jQ+ is equal to
			$1\,\textrm{Q}=0.25\,\textrm{mm}$, where `Q'~(also called `級') is
			a unit used in Japanese phototypesetting. So one should not change the value of this dimension. 
\dim{jH}
There is also a unit called `歯' which equals to $0.25\,\textrm{mm}$ and
			used in Japanese phototypesetting. The dimension
			\verb+\jH+ stores this length, similar to \verb+\jQ+. 
\dim{ltj@zw} A temporal register for the `full-width' of current Japanese font.
\dim{ltj@zh} A temporal register for the `full-height' (usually the sum of height of imaginary body and its depth) of current Japanese font.
\attr{jfam} Current number of Japanese font family for math formulas.
\attr{ltj@curjfnt} The font index of current Japanese font.
\attr{ltj@charclass} The character class of Japanese \textit{glyph\_node}.
\attr{ltj@yablshift} The amount of shifting the baseline of alphabetic
			fonts in scaled point ($2^{-16}\,\textrm{pt}$).
\attr{ltj@ykblshift} The amount of shifting the baseline of Japanese
			fonts in scaled point ($2^{-16}\,\textrm{pt}$).
\attr{ltj@autospc} Whether the auto insertion of \textsf{kanjiskip} is allowed at the node.
\attr{ltj@autoxspc} Whether the auto insertion of \textsf{xkanjiskip} is allowed at the node.
\attr{ltj@icflag} For distinguishing `kinds' of the node. To this
			attribute, one of the following value is
			assigned:
\begin{description}
\item[ITALIC (1)] Glues from an itaric correction
	   (\verb+\/+). This distinction of origins of glues 
	   (from explicit \verb+\kern+, or from \verb+\/+)
	   is needed in the insertion process of \textsf{xkanjiskip}.
\item[PACKED (2)] 
\item[KINSOKU (3)] Penalties inserted for the word-wrapping  process of Japanese characters (\emph{kinsoku}).
\item[FROM\_JFM (4)] Glues/kerns from JFM.
\item[LINE\_END (5)] Kerns for ...
\item[KANJI\_SKIP (6)] Glues for \textsf{kanjiskip}.
\item[XKANJI\_SKIP (7)] Glues for \textsf{xkanjiskip}.
\item[PROCESSED (8)] Nodes which is already processed by ...
\item[IC\_PROCESSED (9)] Glues from an itaric correction, but also already processed.
\item[BOXBDD (15)] Glues/kerns that inserted just the beginning or the ending of an hbox or a paragraph.
\end{description}
\attr{ltj@kcat$i$} Where $i$~is a natural number which is less than~7.
These 7~attributes store bit~vectors indicating which character block is regarded as a block of \textbf{JAchar}s.
\end{list}

\subsection{Stack System of Lua\TeX-ja}
\paragraph{Background}
Lua\TeX-ja has its own stack system, and most parameters of Lua\TeX-ja
are stored in it.  To clarify the reason, imagine the parameter
\textsf{kanjiskip} is stored by a skip, and consider the following
source:
\begin{LTXexample}
\ltjsetparameter{kanjiskip=0pt}ふがふが.%
\setbox0=\hbox{\ltjsetparameter{kanjiskip=5pt}ほげほげ}
\box0.ぴよぴよ\par
\end{LTXexample}

As described in Part~\ref{part-ref}, the only effective value of
\textsf{kanjiskip} in an hbox is the latest value, so the value of
\textsf{kanjiskip} which applied in the entire hbox should be 5\,pt.
However, by the implementation method of Lua\TeX, this `5\,pt' cannot be
known from any callbacks.  In the \texttt{tex/packaging.w} (which is a
file in the source of Lua\TeX), there are the following codes:
\begin{lstlisting}
void package(int c)
{
    scaled h;                   /* height of box */
    halfword p;                 /* first node in a box */
    scaled d;                   /* max depth */
    int grp;
    grp = cur_group;
    d = box_max_depth;
    unsave();
    save_ptr -= 4;
    if (cur_list.mode_field == -hmode) {
        cur_box = filtered_hpack(cur_list.head_field,
                                 cur_list.tail_field, saved_value(1),
                                 saved_level(1), grp, saved_level(2));
        subtype(cur_box) = HLIST_SUBTYPE_HBOX;
\end{lstlisting}
Notice that \verb+unsave+ is executed \emph{before}
\verb+filtered_hpack+ (this is where \verb+hpack_filter+ callback is
executed): so `5\,pt' in the above source is orphaned at
\texttt+unsave+, and hence it can't be accessed from \verb+hpack_filter+
callback.

\paragraph{The method}
The code of stack system is based on that in a post of Dev-luatex mailing list\footnote{%
\texttt{[Dev-luatex] tex.currentgrouplevel}, a post at 2008/8/19 by Jonathan Sauer.}.

These are two \TeX\ count registers for maintaining informations:
\verb+\ltj@@stack+ for the stack level, and \verb+\ltj@@group@level+ for
the \TeX's group level when the last assignment was done.  Parameters
are stored in one big table named \texttt{charprop\_stack\_table}, where
\texttt{charprop\_stack\_table[$i$]} stores data of stack level~$i$. If
a new stack level is created by \verb+\ltjsetparameter+, all data of the
previous level is copied.

To resolve the problem mentioned in `Background' above, Lua\TeX-ja uses
another thing: When a new stack level is about to be created, a whatsit
node whose type, subtype and value are 44~(\textit{user\_defined}),
30112, and current group level respectively is appended to the current
list (we refer this node by \textit{stack\_flag}). This enables us to
know whether assignment is done just inside a hbox. Suppose that the
stack level is~$s$ and the \TeX's group level is~$t$ just after the hbox
group, then:
\begin{itemize}
\item If there is no \textit{stack\_flag} node in the list of hbox, then
      no assignment was occurred inside the hbox. Hence values of
      parameters at the end of the hbox are stored in the stack
      level~$s$.
\item If there is a \textit{stack\_flag} node whose value is~$t+1$, then
      an assignment was occurred just inside the hbox group. Hence
      values of parameters at the end of the hbox are stored in the
      stack level~$s+1$.
\item If there are \textit{stack\_flag} nodes but all of their values
      are more than~$t+1$, then an assignment was occurred in the box,
      but it is done is `more internal' group. Hence values of
      parameters at the end of the hbox are stored in the stack
      level~$s$.
\end{itemize}

Note that to work this trick correctly, assignments to
\verb+\ltj@@stack+ and \verb+\ltj@@group@level+ have to be local always, 
regardless the value of \verb+\globaldefs+.
This problem is resolved by using
\hbox{\verb+\directlua{tex.globaldefs=0}+} (this assignment is local).


\section{Linebreak after Japanese Character}\label{sec-lbreak}
\subsection{Reference: Behavior in p\TeX}
(NOT COMPLETED)

In~p\TeX, a linebreak after a Japanese character doesn't emit a space,
since words are not separated by spaces in Japanese writings. However,
this feature isn't fully implemented in Lua\TeX-ja due to the
specification of callbacks in~Lua\TeX. To clarify the difference between
p\TeX~and~Lua\TeX, We briefly describe the handling of a linebreak in~p\TeX, in
this subsection. 

p\TeX's input processor can be described in terms of a finite state
automaton, as that of~\TeX\ in~Section~2.5 of~\cite{texbytopic}. The
internal states are as follows:
\begin{itemize}
\item State~$N$: new line
\item State~$S$: skipping spaces
\item State~$M$: middle of line
\item State~$K$: after a Japanese character
\end{itemize}
The first three states---$N$, $S$~and~$M$---are as same as \TeX's input
processor.  State~$K$ is similar to state~$M$, and is entered after
Japanese characters.  The diagram of state transitions are indicated in
Figure~\ref{fig-ptexipro}.  Note that p\TeX\ doesn't leave state~$K$
after `beginning/ending of a group' characters.
\begin{figure}[tb]
\label{fig-ptexipro}
\begin{gather*}
 \def\sp{\text{\tt\char32}}
 \xymatrix{&&
   {\text{scan a cs}}\ar@(r,ul)[dr]&\\
\ar[r]&
   *++[o][F-]{N}\ar[ur]^0\ar[dd]_{d,\ g}\ar[u]^{5\ (\text{\tt\char92par})}
     \ar@{->}@(d,l)[ddrr]_(0.45){j}&&
   *++[o][F-]{S}\ar@(l,dr)[ul]^0\ar@(l,ur)[ddll]_{d,\ g}\ar[u]_{5}
     \ar@{->}@(r,r)[dd]^{j}\\&\\&
   *++[o][F-]{M}\ar[uuur]^0\ar@(r,dl)[uurr]_(0.55){10\ (\sp)}
     \ar[d]_{5\ ({\sp})}\ar@{->}@(dr,dl)[rr]_{j}&&
   *++[o][F-]{K}\ar@{->}@(ul,d)[uuul]^0\ar@{->}[ll]^{d}
     \ar@{->}@(ur,dr)[uu]^{10\ (\sp)}\ar@{->}[d]_5\\
   &&&
 }\\
 d:=\{3,4,6,7,8,11,12,13\},\quad g:=\{1,2\},\quad j:=(\text{Japanese characters})
\end{gather*}
\begin{itemize}
\item Numbers represent category codes.
\item Category codes 9~(ignored), 14~(comment)~and~15~(invalid) are omitted in above diagram.
\end{itemize}
\caption{State transitions of p\TeX's input processor}
\end{figure}


\subsection{Behavior in Lua\TeX-ja}
States in the input processoe of Lua\TeX\ is the same as that of \TeX,
and they can't be customized by any callbacks. Hence, we can only use
\verb+process_input_buffer+ and \verb+token_filter+ callbacks for to
suppress a space by a linebreak which is after Japanese characters.

However, \verb+token_filter+ callback cannot be used either, since a
character in category code 5~(end-of-line) is converted into an space
token \emph{in the input processor}.  So we can use only the
\verb+process_input_buffer+ callback.  This means that suppressing a
space must be done \emph{just before} an input line is read.

Considering these situations, handling of a end-of-line in Lua\TeX-ja are as follows:
\begin{quote}
A character U+FFFFF (its category code is set to 14~(comment) by
Lua\TeX-ja) is appended to an input line, before Lua\TeX\ actually
process it, if and only if the following two conditions are satisfied:
\begin{enumerate}
\item The category code of the character $\langle${return}$\rangle$
      (whose character code is 13) is 5~(end-of-line).
\item The input line matches the following `regular expression':
\[
  (\text{any char})^*(\textbf{JAchar})
  \bigl(\{\text{catcode}=1\}\cup\{\text{catcode}=2\}\bigr)^*
\]
\end{enumerate}

\end{quote}

\section{Insertion of JFM glues, \textsf{kanjiskip} and \textsf{xkanjiskip}}
This is the longest section of the document.

jfmglue.tex の内容をここに入れる
\end{document}
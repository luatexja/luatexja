% \iffalse meta-comment
%% File: ltjltxdoc.dtx
%  -------------------------------------
%  Original: jltxdoc.dtx
%
%  Copyright 1995,1996,1997  ASCII Corporation.
%
%  This file is part of pLaTeX2e system.
%  -------------------------------------
%
% \fi
%
% \setcounter{StandardModuleDepth}{1}
% \StopEventually{}
%
% \iffalse
% \changes{v1.0a}{1997/01/23}{\LaTeX \texttt{!<1996/12/01!>}への対応に
%     合わせて修正}
% \changes{v1.0b}{1997/07/29}{\cs{}と\texttt{"}の\cs{xspcode}を変更}
% \changes{v1.0b-ltj}{2011/09/27}{Lua\LaTeX-ja 用に修正}
% \changes{v1.0b-ltj-2}{2014/07/03}{orを意味する縦棒の出力が異常だったので修正}
% \changes{v1.0c}{2016/07/25}{docパッケージが上書きする\cs{verb}を再々定義}
% \changes{v1.0d}{2017/09/24}{\cs{vadjust\{\}}を追加}
% \changes{v1.0d-ltj-3}{2018/01/01}{\cs{Cjascale}を追加}
% \fi
%
% \iffalse
%<class>
%<class>\NeedsTeXFormat{LaTeX2e}
%<class>\ProvidesClass{ltjltxdoc}[2018/01/01 v1.0d-ltj-3 Standard LuaLaTeX-ja file]
%<*driver>
\documentclass{ltjltxdoc}
\GetFileInfo{ltjltxdoc.cls}
\usepackage[kozuka-pr6n]{luatexja-preset}
\usepackage{unicode-math}
\setmathfont{Latin Modern Math}
\title{Lua\LaTeX-jaドキュメント記述用クラス}
\author{Lua\TeX-jaプロジェクト}
\date{\filedate}
\begin{document}
   \maketitle
   \DocInput{ltjltxdoc.dtx}
\end{document}
%</driver>
% \fi
%
% \file{ltjltxdoc}クラスは、\file{ltxdoc}をテンプレートにして、日本語用の
% 修正を加えています。
%    \begin{macrocode}
%<*class>
\DeclareOption*{\PassOptionsToClass{\CurrentOption}{ltxdoc}}
\ProcessOptions
\LoadClass{ltxdoc}
%    \end{macrocode}
% \file{ltxdoc}の読み込み後に\file{luatexja}を読み込みます。
%
% \changes{v1.60d-ltj-3}{2018/01/01}{\cs{Cjascale}を追加しました。
% これは、コミュニティ版p\kern-.05em\LaTeX で導入された、
% 和文スケール($1\,\cs{zw} \div \hbox{要求サイズ}$)を表す実数値マクロです。}
%
%    \begin{macrocode}
\RequirePackage{luatexja}
\def\Cjascale{0.962216}
%    \end{macrocode}
%
% \begin{macro}{\normalsize}
% \begin{macro}{\small}
% \begin{macro}{\parindent}
% \changes{v1.0a}{1997/01/23}{\cs{normalsize}, \cs{small}などの再定義}
% \file{ltxdoc}からロードされる\file{article}クラスでの行間などの設定値で、
% 日本語の文章を組版すると、行間が狭いように思われるので、多少広くするように
% 再設定します。また、段落先頭での字下げ量を全角一文字分とします。
%    \begin{macrocode}
\renewcommand{\normalsize}{%
    \@setfontsize\normalsize\@xpt{15}%
  \abovedisplayskip 10\p@ \@plus2\p@ \@minus5\p@
  \abovedisplayshortskip \z@ \@plus3\p@
  \belowdisplayshortskip 6\p@ \@plus3\p@ \@minus3\p@
   \belowdisplayskip \abovedisplayskip
   \let\@listi\@listI}
\renewcommand{\small}{%
  \@setfontsize\small\@ixpt{11}%
  \abovedisplayskip 8.5\p@ \@plus3\p@ \@minus4\p@
  \abovedisplayshortskip \z@ \@plus2\p@
  \belowdisplayshortskip 4\p@ \@plus2\p@ \@minus2\p@
  \def\@listi{\leftmargin\leftmargini
              \topsep 4\p@ \@plus2\p@ \@minus2\p@
              \parsep 2\p@ \@plus\p@ \@minus\p@
              \itemsep \parsep}%
  \belowdisplayskip \abovedisplayskip}
\normalsize
\setlength\parindent{1\zw}
%    \end{macrocode}
% \end{macro}
% \end{macro}
% \end{macro}
%
% \begin{macro}{\file}
% |\file|マクロは、ファイル名を示すのに用います。
%    \begin{macrocode}
\providecommand*{\file}[1]{\texttt{#1}}
%    \end{macrocode}
% \end{macro}
%
% \begin{macro}{\pstyle}
% |\pstyle|マクロは、ページスタイル名を示すのに用います。
%    \begin{macrocode}
\providecommand*{\pstyle}[1]{\textsl{#1}}
%    \end{macrocode}
% \end{macro}
%
% \begin{macro}{\Lcount}
% |\Lcount|マクロは、カウンタ名を示すのに用います。
%    \begin{macrocode}
\providecommand*{\Lcount}[1]{\textsl{\small#1}}
%    \end{macrocode}
% \end{macro}
%
% \begin{macro}{\Lopt}
% |\Lopt|マクロは、クラスオプションやパッケージオプションを示すのに用います。
%    \begin{macrocode}
\providecommand*{\Lopt}[1]{\textsf{#1}}
%    \end{macrocode}
% \end{macro}
%
% \begin{macro}{\dst}
% |\dst|マクロは、``\dst''を出力する。
%    \begin{macrocode}
\providecommand\dst{{\normalfont\scshape docstrip}}
%    \end{macrocode}
% \end{macro}
%
% \begin{macro}{\NFSS}
% |\NFSS|マクロは、``\NFSS''を出力します。
%    \begin{macrocode}
\providecommand\NFSS{\textsf{NFSS}}
%    \end{macrocode}
% \end{macro}
%
% \begin{macro}{\c@clineno}
% \begin{macro}{\mlineplus}
% |\mlineplus|マクロは、その時点でのマクロコードの行番号に、引数に指定された
% 行数だけを加えた数値を出力します。たとえば|\mlineplus{3}|とすれば、
% 直前のマクロコードの行番号(\arabic{CodelineNo})に3を加えた数、
% ``\mlineplus{3}''が出力されます。
%    \begin{macrocode}
\newcounter{@clineno}
\def\mlineplus#1{\setcounter{@clineno}{\arabic{CodelineNo}}%
   \addtocounter{@clineno}{#1}\arabic{@clineno}}
%    \end{macrocode}
% \end{macro}
% \end{macro}
%
% \begin{environment}{tsample}
% |tsample|環境は、環境内に指定された内容を罫線で囲って出力をします。
% 第一引数は、出力するボックスの高さです。
% このマクロ内では縦組になることに注意してください。
%    \begin{macrocode}
\def\tsample#1{%
  \hbox to\linewidth\bgroup\vrule width.1pt\hss
    \vbox\bgroup\hrule height.1pt
      \vskip.5\baselineskip
      \vbox to\linewidth\bgroup\tate\hsize=#1\relax\vss}
\def\endtsample{%
      \vss\egroup
      \vskip.5\baselineskip
    \hrule height.1pt\egroup
  \hss\vrule width.1pt\egroup}
%    \end{macrocode}
% \end{environment}
%
% \begin{macro}{\verb}
% p\LaTeX{}では、|\verb|コマンドを修正して直前に|\xkanjiskip|が入るように
% しています。しかし、\file{ltxdoc.cls}が読み込む\file{doc.sty}が上書き
% してしまいますので、これを再々定義します。\file{doc.sty}での定義は
%\begin{verbatim}
%   \def\verb{\relax\ifmmode\hbox\else\leavevmode\null\fi
%     \bgroup \let\do\do@noligs \verbatim@nolig@list
%       \ttfamily \verb@eol@error \let\do\@makeother \dospecials
%       \@ifstar{\@sverb}{\@vobeyspaces \frenchspacing \@sverb}}
%\end{verbatim}
% となっていますので、\file{plcore.dtx}と同様に|\null|を外して|\vadjust{}|を
% 入れます。
% \changes{v1.0c}{2016/07/25}{docパッケージが上書きする\cs{verb}を再々定義}
% \changes{v1.0d}{2017/09/24}{\cs{vadjust\{\}}を追加}
%    \begin{macrocode}
\def\verb{\relax\ifmmode\hbox\else\leavevmode\vadjust{}\fi
  \bgroup \let\do\do@noligs \verbatim@nolig@list
    \ttfamily \verb@eol@error \let\do\@makeother \dospecials
    \@ifstar{\@sverb}{\@vobeyspaces \frenchspacing \@sverb}}
%    \end{macrocode}
% \end{macro}
%
% \begin{macro}{alxspmode}
% コマンド名の|\|と16進数を示すための|"|の前にもスペースが入るよう、
% これらの|alxspmode|の値を変更します。
% \changes{v1.0b}{1997/07/29}{\cs{}と\texttt{"}の\cs{xspcode}を変更}
% \changes{v1.0b-ltj}{2011/09/27}{\cs{xspcode}→\cs{ltjsetparameter{alxspmode={...}}}}
%    \begin{macrocode}
\ltjsetparameter{alxspmode={"5C,3}} %% \
\ltjsetparameter{alxspmode={"22,3}} %% "
%</class>
%    \end{macrocode}
% \end{macro}
%
% \begin{macro}{mod@math@codes}
% docパッケージでは,ドライバ指定の表示の部分における\texttt{\char`\|}の
% \cs{mathcode}は\texttt{"226A}になっており,これにより\texttt{\char`\|}が小文字のjで表示されて
% しまう状況になっています.改善するため,\texttt{"207C}に変更します.
%    \begin{macrocode}
\def\mod@math@codes{\mathcode`\|="207C \mathcode`\&="2026
                    \mathcode`\-="702D \mathcode`\+="702B
                    \mathcode`\:="703A \mathcode`\=="703D }
%    \end{macrocode}
% \end{macro}
% \Finale
%
\endinput
